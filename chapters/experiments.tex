% !TEX root = ../main.tex

\chapter{Experiments} % (fold)
\label{cha:Experiments}

\section{DeepWalk}

Deepwalk is an algorithm to learn latent representation in a network.
It is the first of its kind to take successful techniques usually reserved for natural language processing and apply them to network analysis.

To learn the features and relations between vertices, the algorithm walks randomly through a network.
Deepwalk is a form of unsupervised learning: the model learns independently from the labels, if available.
This makes it a powerful algorithm for finding latents representation in very large networks even with sparse data.

Deepwalk is based primarily on the SkipGram model, often referred to as \textit{word2vec}. 
\textit{Word2vec} is a technique created for natural language processing published in 2013.
It is used to perform word embeddings: in a higher dimensional space, words of similar meaning and used in similar context are grouped together, away from the words used in a different context.
Here, Deepwalk applies this technique to learn the relationship between nodes in a network.
The idea behind it is to treat a network as a document.
Since we're working with network, the SkipGram model is trained with each random walk performed.
In natural language processing, \textit{word2vec} represents words with similar meaning close to another in the resulting embedding space; in Deepwalk, this will be the case for nodes with similar features.

Deepwalk has four main hyperparameters which can be tuned to obtain a better representation of the network.
We can choose the size of the context of a SkipGram model: in NLP, this is the number of words included before and a after a word to then chose the next one.
With a network, this will be the number of neighboring vertices to take into account for the next random walk.

In addition, the output embedding can be tuned via the embedding size \textit{d}.
This is the number of latent dimensions we want to have for a graph.
A higher or lower dimensionality can result in a better representation of the network in the embedding space.
However more dimensions does not necessarily mean better results.
It is important to find the right hyperparameter which leads to the best results.

We can set the number of random walk started at each vertex with the parameter \textit{\gamma}, and the length of each random walk with the parameter \textit{t}.
Through these parameters, we can tune DeepWalk's random walk behaviour. 
Depending on the network, we might prefer DeepWalk to do numerous short walk or few long ones.

\subsection{DeepWalk With Disinformation Dataset and Parameters Sensitivity}

We used DeepWalk algorithm on our disinformation dataset.
For this experiment we perform the graph embedding on the adjacency list discussed in subsection~\ref{sub:h2h}.
In a text, more context helps the algorithm to better understand the meaning of a word and how to place them in the embedding space.
Similarly, more vertices helps gain information on the network underlying function.
For this reason, we use the entire network data to perform the embedding.

The resulting embedding representation is then cleaned from users and sources entities so that we score only the news articles.
To transform the network to homogeneous, we assigned a unique index for each entity and mapped it to the graph's edge list.
Conveniently, we can reuse this mapping to now remove all indexes which are not news entities.

\begin{figure}[h!]
    \centering
    \begin{subfigure}[b]{0.49\textwidth}
        \centering
        \scalebox{.5}{\input{figures/plots/deepwalk/deepwalk-plot-dimensions-window-size.pgf}}
        \label{fig:deepwalk:representation window size}
    \end{subfigure}
    \hfill
    \begin{subfigure}[b]{0.49\textwidth}
        \centering
        \scalebox{.5}{%% Creator: Matplotlib, PGF backend
%%
%% To include the figure in your LaTeX document, write
%%   \input{<filename>.pgf}
%%
%% Make sure the required packages are loaded in your preamble
%%   \usepackage{pgf}
%%
%% Also ensure that all the required font packages are loaded; for instance,
%% the lmodern package is sometimes necessary when using math font.
%%   \usepackage{lmodern}
%%
%% Figures using additional raster images can only be included by \input if
%% they are in the same directory as the main LaTeX file. For loading figures
%% from other directories you can use the `import` package
%%   \usepackage{import}
%%
%% and then include the figures with
%%   \import{<path to file>}{<filename>.pgf}
%%
%% Matplotlib used the following preamble
%%   
%%   \makeatletter\@ifpackageloaded{underscore}{}{\usepackage[strings]{underscore}}\makeatother
%%
\begingroup%
\makeatletter%
\begin{pgfpicture}%
\pgfpathrectangle{\pgfpointorigin}{\pgfqpoint{5.590148in}{6.909803in}}%
\pgfusepath{use as bounding box, clip}%
\begin{pgfscope}%
\pgfsetbuttcap%
\pgfsetmiterjoin%
\definecolor{currentfill}{rgb}{1.000000,1.000000,1.000000}%
\pgfsetfillcolor{currentfill}%
\pgfsetlinewidth{0.000000pt}%
\definecolor{currentstroke}{rgb}{1.000000,1.000000,1.000000}%
\pgfsetstrokecolor{currentstroke}%
\pgfsetdash{}{0pt}%
\pgfpathmoveto{\pgfqpoint{0.000000in}{0.000000in}}%
\pgfpathlineto{\pgfqpoint{5.590148in}{0.000000in}}%
\pgfpathlineto{\pgfqpoint{5.590148in}{6.909803in}}%
\pgfpathlineto{\pgfqpoint{0.000000in}{6.909803in}}%
\pgfpathlineto{\pgfqpoint{0.000000in}{0.000000in}}%
\pgfpathclose%
\pgfusepath{fill}%
\end{pgfscope}%
\begin{pgfscope}%
\pgfsetbuttcap%
\pgfsetmiterjoin%
\definecolor{currentfill}{rgb}{0.917647,0.917647,0.949020}%
\pgfsetfillcolor{currentfill}%
\pgfsetlinewidth{0.000000pt}%
\definecolor{currentstroke}{rgb}{0.000000,0.000000,0.000000}%
\pgfsetstrokecolor{currentstroke}%
\pgfsetstrokeopacity{0.000000}%
\pgfsetdash{}{0pt}%
\pgfpathmoveto{\pgfqpoint{0.667731in}{4.015734in}}%
\pgfpathlineto{\pgfqpoint{5.410148in}{4.015734in}}%
\pgfpathlineto{\pgfqpoint{5.410148in}{6.729803in}}%
\pgfpathlineto{\pgfqpoint{0.667731in}{6.729803in}}%
\pgfpathlineto{\pgfqpoint{0.667731in}{4.015734in}}%
\pgfpathclose%
\pgfusepath{fill}%
\end{pgfscope}%
\begin{pgfscope}%
\definecolor{textcolor}{rgb}{0.150000,0.150000,0.150000}%
\pgfsetstrokecolor{textcolor}%
\pgfsetfillcolor{textcolor}%
\pgftext[x=1.141973in,y=3.883790in,,top]{\color{textcolor}\sffamily\fontsize{11.000000}{13.200000}\selectfont 10}%
\end{pgfscope}%
\begin{pgfscope}%
\definecolor{textcolor}{rgb}{0.150000,0.150000,0.150000}%
\pgfsetstrokecolor{textcolor}%
\pgfsetfillcolor{textcolor}%
\pgftext[x=2.090456in,y=3.883790in,,top]{\color{textcolor}\sffamily\fontsize{11.000000}{13.200000}\selectfont 20}%
\end{pgfscope}%
\begin{pgfscope}%
\definecolor{textcolor}{rgb}{0.150000,0.150000,0.150000}%
\pgfsetstrokecolor{textcolor}%
\pgfsetfillcolor{textcolor}%
\pgftext[x=3.038940in,y=3.883790in,,top]{\color{textcolor}\sffamily\fontsize{11.000000}{13.200000}\selectfont 40}%
\end{pgfscope}%
\begin{pgfscope}%
\definecolor{textcolor}{rgb}{0.150000,0.150000,0.150000}%
\pgfsetstrokecolor{textcolor}%
\pgfsetfillcolor{textcolor}%
\pgftext[x=3.987423in,y=3.883790in,,top]{\color{textcolor}\sffamily\fontsize{11.000000}{13.200000}\selectfont 60}%
\end{pgfscope}%
\begin{pgfscope}%
\definecolor{textcolor}{rgb}{0.150000,0.150000,0.150000}%
\pgfsetstrokecolor{textcolor}%
\pgfsetfillcolor{textcolor}%
\pgftext[x=4.935906in,y=3.883790in,,top]{\color{textcolor}\sffamily\fontsize{11.000000}{13.200000}\selectfont 80}%
\end{pgfscope}%
\begin{pgfscope}%
\definecolor{textcolor}{rgb}{0.150000,0.150000,0.150000}%
\pgfsetstrokecolor{textcolor}%
\pgfsetfillcolor{textcolor}%
\pgftext[x=3.038940in,y=3.693049in,,top]{\color{textcolor}\sffamily\fontsize{12.000000}{14.400000}\selectfont random walk length}%
\end{pgfscope}%
\begin{pgfscope}%
\pgfpathrectangle{\pgfqpoint{0.667731in}{4.015734in}}{\pgfqpoint{4.742417in}{2.714069in}}%
\pgfusepath{clip}%
\pgfsetroundcap%
\pgfsetroundjoin%
\pgfsetlinewidth{1.003750pt}%
\definecolor{currentstroke}{rgb}{1.000000,1.000000,1.000000}%
\pgfsetstrokecolor{currentstroke}%
\pgfsetdash{}{0pt}%
\pgfpathmoveto{\pgfqpoint{0.667731in}{4.080654in}}%
\pgfpathlineto{\pgfqpoint{5.410148in}{4.080654in}}%
\pgfusepath{stroke}%
\end{pgfscope}%
\begin{pgfscope}%
\definecolor{textcolor}{rgb}{0.150000,0.150000,0.150000}%
\pgfsetstrokecolor{textcolor}%
\pgfsetfillcolor{textcolor}%
\pgftext[x=0.383703in, y=4.027848in, left, base]{\color{textcolor}\sffamily\fontsize{11.000000}{13.200000}\selectfont \(\displaystyle {62}\)}%
\end{pgfscope}%
\begin{pgfscope}%
\pgfpathrectangle{\pgfqpoint{0.667731in}{4.015734in}}{\pgfqpoint{4.742417in}{2.714069in}}%
\pgfusepath{clip}%
\pgfsetroundcap%
\pgfsetroundjoin%
\pgfsetlinewidth{1.003750pt}%
\definecolor{currentstroke}{rgb}{1.000000,1.000000,1.000000}%
\pgfsetstrokecolor{currentstroke}%
\pgfsetdash{}{0pt}%
\pgfpathmoveto{\pgfqpoint{0.667731in}{4.429265in}}%
\pgfpathlineto{\pgfqpoint{5.410148in}{4.429265in}}%
\pgfusepath{stroke}%
\end{pgfscope}%
\begin{pgfscope}%
\definecolor{textcolor}{rgb}{0.150000,0.150000,0.150000}%
\pgfsetstrokecolor{textcolor}%
\pgfsetfillcolor{textcolor}%
\pgftext[x=0.383703in, y=4.376458in, left, base]{\color{textcolor}\sffamily\fontsize{11.000000}{13.200000}\selectfont \(\displaystyle {63}\)}%
\end{pgfscope}%
\begin{pgfscope}%
\pgfpathrectangle{\pgfqpoint{0.667731in}{4.015734in}}{\pgfqpoint{4.742417in}{2.714069in}}%
\pgfusepath{clip}%
\pgfsetroundcap%
\pgfsetroundjoin%
\pgfsetlinewidth{1.003750pt}%
\definecolor{currentstroke}{rgb}{1.000000,1.000000,1.000000}%
\pgfsetstrokecolor{currentstroke}%
\pgfsetdash{}{0pt}%
\pgfpathmoveto{\pgfqpoint{0.667731in}{4.777875in}}%
\pgfpathlineto{\pgfqpoint{5.410148in}{4.777875in}}%
\pgfusepath{stroke}%
\end{pgfscope}%
\begin{pgfscope}%
\definecolor{textcolor}{rgb}{0.150000,0.150000,0.150000}%
\pgfsetstrokecolor{textcolor}%
\pgfsetfillcolor{textcolor}%
\pgftext[x=0.383703in, y=4.725069in, left, base]{\color{textcolor}\sffamily\fontsize{11.000000}{13.200000}\selectfont \(\displaystyle {64}\)}%
\end{pgfscope}%
\begin{pgfscope}%
\pgfpathrectangle{\pgfqpoint{0.667731in}{4.015734in}}{\pgfqpoint{4.742417in}{2.714069in}}%
\pgfusepath{clip}%
\pgfsetroundcap%
\pgfsetroundjoin%
\pgfsetlinewidth{1.003750pt}%
\definecolor{currentstroke}{rgb}{1.000000,1.000000,1.000000}%
\pgfsetstrokecolor{currentstroke}%
\pgfsetdash{}{0pt}%
\pgfpathmoveto{\pgfqpoint{0.667731in}{5.126486in}}%
\pgfpathlineto{\pgfqpoint{5.410148in}{5.126486in}}%
\pgfusepath{stroke}%
\end{pgfscope}%
\begin{pgfscope}%
\definecolor{textcolor}{rgb}{0.150000,0.150000,0.150000}%
\pgfsetstrokecolor{textcolor}%
\pgfsetfillcolor{textcolor}%
\pgftext[x=0.383703in, y=5.073679in, left, base]{\color{textcolor}\sffamily\fontsize{11.000000}{13.200000}\selectfont \(\displaystyle {65}\)}%
\end{pgfscope}%
\begin{pgfscope}%
\pgfpathrectangle{\pgfqpoint{0.667731in}{4.015734in}}{\pgfqpoint{4.742417in}{2.714069in}}%
\pgfusepath{clip}%
\pgfsetroundcap%
\pgfsetroundjoin%
\pgfsetlinewidth{1.003750pt}%
\definecolor{currentstroke}{rgb}{1.000000,1.000000,1.000000}%
\pgfsetstrokecolor{currentstroke}%
\pgfsetdash{}{0pt}%
\pgfpathmoveto{\pgfqpoint{0.667731in}{5.475096in}}%
\pgfpathlineto{\pgfqpoint{5.410148in}{5.475096in}}%
\pgfusepath{stroke}%
\end{pgfscope}%
\begin{pgfscope}%
\definecolor{textcolor}{rgb}{0.150000,0.150000,0.150000}%
\pgfsetstrokecolor{textcolor}%
\pgfsetfillcolor{textcolor}%
\pgftext[x=0.383703in, y=5.422290in, left, base]{\color{textcolor}\sffamily\fontsize{11.000000}{13.200000}\selectfont \(\displaystyle {66}\)}%
\end{pgfscope}%
\begin{pgfscope}%
\pgfpathrectangle{\pgfqpoint{0.667731in}{4.015734in}}{\pgfqpoint{4.742417in}{2.714069in}}%
\pgfusepath{clip}%
\pgfsetroundcap%
\pgfsetroundjoin%
\pgfsetlinewidth{1.003750pt}%
\definecolor{currentstroke}{rgb}{1.000000,1.000000,1.000000}%
\pgfsetstrokecolor{currentstroke}%
\pgfsetdash{}{0pt}%
\pgfpathmoveto{\pgfqpoint{0.667731in}{5.823707in}}%
\pgfpathlineto{\pgfqpoint{5.410148in}{5.823707in}}%
\pgfusepath{stroke}%
\end{pgfscope}%
\begin{pgfscope}%
\definecolor{textcolor}{rgb}{0.150000,0.150000,0.150000}%
\pgfsetstrokecolor{textcolor}%
\pgfsetfillcolor{textcolor}%
\pgftext[x=0.383703in, y=5.770900in, left, base]{\color{textcolor}\sffamily\fontsize{11.000000}{13.200000}\selectfont \(\displaystyle {67}\)}%
\end{pgfscope}%
\begin{pgfscope}%
\pgfpathrectangle{\pgfqpoint{0.667731in}{4.015734in}}{\pgfqpoint{4.742417in}{2.714069in}}%
\pgfusepath{clip}%
\pgfsetroundcap%
\pgfsetroundjoin%
\pgfsetlinewidth{1.003750pt}%
\definecolor{currentstroke}{rgb}{1.000000,1.000000,1.000000}%
\pgfsetstrokecolor{currentstroke}%
\pgfsetdash{}{0pt}%
\pgfpathmoveto{\pgfqpoint{0.667731in}{6.172317in}}%
\pgfpathlineto{\pgfqpoint{5.410148in}{6.172317in}}%
\pgfusepath{stroke}%
\end{pgfscope}%
\begin{pgfscope}%
\definecolor{textcolor}{rgb}{0.150000,0.150000,0.150000}%
\pgfsetstrokecolor{textcolor}%
\pgfsetfillcolor{textcolor}%
\pgftext[x=0.383703in, y=6.119511in, left, base]{\color{textcolor}\sffamily\fontsize{11.000000}{13.200000}\selectfont \(\displaystyle {68}\)}%
\end{pgfscope}%
\begin{pgfscope}%
\pgfpathrectangle{\pgfqpoint{0.667731in}{4.015734in}}{\pgfqpoint{4.742417in}{2.714069in}}%
\pgfusepath{clip}%
\pgfsetroundcap%
\pgfsetroundjoin%
\pgfsetlinewidth{1.003750pt}%
\definecolor{currentstroke}{rgb}{1.000000,1.000000,1.000000}%
\pgfsetstrokecolor{currentstroke}%
\pgfsetdash{}{0pt}%
\pgfpathmoveto{\pgfqpoint{0.667731in}{6.520928in}}%
\pgfpathlineto{\pgfqpoint{5.410148in}{6.520928in}}%
\pgfusepath{stroke}%
\end{pgfscope}%
\begin{pgfscope}%
\definecolor{textcolor}{rgb}{0.150000,0.150000,0.150000}%
\pgfsetstrokecolor{textcolor}%
\pgfsetfillcolor{textcolor}%
\pgftext[x=0.383703in, y=6.468121in, left, base]{\color{textcolor}\sffamily\fontsize{11.000000}{13.200000}\selectfont \(\displaystyle {69}\)}%
\end{pgfscope}%
\begin{pgfscope}%
\definecolor{textcolor}{rgb}{0.150000,0.150000,0.150000}%
\pgfsetstrokecolor{textcolor}%
\pgfsetfillcolor{textcolor}%
\pgftext[x=0.328148in,y=5.372769in,,bottom,rotate=90.000000]{\color{textcolor}\sffamily\fontsize{12.000000}{14.400000}\selectfont Average score}%
\end{pgfscope}%
\begin{pgfscope}%
\pgfpathrectangle{\pgfqpoint{0.667731in}{4.015734in}}{\pgfqpoint{4.742417in}{2.714069in}}%
\pgfusepath{clip}%
\pgfsetbuttcap%
\pgfsetroundjoin%
\pgfsetlinewidth{2.710125pt}%
\definecolor{currentstroke}{rgb}{0.298039,0.447059,0.690196}%
\pgfsetstrokecolor{currentstroke}%
\pgfsetdash{{9.990000pt}{4.320000pt}}{0.000000pt}%
\pgfpathmoveto{\pgfqpoint{1.141973in}{5.981568in}}%
\pgfpathlineto{\pgfqpoint{2.090456in}{6.606436in}}%
\pgfpathlineto{\pgfqpoint{3.038940in}{5.455364in}}%
\pgfpathlineto{\pgfqpoint{3.987423in}{5.027823in}}%
\pgfpathlineto{\pgfqpoint{4.935906in}{4.962047in}}%
\pgfusepath{stroke}%
\end{pgfscope}%
\begin{pgfscope}%
\pgfpathrectangle{\pgfqpoint{0.667731in}{4.015734in}}{\pgfqpoint{4.742417in}{2.714069in}}%
\pgfusepath{clip}%
\pgfsetroundcap%
\pgfsetroundjoin%
\pgfsetlinewidth{2.710125pt}%
\definecolor{currentstroke}{rgb}{0.298039,0.447059,0.690196}%
\pgfsetstrokecolor{currentstroke}%
\pgfsetdash{}{0pt}%
\pgfusepath{stroke}%
\end{pgfscope}%
\begin{pgfscope}%
\pgfpathrectangle{\pgfqpoint{0.667731in}{4.015734in}}{\pgfqpoint{4.742417in}{2.714069in}}%
\pgfusepath{clip}%
\pgfsetroundcap%
\pgfsetroundjoin%
\pgfsetlinewidth{2.710125pt}%
\definecolor{currentstroke}{rgb}{0.298039,0.447059,0.690196}%
\pgfsetstrokecolor{currentstroke}%
\pgfsetdash{}{0pt}%
\pgfusepath{stroke}%
\end{pgfscope}%
\begin{pgfscope}%
\pgfpathrectangle{\pgfqpoint{0.667731in}{4.015734in}}{\pgfqpoint{4.742417in}{2.714069in}}%
\pgfusepath{clip}%
\pgfsetroundcap%
\pgfsetroundjoin%
\pgfsetlinewidth{2.710125pt}%
\definecolor{currentstroke}{rgb}{0.298039,0.447059,0.690196}%
\pgfsetstrokecolor{currentstroke}%
\pgfsetdash{}{0pt}%
\pgfusepath{stroke}%
\end{pgfscope}%
\begin{pgfscope}%
\pgfpathrectangle{\pgfqpoint{0.667731in}{4.015734in}}{\pgfqpoint{4.742417in}{2.714069in}}%
\pgfusepath{clip}%
\pgfsetroundcap%
\pgfsetroundjoin%
\pgfsetlinewidth{2.710125pt}%
\definecolor{currentstroke}{rgb}{0.298039,0.447059,0.690196}%
\pgfsetstrokecolor{currentstroke}%
\pgfsetdash{}{0pt}%
\pgfusepath{stroke}%
\end{pgfscope}%
\begin{pgfscope}%
\pgfpathrectangle{\pgfqpoint{0.667731in}{4.015734in}}{\pgfqpoint{4.742417in}{2.714069in}}%
\pgfusepath{clip}%
\pgfsetroundcap%
\pgfsetroundjoin%
\pgfsetlinewidth{2.710125pt}%
\definecolor{currentstroke}{rgb}{0.298039,0.447059,0.690196}%
\pgfsetstrokecolor{currentstroke}%
\pgfsetdash{}{0pt}%
\pgfusepath{stroke}%
\end{pgfscope}%
\begin{pgfscope}%
\pgfpathrectangle{\pgfqpoint{0.667731in}{4.015734in}}{\pgfqpoint{4.742417in}{2.714069in}}%
\pgfusepath{clip}%
\pgfsetbuttcap%
\pgfsetroundjoin%
\definecolor{currentfill}{rgb}{0.298039,0.447059,0.690196}%
\pgfsetfillcolor{currentfill}%
\pgfsetlinewidth{2.032594pt}%
\definecolor{currentstroke}{rgb}{0.298039,0.447059,0.690196}%
\pgfsetstrokecolor{currentstroke}%
\pgfsetdash{}{0pt}%
\pgfsys@defobject{currentmarker}{\pgfqpoint{-0.046999in}{-0.046999in}}{\pgfqpoint{0.046999in}{0.046999in}}{%
\pgfpathmoveto{\pgfqpoint{0.000000in}{-0.046999in}}%
\pgfpathcurveto{\pgfqpoint{0.012464in}{-0.046999in}}{\pgfqpoint{0.024420in}{-0.042047in}}{\pgfqpoint{0.033234in}{-0.033234in}}%
\pgfpathcurveto{\pgfqpoint{0.042047in}{-0.024420in}}{\pgfqpoint{0.046999in}{-0.012464in}}{\pgfqpoint{0.046999in}{0.000000in}}%
\pgfpathcurveto{\pgfqpoint{0.046999in}{0.012464in}}{\pgfqpoint{0.042047in}{0.024420in}}{\pgfqpoint{0.033234in}{0.033234in}}%
\pgfpathcurveto{\pgfqpoint{0.024420in}{0.042047in}}{\pgfqpoint{0.012464in}{0.046999in}}{\pgfqpoint{0.000000in}{0.046999in}}%
\pgfpathcurveto{\pgfqpoint{-0.012464in}{0.046999in}}{\pgfqpoint{-0.024420in}{0.042047in}}{\pgfqpoint{-0.033234in}{0.033234in}}%
\pgfpathcurveto{\pgfqpoint{-0.042047in}{0.024420in}}{\pgfqpoint{-0.046999in}{0.012464in}}{\pgfqpoint{-0.046999in}{0.000000in}}%
\pgfpathcurveto{\pgfqpoint{-0.046999in}{-0.012464in}}{\pgfqpoint{-0.042047in}{-0.024420in}}{\pgfqpoint{-0.033234in}{-0.033234in}}%
\pgfpathcurveto{\pgfqpoint{-0.024420in}{-0.042047in}}{\pgfqpoint{-0.012464in}{-0.046999in}}{\pgfqpoint{0.000000in}{-0.046999in}}%
\pgfpathlineto{\pgfqpoint{0.000000in}{-0.046999in}}%
\pgfpathclose%
\pgfusepath{stroke,fill}%
}%
\begin{pgfscope}%
\pgfsys@transformshift{1.141973in}{5.981568in}%
\pgfsys@useobject{currentmarker}{}%
\end{pgfscope}%
\begin{pgfscope}%
\pgfsys@transformshift{2.090456in}{6.606436in}%
\pgfsys@useobject{currentmarker}{}%
\end{pgfscope}%
\begin{pgfscope}%
\pgfsys@transformshift{3.038940in}{5.455364in}%
\pgfsys@useobject{currentmarker}{}%
\end{pgfscope}%
\begin{pgfscope}%
\pgfsys@transformshift{3.987423in}{5.027823in}%
\pgfsys@useobject{currentmarker}{}%
\end{pgfscope}%
\begin{pgfscope}%
\pgfsys@transformshift{4.935906in}{4.962047in}%
\pgfsys@useobject{currentmarker}{}%
\end{pgfscope}%
\end{pgfscope}%
\begin{pgfscope}%
\pgfpathrectangle{\pgfqpoint{0.667731in}{4.015734in}}{\pgfqpoint{4.742417in}{2.714069in}}%
\pgfusepath{clip}%
\pgfsetbuttcap%
\pgfsetroundjoin%
\pgfsetlinewidth{2.710125pt}%
\definecolor{currentstroke}{rgb}{0.866667,0.517647,0.321569}%
\pgfsetstrokecolor{currentstroke}%
\pgfsetdash{{9.990000pt}{4.320000pt}}{0.000000pt}%
\pgfpathmoveto{\pgfqpoint{1.141973in}{5.134986in}}%
\pgfpathlineto{\pgfqpoint{2.090456in}{5.929632in}}%
\pgfpathlineto{\pgfqpoint{3.038940in}{4.981911in}}%
\pgfpathlineto{\pgfqpoint{3.987423in}{4.592100in}}%
\pgfpathlineto{\pgfqpoint{4.935906in}{4.139101in}}%
\pgfusepath{stroke}%
\end{pgfscope}%
\begin{pgfscope}%
\pgfpathrectangle{\pgfqpoint{0.667731in}{4.015734in}}{\pgfqpoint{4.742417in}{2.714069in}}%
\pgfusepath{clip}%
\pgfsetroundcap%
\pgfsetroundjoin%
\pgfsetlinewidth{2.710125pt}%
\definecolor{currentstroke}{rgb}{0.866667,0.517647,0.321569}%
\pgfsetstrokecolor{currentstroke}%
\pgfsetdash{}{0pt}%
\pgfusepath{stroke}%
\end{pgfscope}%
\begin{pgfscope}%
\pgfpathrectangle{\pgfqpoint{0.667731in}{4.015734in}}{\pgfqpoint{4.742417in}{2.714069in}}%
\pgfusepath{clip}%
\pgfsetroundcap%
\pgfsetroundjoin%
\pgfsetlinewidth{2.710125pt}%
\definecolor{currentstroke}{rgb}{0.866667,0.517647,0.321569}%
\pgfsetstrokecolor{currentstroke}%
\pgfsetdash{}{0pt}%
\pgfusepath{stroke}%
\end{pgfscope}%
\begin{pgfscope}%
\pgfpathrectangle{\pgfqpoint{0.667731in}{4.015734in}}{\pgfqpoint{4.742417in}{2.714069in}}%
\pgfusepath{clip}%
\pgfsetroundcap%
\pgfsetroundjoin%
\pgfsetlinewidth{2.710125pt}%
\definecolor{currentstroke}{rgb}{0.866667,0.517647,0.321569}%
\pgfsetstrokecolor{currentstroke}%
\pgfsetdash{}{0pt}%
\pgfusepath{stroke}%
\end{pgfscope}%
\begin{pgfscope}%
\pgfpathrectangle{\pgfqpoint{0.667731in}{4.015734in}}{\pgfqpoint{4.742417in}{2.714069in}}%
\pgfusepath{clip}%
\pgfsetroundcap%
\pgfsetroundjoin%
\pgfsetlinewidth{2.710125pt}%
\definecolor{currentstroke}{rgb}{0.866667,0.517647,0.321569}%
\pgfsetstrokecolor{currentstroke}%
\pgfsetdash{}{0pt}%
\pgfusepath{stroke}%
\end{pgfscope}%
\begin{pgfscope}%
\pgfpathrectangle{\pgfqpoint{0.667731in}{4.015734in}}{\pgfqpoint{4.742417in}{2.714069in}}%
\pgfusepath{clip}%
\pgfsetroundcap%
\pgfsetroundjoin%
\pgfsetlinewidth{2.710125pt}%
\definecolor{currentstroke}{rgb}{0.866667,0.517647,0.321569}%
\pgfsetstrokecolor{currentstroke}%
\pgfsetdash{}{0pt}%
\pgfusepath{stroke}%
\end{pgfscope}%
\begin{pgfscope}%
\pgfpathrectangle{\pgfqpoint{0.667731in}{4.015734in}}{\pgfqpoint{4.742417in}{2.714069in}}%
\pgfusepath{clip}%
\pgfsetbuttcap%
\pgfsetroundjoin%
\definecolor{currentfill}{rgb}{0.866667,0.517647,0.321569}%
\pgfsetfillcolor{currentfill}%
\pgfsetlinewidth{2.032594pt}%
\definecolor{currentstroke}{rgb}{0.866667,0.517647,0.321569}%
\pgfsetstrokecolor{currentstroke}%
\pgfsetdash{}{0pt}%
\pgfsys@defobject{currentmarker}{\pgfqpoint{-0.046999in}{-0.046999in}}{\pgfqpoint{0.046999in}{0.046999in}}{%
\pgfpathmoveto{\pgfqpoint{0.000000in}{-0.046999in}}%
\pgfpathcurveto{\pgfqpoint{0.012464in}{-0.046999in}}{\pgfqpoint{0.024420in}{-0.042047in}}{\pgfqpoint{0.033234in}{-0.033234in}}%
\pgfpathcurveto{\pgfqpoint{0.042047in}{-0.024420in}}{\pgfqpoint{0.046999in}{-0.012464in}}{\pgfqpoint{0.046999in}{0.000000in}}%
\pgfpathcurveto{\pgfqpoint{0.046999in}{0.012464in}}{\pgfqpoint{0.042047in}{0.024420in}}{\pgfqpoint{0.033234in}{0.033234in}}%
\pgfpathcurveto{\pgfqpoint{0.024420in}{0.042047in}}{\pgfqpoint{0.012464in}{0.046999in}}{\pgfqpoint{0.000000in}{0.046999in}}%
\pgfpathcurveto{\pgfqpoint{-0.012464in}{0.046999in}}{\pgfqpoint{-0.024420in}{0.042047in}}{\pgfqpoint{-0.033234in}{0.033234in}}%
\pgfpathcurveto{\pgfqpoint{-0.042047in}{0.024420in}}{\pgfqpoint{-0.046999in}{0.012464in}}{\pgfqpoint{-0.046999in}{0.000000in}}%
\pgfpathcurveto{\pgfqpoint{-0.046999in}{-0.012464in}}{\pgfqpoint{-0.042047in}{-0.024420in}}{\pgfqpoint{-0.033234in}{-0.033234in}}%
\pgfpathcurveto{\pgfqpoint{-0.024420in}{-0.042047in}}{\pgfqpoint{-0.012464in}{-0.046999in}}{\pgfqpoint{0.000000in}{-0.046999in}}%
\pgfpathlineto{\pgfqpoint{0.000000in}{-0.046999in}}%
\pgfpathclose%
\pgfusepath{stroke,fill}%
}%
\begin{pgfscope}%
\pgfsys@transformshift{1.141973in}{5.134986in}%
\pgfsys@useobject{currentmarker}{}%
\end{pgfscope}%
\begin{pgfscope}%
\pgfsys@transformshift{2.090456in}{5.929632in}%
\pgfsys@useobject{currentmarker}{}%
\end{pgfscope}%
\begin{pgfscope}%
\pgfsys@transformshift{3.038940in}{4.981911in}%
\pgfsys@useobject{currentmarker}{}%
\end{pgfscope}%
\begin{pgfscope}%
\pgfsys@transformshift{3.987423in}{4.592100in}%
\pgfsys@useobject{currentmarker}{}%
\end{pgfscope}%
\begin{pgfscope}%
\pgfsys@transformshift{4.935906in}{4.139101in}%
\pgfsys@useobject{currentmarker}{}%
\end{pgfscope}%
\end{pgfscope}%
\begin{pgfscope}%
\pgfsetrectcap%
\pgfsetmiterjoin%
\pgfsetlinewidth{1.254687pt}%
\definecolor{currentstroke}{rgb}{1.000000,1.000000,1.000000}%
\pgfsetstrokecolor{currentstroke}%
\pgfsetdash{}{0pt}%
\pgfpathmoveto{\pgfqpoint{0.667731in}{4.015734in}}%
\pgfpathlineto{\pgfqpoint{0.667731in}{6.729803in}}%
\pgfusepath{stroke}%
\end{pgfscope}%
\begin{pgfscope}%
\pgfsetrectcap%
\pgfsetmiterjoin%
\pgfsetlinewidth{1.254687pt}%
\definecolor{currentstroke}{rgb}{1.000000,1.000000,1.000000}%
\pgfsetstrokecolor{currentstroke}%
\pgfsetdash{}{0pt}%
\pgfpathmoveto{\pgfqpoint{5.410148in}{4.015734in}}%
\pgfpathlineto{\pgfqpoint{5.410148in}{6.729803in}}%
\pgfusepath{stroke}%
\end{pgfscope}%
\begin{pgfscope}%
\pgfsetrectcap%
\pgfsetmiterjoin%
\pgfsetlinewidth{1.254687pt}%
\definecolor{currentstroke}{rgb}{1.000000,1.000000,1.000000}%
\pgfsetstrokecolor{currentstroke}%
\pgfsetdash{}{0pt}%
\pgfpathmoveto{\pgfqpoint{0.667731in}{4.015734in}}%
\pgfpathlineto{\pgfqpoint{5.410148in}{4.015734in}}%
\pgfusepath{stroke}%
\end{pgfscope}%
\begin{pgfscope}%
\pgfsetrectcap%
\pgfsetmiterjoin%
\pgfsetlinewidth{1.254687pt}%
\definecolor{currentstroke}{rgb}{1.000000,1.000000,1.000000}%
\pgfsetstrokecolor{currentstroke}%
\pgfsetdash{}{0pt}%
\pgfpathmoveto{\pgfqpoint{0.667731in}{6.729803in}}%
\pgfpathlineto{\pgfqpoint{5.410148in}{6.729803in}}%
\pgfusepath{stroke}%
\end{pgfscope}%
\begin{pgfscope}%
\pgfsetbuttcap%
\pgfsetmiterjoin%
\definecolor{currentfill}{rgb}{0.917647,0.917647,0.949020}%
\pgfsetfillcolor{currentfill}%
\pgfsetfillopacity{0.800000}%
\pgfsetlinewidth{1.003750pt}%
\definecolor{currentstroke}{rgb}{0.800000,0.800000,0.800000}%
\pgfsetstrokecolor{currentstroke}%
\pgfsetstrokeopacity{0.800000}%
\pgfsetdash{}{0pt}%
\pgfpathmoveto{\pgfqpoint{4.211472in}{5.957234in}}%
\pgfpathlineto{\pgfqpoint{5.303204in}{5.957234in}}%
\pgfpathquadraticcurveto{\pgfqpoint{5.333759in}{5.957234in}}{\pgfqpoint{5.333759in}{5.987790in}}%
\pgfpathlineto{\pgfqpoint{5.333759in}{6.622859in}}%
\pgfpathquadraticcurveto{\pgfqpoint{5.333759in}{6.653414in}}{\pgfqpoint{5.303204in}{6.653414in}}%
\pgfpathlineto{\pgfqpoint{4.211472in}{6.653414in}}%
\pgfpathquadraticcurveto{\pgfqpoint{4.180916in}{6.653414in}}{\pgfqpoint{4.180916in}{6.622859in}}%
\pgfpathlineto{\pgfqpoint{4.180916in}{5.987790in}}%
\pgfpathquadraticcurveto{\pgfqpoint{4.180916in}{5.957234in}}{\pgfqpoint{4.211472in}{5.957234in}}%
\pgfpathlineto{\pgfqpoint{4.211472in}{5.957234in}}%
\pgfpathclose%
\pgfusepath{stroke,fill}%
\end{pgfscope}%
\begin{pgfscope}%
\definecolor{textcolor}{rgb}{0.150000,0.150000,0.150000}%
\pgfsetstrokecolor{textcolor}%
\pgfsetfillcolor{textcolor}%
\pgftext[x=4.513726in,y=6.476563in,left,base]{\color{textcolor}\sffamily\fontsize{12.000000}{14.400000}\selectfont F-score}%
\end{pgfscope}%
\begin{pgfscope}%
\pgfsetbuttcap%
\pgfsetroundjoin%
\definecolor{currentfill}{rgb}{0.298039,0.447059,0.690196}%
\pgfsetfillcolor{currentfill}%
\pgfsetlinewidth{2.032594pt}%
\definecolor{currentstroke}{rgb}{0.298039,0.447059,0.690196}%
\pgfsetstrokecolor{currentstroke}%
\pgfsetdash{}{0pt}%
\pgfsys@defobject{currentmarker}{\pgfqpoint{-0.046999in}{-0.046999in}}{\pgfqpoint{0.046999in}{0.046999in}}{%
\pgfpathmoveto{\pgfqpoint{0.000000in}{-0.046999in}}%
\pgfpathcurveto{\pgfqpoint{0.012464in}{-0.046999in}}{\pgfqpoint{0.024420in}{-0.042047in}}{\pgfqpoint{0.033234in}{-0.033234in}}%
\pgfpathcurveto{\pgfqpoint{0.042047in}{-0.024420in}}{\pgfqpoint{0.046999in}{-0.012464in}}{\pgfqpoint{0.046999in}{0.000000in}}%
\pgfpathcurveto{\pgfqpoint{0.046999in}{0.012464in}}{\pgfqpoint{0.042047in}{0.024420in}}{\pgfqpoint{0.033234in}{0.033234in}}%
\pgfpathcurveto{\pgfqpoint{0.024420in}{0.042047in}}{\pgfqpoint{0.012464in}{0.046999in}}{\pgfqpoint{0.000000in}{0.046999in}}%
\pgfpathcurveto{\pgfqpoint{-0.012464in}{0.046999in}}{\pgfqpoint{-0.024420in}{0.042047in}}{\pgfqpoint{-0.033234in}{0.033234in}}%
\pgfpathcurveto{\pgfqpoint{-0.042047in}{0.024420in}}{\pgfqpoint{-0.046999in}{0.012464in}}{\pgfqpoint{-0.046999in}{0.000000in}}%
\pgfpathcurveto{\pgfqpoint{-0.046999in}{-0.012464in}}{\pgfqpoint{-0.042047in}{-0.024420in}}{\pgfqpoint{-0.033234in}{-0.033234in}}%
\pgfpathcurveto{\pgfqpoint{-0.024420in}{-0.042047in}}{\pgfqpoint{-0.012464in}{-0.046999in}}{\pgfqpoint{0.000000in}{-0.046999in}}%
\pgfpathlineto{\pgfqpoint{0.000000in}{-0.046999in}}%
\pgfpathclose%
\pgfusepath{stroke,fill}%
}%
\begin{pgfscope}%
\pgfsys@transformshift{4.394805in}{6.300926in}%
\pgfsys@useobject{currentmarker}{}%
\end{pgfscope}%
\end{pgfscope}%
\begin{pgfscope}%
\definecolor{textcolor}{rgb}{0.150000,0.150000,0.150000}%
\pgfsetstrokecolor{textcolor}%
\pgfsetfillcolor{textcolor}%
\pgftext[x=4.669805in,y=6.260822in,left,base]{\color{textcolor}\sffamily\fontsize{11.000000}{13.200000}\selectfont F1 micro}%
\end{pgfscope}%
\begin{pgfscope}%
\pgfsetbuttcap%
\pgfsetroundjoin%
\definecolor{currentfill}{rgb}{0.866667,0.517647,0.321569}%
\pgfsetfillcolor{currentfill}%
\pgfsetlinewidth{2.032594pt}%
\definecolor{currentstroke}{rgb}{0.866667,0.517647,0.321569}%
\pgfsetstrokecolor{currentstroke}%
\pgfsetdash{}{0pt}%
\pgfsys@defobject{currentmarker}{\pgfqpoint{-0.046999in}{-0.046999in}}{\pgfqpoint{0.046999in}{0.046999in}}{%
\pgfpathmoveto{\pgfqpoint{0.000000in}{-0.046999in}}%
\pgfpathcurveto{\pgfqpoint{0.012464in}{-0.046999in}}{\pgfqpoint{0.024420in}{-0.042047in}}{\pgfqpoint{0.033234in}{-0.033234in}}%
\pgfpathcurveto{\pgfqpoint{0.042047in}{-0.024420in}}{\pgfqpoint{0.046999in}{-0.012464in}}{\pgfqpoint{0.046999in}{0.000000in}}%
\pgfpathcurveto{\pgfqpoint{0.046999in}{0.012464in}}{\pgfqpoint{0.042047in}{0.024420in}}{\pgfqpoint{0.033234in}{0.033234in}}%
\pgfpathcurveto{\pgfqpoint{0.024420in}{0.042047in}}{\pgfqpoint{0.012464in}{0.046999in}}{\pgfqpoint{0.000000in}{0.046999in}}%
\pgfpathcurveto{\pgfqpoint{-0.012464in}{0.046999in}}{\pgfqpoint{-0.024420in}{0.042047in}}{\pgfqpoint{-0.033234in}{0.033234in}}%
\pgfpathcurveto{\pgfqpoint{-0.042047in}{0.024420in}}{\pgfqpoint{-0.046999in}{0.012464in}}{\pgfqpoint{-0.046999in}{0.000000in}}%
\pgfpathcurveto{\pgfqpoint{-0.046999in}{-0.012464in}}{\pgfqpoint{-0.042047in}{-0.024420in}}{\pgfqpoint{-0.033234in}{-0.033234in}}%
\pgfpathcurveto{\pgfqpoint{-0.024420in}{-0.042047in}}{\pgfqpoint{-0.012464in}{-0.046999in}}{\pgfqpoint{0.000000in}{-0.046999in}}%
\pgfpathlineto{\pgfqpoint{0.000000in}{-0.046999in}}%
\pgfpathclose%
\pgfusepath{stroke,fill}%
}%
\begin{pgfscope}%
\pgfsys@transformshift{4.394805in}{6.088021in}%
\pgfsys@useobject{currentmarker}{}%
\end{pgfscope}%
\end{pgfscope}%
\begin{pgfscope}%
\definecolor{textcolor}{rgb}{0.150000,0.150000,0.150000}%
\pgfsetstrokecolor{textcolor}%
\pgfsetfillcolor{textcolor}%
\pgftext[x=4.669805in,y=6.047917in,left,base]{\color{textcolor}\sffamily\fontsize{11.000000}{13.200000}\selectfont F1 macro}%
\end{pgfscope}%
\begin{pgfscope}%
\pgfsetbuttcap%
\pgfsetmiterjoin%
\definecolor{currentfill}{rgb}{0.917647,0.917647,0.949020}%
\pgfsetfillcolor{currentfill}%
\pgfsetlinewidth{0.000000pt}%
\definecolor{currentstroke}{rgb}{0.000000,0.000000,0.000000}%
\pgfsetstrokecolor{currentstroke}%
\pgfsetstrokeopacity{0.000000}%
\pgfsetdash{}{0pt}%
\pgfpathmoveto{\pgfqpoint{0.667731in}{0.650833in}}%
\pgfpathlineto{\pgfqpoint{5.410148in}{0.650833in}}%
\pgfpathlineto{\pgfqpoint{5.410148in}{3.364902in}}%
\pgfpathlineto{\pgfqpoint{0.667731in}{3.364902in}}%
\pgfpathlineto{\pgfqpoint{0.667731in}{0.650833in}}%
\pgfpathclose%
\pgfusepath{fill}%
\end{pgfscope}%
\begin{pgfscope}%
\definecolor{textcolor}{rgb}{0.150000,0.150000,0.150000}%
\pgfsetstrokecolor{textcolor}%
\pgfsetfillcolor{textcolor}%
\pgftext[x=0.964132in,y=0.518888in,,top]{\color{textcolor}\sffamily\fontsize{11.000000}{13.200000}\selectfont 5}%
\end{pgfscope}%
\begin{pgfscope}%
\definecolor{textcolor}{rgb}{0.150000,0.150000,0.150000}%
\pgfsetstrokecolor{textcolor}%
\pgfsetfillcolor{textcolor}%
\pgftext[x=1.556934in,y=0.518888in,,top]{\color{textcolor}\sffamily\fontsize{11.000000}{13.200000}\selectfont 10}%
\end{pgfscope}%
\begin{pgfscope}%
\definecolor{textcolor}{rgb}{0.150000,0.150000,0.150000}%
\pgfsetstrokecolor{textcolor}%
\pgfsetfillcolor{textcolor}%
\pgftext[x=2.149736in,y=0.518888in,,top]{\color{textcolor}\sffamily\fontsize{11.000000}{13.200000}\selectfont 15}%
\end{pgfscope}%
\begin{pgfscope}%
\definecolor{textcolor}{rgb}{0.150000,0.150000,0.150000}%
\pgfsetstrokecolor{textcolor}%
\pgfsetfillcolor{textcolor}%
\pgftext[x=2.742538in,y=0.518888in,,top]{\color{textcolor}\sffamily\fontsize{11.000000}{13.200000}\selectfont 20}%
\end{pgfscope}%
\begin{pgfscope}%
\definecolor{textcolor}{rgb}{0.150000,0.150000,0.150000}%
\pgfsetstrokecolor{textcolor}%
\pgfsetfillcolor{textcolor}%
\pgftext[x=3.335341in,y=0.518888in,,top]{\color{textcolor}\sffamily\fontsize{11.000000}{13.200000}\selectfont 30}%
\end{pgfscope}%
\begin{pgfscope}%
\definecolor{textcolor}{rgb}{0.150000,0.150000,0.150000}%
\pgfsetstrokecolor{textcolor}%
\pgfsetfillcolor{textcolor}%
\pgftext[x=3.928143in,y=0.518888in,,top]{\color{textcolor}\sffamily\fontsize{11.000000}{13.200000}\selectfont 40}%
\end{pgfscope}%
\begin{pgfscope}%
\definecolor{textcolor}{rgb}{0.150000,0.150000,0.150000}%
\pgfsetstrokecolor{textcolor}%
\pgfsetfillcolor{textcolor}%
\pgftext[x=4.520945in,y=0.518888in,,top]{\color{textcolor}\sffamily\fontsize{11.000000}{13.200000}\selectfont 60}%
\end{pgfscope}%
\begin{pgfscope}%
\definecolor{textcolor}{rgb}{0.150000,0.150000,0.150000}%
\pgfsetstrokecolor{textcolor}%
\pgfsetfillcolor{textcolor}%
\pgftext[x=5.113747in,y=0.518888in,,top]{\color{textcolor}\sffamily\fontsize{11.000000}{13.200000}\selectfont 80}%
\end{pgfscope}%
\begin{pgfscope}%
\definecolor{textcolor}{rgb}{0.150000,0.150000,0.150000}%
\pgfsetstrokecolor{textcolor}%
\pgfsetfillcolor{textcolor}%
\pgftext[x=3.038940in,y=0.328148in,,top]{\color{textcolor}\sffamily\fontsize{12.000000}{14.400000}\selectfont number of random walks}%
\end{pgfscope}%
\begin{pgfscope}%
\pgfpathrectangle{\pgfqpoint{0.667731in}{0.650833in}}{\pgfqpoint{4.742417in}{2.714069in}}%
\pgfusepath{clip}%
\pgfsetroundcap%
\pgfsetroundjoin%
\pgfsetlinewidth{1.003750pt}%
\definecolor{currentstroke}{rgb}{1.000000,1.000000,1.000000}%
\pgfsetstrokecolor{currentstroke}%
\pgfsetdash{}{0pt}%
\pgfpathmoveto{\pgfqpoint{0.667731in}{0.691163in}}%
\pgfpathlineto{\pgfqpoint{5.410148in}{0.691163in}}%
\pgfusepath{stroke}%
\end{pgfscope}%
\begin{pgfscope}%
\definecolor{textcolor}{rgb}{0.150000,0.150000,0.150000}%
\pgfsetstrokecolor{textcolor}%
\pgfsetfillcolor{textcolor}%
\pgftext[x=0.383703in, y=0.638357in, left, base]{\color{textcolor}\sffamily\fontsize{11.000000}{13.200000}\selectfont \(\displaystyle {63}\)}%
\end{pgfscope}%
\begin{pgfscope}%
\pgfpathrectangle{\pgfqpoint{0.667731in}{0.650833in}}{\pgfqpoint{4.742417in}{2.714069in}}%
\pgfusepath{clip}%
\pgfsetroundcap%
\pgfsetroundjoin%
\pgfsetlinewidth{1.003750pt}%
\definecolor{currentstroke}{rgb}{1.000000,1.000000,1.000000}%
\pgfsetstrokecolor{currentstroke}%
\pgfsetdash{}{0pt}%
\pgfpathmoveto{\pgfqpoint{0.667731in}{1.093454in}}%
\pgfpathlineto{\pgfqpoint{5.410148in}{1.093454in}}%
\pgfusepath{stroke}%
\end{pgfscope}%
\begin{pgfscope}%
\definecolor{textcolor}{rgb}{0.150000,0.150000,0.150000}%
\pgfsetstrokecolor{textcolor}%
\pgfsetfillcolor{textcolor}%
\pgftext[x=0.383703in, y=1.040648in, left, base]{\color{textcolor}\sffamily\fontsize{11.000000}{13.200000}\selectfont \(\displaystyle {64}\)}%
\end{pgfscope}%
\begin{pgfscope}%
\pgfpathrectangle{\pgfqpoint{0.667731in}{0.650833in}}{\pgfqpoint{4.742417in}{2.714069in}}%
\pgfusepath{clip}%
\pgfsetroundcap%
\pgfsetroundjoin%
\pgfsetlinewidth{1.003750pt}%
\definecolor{currentstroke}{rgb}{1.000000,1.000000,1.000000}%
\pgfsetstrokecolor{currentstroke}%
\pgfsetdash{}{0pt}%
\pgfpathmoveto{\pgfqpoint{0.667731in}{1.495745in}}%
\pgfpathlineto{\pgfqpoint{5.410148in}{1.495745in}}%
\pgfusepath{stroke}%
\end{pgfscope}%
\begin{pgfscope}%
\definecolor{textcolor}{rgb}{0.150000,0.150000,0.150000}%
\pgfsetstrokecolor{textcolor}%
\pgfsetfillcolor{textcolor}%
\pgftext[x=0.383703in, y=1.442938in, left, base]{\color{textcolor}\sffamily\fontsize{11.000000}{13.200000}\selectfont \(\displaystyle {65}\)}%
\end{pgfscope}%
\begin{pgfscope}%
\pgfpathrectangle{\pgfqpoint{0.667731in}{0.650833in}}{\pgfqpoint{4.742417in}{2.714069in}}%
\pgfusepath{clip}%
\pgfsetroundcap%
\pgfsetroundjoin%
\pgfsetlinewidth{1.003750pt}%
\definecolor{currentstroke}{rgb}{1.000000,1.000000,1.000000}%
\pgfsetstrokecolor{currentstroke}%
\pgfsetdash{}{0pt}%
\pgfpathmoveto{\pgfqpoint{0.667731in}{1.898036in}}%
\pgfpathlineto{\pgfqpoint{5.410148in}{1.898036in}}%
\pgfusepath{stroke}%
\end{pgfscope}%
\begin{pgfscope}%
\definecolor{textcolor}{rgb}{0.150000,0.150000,0.150000}%
\pgfsetstrokecolor{textcolor}%
\pgfsetfillcolor{textcolor}%
\pgftext[x=0.383703in, y=1.845229in, left, base]{\color{textcolor}\sffamily\fontsize{11.000000}{13.200000}\selectfont \(\displaystyle {66}\)}%
\end{pgfscope}%
\begin{pgfscope}%
\pgfpathrectangle{\pgfqpoint{0.667731in}{0.650833in}}{\pgfqpoint{4.742417in}{2.714069in}}%
\pgfusepath{clip}%
\pgfsetroundcap%
\pgfsetroundjoin%
\pgfsetlinewidth{1.003750pt}%
\definecolor{currentstroke}{rgb}{1.000000,1.000000,1.000000}%
\pgfsetstrokecolor{currentstroke}%
\pgfsetdash{}{0pt}%
\pgfpathmoveto{\pgfqpoint{0.667731in}{2.300326in}}%
\pgfpathlineto{\pgfqpoint{5.410148in}{2.300326in}}%
\pgfusepath{stroke}%
\end{pgfscope}%
\begin{pgfscope}%
\definecolor{textcolor}{rgb}{0.150000,0.150000,0.150000}%
\pgfsetstrokecolor{textcolor}%
\pgfsetfillcolor{textcolor}%
\pgftext[x=0.383703in, y=2.247520in, left, base]{\color{textcolor}\sffamily\fontsize{11.000000}{13.200000}\selectfont \(\displaystyle {67}\)}%
\end{pgfscope}%
\begin{pgfscope}%
\pgfpathrectangle{\pgfqpoint{0.667731in}{0.650833in}}{\pgfqpoint{4.742417in}{2.714069in}}%
\pgfusepath{clip}%
\pgfsetroundcap%
\pgfsetroundjoin%
\pgfsetlinewidth{1.003750pt}%
\definecolor{currentstroke}{rgb}{1.000000,1.000000,1.000000}%
\pgfsetstrokecolor{currentstroke}%
\pgfsetdash{}{0pt}%
\pgfpathmoveto{\pgfqpoint{0.667731in}{2.702617in}}%
\pgfpathlineto{\pgfqpoint{5.410148in}{2.702617in}}%
\pgfusepath{stroke}%
\end{pgfscope}%
\begin{pgfscope}%
\definecolor{textcolor}{rgb}{0.150000,0.150000,0.150000}%
\pgfsetstrokecolor{textcolor}%
\pgfsetfillcolor{textcolor}%
\pgftext[x=0.383703in, y=2.649810in, left, base]{\color{textcolor}\sffamily\fontsize{11.000000}{13.200000}\selectfont \(\displaystyle {68}\)}%
\end{pgfscope}%
\begin{pgfscope}%
\pgfpathrectangle{\pgfqpoint{0.667731in}{0.650833in}}{\pgfqpoint{4.742417in}{2.714069in}}%
\pgfusepath{clip}%
\pgfsetroundcap%
\pgfsetroundjoin%
\pgfsetlinewidth{1.003750pt}%
\definecolor{currentstroke}{rgb}{1.000000,1.000000,1.000000}%
\pgfsetstrokecolor{currentstroke}%
\pgfsetdash{}{0pt}%
\pgfpathmoveto{\pgfqpoint{0.667731in}{3.104908in}}%
\pgfpathlineto{\pgfqpoint{5.410148in}{3.104908in}}%
\pgfusepath{stroke}%
\end{pgfscope}%
\begin{pgfscope}%
\definecolor{textcolor}{rgb}{0.150000,0.150000,0.150000}%
\pgfsetstrokecolor{textcolor}%
\pgfsetfillcolor{textcolor}%
\pgftext[x=0.383703in, y=3.052101in, left, base]{\color{textcolor}\sffamily\fontsize{11.000000}{13.200000}\selectfont \(\displaystyle {69}\)}%
\end{pgfscope}%
\begin{pgfscope}%
\definecolor{textcolor}{rgb}{0.150000,0.150000,0.150000}%
\pgfsetstrokecolor{textcolor}%
\pgfsetfillcolor{textcolor}%
\pgftext[x=0.328148in,y=2.007867in,,bottom,rotate=90.000000]{\color{textcolor}\sffamily\fontsize{12.000000}{14.400000}\selectfont Average score}%
\end{pgfscope}%
\begin{pgfscope}%
\pgfpathrectangle{\pgfqpoint{0.667731in}{0.650833in}}{\pgfqpoint{4.742417in}{2.714069in}}%
\pgfusepath{clip}%
\pgfsetbuttcap%
\pgfsetroundjoin%
\pgfsetlinewidth{2.710125pt}%
\definecolor{currentstroke}{rgb}{0.298039,0.447059,0.690196}%
\pgfsetstrokecolor{currentstroke}%
\pgfsetdash{{9.990000pt}{4.320000pt}}{0.000000pt}%
\pgfpathmoveto{\pgfqpoint{0.964132in}{1.419841in}}%
\pgfpathlineto{\pgfqpoint{1.556934in}{2.406592in}}%
\pgfpathlineto{\pgfqpoint{2.149736in}{1.647553in}}%
\pgfpathlineto{\pgfqpoint{2.742538in}{1.799361in}}%
\pgfpathlineto{\pgfqpoint{3.335341in}{3.013823in}}%
\pgfpathlineto{\pgfqpoint{3.928143in}{3.241535in}}%
\pgfpathlineto{\pgfqpoint{4.520945in}{2.216832in}}%
\pgfpathlineto{\pgfqpoint{5.113747in}{2.824063in}}%
\pgfusepath{stroke}%
\end{pgfscope}%
\begin{pgfscope}%
\pgfpathrectangle{\pgfqpoint{0.667731in}{0.650833in}}{\pgfqpoint{4.742417in}{2.714069in}}%
\pgfusepath{clip}%
\pgfsetroundcap%
\pgfsetroundjoin%
\pgfsetlinewidth{2.710125pt}%
\definecolor{currentstroke}{rgb}{0.298039,0.447059,0.690196}%
\pgfsetstrokecolor{currentstroke}%
\pgfsetdash{}{0pt}%
\pgfusepath{stroke}%
\end{pgfscope}%
\begin{pgfscope}%
\pgfpathrectangle{\pgfqpoint{0.667731in}{0.650833in}}{\pgfqpoint{4.742417in}{2.714069in}}%
\pgfusepath{clip}%
\pgfsetroundcap%
\pgfsetroundjoin%
\pgfsetlinewidth{2.710125pt}%
\definecolor{currentstroke}{rgb}{0.298039,0.447059,0.690196}%
\pgfsetstrokecolor{currentstroke}%
\pgfsetdash{}{0pt}%
\pgfusepath{stroke}%
\end{pgfscope}%
\begin{pgfscope}%
\pgfpathrectangle{\pgfqpoint{0.667731in}{0.650833in}}{\pgfqpoint{4.742417in}{2.714069in}}%
\pgfusepath{clip}%
\pgfsetroundcap%
\pgfsetroundjoin%
\pgfsetlinewidth{2.710125pt}%
\definecolor{currentstroke}{rgb}{0.298039,0.447059,0.690196}%
\pgfsetstrokecolor{currentstroke}%
\pgfsetdash{}{0pt}%
\pgfusepath{stroke}%
\end{pgfscope}%
\begin{pgfscope}%
\pgfpathrectangle{\pgfqpoint{0.667731in}{0.650833in}}{\pgfqpoint{4.742417in}{2.714069in}}%
\pgfusepath{clip}%
\pgfsetroundcap%
\pgfsetroundjoin%
\pgfsetlinewidth{2.710125pt}%
\definecolor{currentstroke}{rgb}{0.298039,0.447059,0.690196}%
\pgfsetstrokecolor{currentstroke}%
\pgfsetdash{}{0pt}%
\pgfusepath{stroke}%
\end{pgfscope}%
\begin{pgfscope}%
\pgfpathrectangle{\pgfqpoint{0.667731in}{0.650833in}}{\pgfqpoint{4.742417in}{2.714069in}}%
\pgfusepath{clip}%
\pgfsetroundcap%
\pgfsetroundjoin%
\pgfsetlinewidth{2.710125pt}%
\definecolor{currentstroke}{rgb}{0.298039,0.447059,0.690196}%
\pgfsetstrokecolor{currentstroke}%
\pgfsetdash{}{0pt}%
\pgfusepath{stroke}%
\end{pgfscope}%
\begin{pgfscope}%
\pgfpathrectangle{\pgfqpoint{0.667731in}{0.650833in}}{\pgfqpoint{4.742417in}{2.714069in}}%
\pgfusepath{clip}%
\pgfsetroundcap%
\pgfsetroundjoin%
\pgfsetlinewidth{2.710125pt}%
\definecolor{currentstroke}{rgb}{0.298039,0.447059,0.690196}%
\pgfsetstrokecolor{currentstroke}%
\pgfsetdash{}{0pt}%
\pgfusepath{stroke}%
\end{pgfscope}%
\begin{pgfscope}%
\pgfpathrectangle{\pgfqpoint{0.667731in}{0.650833in}}{\pgfqpoint{4.742417in}{2.714069in}}%
\pgfusepath{clip}%
\pgfsetroundcap%
\pgfsetroundjoin%
\pgfsetlinewidth{2.710125pt}%
\definecolor{currentstroke}{rgb}{0.298039,0.447059,0.690196}%
\pgfsetstrokecolor{currentstroke}%
\pgfsetdash{}{0pt}%
\pgfusepath{stroke}%
\end{pgfscope}%
\begin{pgfscope}%
\pgfpathrectangle{\pgfqpoint{0.667731in}{0.650833in}}{\pgfqpoint{4.742417in}{2.714069in}}%
\pgfusepath{clip}%
\pgfsetroundcap%
\pgfsetroundjoin%
\pgfsetlinewidth{2.710125pt}%
\definecolor{currentstroke}{rgb}{0.298039,0.447059,0.690196}%
\pgfsetstrokecolor{currentstroke}%
\pgfsetdash{}{0pt}%
\pgfusepath{stroke}%
\end{pgfscope}%
\begin{pgfscope}%
\pgfpathrectangle{\pgfqpoint{0.667731in}{0.650833in}}{\pgfqpoint{4.742417in}{2.714069in}}%
\pgfusepath{clip}%
\pgfsetbuttcap%
\pgfsetroundjoin%
\definecolor{currentfill}{rgb}{0.298039,0.447059,0.690196}%
\pgfsetfillcolor{currentfill}%
\pgfsetlinewidth{2.032594pt}%
\definecolor{currentstroke}{rgb}{0.298039,0.447059,0.690196}%
\pgfsetstrokecolor{currentstroke}%
\pgfsetdash{}{0pt}%
\pgfsys@defobject{currentmarker}{\pgfqpoint{-0.046999in}{-0.046999in}}{\pgfqpoint{0.046999in}{0.046999in}}{%
\pgfpathmoveto{\pgfqpoint{0.000000in}{-0.046999in}}%
\pgfpathcurveto{\pgfqpoint{0.012464in}{-0.046999in}}{\pgfqpoint{0.024420in}{-0.042047in}}{\pgfqpoint{0.033234in}{-0.033234in}}%
\pgfpathcurveto{\pgfqpoint{0.042047in}{-0.024420in}}{\pgfqpoint{0.046999in}{-0.012464in}}{\pgfqpoint{0.046999in}{0.000000in}}%
\pgfpathcurveto{\pgfqpoint{0.046999in}{0.012464in}}{\pgfqpoint{0.042047in}{0.024420in}}{\pgfqpoint{0.033234in}{0.033234in}}%
\pgfpathcurveto{\pgfqpoint{0.024420in}{0.042047in}}{\pgfqpoint{0.012464in}{0.046999in}}{\pgfqpoint{0.000000in}{0.046999in}}%
\pgfpathcurveto{\pgfqpoint{-0.012464in}{0.046999in}}{\pgfqpoint{-0.024420in}{0.042047in}}{\pgfqpoint{-0.033234in}{0.033234in}}%
\pgfpathcurveto{\pgfqpoint{-0.042047in}{0.024420in}}{\pgfqpoint{-0.046999in}{0.012464in}}{\pgfqpoint{-0.046999in}{0.000000in}}%
\pgfpathcurveto{\pgfqpoint{-0.046999in}{-0.012464in}}{\pgfqpoint{-0.042047in}{-0.024420in}}{\pgfqpoint{-0.033234in}{-0.033234in}}%
\pgfpathcurveto{\pgfqpoint{-0.024420in}{-0.042047in}}{\pgfqpoint{-0.012464in}{-0.046999in}}{\pgfqpoint{0.000000in}{-0.046999in}}%
\pgfpathlineto{\pgfqpoint{0.000000in}{-0.046999in}}%
\pgfpathclose%
\pgfusepath{stroke,fill}%
}%
\begin{pgfscope}%
\pgfsys@transformshift{0.964132in}{1.419841in}%
\pgfsys@useobject{currentmarker}{}%
\end{pgfscope}%
\begin{pgfscope}%
\pgfsys@transformshift{1.556934in}{2.406592in}%
\pgfsys@useobject{currentmarker}{}%
\end{pgfscope}%
\begin{pgfscope}%
\pgfsys@transformshift{2.149736in}{1.647553in}%
\pgfsys@useobject{currentmarker}{}%
\end{pgfscope}%
\begin{pgfscope}%
\pgfsys@transformshift{2.742538in}{1.799361in}%
\pgfsys@useobject{currentmarker}{}%
\end{pgfscope}%
\begin{pgfscope}%
\pgfsys@transformshift{3.335341in}{3.013823in}%
\pgfsys@useobject{currentmarker}{}%
\end{pgfscope}%
\begin{pgfscope}%
\pgfsys@transformshift{3.928143in}{3.241535in}%
\pgfsys@useobject{currentmarker}{}%
\end{pgfscope}%
\begin{pgfscope}%
\pgfsys@transformshift{4.520945in}{2.216832in}%
\pgfsys@useobject{currentmarker}{}%
\end{pgfscope}%
\begin{pgfscope}%
\pgfsys@transformshift{5.113747in}{2.824063in}%
\pgfsys@useobject{currentmarker}{}%
\end{pgfscope}%
\end{pgfscope}%
\begin{pgfscope}%
\pgfpathrectangle{\pgfqpoint{0.667731in}{0.650833in}}{\pgfqpoint{4.742417in}{2.714069in}}%
\pgfusepath{clip}%
\pgfsetbuttcap%
\pgfsetroundjoin%
\pgfsetlinewidth{2.710125pt}%
\definecolor{currentstroke}{rgb}{0.866667,0.517647,0.321569}%
\pgfsetstrokecolor{currentstroke}%
\pgfsetdash{{9.990000pt}{4.320000pt}}{0.000000pt}%
\pgfpathmoveto{\pgfqpoint{0.964132in}{0.774199in}}%
\pgfpathlineto{\pgfqpoint{1.556934in}{1.375127in}}%
\pgfpathlineto{\pgfqpoint{2.149736in}{0.807932in}}%
\pgfpathlineto{\pgfqpoint{2.742538in}{1.052932in}}%
\pgfpathlineto{\pgfqpoint{3.335341in}{2.339383in}}%
\pgfpathlineto{\pgfqpoint{3.928143in}{2.641892in}}%
\pgfpathlineto{\pgfqpoint{4.520945in}{1.468794in}}%
\pgfpathlineto{\pgfqpoint{5.113747in}{2.204825in}}%
\pgfusepath{stroke}%
\end{pgfscope}%
\begin{pgfscope}%
\pgfpathrectangle{\pgfqpoint{0.667731in}{0.650833in}}{\pgfqpoint{4.742417in}{2.714069in}}%
\pgfusepath{clip}%
\pgfsetroundcap%
\pgfsetroundjoin%
\pgfsetlinewidth{2.710125pt}%
\definecolor{currentstroke}{rgb}{0.866667,0.517647,0.321569}%
\pgfsetstrokecolor{currentstroke}%
\pgfsetdash{}{0pt}%
\pgfusepath{stroke}%
\end{pgfscope}%
\begin{pgfscope}%
\pgfpathrectangle{\pgfqpoint{0.667731in}{0.650833in}}{\pgfqpoint{4.742417in}{2.714069in}}%
\pgfusepath{clip}%
\pgfsetroundcap%
\pgfsetroundjoin%
\pgfsetlinewidth{2.710125pt}%
\definecolor{currentstroke}{rgb}{0.866667,0.517647,0.321569}%
\pgfsetstrokecolor{currentstroke}%
\pgfsetdash{}{0pt}%
\pgfusepath{stroke}%
\end{pgfscope}%
\begin{pgfscope}%
\pgfpathrectangle{\pgfqpoint{0.667731in}{0.650833in}}{\pgfqpoint{4.742417in}{2.714069in}}%
\pgfusepath{clip}%
\pgfsetroundcap%
\pgfsetroundjoin%
\pgfsetlinewidth{2.710125pt}%
\definecolor{currentstroke}{rgb}{0.866667,0.517647,0.321569}%
\pgfsetstrokecolor{currentstroke}%
\pgfsetdash{}{0pt}%
\pgfusepath{stroke}%
\end{pgfscope}%
\begin{pgfscope}%
\pgfpathrectangle{\pgfqpoint{0.667731in}{0.650833in}}{\pgfqpoint{4.742417in}{2.714069in}}%
\pgfusepath{clip}%
\pgfsetroundcap%
\pgfsetroundjoin%
\pgfsetlinewidth{2.710125pt}%
\definecolor{currentstroke}{rgb}{0.866667,0.517647,0.321569}%
\pgfsetstrokecolor{currentstroke}%
\pgfsetdash{}{0pt}%
\pgfusepath{stroke}%
\end{pgfscope}%
\begin{pgfscope}%
\pgfpathrectangle{\pgfqpoint{0.667731in}{0.650833in}}{\pgfqpoint{4.742417in}{2.714069in}}%
\pgfusepath{clip}%
\pgfsetroundcap%
\pgfsetroundjoin%
\pgfsetlinewidth{2.710125pt}%
\definecolor{currentstroke}{rgb}{0.866667,0.517647,0.321569}%
\pgfsetstrokecolor{currentstroke}%
\pgfsetdash{}{0pt}%
\pgfusepath{stroke}%
\end{pgfscope}%
\begin{pgfscope}%
\pgfpathrectangle{\pgfqpoint{0.667731in}{0.650833in}}{\pgfqpoint{4.742417in}{2.714069in}}%
\pgfusepath{clip}%
\pgfsetroundcap%
\pgfsetroundjoin%
\pgfsetlinewidth{2.710125pt}%
\definecolor{currentstroke}{rgb}{0.866667,0.517647,0.321569}%
\pgfsetstrokecolor{currentstroke}%
\pgfsetdash{}{0pt}%
\pgfusepath{stroke}%
\end{pgfscope}%
\begin{pgfscope}%
\pgfpathrectangle{\pgfqpoint{0.667731in}{0.650833in}}{\pgfqpoint{4.742417in}{2.714069in}}%
\pgfusepath{clip}%
\pgfsetroundcap%
\pgfsetroundjoin%
\pgfsetlinewidth{2.710125pt}%
\definecolor{currentstroke}{rgb}{0.866667,0.517647,0.321569}%
\pgfsetstrokecolor{currentstroke}%
\pgfsetdash{}{0pt}%
\pgfusepath{stroke}%
\end{pgfscope}%
\begin{pgfscope}%
\pgfpathrectangle{\pgfqpoint{0.667731in}{0.650833in}}{\pgfqpoint{4.742417in}{2.714069in}}%
\pgfusepath{clip}%
\pgfsetroundcap%
\pgfsetroundjoin%
\pgfsetlinewidth{2.710125pt}%
\definecolor{currentstroke}{rgb}{0.866667,0.517647,0.321569}%
\pgfsetstrokecolor{currentstroke}%
\pgfsetdash{}{0pt}%
\pgfusepath{stroke}%
\end{pgfscope}%
\begin{pgfscope}%
\pgfpathrectangle{\pgfqpoint{0.667731in}{0.650833in}}{\pgfqpoint{4.742417in}{2.714069in}}%
\pgfusepath{clip}%
\pgfsetbuttcap%
\pgfsetroundjoin%
\definecolor{currentfill}{rgb}{0.866667,0.517647,0.321569}%
\pgfsetfillcolor{currentfill}%
\pgfsetlinewidth{2.032594pt}%
\definecolor{currentstroke}{rgb}{0.866667,0.517647,0.321569}%
\pgfsetstrokecolor{currentstroke}%
\pgfsetdash{}{0pt}%
\pgfsys@defobject{currentmarker}{\pgfqpoint{-0.046999in}{-0.046999in}}{\pgfqpoint{0.046999in}{0.046999in}}{%
\pgfpathmoveto{\pgfqpoint{0.000000in}{-0.046999in}}%
\pgfpathcurveto{\pgfqpoint{0.012464in}{-0.046999in}}{\pgfqpoint{0.024420in}{-0.042047in}}{\pgfqpoint{0.033234in}{-0.033234in}}%
\pgfpathcurveto{\pgfqpoint{0.042047in}{-0.024420in}}{\pgfqpoint{0.046999in}{-0.012464in}}{\pgfqpoint{0.046999in}{0.000000in}}%
\pgfpathcurveto{\pgfqpoint{0.046999in}{0.012464in}}{\pgfqpoint{0.042047in}{0.024420in}}{\pgfqpoint{0.033234in}{0.033234in}}%
\pgfpathcurveto{\pgfqpoint{0.024420in}{0.042047in}}{\pgfqpoint{0.012464in}{0.046999in}}{\pgfqpoint{0.000000in}{0.046999in}}%
\pgfpathcurveto{\pgfqpoint{-0.012464in}{0.046999in}}{\pgfqpoint{-0.024420in}{0.042047in}}{\pgfqpoint{-0.033234in}{0.033234in}}%
\pgfpathcurveto{\pgfqpoint{-0.042047in}{0.024420in}}{\pgfqpoint{-0.046999in}{0.012464in}}{\pgfqpoint{-0.046999in}{0.000000in}}%
\pgfpathcurveto{\pgfqpoint{-0.046999in}{-0.012464in}}{\pgfqpoint{-0.042047in}{-0.024420in}}{\pgfqpoint{-0.033234in}{-0.033234in}}%
\pgfpathcurveto{\pgfqpoint{-0.024420in}{-0.042047in}}{\pgfqpoint{-0.012464in}{-0.046999in}}{\pgfqpoint{0.000000in}{-0.046999in}}%
\pgfpathlineto{\pgfqpoint{0.000000in}{-0.046999in}}%
\pgfpathclose%
\pgfusepath{stroke,fill}%
}%
\begin{pgfscope}%
\pgfsys@transformshift{0.964132in}{0.774199in}%
\pgfsys@useobject{currentmarker}{}%
\end{pgfscope}%
\begin{pgfscope}%
\pgfsys@transformshift{1.556934in}{1.375127in}%
\pgfsys@useobject{currentmarker}{}%
\end{pgfscope}%
\begin{pgfscope}%
\pgfsys@transformshift{2.149736in}{0.807932in}%
\pgfsys@useobject{currentmarker}{}%
\end{pgfscope}%
\begin{pgfscope}%
\pgfsys@transformshift{2.742538in}{1.052932in}%
\pgfsys@useobject{currentmarker}{}%
\end{pgfscope}%
\begin{pgfscope}%
\pgfsys@transformshift{3.335341in}{2.339383in}%
\pgfsys@useobject{currentmarker}{}%
\end{pgfscope}%
\begin{pgfscope}%
\pgfsys@transformshift{3.928143in}{2.641892in}%
\pgfsys@useobject{currentmarker}{}%
\end{pgfscope}%
\begin{pgfscope}%
\pgfsys@transformshift{4.520945in}{1.468794in}%
\pgfsys@useobject{currentmarker}{}%
\end{pgfscope}%
\begin{pgfscope}%
\pgfsys@transformshift{5.113747in}{2.204825in}%
\pgfsys@useobject{currentmarker}{}%
\end{pgfscope}%
\end{pgfscope}%
\begin{pgfscope}%
\pgfsetrectcap%
\pgfsetmiterjoin%
\pgfsetlinewidth{1.254687pt}%
\definecolor{currentstroke}{rgb}{1.000000,1.000000,1.000000}%
\pgfsetstrokecolor{currentstroke}%
\pgfsetdash{}{0pt}%
\pgfpathmoveto{\pgfqpoint{0.667731in}{0.650833in}}%
\pgfpathlineto{\pgfqpoint{0.667731in}{3.364902in}}%
\pgfusepath{stroke}%
\end{pgfscope}%
\begin{pgfscope}%
\pgfsetrectcap%
\pgfsetmiterjoin%
\pgfsetlinewidth{1.254687pt}%
\definecolor{currentstroke}{rgb}{1.000000,1.000000,1.000000}%
\pgfsetstrokecolor{currentstroke}%
\pgfsetdash{}{0pt}%
\pgfpathmoveto{\pgfqpoint{5.410148in}{0.650833in}}%
\pgfpathlineto{\pgfqpoint{5.410148in}{3.364902in}}%
\pgfusepath{stroke}%
\end{pgfscope}%
\begin{pgfscope}%
\pgfsetrectcap%
\pgfsetmiterjoin%
\pgfsetlinewidth{1.254687pt}%
\definecolor{currentstroke}{rgb}{1.000000,1.000000,1.000000}%
\pgfsetstrokecolor{currentstroke}%
\pgfsetdash{}{0pt}%
\pgfpathmoveto{\pgfqpoint{0.667731in}{0.650833in}}%
\pgfpathlineto{\pgfqpoint{5.410148in}{0.650833in}}%
\pgfusepath{stroke}%
\end{pgfscope}%
\begin{pgfscope}%
\pgfsetrectcap%
\pgfsetmiterjoin%
\pgfsetlinewidth{1.254687pt}%
\definecolor{currentstroke}{rgb}{1.000000,1.000000,1.000000}%
\pgfsetstrokecolor{currentstroke}%
\pgfsetdash{}{0pt}%
\pgfpathmoveto{\pgfqpoint{0.667731in}{3.364902in}}%
\pgfpathlineto{\pgfqpoint{5.410148in}{3.364902in}}%
\pgfusepath{stroke}%
\end{pgfscope}%
\begin{pgfscope}%
\pgfsetbuttcap%
\pgfsetmiterjoin%
\definecolor{currentfill}{rgb}{0.917647,0.917647,0.949020}%
\pgfsetfillcolor{currentfill}%
\pgfsetfillopacity{0.800000}%
\pgfsetlinewidth{1.003750pt}%
\definecolor{currentstroke}{rgb}{0.800000,0.800000,0.800000}%
\pgfsetstrokecolor{currentstroke}%
\pgfsetstrokeopacity{0.800000}%
\pgfsetdash{}{0pt}%
\pgfpathmoveto{\pgfqpoint{0.774675in}{2.592333in}}%
\pgfpathlineto{\pgfqpoint{1.866407in}{2.592333in}}%
\pgfpathquadraticcurveto{\pgfqpoint{1.896963in}{2.592333in}}{\pgfqpoint{1.896963in}{2.622888in}}%
\pgfpathlineto{\pgfqpoint{1.896963in}{3.257957in}}%
\pgfpathquadraticcurveto{\pgfqpoint{1.896963in}{3.288513in}}{\pgfqpoint{1.866407in}{3.288513in}}%
\pgfpathlineto{\pgfqpoint{0.774675in}{3.288513in}}%
\pgfpathquadraticcurveto{\pgfqpoint{0.744120in}{3.288513in}}{\pgfqpoint{0.744120in}{3.257957in}}%
\pgfpathlineto{\pgfqpoint{0.744120in}{2.622888in}}%
\pgfpathquadraticcurveto{\pgfqpoint{0.744120in}{2.592333in}}{\pgfqpoint{0.774675in}{2.592333in}}%
\pgfpathlineto{\pgfqpoint{0.774675in}{2.592333in}}%
\pgfpathclose%
\pgfusepath{stroke,fill}%
\end{pgfscope}%
\begin{pgfscope}%
\definecolor{textcolor}{rgb}{0.150000,0.150000,0.150000}%
\pgfsetstrokecolor{textcolor}%
\pgfsetfillcolor{textcolor}%
\pgftext[x=1.076929in,y=3.111661in,left,base]{\color{textcolor}\sffamily\fontsize{12.000000}{14.400000}\selectfont F-score}%
\end{pgfscope}%
\begin{pgfscope}%
\pgfsetbuttcap%
\pgfsetroundjoin%
\definecolor{currentfill}{rgb}{0.298039,0.447059,0.690196}%
\pgfsetfillcolor{currentfill}%
\pgfsetlinewidth{2.032594pt}%
\definecolor{currentstroke}{rgb}{0.298039,0.447059,0.690196}%
\pgfsetstrokecolor{currentstroke}%
\pgfsetdash{}{0pt}%
\pgfsys@defobject{currentmarker}{\pgfqpoint{-0.046999in}{-0.046999in}}{\pgfqpoint{0.046999in}{0.046999in}}{%
\pgfpathmoveto{\pgfqpoint{0.000000in}{-0.046999in}}%
\pgfpathcurveto{\pgfqpoint{0.012464in}{-0.046999in}}{\pgfqpoint{0.024420in}{-0.042047in}}{\pgfqpoint{0.033234in}{-0.033234in}}%
\pgfpathcurveto{\pgfqpoint{0.042047in}{-0.024420in}}{\pgfqpoint{0.046999in}{-0.012464in}}{\pgfqpoint{0.046999in}{0.000000in}}%
\pgfpathcurveto{\pgfqpoint{0.046999in}{0.012464in}}{\pgfqpoint{0.042047in}{0.024420in}}{\pgfqpoint{0.033234in}{0.033234in}}%
\pgfpathcurveto{\pgfqpoint{0.024420in}{0.042047in}}{\pgfqpoint{0.012464in}{0.046999in}}{\pgfqpoint{0.000000in}{0.046999in}}%
\pgfpathcurveto{\pgfqpoint{-0.012464in}{0.046999in}}{\pgfqpoint{-0.024420in}{0.042047in}}{\pgfqpoint{-0.033234in}{0.033234in}}%
\pgfpathcurveto{\pgfqpoint{-0.042047in}{0.024420in}}{\pgfqpoint{-0.046999in}{0.012464in}}{\pgfqpoint{-0.046999in}{0.000000in}}%
\pgfpathcurveto{\pgfqpoint{-0.046999in}{-0.012464in}}{\pgfqpoint{-0.042047in}{-0.024420in}}{\pgfqpoint{-0.033234in}{-0.033234in}}%
\pgfpathcurveto{\pgfqpoint{-0.024420in}{-0.042047in}}{\pgfqpoint{-0.012464in}{-0.046999in}}{\pgfqpoint{0.000000in}{-0.046999in}}%
\pgfpathlineto{\pgfqpoint{0.000000in}{-0.046999in}}%
\pgfpathclose%
\pgfusepath{stroke,fill}%
}%
\begin{pgfscope}%
\pgfsys@transformshift{0.958009in}{2.936025in}%
\pgfsys@useobject{currentmarker}{}%
\end{pgfscope}%
\end{pgfscope}%
\begin{pgfscope}%
\definecolor{textcolor}{rgb}{0.150000,0.150000,0.150000}%
\pgfsetstrokecolor{textcolor}%
\pgfsetfillcolor{textcolor}%
\pgftext[x=1.233009in,y=2.895921in,left,base]{\color{textcolor}\sffamily\fontsize{11.000000}{13.200000}\selectfont F1 micro}%
\end{pgfscope}%
\begin{pgfscope}%
\pgfsetbuttcap%
\pgfsetroundjoin%
\definecolor{currentfill}{rgb}{0.866667,0.517647,0.321569}%
\pgfsetfillcolor{currentfill}%
\pgfsetlinewidth{2.032594pt}%
\definecolor{currentstroke}{rgb}{0.866667,0.517647,0.321569}%
\pgfsetstrokecolor{currentstroke}%
\pgfsetdash{}{0pt}%
\pgfsys@defobject{currentmarker}{\pgfqpoint{-0.046999in}{-0.046999in}}{\pgfqpoint{0.046999in}{0.046999in}}{%
\pgfpathmoveto{\pgfqpoint{0.000000in}{-0.046999in}}%
\pgfpathcurveto{\pgfqpoint{0.012464in}{-0.046999in}}{\pgfqpoint{0.024420in}{-0.042047in}}{\pgfqpoint{0.033234in}{-0.033234in}}%
\pgfpathcurveto{\pgfqpoint{0.042047in}{-0.024420in}}{\pgfqpoint{0.046999in}{-0.012464in}}{\pgfqpoint{0.046999in}{0.000000in}}%
\pgfpathcurveto{\pgfqpoint{0.046999in}{0.012464in}}{\pgfqpoint{0.042047in}{0.024420in}}{\pgfqpoint{0.033234in}{0.033234in}}%
\pgfpathcurveto{\pgfqpoint{0.024420in}{0.042047in}}{\pgfqpoint{0.012464in}{0.046999in}}{\pgfqpoint{0.000000in}{0.046999in}}%
\pgfpathcurveto{\pgfqpoint{-0.012464in}{0.046999in}}{\pgfqpoint{-0.024420in}{0.042047in}}{\pgfqpoint{-0.033234in}{0.033234in}}%
\pgfpathcurveto{\pgfqpoint{-0.042047in}{0.024420in}}{\pgfqpoint{-0.046999in}{0.012464in}}{\pgfqpoint{-0.046999in}{0.000000in}}%
\pgfpathcurveto{\pgfqpoint{-0.046999in}{-0.012464in}}{\pgfqpoint{-0.042047in}{-0.024420in}}{\pgfqpoint{-0.033234in}{-0.033234in}}%
\pgfpathcurveto{\pgfqpoint{-0.024420in}{-0.042047in}}{\pgfqpoint{-0.012464in}{-0.046999in}}{\pgfqpoint{0.000000in}{-0.046999in}}%
\pgfpathlineto{\pgfqpoint{0.000000in}{-0.046999in}}%
\pgfpathclose%
\pgfusepath{stroke,fill}%
}%
\begin{pgfscope}%
\pgfsys@transformshift{0.958009in}{2.723120in}%
\pgfsys@useobject{currentmarker}{}%
\end{pgfscope}%
\end{pgfscope}%
\begin{pgfscope}%
\definecolor{textcolor}{rgb}{0.150000,0.150000,0.150000}%
\pgfsetstrokecolor{textcolor}%
\pgfsetfillcolor{textcolor}%
\pgftext[x=1.233009in,y=2.683016in,left,base]{\color{textcolor}\sffamily\fontsize{11.000000}{13.200000}\selectfont F1 macro}%
\end{pgfscope}%
\end{pgfpicture}%
\makeatother%
\endgroup%
}
        \label{fig:deepwalk:number walk length}
    \end{subfigure}
    \caption{DeepWalk results.}
    \label{fig:deepwalk:plots}
\end{figure}

A new classification model is trained on the cleaned embedding to classify fake and real news.
We use the news labels provided with the dataset as targets.
Ideally news should be grouped together by veracity in the resulting embedding space since they share similar characteristics.
Here, we want to test if DeepWalk was able to group them well.

To obtain the best performance, each hyperparameter of the algorithm is tuned one after the other.
We first tune the number of dimensions \textit{d}, with all the other parameters set to their respective default.
We select the number of dimensions which resulted in the best F-score.
This value is then used for all remaining parameters tuning.
This process is repeated for the \textit{w}, \textit{\gamma} and \textit{t} parameters in this order.
We always use the best parameter value for the parameters previously tuned, and leave the other left to tune to their default values.

Figure~\ref{fig:deepwalk:plots} showcases the F-score obtained for all 4 parameters tuning.

For DeepWalk, the best parameters are $d=64, w=10, \gamma=20, t=40$.
For the misinformation network, DeepWalks prefer few but quite short random walks with an average number of vertices to chose from.

\newpage
\section{Node2vec}

The node2vec algorithm uses a similar embedding approach as DeepWalk: random walks through the network to learn latent features.

However node2vec's authors found that purely random walk sometimes did not give the best possible result.
Indeed if the walks are purely random, we could obtain very different embeddings between two instances of DeepWalk on the same network.
For this reason, they decided to add two additional hyperparameters: tuning for breadth first search and depth first search.
These two concepts can be tuned through the return parameter denoted \textit{p} and in-out parameter denoted \textit{q}.
These two parameters are probability and influences the exploration of the network.
They are used as $1/p$ and $1/p$ by the algorithm in the graph.

The return parameter controls how often the algorithm revisits a node previously visited in the network.
Thanks to this parameter, we can tune the random walks behaviour.
A high value for this parameter makes it less likely we revisit a node and encourages exploration.
While A low value ensure the algorithm walks closely to the starting node.

The in-out parameter controls the probability to visit not yet visited neighbours. A high value will encourage exploration away from the neighborhood while a low value will lead to a more conservative walk on the graph. 

\subsection{Node2vec with Disinformation Dataset and Parameters Sensitivity}

Just like for Deepwalk, we again perform the embedding with node2vec on the integrity of the network to gain from the additional context provided by users and news source.
We always clean the resulting embedding, since we want to perform the classification on the news entity only.

Before tuning the four hyperparameters for the output embedding, we must first tune the \textit{p} and \textit{q} parameters.
We obtained embeddings by iterating over these two parameters with a set of 5 values for each, for a total of 25 runs. All other parameters are set to their default.
Table~\ref{tab:node2vec:pqMicro} shows us the micro F1 score obtained by tuning these parameters.
We obtain $p=2$ and $q=1$ as the best parameters for the node2vec with the fake news dataset.
\begin{table}[h]
    \centering
    \caption{Micro F1 score for $p$ and $q$.}
    \label{tab:node2vec:pqMicro}
    \begin{tabular}{llllll}
        \diagbox[height=0.8cm]{p}{q} & 0.25 & 0.5 & 1.0 & 2.0 & 4.0 \\
        0.25 & 0.6564 & 0.6720 & 0.6550 & 0.6583 & 0.6564 \\
        0.5 & 0.6493 & 0.6635 & 0.6687 & 0.6408 & 0.6450 \\
        1.0 & 0.6649 & 0.6848 & 0.6536 & 0.6640 & 0.6588 \\
        2.0 & 0.6829 & 0.6716 & \bfseries 0.6943 & 0.6744 & 0.6654 \\
        4.0 & 0.6749 & 0.6739 & 0.6773 & 0.6540 & 0.6815 \\
    \end{tabular}
\end{table}

Just like for DeepWalk, when then tune each four hyperparameters so we can obtain the best possible results for this algorithm on our dataset.
We first tune the number of dimensions and leave all other parameters left to tune to their default. 
We pick the parameter value for the number of dimensions which gave the best F1 score and use this value for all next runs. 
The parameters which gave the best F1 score for each run is picked for the tuning of the following parameters.
We continue doing this for the window size, the random walk length and the number of random walk. 

\begin{figure}[h]
    \centering
    \begin{subfigure}[b]{0.49\textwidth}
        \centering
        \scalebox{.5}{%% Creator: Matplotlib, PGF backend
%%
%% To include the figure in your LaTeX document, write
%%   \input{<filename>.pgf}
%%
%% Make sure the required packages are loaded in your preamble
%%   \usepackage{pgf}
%%
%% Also ensure that all the required font packages are loaded; for instance,
%% the lmodern package is sometimes necessary when using math font.
%%   \usepackage{lmodern}
%%
%% Figures using additional raster images can only be included by \input if
%% they are in the same directory as the main LaTeX file. For loading figures
%% from other directories you can use the `import` package
%%   \usepackage{import}
%%
%% and then include the figures with
%%   \import{<path to file>}{<filename>.pgf}
%%
%% Matplotlib used the following preamble
%%   
%%   \makeatletter\@ifpackageloaded{underscore}{}{\usepackage[strings]{underscore}}\makeatother
%%
\begingroup%
\makeatletter%
\begin{pgfpicture}%
\pgfpathrectangle{\pgfpointorigin}{\pgfqpoint{5.590148in}{6.909803in}}%
\pgfusepath{use as bounding box, clip}%
\begin{pgfscope}%
\pgfsetbuttcap%
\pgfsetmiterjoin%
\definecolor{currentfill}{rgb}{1.000000,1.000000,1.000000}%
\pgfsetfillcolor{currentfill}%
\pgfsetlinewidth{0.000000pt}%
\definecolor{currentstroke}{rgb}{1.000000,1.000000,1.000000}%
\pgfsetstrokecolor{currentstroke}%
\pgfsetdash{}{0pt}%
\pgfpathmoveto{\pgfqpoint{0.000000in}{0.000000in}}%
\pgfpathlineto{\pgfqpoint{5.590148in}{0.000000in}}%
\pgfpathlineto{\pgfqpoint{5.590148in}{6.909803in}}%
\pgfpathlineto{\pgfqpoint{0.000000in}{6.909803in}}%
\pgfpathlineto{\pgfqpoint{0.000000in}{0.000000in}}%
\pgfpathclose%
\pgfusepath{fill}%
\end{pgfscope}%
\begin{pgfscope}%
\pgfsetbuttcap%
\pgfsetmiterjoin%
\definecolor{currentfill}{rgb}{0.917647,0.917647,0.949020}%
\pgfsetfillcolor{currentfill}%
\pgfsetlinewidth{0.000000pt}%
\definecolor{currentstroke}{rgb}{0.000000,0.000000,0.000000}%
\pgfsetstrokecolor{currentstroke}%
\pgfsetstrokeopacity{0.000000}%
\pgfsetdash{}{0pt}%
\pgfpathmoveto{\pgfqpoint{0.667731in}{4.015734in}}%
\pgfpathlineto{\pgfqpoint{5.410148in}{4.015734in}}%
\pgfpathlineto{\pgfqpoint{5.410148in}{6.729803in}}%
\pgfpathlineto{\pgfqpoint{0.667731in}{6.729803in}}%
\pgfpathlineto{\pgfqpoint{0.667731in}{4.015734in}}%
\pgfpathclose%
\pgfusepath{fill}%
\end{pgfscope}%
\begin{pgfscope}%
\definecolor{textcolor}{rgb}{0.150000,0.150000,0.150000}%
\pgfsetstrokecolor{textcolor}%
\pgfsetfillcolor{textcolor}%
\pgftext[x=1.260533in,y=3.883790in,,top]{\color{textcolor}\sffamily\fontsize{11.000000}{13.200000}\selectfont 32}%
\end{pgfscope}%
\begin{pgfscope}%
\definecolor{textcolor}{rgb}{0.150000,0.150000,0.150000}%
\pgfsetstrokecolor{textcolor}%
\pgfsetfillcolor{textcolor}%
\pgftext[x=2.446137in,y=3.883790in,,top]{\color{textcolor}\sffamily\fontsize{11.000000}{13.200000}\selectfont 64}%
\end{pgfscope}%
\begin{pgfscope}%
\definecolor{textcolor}{rgb}{0.150000,0.150000,0.150000}%
\pgfsetstrokecolor{textcolor}%
\pgfsetfillcolor{textcolor}%
\pgftext[x=3.631742in,y=3.883790in,,top]{\color{textcolor}\sffamily\fontsize{11.000000}{13.200000}\selectfont 128}%
\end{pgfscope}%
\begin{pgfscope}%
\definecolor{textcolor}{rgb}{0.150000,0.150000,0.150000}%
\pgfsetstrokecolor{textcolor}%
\pgfsetfillcolor{textcolor}%
\pgftext[x=4.817346in,y=3.883790in,,top]{\color{textcolor}\sffamily\fontsize{11.000000}{13.200000}\selectfont 256}%
\end{pgfscope}%
\begin{pgfscope}%
\definecolor{textcolor}{rgb}{0.150000,0.150000,0.150000}%
\pgfsetstrokecolor{textcolor}%
\pgfsetfillcolor{textcolor}%
\pgftext[x=3.038940in,y=3.693049in,,top]{\color{textcolor}\sffamily\fontsize{12.000000}{14.400000}\selectfont number of dimensions}%
\end{pgfscope}%
\begin{pgfscope}%
\pgfpathrectangle{\pgfqpoint{0.667731in}{4.015734in}}{\pgfqpoint{4.742417in}{2.714069in}}%
\pgfusepath{clip}%
\pgfsetroundcap%
\pgfsetroundjoin%
\pgfsetlinewidth{1.003750pt}%
\definecolor{currentstroke}{rgb}{1.000000,1.000000,1.000000}%
\pgfsetstrokecolor{currentstroke}%
\pgfsetdash{}{0pt}%
\pgfpathmoveto{\pgfqpoint{0.667731in}{4.039097in}}%
\pgfpathlineto{\pgfqpoint{5.410148in}{4.039097in}}%
\pgfusepath{stroke}%
\end{pgfscope}%
\begin{pgfscope}%
\definecolor{textcolor}{rgb}{0.150000,0.150000,0.150000}%
\pgfsetstrokecolor{textcolor}%
\pgfsetfillcolor{textcolor}%
\pgftext[x=0.383703in, y=3.986290in, left, base]{\color{textcolor}\sffamily\fontsize{11.000000}{13.200000}\selectfont \(\displaystyle {63}\)}%
\end{pgfscope}%
\begin{pgfscope}%
\pgfpathrectangle{\pgfqpoint{0.667731in}{4.015734in}}{\pgfqpoint{4.742417in}{2.714069in}}%
\pgfusepath{clip}%
\pgfsetroundcap%
\pgfsetroundjoin%
\pgfsetlinewidth{1.003750pt}%
\definecolor{currentstroke}{rgb}{1.000000,1.000000,1.000000}%
\pgfsetstrokecolor{currentstroke}%
\pgfsetdash{}{0pt}%
\pgfpathmoveto{\pgfqpoint{0.667731in}{4.532101in}}%
\pgfpathlineto{\pgfqpoint{5.410148in}{4.532101in}}%
\pgfusepath{stroke}%
\end{pgfscope}%
\begin{pgfscope}%
\definecolor{textcolor}{rgb}{0.150000,0.150000,0.150000}%
\pgfsetstrokecolor{textcolor}%
\pgfsetfillcolor{textcolor}%
\pgftext[x=0.383703in, y=4.479294in, left, base]{\color{textcolor}\sffamily\fontsize{11.000000}{13.200000}\selectfont \(\displaystyle {64}\)}%
\end{pgfscope}%
\begin{pgfscope}%
\pgfpathrectangle{\pgfqpoint{0.667731in}{4.015734in}}{\pgfqpoint{4.742417in}{2.714069in}}%
\pgfusepath{clip}%
\pgfsetroundcap%
\pgfsetroundjoin%
\pgfsetlinewidth{1.003750pt}%
\definecolor{currentstroke}{rgb}{1.000000,1.000000,1.000000}%
\pgfsetstrokecolor{currentstroke}%
\pgfsetdash{}{0pt}%
\pgfpathmoveto{\pgfqpoint{0.667731in}{5.025104in}}%
\pgfpathlineto{\pgfqpoint{5.410148in}{5.025104in}}%
\pgfusepath{stroke}%
\end{pgfscope}%
\begin{pgfscope}%
\definecolor{textcolor}{rgb}{0.150000,0.150000,0.150000}%
\pgfsetstrokecolor{textcolor}%
\pgfsetfillcolor{textcolor}%
\pgftext[x=0.383703in, y=4.972297in, left, base]{\color{textcolor}\sffamily\fontsize{11.000000}{13.200000}\selectfont \(\displaystyle {65}\)}%
\end{pgfscope}%
\begin{pgfscope}%
\pgfpathrectangle{\pgfqpoint{0.667731in}{4.015734in}}{\pgfqpoint{4.742417in}{2.714069in}}%
\pgfusepath{clip}%
\pgfsetroundcap%
\pgfsetroundjoin%
\pgfsetlinewidth{1.003750pt}%
\definecolor{currentstroke}{rgb}{1.000000,1.000000,1.000000}%
\pgfsetstrokecolor{currentstroke}%
\pgfsetdash{}{0pt}%
\pgfpathmoveto{\pgfqpoint{0.667731in}{5.518108in}}%
\pgfpathlineto{\pgfqpoint{5.410148in}{5.518108in}}%
\pgfusepath{stroke}%
\end{pgfscope}%
\begin{pgfscope}%
\definecolor{textcolor}{rgb}{0.150000,0.150000,0.150000}%
\pgfsetstrokecolor{textcolor}%
\pgfsetfillcolor{textcolor}%
\pgftext[x=0.383703in, y=5.465301in, left, base]{\color{textcolor}\sffamily\fontsize{11.000000}{13.200000}\selectfont \(\displaystyle {66}\)}%
\end{pgfscope}%
\begin{pgfscope}%
\pgfpathrectangle{\pgfqpoint{0.667731in}{4.015734in}}{\pgfqpoint{4.742417in}{2.714069in}}%
\pgfusepath{clip}%
\pgfsetroundcap%
\pgfsetroundjoin%
\pgfsetlinewidth{1.003750pt}%
\definecolor{currentstroke}{rgb}{1.000000,1.000000,1.000000}%
\pgfsetstrokecolor{currentstroke}%
\pgfsetdash{}{0pt}%
\pgfpathmoveto{\pgfqpoint{0.667731in}{6.011111in}}%
\pgfpathlineto{\pgfqpoint{5.410148in}{6.011111in}}%
\pgfusepath{stroke}%
\end{pgfscope}%
\begin{pgfscope}%
\definecolor{textcolor}{rgb}{0.150000,0.150000,0.150000}%
\pgfsetstrokecolor{textcolor}%
\pgfsetfillcolor{textcolor}%
\pgftext[x=0.383703in, y=5.958305in, left, base]{\color{textcolor}\sffamily\fontsize{11.000000}{13.200000}\selectfont \(\displaystyle {67}\)}%
\end{pgfscope}%
\begin{pgfscope}%
\pgfpathrectangle{\pgfqpoint{0.667731in}{4.015734in}}{\pgfqpoint{4.742417in}{2.714069in}}%
\pgfusepath{clip}%
\pgfsetroundcap%
\pgfsetroundjoin%
\pgfsetlinewidth{1.003750pt}%
\definecolor{currentstroke}{rgb}{1.000000,1.000000,1.000000}%
\pgfsetstrokecolor{currentstroke}%
\pgfsetdash{}{0pt}%
\pgfpathmoveto{\pgfqpoint{0.667731in}{6.504115in}}%
\pgfpathlineto{\pgfqpoint{5.410148in}{6.504115in}}%
\pgfusepath{stroke}%
\end{pgfscope}%
\begin{pgfscope}%
\definecolor{textcolor}{rgb}{0.150000,0.150000,0.150000}%
\pgfsetstrokecolor{textcolor}%
\pgfsetfillcolor{textcolor}%
\pgftext[x=0.383703in, y=6.451308in, left, base]{\color{textcolor}\sffamily\fontsize{11.000000}{13.200000}\selectfont \(\displaystyle {68}\)}%
\end{pgfscope}%
\begin{pgfscope}%
\definecolor{textcolor}{rgb}{0.150000,0.150000,0.150000}%
\pgfsetstrokecolor{textcolor}%
\pgfsetfillcolor{textcolor}%
\pgftext[x=0.328148in,y=5.372769in,,bottom,rotate=90.000000]{\color{textcolor}\sffamily\fontsize{12.000000}{14.400000}\selectfont Average score}%
\end{pgfscope}%
\begin{pgfscope}%
\pgfpathrectangle{\pgfqpoint{0.667731in}{4.015734in}}{\pgfqpoint{4.742417in}{2.714069in}}%
\pgfusepath{clip}%
\pgfsetbuttcap%
\pgfsetroundjoin%
\pgfsetlinewidth{2.710125pt}%
\definecolor{currentstroke}{rgb}{0.298039,0.447059,0.690196}%
\pgfsetstrokecolor{currentstroke}%
\pgfsetdash{{9.990000pt}{4.320000pt}}{0.000000pt}%
\pgfpathmoveto{\pgfqpoint{1.260533in}{6.606436in}}%
\pgfpathlineto{\pgfqpoint{2.446137in}{6.141339in}}%
\pgfpathlineto{\pgfqpoint{3.631742in}{5.211143in}}%
\pgfpathlineto{\pgfqpoint{4.817346in}{6.187848in}}%
\pgfusepath{stroke}%
\end{pgfscope}%
\begin{pgfscope}%
\pgfpathrectangle{\pgfqpoint{0.667731in}{4.015734in}}{\pgfqpoint{4.742417in}{2.714069in}}%
\pgfusepath{clip}%
\pgfsetroundcap%
\pgfsetroundjoin%
\pgfsetlinewidth{2.710125pt}%
\definecolor{currentstroke}{rgb}{0.298039,0.447059,0.690196}%
\pgfsetstrokecolor{currentstroke}%
\pgfsetdash{}{0pt}%
\pgfusepath{stroke}%
\end{pgfscope}%
\begin{pgfscope}%
\pgfpathrectangle{\pgfqpoint{0.667731in}{4.015734in}}{\pgfqpoint{4.742417in}{2.714069in}}%
\pgfusepath{clip}%
\pgfsetroundcap%
\pgfsetroundjoin%
\pgfsetlinewidth{2.710125pt}%
\definecolor{currentstroke}{rgb}{0.298039,0.447059,0.690196}%
\pgfsetstrokecolor{currentstroke}%
\pgfsetdash{}{0pt}%
\pgfusepath{stroke}%
\end{pgfscope}%
\begin{pgfscope}%
\pgfpathrectangle{\pgfqpoint{0.667731in}{4.015734in}}{\pgfqpoint{4.742417in}{2.714069in}}%
\pgfusepath{clip}%
\pgfsetroundcap%
\pgfsetroundjoin%
\pgfsetlinewidth{2.710125pt}%
\definecolor{currentstroke}{rgb}{0.298039,0.447059,0.690196}%
\pgfsetstrokecolor{currentstroke}%
\pgfsetdash{}{0pt}%
\pgfusepath{stroke}%
\end{pgfscope}%
\begin{pgfscope}%
\pgfpathrectangle{\pgfqpoint{0.667731in}{4.015734in}}{\pgfqpoint{4.742417in}{2.714069in}}%
\pgfusepath{clip}%
\pgfsetroundcap%
\pgfsetroundjoin%
\pgfsetlinewidth{2.710125pt}%
\definecolor{currentstroke}{rgb}{0.298039,0.447059,0.690196}%
\pgfsetstrokecolor{currentstroke}%
\pgfsetdash{}{0pt}%
\pgfusepath{stroke}%
\end{pgfscope}%
\begin{pgfscope}%
\pgfpathrectangle{\pgfqpoint{0.667731in}{4.015734in}}{\pgfqpoint{4.742417in}{2.714069in}}%
\pgfusepath{clip}%
\pgfsetbuttcap%
\pgfsetroundjoin%
\definecolor{currentfill}{rgb}{0.298039,0.447059,0.690196}%
\pgfsetfillcolor{currentfill}%
\pgfsetlinewidth{2.032594pt}%
\definecolor{currentstroke}{rgb}{0.298039,0.447059,0.690196}%
\pgfsetstrokecolor{currentstroke}%
\pgfsetdash{}{0pt}%
\pgfsys@defobject{currentmarker}{\pgfqpoint{-0.046999in}{-0.046999in}}{\pgfqpoint{0.046999in}{0.046999in}}{%
\pgfpathmoveto{\pgfqpoint{0.000000in}{-0.046999in}}%
\pgfpathcurveto{\pgfqpoint{0.012464in}{-0.046999in}}{\pgfqpoint{0.024420in}{-0.042047in}}{\pgfqpoint{0.033234in}{-0.033234in}}%
\pgfpathcurveto{\pgfqpoint{0.042047in}{-0.024420in}}{\pgfqpoint{0.046999in}{-0.012464in}}{\pgfqpoint{0.046999in}{0.000000in}}%
\pgfpathcurveto{\pgfqpoint{0.046999in}{0.012464in}}{\pgfqpoint{0.042047in}{0.024420in}}{\pgfqpoint{0.033234in}{0.033234in}}%
\pgfpathcurveto{\pgfqpoint{0.024420in}{0.042047in}}{\pgfqpoint{0.012464in}{0.046999in}}{\pgfqpoint{0.000000in}{0.046999in}}%
\pgfpathcurveto{\pgfqpoint{-0.012464in}{0.046999in}}{\pgfqpoint{-0.024420in}{0.042047in}}{\pgfqpoint{-0.033234in}{0.033234in}}%
\pgfpathcurveto{\pgfqpoint{-0.042047in}{0.024420in}}{\pgfqpoint{-0.046999in}{0.012464in}}{\pgfqpoint{-0.046999in}{0.000000in}}%
\pgfpathcurveto{\pgfqpoint{-0.046999in}{-0.012464in}}{\pgfqpoint{-0.042047in}{-0.024420in}}{\pgfqpoint{-0.033234in}{-0.033234in}}%
\pgfpathcurveto{\pgfqpoint{-0.024420in}{-0.042047in}}{\pgfqpoint{-0.012464in}{-0.046999in}}{\pgfqpoint{0.000000in}{-0.046999in}}%
\pgfpathlineto{\pgfqpoint{0.000000in}{-0.046999in}}%
\pgfpathclose%
\pgfusepath{stroke,fill}%
}%
\begin{pgfscope}%
\pgfsys@transformshift{1.260533in}{6.606436in}%
\pgfsys@useobject{currentmarker}{}%
\end{pgfscope}%
\begin{pgfscope}%
\pgfsys@transformshift{2.446137in}{6.141339in}%
\pgfsys@useobject{currentmarker}{}%
\end{pgfscope}%
\begin{pgfscope}%
\pgfsys@transformshift{3.631742in}{5.211143in}%
\pgfsys@useobject{currentmarker}{}%
\end{pgfscope}%
\begin{pgfscope}%
\pgfsys@transformshift{4.817346in}{6.187848in}%
\pgfsys@useobject{currentmarker}{}%
\end{pgfscope}%
\end{pgfscope}%
\begin{pgfscope}%
\pgfpathrectangle{\pgfqpoint{0.667731in}{4.015734in}}{\pgfqpoint{4.742417in}{2.714069in}}%
\pgfusepath{clip}%
\pgfsetbuttcap%
\pgfsetroundjoin%
\pgfsetlinewidth{2.710125pt}%
\definecolor{currentstroke}{rgb}{0.866667,0.517647,0.321569}%
\pgfsetstrokecolor{currentstroke}%
\pgfsetdash{{9.990000pt}{4.320000pt}}{0.000000pt}%
\pgfpathmoveto{\pgfqpoint{1.260533in}{5.748086in}}%
\pgfpathlineto{\pgfqpoint{2.446137in}{4.807854in}}%
\pgfpathlineto{\pgfqpoint{3.631742in}{4.139101in}}%
\pgfpathlineto{\pgfqpoint{4.817346in}{4.980048in}}%
\pgfusepath{stroke}%
\end{pgfscope}%
\begin{pgfscope}%
\pgfpathrectangle{\pgfqpoint{0.667731in}{4.015734in}}{\pgfqpoint{4.742417in}{2.714069in}}%
\pgfusepath{clip}%
\pgfsetroundcap%
\pgfsetroundjoin%
\pgfsetlinewidth{2.710125pt}%
\definecolor{currentstroke}{rgb}{0.866667,0.517647,0.321569}%
\pgfsetstrokecolor{currentstroke}%
\pgfsetdash{}{0pt}%
\pgfusepath{stroke}%
\end{pgfscope}%
\begin{pgfscope}%
\pgfpathrectangle{\pgfqpoint{0.667731in}{4.015734in}}{\pgfqpoint{4.742417in}{2.714069in}}%
\pgfusepath{clip}%
\pgfsetroundcap%
\pgfsetroundjoin%
\pgfsetlinewidth{2.710125pt}%
\definecolor{currentstroke}{rgb}{0.866667,0.517647,0.321569}%
\pgfsetstrokecolor{currentstroke}%
\pgfsetdash{}{0pt}%
\pgfusepath{stroke}%
\end{pgfscope}%
\begin{pgfscope}%
\pgfpathrectangle{\pgfqpoint{0.667731in}{4.015734in}}{\pgfqpoint{4.742417in}{2.714069in}}%
\pgfusepath{clip}%
\pgfsetroundcap%
\pgfsetroundjoin%
\pgfsetlinewidth{2.710125pt}%
\definecolor{currentstroke}{rgb}{0.866667,0.517647,0.321569}%
\pgfsetstrokecolor{currentstroke}%
\pgfsetdash{}{0pt}%
\pgfusepath{stroke}%
\end{pgfscope}%
\begin{pgfscope}%
\pgfpathrectangle{\pgfqpoint{0.667731in}{4.015734in}}{\pgfqpoint{4.742417in}{2.714069in}}%
\pgfusepath{clip}%
\pgfsetroundcap%
\pgfsetroundjoin%
\pgfsetlinewidth{2.710125pt}%
\definecolor{currentstroke}{rgb}{0.866667,0.517647,0.321569}%
\pgfsetstrokecolor{currentstroke}%
\pgfsetdash{}{0pt}%
\pgfusepath{stroke}%
\end{pgfscope}%
\begin{pgfscope}%
\pgfpathrectangle{\pgfqpoint{0.667731in}{4.015734in}}{\pgfqpoint{4.742417in}{2.714069in}}%
\pgfusepath{clip}%
\pgfsetbuttcap%
\pgfsetroundjoin%
\definecolor{currentfill}{rgb}{0.866667,0.517647,0.321569}%
\pgfsetfillcolor{currentfill}%
\pgfsetlinewidth{2.032594pt}%
\definecolor{currentstroke}{rgb}{0.866667,0.517647,0.321569}%
\pgfsetstrokecolor{currentstroke}%
\pgfsetdash{}{0pt}%
\pgfsys@defobject{currentmarker}{\pgfqpoint{-0.046999in}{-0.046999in}}{\pgfqpoint{0.046999in}{0.046999in}}{%
\pgfpathmoveto{\pgfqpoint{0.000000in}{-0.046999in}}%
\pgfpathcurveto{\pgfqpoint{0.012464in}{-0.046999in}}{\pgfqpoint{0.024420in}{-0.042047in}}{\pgfqpoint{0.033234in}{-0.033234in}}%
\pgfpathcurveto{\pgfqpoint{0.042047in}{-0.024420in}}{\pgfqpoint{0.046999in}{-0.012464in}}{\pgfqpoint{0.046999in}{0.000000in}}%
\pgfpathcurveto{\pgfqpoint{0.046999in}{0.012464in}}{\pgfqpoint{0.042047in}{0.024420in}}{\pgfqpoint{0.033234in}{0.033234in}}%
\pgfpathcurveto{\pgfqpoint{0.024420in}{0.042047in}}{\pgfqpoint{0.012464in}{0.046999in}}{\pgfqpoint{0.000000in}{0.046999in}}%
\pgfpathcurveto{\pgfqpoint{-0.012464in}{0.046999in}}{\pgfqpoint{-0.024420in}{0.042047in}}{\pgfqpoint{-0.033234in}{0.033234in}}%
\pgfpathcurveto{\pgfqpoint{-0.042047in}{0.024420in}}{\pgfqpoint{-0.046999in}{0.012464in}}{\pgfqpoint{-0.046999in}{0.000000in}}%
\pgfpathcurveto{\pgfqpoint{-0.046999in}{-0.012464in}}{\pgfqpoint{-0.042047in}{-0.024420in}}{\pgfqpoint{-0.033234in}{-0.033234in}}%
\pgfpathcurveto{\pgfqpoint{-0.024420in}{-0.042047in}}{\pgfqpoint{-0.012464in}{-0.046999in}}{\pgfqpoint{0.000000in}{-0.046999in}}%
\pgfpathlineto{\pgfqpoint{0.000000in}{-0.046999in}}%
\pgfpathclose%
\pgfusepath{stroke,fill}%
}%
\begin{pgfscope}%
\pgfsys@transformshift{1.260533in}{5.748086in}%
\pgfsys@useobject{currentmarker}{}%
\end{pgfscope}%
\begin{pgfscope}%
\pgfsys@transformshift{2.446137in}{4.807854in}%
\pgfsys@useobject{currentmarker}{}%
\end{pgfscope}%
\begin{pgfscope}%
\pgfsys@transformshift{3.631742in}{4.139101in}%
\pgfsys@useobject{currentmarker}{}%
\end{pgfscope}%
\begin{pgfscope}%
\pgfsys@transformshift{4.817346in}{4.980048in}%
\pgfsys@useobject{currentmarker}{}%
\end{pgfscope}%
\end{pgfscope}%
\begin{pgfscope}%
\pgfsetrectcap%
\pgfsetmiterjoin%
\pgfsetlinewidth{1.254687pt}%
\definecolor{currentstroke}{rgb}{1.000000,1.000000,1.000000}%
\pgfsetstrokecolor{currentstroke}%
\pgfsetdash{}{0pt}%
\pgfpathmoveto{\pgfqpoint{0.667731in}{4.015734in}}%
\pgfpathlineto{\pgfqpoint{0.667731in}{6.729803in}}%
\pgfusepath{stroke}%
\end{pgfscope}%
\begin{pgfscope}%
\pgfsetrectcap%
\pgfsetmiterjoin%
\pgfsetlinewidth{1.254687pt}%
\definecolor{currentstroke}{rgb}{1.000000,1.000000,1.000000}%
\pgfsetstrokecolor{currentstroke}%
\pgfsetdash{}{0pt}%
\pgfpathmoveto{\pgfqpoint{5.410148in}{4.015734in}}%
\pgfpathlineto{\pgfqpoint{5.410148in}{6.729803in}}%
\pgfusepath{stroke}%
\end{pgfscope}%
\begin{pgfscope}%
\pgfsetrectcap%
\pgfsetmiterjoin%
\pgfsetlinewidth{1.254687pt}%
\definecolor{currentstroke}{rgb}{1.000000,1.000000,1.000000}%
\pgfsetstrokecolor{currentstroke}%
\pgfsetdash{}{0pt}%
\pgfpathmoveto{\pgfqpoint{0.667731in}{4.015734in}}%
\pgfpathlineto{\pgfqpoint{5.410148in}{4.015734in}}%
\pgfusepath{stroke}%
\end{pgfscope}%
\begin{pgfscope}%
\pgfsetrectcap%
\pgfsetmiterjoin%
\pgfsetlinewidth{1.254687pt}%
\definecolor{currentstroke}{rgb}{1.000000,1.000000,1.000000}%
\pgfsetstrokecolor{currentstroke}%
\pgfsetdash{}{0pt}%
\pgfpathmoveto{\pgfqpoint{0.667731in}{6.729803in}}%
\pgfpathlineto{\pgfqpoint{5.410148in}{6.729803in}}%
\pgfusepath{stroke}%
\end{pgfscope}%
\begin{pgfscope}%
\pgfsetbuttcap%
\pgfsetmiterjoin%
\definecolor{currentfill}{rgb}{0.917647,0.917647,0.949020}%
\pgfsetfillcolor{currentfill}%
\pgfsetfillopacity{0.800000}%
\pgfsetlinewidth{1.003750pt}%
\definecolor{currentstroke}{rgb}{0.800000,0.800000,0.800000}%
\pgfsetstrokecolor{currentstroke}%
\pgfsetstrokeopacity{0.800000}%
\pgfsetdash{}{0pt}%
\pgfpathmoveto{\pgfqpoint{0.774675in}{4.092123in}}%
\pgfpathlineto{\pgfqpoint{1.866407in}{4.092123in}}%
\pgfpathquadraticcurveto{\pgfqpoint{1.896963in}{4.092123in}}{\pgfqpoint{1.896963in}{4.122679in}}%
\pgfpathlineto{\pgfqpoint{1.896963in}{4.757747in}}%
\pgfpathquadraticcurveto{\pgfqpoint{1.896963in}{4.788303in}}{\pgfqpoint{1.866407in}{4.788303in}}%
\pgfpathlineto{\pgfqpoint{0.774675in}{4.788303in}}%
\pgfpathquadraticcurveto{\pgfqpoint{0.744120in}{4.788303in}}{\pgfqpoint{0.744120in}{4.757747in}}%
\pgfpathlineto{\pgfqpoint{0.744120in}{4.122679in}}%
\pgfpathquadraticcurveto{\pgfqpoint{0.744120in}{4.092123in}}{\pgfqpoint{0.774675in}{4.092123in}}%
\pgfpathlineto{\pgfqpoint{0.774675in}{4.092123in}}%
\pgfpathclose%
\pgfusepath{stroke,fill}%
\end{pgfscope}%
\begin{pgfscope}%
\definecolor{textcolor}{rgb}{0.150000,0.150000,0.150000}%
\pgfsetstrokecolor{textcolor}%
\pgfsetfillcolor{textcolor}%
\pgftext[x=1.076929in,y=4.611452in,left,base]{\color{textcolor}\sffamily\fontsize{12.000000}{14.400000}\selectfont F-score}%
\end{pgfscope}%
\begin{pgfscope}%
\pgfsetbuttcap%
\pgfsetroundjoin%
\definecolor{currentfill}{rgb}{0.298039,0.447059,0.690196}%
\pgfsetfillcolor{currentfill}%
\pgfsetlinewidth{2.032594pt}%
\definecolor{currentstroke}{rgb}{0.298039,0.447059,0.690196}%
\pgfsetstrokecolor{currentstroke}%
\pgfsetdash{}{0pt}%
\pgfsys@defobject{currentmarker}{\pgfqpoint{-0.046999in}{-0.046999in}}{\pgfqpoint{0.046999in}{0.046999in}}{%
\pgfpathmoveto{\pgfqpoint{0.000000in}{-0.046999in}}%
\pgfpathcurveto{\pgfqpoint{0.012464in}{-0.046999in}}{\pgfqpoint{0.024420in}{-0.042047in}}{\pgfqpoint{0.033234in}{-0.033234in}}%
\pgfpathcurveto{\pgfqpoint{0.042047in}{-0.024420in}}{\pgfqpoint{0.046999in}{-0.012464in}}{\pgfqpoint{0.046999in}{0.000000in}}%
\pgfpathcurveto{\pgfqpoint{0.046999in}{0.012464in}}{\pgfqpoint{0.042047in}{0.024420in}}{\pgfqpoint{0.033234in}{0.033234in}}%
\pgfpathcurveto{\pgfqpoint{0.024420in}{0.042047in}}{\pgfqpoint{0.012464in}{0.046999in}}{\pgfqpoint{0.000000in}{0.046999in}}%
\pgfpathcurveto{\pgfqpoint{-0.012464in}{0.046999in}}{\pgfqpoint{-0.024420in}{0.042047in}}{\pgfqpoint{-0.033234in}{0.033234in}}%
\pgfpathcurveto{\pgfqpoint{-0.042047in}{0.024420in}}{\pgfqpoint{-0.046999in}{0.012464in}}{\pgfqpoint{-0.046999in}{0.000000in}}%
\pgfpathcurveto{\pgfqpoint{-0.046999in}{-0.012464in}}{\pgfqpoint{-0.042047in}{-0.024420in}}{\pgfqpoint{-0.033234in}{-0.033234in}}%
\pgfpathcurveto{\pgfqpoint{-0.024420in}{-0.042047in}}{\pgfqpoint{-0.012464in}{-0.046999in}}{\pgfqpoint{0.000000in}{-0.046999in}}%
\pgfpathlineto{\pgfqpoint{0.000000in}{-0.046999in}}%
\pgfpathclose%
\pgfusepath{stroke,fill}%
}%
\begin{pgfscope}%
\pgfsys@transformshift{0.958009in}{4.435815in}%
\pgfsys@useobject{currentmarker}{}%
\end{pgfscope}%
\end{pgfscope}%
\begin{pgfscope}%
\definecolor{textcolor}{rgb}{0.150000,0.150000,0.150000}%
\pgfsetstrokecolor{textcolor}%
\pgfsetfillcolor{textcolor}%
\pgftext[x=1.233009in,y=4.395711in,left,base]{\color{textcolor}\sffamily\fontsize{11.000000}{13.200000}\selectfont F1 micro}%
\end{pgfscope}%
\begin{pgfscope}%
\pgfsetbuttcap%
\pgfsetroundjoin%
\definecolor{currentfill}{rgb}{0.866667,0.517647,0.321569}%
\pgfsetfillcolor{currentfill}%
\pgfsetlinewidth{2.032594pt}%
\definecolor{currentstroke}{rgb}{0.866667,0.517647,0.321569}%
\pgfsetstrokecolor{currentstroke}%
\pgfsetdash{}{0pt}%
\pgfsys@defobject{currentmarker}{\pgfqpoint{-0.046999in}{-0.046999in}}{\pgfqpoint{0.046999in}{0.046999in}}{%
\pgfpathmoveto{\pgfqpoint{0.000000in}{-0.046999in}}%
\pgfpathcurveto{\pgfqpoint{0.012464in}{-0.046999in}}{\pgfqpoint{0.024420in}{-0.042047in}}{\pgfqpoint{0.033234in}{-0.033234in}}%
\pgfpathcurveto{\pgfqpoint{0.042047in}{-0.024420in}}{\pgfqpoint{0.046999in}{-0.012464in}}{\pgfqpoint{0.046999in}{0.000000in}}%
\pgfpathcurveto{\pgfqpoint{0.046999in}{0.012464in}}{\pgfqpoint{0.042047in}{0.024420in}}{\pgfqpoint{0.033234in}{0.033234in}}%
\pgfpathcurveto{\pgfqpoint{0.024420in}{0.042047in}}{\pgfqpoint{0.012464in}{0.046999in}}{\pgfqpoint{0.000000in}{0.046999in}}%
\pgfpathcurveto{\pgfqpoint{-0.012464in}{0.046999in}}{\pgfqpoint{-0.024420in}{0.042047in}}{\pgfqpoint{-0.033234in}{0.033234in}}%
\pgfpathcurveto{\pgfqpoint{-0.042047in}{0.024420in}}{\pgfqpoint{-0.046999in}{0.012464in}}{\pgfqpoint{-0.046999in}{0.000000in}}%
\pgfpathcurveto{\pgfqpoint{-0.046999in}{-0.012464in}}{\pgfqpoint{-0.042047in}{-0.024420in}}{\pgfqpoint{-0.033234in}{-0.033234in}}%
\pgfpathcurveto{\pgfqpoint{-0.024420in}{-0.042047in}}{\pgfqpoint{-0.012464in}{-0.046999in}}{\pgfqpoint{0.000000in}{-0.046999in}}%
\pgfpathlineto{\pgfqpoint{0.000000in}{-0.046999in}}%
\pgfpathclose%
\pgfusepath{stroke,fill}%
}%
\begin{pgfscope}%
\pgfsys@transformshift{0.958009in}{4.222910in}%
\pgfsys@useobject{currentmarker}{}%
\end{pgfscope}%
\end{pgfscope}%
\begin{pgfscope}%
\definecolor{textcolor}{rgb}{0.150000,0.150000,0.150000}%
\pgfsetstrokecolor{textcolor}%
\pgfsetfillcolor{textcolor}%
\pgftext[x=1.233009in,y=4.182806in,left,base]{\color{textcolor}\sffamily\fontsize{11.000000}{13.200000}\selectfont F1 macro}%
\end{pgfscope}%
\begin{pgfscope}%
\pgfsetbuttcap%
\pgfsetmiterjoin%
\definecolor{currentfill}{rgb}{0.917647,0.917647,0.949020}%
\pgfsetfillcolor{currentfill}%
\pgfsetlinewidth{0.000000pt}%
\definecolor{currentstroke}{rgb}{0.000000,0.000000,0.000000}%
\pgfsetstrokecolor{currentstroke}%
\pgfsetstrokeopacity{0.000000}%
\pgfsetdash{}{0pt}%
\pgfpathmoveto{\pgfqpoint{0.667731in}{0.650833in}}%
\pgfpathlineto{\pgfqpoint{5.410148in}{0.650833in}}%
\pgfpathlineto{\pgfqpoint{5.410148in}{3.364902in}}%
\pgfpathlineto{\pgfqpoint{0.667731in}{3.364902in}}%
\pgfpathlineto{\pgfqpoint{0.667731in}{0.650833in}}%
\pgfpathclose%
\pgfusepath{fill}%
\end{pgfscope}%
\begin{pgfscope}%
\definecolor{textcolor}{rgb}{0.150000,0.150000,0.150000}%
\pgfsetstrokecolor{textcolor}%
\pgfsetfillcolor{textcolor}%
\pgftext[x=1.141973in,y=0.518888in,,top]{\color{textcolor}\sffamily\fontsize{11.000000}{13.200000}\selectfont 1}%
\end{pgfscope}%
\begin{pgfscope}%
\definecolor{textcolor}{rgb}{0.150000,0.150000,0.150000}%
\pgfsetstrokecolor{textcolor}%
\pgfsetfillcolor{textcolor}%
\pgftext[x=2.090456in,y=0.518888in,,top]{\color{textcolor}\sffamily\fontsize{11.000000}{13.200000}\selectfont 5}%
\end{pgfscope}%
\begin{pgfscope}%
\definecolor{textcolor}{rgb}{0.150000,0.150000,0.150000}%
\pgfsetstrokecolor{textcolor}%
\pgfsetfillcolor{textcolor}%
\pgftext[x=3.038940in,y=0.518888in,,top]{\color{textcolor}\sffamily\fontsize{11.000000}{13.200000}\selectfont 10}%
\end{pgfscope}%
\begin{pgfscope}%
\definecolor{textcolor}{rgb}{0.150000,0.150000,0.150000}%
\pgfsetstrokecolor{textcolor}%
\pgfsetfillcolor{textcolor}%
\pgftext[x=3.987423in,y=0.518888in,,top]{\color{textcolor}\sffamily\fontsize{11.000000}{13.200000}\selectfont 15}%
\end{pgfscope}%
\begin{pgfscope}%
\definecolor{textcolor}{rgb}{0.150000,0.150000,0.150000}%
\pgfsetstrokecolor{textcolor}%
\pgfsetfillcolor{textcolor}%
\pgftext[x=4.935906in,y=0.518888in,,top]{\color{textcolor}\sffamily\fontsize{11.000000}{13.200000}\selectfont 20}%
\end{pgfscope}%
\begin{pgfscope}%
\definecolor{textcolor}{rgb}{0.150000,0.150000,0.150000}%
\pgfsetstrokecolor{textcolor}%
\pgfsetfillcolor{textcolor}%
\pgftext[x=3.038940in,y=0.328148in,,top]{\color{textcolor}\sffamily\fontsize{12.000000}{14.400000}\selectfont window size}%
\end{pgfscope}%
\begin{pgfscope}%
\pgfpathrectangle{\pgfqpoint{0.667731in}{0.650833in}}{\pgfqpoint{4.742417in}{2.714069in}}%
\pgfusepath{clip}%
\pgfsetroundcap%
\pgfsetroundjoin%
\pgfsetlinewidth{1.003750pt}%
\definecolor{currentstroke}{rgb}{1.000000,1.000000,1.000000}%
\pgfsetstrokecolor{currentstroke}%
\pgfsetdash{}{0pt}%
\pgfpathmoveto{\pgfqpoint{0.667731in}{0.776442in}}%
\pgfpathlineto{\pgfqpoint{5.410148in}{0.776442in}}%
\pgfusepath{stroke}%
\end{pgfscope}%
\begin{pgfscope}%
\definecolor{textcolor}{rgb}{0.150000,0.150000,0.150000}%
\pgfsetstrokecolor{textcolor}%
\pgfsetfillcolor{textcolor}%
\pgftext[x=0.383703in, y=0.723635in, left, base]{\color{textcolor}\sffamily\fontsize{11.000000}{13.200000}\selectfont \(\displaystyle {63}\)}%
\end{pgfscope}%
\begin{pgfscope}%
\pgfpathrectangle{\pgfqpoint{0.667731in}{0.650833in}}{\pgfqpoint{4.742417in}{2.714069in}}%
\pgfusepath{clip}%
\pgfsetroundcap%
\pgfsetroundjoin%
\pgfsetlinewidth{1.003750pt}%
\definecolor{currentstroke}{rgb}{1.000000,1.000000,1.000000}%
\pgfsetstrokecolor{currentstroke}%
\pgfsetdash{}{0pt}%
\pgfpathmoveto{\pgfqpoint{0.667731in}{1.177209in}}%
\pgfpathlineto{\pgfqpoint{5.410148in}{1.177209in}}%
\pgfusepath{stroke}%
\end{pgfscope}%
\begin{pgfscope}%
\definecolor{textcolor}{rgb}{0.150000,0.150000,0.150000}%
\pgfsetstrokecolor{textcolor}%
\pgfsetfillcolor{textcolor}%
\pgftext[x=0.383703in, y=1.124402in, left, base]{\color{textcolor}\sffamily\fontsize{11.000000}{13.200000}\selectfont \(\displaystyle {64}\)}%
\end{pgfscope}%
\begin{pgfscope}%
\pgfpathrectangle{\pgfqpoint{0.667731in}{0.650833in}}{\pgfqpoint{4.742417in}{2.714069in}}%
\pgfusepath{clip}%
\pgfsetroundcap%
\pgfsetroundjoin%
\pgfsetlinewidth{1.003750pt}%
\definecolor{currentstroke}{rgb}{1.000000,1.000000,1.000000}%
\pgfsetstrokecolor{currentstroke}%
\pgfsetdash{}{0pt}%
\pgfpathmoveto{\pgfqpoint{0.667731in}{1.577975in}}%
\pgfpathlineto{\pgfqpoint{5.410148in}{1.577975in}}%
\pgfusepath{stroke}%
\end{pgfscope}%
\begin{pgfscope}%
\definecolor{textcolor}{rgb}{0.150000,0.150000,0.150000}%
\pgfsetstrokecolor{textcolor}%
\pgfsetfillcolor{textcolor}%
\pgftext[x=0.383703in, y=1.525169in, left, base]{\color{textcolor}\sffamily\fontsize{11.000000}{13.200000}\selectfont \(\displaystyle {65}\)}%
\end{pgfscope}%
\begin{pgfscope}%
\pgfpathrectangle{\pgfqpoint{0.667731in}{0.650833in}}{\pgfqpoint{4.742417in}{2.714069in}}%
\pgfusepath{clip}%
\pgfsetroundcap%
\pgfsetroundjoin%
\pgfsetlinewidth{1.003750pt}%
\definecolor{currentstroke}{rgb}{1.000000,1.000000,1.000000}%
\pgfsetstrokecolor{currentstroke}%
\pgfsetdash{}{0pt}%
\pgfpathmoveto{\pgfqpoint{0.667731in}{1.978742in}}%
\pgfpathlineto{\pgfqpoint{5.410148in}{1.978742in}}%
\pgfusepath{stroke}%
\end{pgfscope}%
\begin{pgfscope}%
\definecolor{textcolor}{rgb}{0.150000,0.150000,0.150000}%
\pgfsetstrokecolor{textcolor}%
\pgfsetfillcolor{textcolor}%
\pgftext[x=0.383703in, y=1.925935in, left, base]{\color{textcolor}\sffamily\fontsize{11.000000}{13.200000}\selectfont \(\displaystyle {66}\)}%
\end{pgfscope}%
\begin{pgfscope}%
\pgfpathrectangle{\pgfqpoint{0.667731in}{0.650833in}}{\pgfqpoint{4.742417in}{2.714069in}}%
\pgfusepath{clip}%
\pgfsetroundcap%
\pgfsetroundjoin%
\pgfsetlinewidth{1.003750pt}%
\definecolor{currentstroke}{rgb}{1.000000,1.000000,1.000000}%
\pgfsetstrokecolor{currentstroke}%
\pgfsetdash{}{0pt}%
\pgfpathmoveto{\pgfqpoint{0.667731in}{2.379508in}}%
\pgfpathlineto{\pgfqpoint{5.410148in}{2.379508in}}%
\pgfusepath{stroke}%
\end{pgfscope}%
\begin{pgfscope}%
\definecolor{textcolor}{rgb}{0.150000,0.150000,0.150000}%
\pgfsetstrokecolor{textcolor}%
\pgfsetfillcolor{textcolor}%
\pgftext[x=0.383703in, y=2.326702in, left, base]{\color{textcolor}\sffamily\fontsize{11.000000}{13.200000}\selectfont \(\displaystyle {67}\)}%
\end{pgfscope}%
\begin{pgfscope}%
\pgfpathrectangle{\pgfqpoint{0.667731in}{0.650833in}}{\pgfqpoint{4.742417in}{2.714069in}}%
\pgfusepath{clip}%
\pgfsetroundcap%
\pgfsetroundjoin%
\pgfsetlinewidth{1.003750pt}%
\definecolor{currentstroke}{rgb}{1.000000,1.000000,1.000000}%
\pgfsetstrokecolor{currentstroke}%
\pgfsetdash{}{0pt}%
\pgfpathmoveto{\pgfqpoint{0.667731in}{2.780275in}}%
\pgfpathlineto{\pgfqpoint{5.410148in}{2.780275in}}%
\pgfusepath{stroke}%
\end{pgfscope}%
\begin{pgfscope}%
\definecolor{textcolor}{rgb}{0.150000,0.150000,0.150000}%
\pgfsetstrokecolor{textcolor}%
\pgfsetfillcolor{textcolor}%
\pgftext[x=0.383703in, y=2.727468in, left, base]{\color{textcolor}\sffamily\fontsize{11.000000}{13.200000}\selectfont \(\displaystyle {68}\)}%
\end{pgfscope}%
\begin{pgfscope}%
\pgfpathrectangle{\pgfqpoint{0.667731in}{0.650833in}}{\pgfqpoint{4.742417in}{2.714069in}}%
\pgfusepath{clip}%
\pgfsetroundcap%
\pgfsetroundjoin%
\pgfsetlinewidth{1.003750pt}%
\definecolor{currentstroke}{rgb}{1.000000,1.000000,1.000000}%
\pgfsetstrokecolor{currentstroke}%
\pgfsetdash{}{0pt}%
\pgfpathmoveto{\pgfqpoint{0.667731in}{3.181042in}}%
\pgfpathlineto{\pgfqpoint{5.410148in}{3.181042in}}%
\pgfusepath{stroke}%
\end{pgfscope}%
\begin{pgfscope}%
\definecolor{textcolor}{rgb}{0.150000,0.150000,0.150000}%
\pgfsetstrokecolor{textcolor}%
\pgfsetfillcolor{textcolor}%
\pgftext[x=0.383703in, y=3.128235in, left, base]{\color{textcolor}\sffamily\fontsize{11.000000}{13.200000}\selectfont \(\displaystyle {69}\)}%
\end{pgfscope}%
\begin{pgfscope}%
\definecolor{textcolor}{rgb}{0.150000,0.150000,0.150000}%
\pgfsetstrokecolor{textcolor}%
\pgfsetfillcolor{textcolor}%
\pgftext[x=0.328148in,y=2.007867in,,bottom,rotate=90.000000]{\color{textcolor}\sffamily\fontsize{12.000000}{14.400000}\selectfont Average score}%
\end{pgfscope}%
\begin{pgfscope}%
\pgfpathrectangle{\pgfqpoint{0.667731in}{0.650833in}}{\pgfqpoint{4.742417in}{2.714069in}}%
\pgfusepath{clip}%
\pgfsetbuttcap%
\pgfsetroundjoin%
\pgfsetlinewidth{2.710125pt}%
\definecolor{currentstroke}{rgb}{0.298039,0.447059,0.690196}%
\pgfsetstrokecolor{currentstroke}%
\pgfsetdash{{9.990000pt}{4.320000pt}}{0.000000pt}%
\pgfpathmoveto{\pgfqpoint{1.141973in}{2.939069in}}%
\pgfpathlineto{\pgfqpoint{2.090456in}{3.241535in}}%
\pgfpathlineto{\pgfqpoint{3.038940in}{2.031673in}}%
\pgfpathlineto{\pgfqpoint{3.987423in}{2.107290in}}%
\pgfpathlineto{\pgfqpoint{4.935906in}{2.334139in}}%
\pgfusepath{stroke}%
\end{pgfscope}%
\begin{pgfscope}%
\pgfpathrectangle{\pgfqpoint{0.667731in}{0.650833in}}{\pgfqpoint{4.742417in}{2.714069in}}%
\pgfusepath{clip}%
\pgfsetroundcap%
\pgfsetroundjoin%
\pgfsetlinewidth{2.710125pt}%
\definecolor{currentstroke}{rgb}{0.298039,0.447059,0.690196}%
\pgfsetstrokecolor{currentstroke}%
\pgfsetdash{}{0pt}%
\pgfusepath{stroke}%
\end{pgfscope}%
\begin{pgfscope}%
\pgfpathrectangle{\pgfqpoint{0.667731in}{0.650833in}}{\pgfqpoint{4.742417in}{2.714069in}}%
\pgfusepath{clip}%
\pgfsetroundcap%
\pgfsetroundjoin%
\pgfsetlinewidth{2.710125pt}%
\definecolor{currentstroke}{rgb}{0.298039,0.447059,0.690196}%
\pgfsetstrokecolor{currentstroke}%
\pgfsetdash{}{0pt}%
\pgfusepath{stroke}%
\end{pgfscope}%
\begin{pgfscope}%
\pgfpathrectangle{\pgfqpoint{0.667731in}{0.650833in}}{\pgfqpoint{4.742417in}{2.714069in}}%
\pgfusepath{clip}%
\pgfsetroundcap%
\pgfsetroundjoin%
\pgfsetlinewidth{2.710125pt}%
\definecolor{currentstroke}{rgb}{0.298039,0.447059,0.690196}%
\pgfsetstrokecolor{currentstroke}%
\pgfsetdash{}{0pt}%
\pgfusepath{stroke}%
\end{pgfscope}%
\begin{pgfscope}%
\pgfpathrectangle{\pgfqpoint{0.667731in}{0.650833in}}{\pgfqpoint{4.742417in}{2.714069in}}%
\pgfusepath{clip}%
\pgfsetroundcap%
\pgfsetroundjoin%
\pgfsetlinewidth{2.710125pt}%
\definecolor{currentstroke}{rgb}{0.298039,0.447059,0.690196}%
\pgfsetstrokecolor{currentstroke}%
\pgfsetdash{}{0pt}%
\pgfusepath{stroke}%
\end{pgfscope}%
\begin{pgfscope}%
\pgfpathrectangle{\pgfqpoint{0.667731in}{0.650833in}}{\pgfqpoint{4.742417in}{2.714069in}}%
\pgfusepath{clip}%
\pgfsetroundcap%
\pgfsetroundjoin%
\pgfsetlinewidth{2.710125pt}%
\definecolor{currentstroke}{rgb}{0.298039,0.447059,0.690196}%
\pgfsetstrokecolor{currentstroke}%
\pgfsetdash{}{0pt}%
\pgfusepath{stroke}%
\end{pgfscope}%
\begin{pgfscope}%
\pgfpathrectangle{\pgfqpoint{0.667731in}{0.650833in}}{\pgfqpoint{4.742417in}{2.714069in}}%
\pgfusepath{clip}%
\pgfsetbuttcap%
\pgfsetroundjoin%
\definecolor{currentfill}{rgb}{0.298039,0.447059,0.690196}%
\pgfsetfillcolor{currentfill}%
\pgfsetlinewidth{2.032594pt}%
\definecolor{currentstroke}{rgb}{0.298039,0.447059,0.690196}%
\pgfsetstrokecolor{currentstroke}%
\pgfsetdash{}{0pt}%
\pgfsys@defobject{currentmarker}{\pgfqpoint{-0.046999in}{-0.046999in}}{\pgfqpoint{0.046999in}{0.046999in}}{%
\pgfpathmoveto{\pgfqpoint{0.000000in}{-0.046999in}}%
\pgfpathcurveto{\pgfqpoint{0.012464in}{-0.046999in}}{\pgfqpoint{0.024420in}{-0.042047in}}{\pgfqpoint{0.033234in}{-0.033234in}}%
\pgfpathcurveto{\pgfqpoint{0.042047in}{-0.024420in}}{\pgfqpoint{0.046999in}{-0.012464in}}{\pgfqpoint{0.046999in}{0.000000in}}%
\pgfpathcurveto{\pgfqpoint{0.046999in}{0.012464in}}{\pgfqpoint{0.042047in}{0.024420in}}{\pgfqpoint{0.033234in}{0.033234in}}%
\pgfpathcurveto{\pgfqpoint{0.024420in}{0.042047in}}{\pgfqpoint{0.012464in}{0.046999in}}{\pgfqpoint{0.000000in}{0.046999in}}%
\pgfpathcurveto{\pgfqpoint{-0.012464in}{0.046999in}}{\pgfqpoint{-0.024420in}{0.042047in}}{\pgfqpoint{-0.033234in}{0.033234in}}%
\pgfpathcurveto{\pgfqpoint{-0.042047in}{0.024420in}}{\pgfqpoint{-0.046999in}{0.012464in}}{\pgfqpoint{-0.046999in}{0.000000in}}%
\pgfpathcurveto{\pgfqpoint{-0.046999in}{-0.012464in}}{\pgfqpoint{-0.042047in}{-0.024420in}}{\pgfqpoint{-0.033234in}{-0.033234in}}%
\pgfpathcurveto{\pgfqpoint{-0.024420in}{-0.042047in}}{\pgfqpoint{-0.012464in}{-0.046999in}}{\pgfqpoint{0.000000in}{-0.046999in}}%
\pgfpathlineto{\pgfqpoint{0.000000in}{-0.046999in}}%
\pgfpathclose%
\pgfusepath{stroke,fill}%
}%
\begin{pgfscope}%
\pgfsys@transformshift{1.141973in}{2.939069in}%
\pgfsys@useobject{currentmarker}{}%
\end{pgfscope}%
\begin{pgfscope}%
\pgfsys@transformshift{2.090456in}{3.241535in}%
\pgfsys@useobject{currentmarker}{}%
\end{pgfscope}%
\begin{pgfscope}%
\pgfsys@transformshift{3.038940in}{2.031673in}%
\pgfsys@useobject{currentmarker}{}%
\end{pgfscope}%
\begin{pgfscope}%
\pgfsys@transformshift{3.987423in}{2.107290in}%
\pgfsys@useobject{currentmarker}{}%
\end{pgfscope}%
\begin{pgfscope}%
\pgfsys@transformshift{4.935906in}{2.334139in}%
\pgfsys@useobject{currentmarker}{}%
\end{pgfscope}%
\end{pgfscope}%
\begin{pgfscope}%
\pgfpathrectangle{\pgfqpoint{0.667731in}{0.650833in}}{\pgfqpoint{4.742417in}{2.714069in}}%
\pgfusepath{clip}%
\pgfsetbuttcap%
\pgfsetroundjoin%
\pgfsetlinewidth{2.710125pt}%
\definecolor{currentstroke}{rgb}{0.866667,0.517647,0.321569}%
\pgfsetstrokecolor{currentstroke}%
\pgfsetdash{{9.990000pt}{4.320000pt}}{0.000000pt}%
\pgfpathmoveto{\pgfqpoint{1.141973in}{1.797423in}}%
\pgfpathlineto{\pgfqpoint{2.090456in}{2.087194in}}%
\pgfpathlineto{\pgfqpoint{3.038940in}{0.863657in}}%
\pgfpathlineto{\pgfqpoint{3.987423in}{0.774199in}}%
\pgfpathlineto{\pgfqpoint{4.935906in}{1.093456in}}%
\pgfusepath{stroke}%
\end{pgfscope}%
\begin{pgfscope}%
\pgfpathrectangle{\pgfqpoint{0.667731in}{0.650833in}}{\pgfqpoint{4.742417in}{2.714069in}}%
\pgfusepath{clip}%
\pgfsetroundcap%
\pgfsetroundjoin%
\pgfsetlinewidth{2.710125pt}%
\definecolor{currentstroke}{rgb}{0.866667,0.517647,0.321569}%
\pgfsetstrokecolor{currentstroke}%
\pgfsetdash{}{0pt}%
\pgfusepath{stroke}%
\end{pgfscope}%
\begin{pgfscope}%
\pgfpathrectangle{\pgfqpoint{0.667731in}{0.650833in}}{\pgfqpoint{4.742417in}{2.714069in}}%
\pgfusepath{clip}%
\pgfsetroundcap%
\pgfsetroundjoin%
\pgfsetlinewidth{2.710125pt}%
\definecolor{currentstroke}{rgb}{0.866667,0.517647,0.321569}%
\pgfsetstrokecolor{currentstroke}%
\pgfsetdash{}{0pt}%
\pgfusepath{stroke}%
\end{pgfscope}%
\begin{pgfscope}%
\pgfpathrectangle{\pgfqpoint{0.667731in}{0.650833in}}{\pgfqpoint{4.742417in}{2.714069in}}%
\pgfusepath{clip}%
\pgfsetroundcap%
\pgfsetroundjoin%
\pgfsetlinewidth{2.710125pt}%
\definecolor{currentstroke}{rgb}{0.866667,0.517647,0.321569}%
\pgfsetstrokecolor{currentstroke}%
\pgfsetdash{}{0pt}%
\pgfusepath{stroke}%
\end{pgfscope}%
\begin{pgfscope}%
\pgfpathrectangle{\pgfqpoint{0.667731in}{0.650833in}}{\pgfqpoint{4.742417in}{2.714069in}}%
\pgfusepath{clip}%
\pgfsetroundcap%
\pgfsetroundjoin%
\pgfsetlinewidth{2.710125pt}%
\definecolor{currentstroke}{rgb}{0.866667,0.517647,0.321569}%
\pgfsetstrokecolor{currentstroke}%
\pgfsetdash{}{0pt}%
\pgfusepath{stroke}%
\end{pgfscope}%
\begin{pgfscope}%
\pgfpathrectangle{\pgfqpoint{0.667731in}{0.650833in}}{\pgfqpoint{4.742417in}{2.714069in}}%
\pgfusepath{clip}%
\pgfsetroundcap%
\pgfsetroundjoin%
\pgfsetlinewidth{2.710125pt}%
\definecolor{currentstroke}{rgb}{0.866667,0.517647,0.321569}%
\pgfsetstrokecolor{currentstroke}%
\pgfsetdash{}{0pt}%
\pgfusepath{stroke}%
\end{pgfscope}%
\begin{pgfscope}%
\pgfpathrectangle{\pgfqpoint{0.667731in}{0.650833in}}{\pgfqpoint{4.742417in}{2.714069in}}%
\pgfusepath{clip}%
\pgfsetbuttcap%
\pgfsetroundjoin%
\definecolor{currentfill}{rgb}{0.866667,0.517647,0.321569}%
\pgfsetfillcolor{currentfill}%
\pgfsetlinewidth{2.032594pt}%
\definecolor{currentstroke}{rgb}{0.866667,0.517647,0.321569}%
\pgfsetstrokecolor{currentstroke}%
\pgfsetdash{}{0pt}%
\pgfsys@defobject{currentmarker}{\pgfqpoint{-0.046999in}{-0.046999in}}{\pgfqpoint{0.046999in}{0.046999in}}{%
\pgfpathmoveto{\pgfqpoint{0.000000in}{-0.046999in}}%
\pgfpathcurveto{\pgfqpoint{0.012464in}{-0.046999in}}{\pgfqpoint{0.024420in}{-0.042047in}}{\pgfqpoint{0.033234in}{-0.033234in}}%
\pgfpathcurveto{\pgfqpoint{0.042047in}{-0.024420in}}{\pgfqpoint{0.046999in}{-0.012464in}}{\pgfqpoint{0.046999in}{0.000000in}}%
\pgfpathcurveto{\pgfqpoint{0.046999in}{0.012464in}}{\pgfqpoint{0.042047in}{0.024420in}}{\pgfqpoint{0.033234in}{0.033234in}}%
\pgfpathcurveto{\pgfqpoint{0.024420in}{0.042047in}}{\pgfqpoint{0.012464in}{0.046999in}}{\pgfqpoint{0.000000in}{0.046999in}}%
\pgfpathcurveto{\pgfqpoint{-0.012464in}{0.046999in}}{\pgfqpoint{-0.024420in}{0.042047in}}{\pgfqpoint{-0.033234in}{0.033234in}}%
\pgfpathcurveto{\pgfqpoint{-0.042047in}{0.024420in}}{\pgfqpoint{-0.046999in}{0.012464in}}{\pgfqpoint{-0.046999in}{0.000000in}}%
\pgfpathcurveto{\pgfqpoint{-0.046999in}{-0.012464in}}{\pgfqpoint{-0.042047in}{-0.024420in}}{\pgfqpoint{-0.033234in}{-0.033234in}}%
\pgfpathcurveto{\pgfqpoint{-0.024420in}{-0.042047in}}{\pgfqpoint{-0.012464in}{-0.046999in}}{\pgfqpoint{0.000000in}{-0.046999in}}%
\pgfpathlineto{\pgfqpoint{0.000000in}{-0.046999in}}%
\pgfpathclose%
\pgfusepath{stroke,fill}%
}%
\begin{pgfscope}%
\pgfsys@transformshift{1.141973in}{1.797423in}%
\pgfsys@useobject{currentmarker}{}%
\end{pgfscope}%
\begin{pgfscope}%
\pgfsys@transformshift{2.090456in}{2.087194in}%
\pgfsys@useobject{currentmarker}{}%
\end{pgfscope}%
\begin{pgfscope}%
\pgfsys@transformshift{3.038940in}{0.863657in}%
\pgfsys@useobject{currentmarker}{}%
\end{pgfscope}%
\begin{pgfscope}%
\pgfsys@transformshift{3.987423in}{0.774199in}%
\pgfsys@useobject{currentmarker}{}%
\end{pgfscope}%
\begin{pgfscope}%
\pgfsys@transformshift{4.935906in}{1.093456in}%
\pgfsys@useobject{currentmarker}{}%
\end{pgfscope}%
\end{pgfscope}%
\begin{pgfscope}%
\pgfsetrectcap%
\pgfsetmiterjoin%
\pgfsetlinewidth{1.254687pt}%
\definecolor{currentstroke}{rgb}{1.000000,1.000000,1.000000}%
\pgfsetstrokecolor{currentstroke}%
\pgfsetdash{}{0pt}%
\pgfpathmoveto{\pgfqpoint{0.667731in}{0.650833in}}%
\pgfpathlineto{\pgfqpoint{0.667731in}{3.364902in}}%
\pgfusepath{stroke}%
\end{pgfscope}%
\begin{pgfscope}%
\pgfsetrectcap%
\pgfsetmiterjoin%
\pgfsetlinewidth{1.254687pt}%
\definecolor{currentstroke}{rgb}{1.000000,1.000000,1.000000}%
\pgfsetstrokecolor{currentstroke}%
\pgfsetdash{}{0pt}%
\pgfpathmoveto{\pgfqpoint{5.410148in}{0.650833in}}%
\pgfpathlineto{\pgfqpoint{5.410148in}{3.364902in}}%
\pgfusepath{stroke}%
\end{pgfscope}%
\begin{pgfscope}%
\pgfsetrectcap%
\pgfsetmiterjoin%
\pgfsetlinewidth{1.254687pt}%
\definecolor{currentstroke}{rgb}{1.000000,1.000000,1.000000}%
\pgfsetstrokecolor{currentstroke}%
\pgfsetdash{}{0pt}%
\pgfpathmoveto{\pgfqpoint{0.667731in}{0.650833in}}%
\pgfpathlineto{\pgfqpoint{5.410148in}{0.650833in}}%
\pgfusepath{stroke}%
\end{pgfscope}%
\begin{pgfscope}%
\pgfsetrectcap%
\pgfsetmiterjoin%
\pgfsetlinewidth{1.254687pt}%
\definecolor{currentstroke}{rgb}{1.000000,1.000000,1.000000}%
\pgfsetstrokecolor{currentstroke}%
\pgfsetdash{}{0pt}%
\pgfpathmoveto{\pgfqpoint{0.667731in}{3.364902in}}%
\pgfpathlineto{\pgfqpoint{5.410148in}{3.364902in}}%
\pgfusepath{stroke}%
\end{pgfscope}%
\begin{pgfscope}%
\pgfsetbuttcap%
\pgfsetmiterjoin%
\definecolor{currentfill}{rgb}{0.917647,0.917647,0.949020}%
\pgfsetfillcolor{currentfill}%
\pgfsetfillopacity{0.800000}%
\pgfsetlinewidth{1.003750pt}%
\definecolor{currentstroke}{rgb}{0.800000,0.800000,0.800000}%
\pgfsetstrokecolor{currentstroke}%
\pgfsetstrokeopacity{0.800000}%
\pgfsetdash{}{0pt}%
\pgfpathmoveto{\pgfqpoint{4.211472in}{2.592333in}}%
\pgfpathlineto{\pgfqpoint{5.303204in}{2.592333in}}%
\pgfpathquadraticcurveto{\pgfqpoint{5.333759in}{2.592333in}}{\pgfqpoint{5.333759in}{2.622888in}}%
\pgfpathlineto{\pgfqpoint{5.333759in}{3.257957in}}%
\pgfpathquadraticcurveto{\pgfqpoint{5.333759in}{3.288513in}}{\pgfqpoint{5.303204in}{3.288513in}}%
\pgfpathlineto{\pgfqpoint{4.211472in}{3.288513in}}%
\pgfpathquadraticcurveto{\pgfqpoint{4.180916in}{3.288513in}}{\pgfqpoint{4.180916in}{3.257957in}}%
\pgfpathlineto{\pgfqpoint{4.180916in}{2.622888in}}%
\pgfpathquadraticcurveto{\pgfqpoint{4.180916in}{2.592333in}}{\pgfqpoint{4.211472in}{2.592333in}}%
\pgfpathlineto{\pgfqpoint{4.211472in}{2.592333in}}%
\pgfpathclose%
\pgfusepath{stroke,fill}%
\end{pgfscope}%
\begin{pgfscope}%
\definecolor{textcolor}{rgb}{0.150000,0.150000,0.150000}%
\pgfsetstrokecolor{textcolor}%
\pgfsetfillcolor{textcolor}%
\pgftext[x=4.513726in,y=3.111661in,left,base]{\color{textcolor}\sffamily\fontsize{12.000000}{14.400000}\selectfont F-score}%
\end{pgfscope}%
\begin{pgfscope}%
\pgfsetbuttcap%
\pgfsetroundjoin%
\definecolor{currentfill}{rgb}{0.298039,0.447059,0.690196}%
\pgfsetfillcolor{currentfill}%
\pgfsetlinewidth{2.032594pt}%
\definecolor{currentstroke}{rgb}{0.298039,0.447059,0.690196}%
\pgfsetstrokecolor{currentstroke}%
\pgfsetdash{}{0pt}%
\pgfsys@defobject{currentmarker}{\pgfqpoint{-0.046999in}{-0.046999in}}{\pgfqpoint{0.046999in}{0.046999in}}{%
\pgfpathmoveto{\pgfqpoint{0.000000in}{-0.046999in}}%
\pgfpathcurveto{\pgfqpoint{0.012464in}{-0.046999in}}{\pgfqpoint{0.024420in}{-0.042047in}}{\pgfqpoint{0.033234in}{-0.033234in}}%
\pgfpathcurveto{\pgfqpoint{0.042047in}{-0.024420in}}{\pgfqpoint{0.046999in}{-0.012464in}}{\pgfqpoint{0.046999in}{0.000000in}}%
\pgfpathcurveto{\pgfqpoint{0.046999in}{0.012464in}}{\pgfqpoint{0.042047in}{0.024420in}}{\pgfqpoint{0.033234in}{0.033234in}}%
\pgfpathcurveto{\pgfqpoint{0.024420in}{0.042047in}}{\pgfqpoint{0.012464in}{0.046999in}}{\pgfqpoint{0.000000in}{0.046999in}}%
\pgfpathcurveto{\pgfqpoint{-0.012464in}{0.046999in}}{\pgfqpoint{-0.024420in}{0.042047in}}{\pgfqpoint{-0.033234in}{0.033234in}}%
\pgfpathcurveto{\pgfqpoint{-0.042047in}{0.024420in}}{\pgfqpoint{-0.046999in}{0.012464in}}{\pgfqpoint{-0.046999in}{0.000000in}}%
\pgfpathcurveto{\pgfqpoint{-0.046999in}{-0.012464in}}{\pgfqpoint{-0.042047in}{-0.024420in}}{\pgfqpoint{-0.033234in}{-0.033234in}}%
\pgfpathcurveto{\pgfqpoint{-0.024420in}{-0.042047in}}{\pgfqpoint{-0.012464in}{-0.046999in}}{\pgfqpoint{0.000000in}{-0.046999in}}%
\pgfpathlineto{\pgfqpoint{0.000000in}{-0.046999in}}%
\pgfpathclose%
\pgfusepath{stroke,fill}%
}%
\begin{pgfscope}%
\pgfsys@transformshift{4.394805in}{2.936025in}%
\pgfsys@useobject{currentmarker}{}%
\end{pgfscope}%
\end{pgfscope}%
\begin{pgfscope}%
\definecolor{textcolor}{rgb}{0.150000,0.150000,0.150000}%
\pgfsetstrokecolor{textcolor}%
\pgfsetfillcolor{textcolor}%
\pgftext[x=4.669805in,y=2.895921in,left,base]{\color{textcolor}\sffamily\fontsize{11.000000}{13.200000}\selectfont F1 micro}%
\end{pgfscope}%
\begin{pgfscope}%
\pgfsetbuttcap%
\pgfsetroundjoin%
\definecolor{currentfill}{rgb}{0.866667,0.517647,0.321569}%
\pgfsetfillcolor{currentfill}%
\pgfsetlinewidth{2.032594pt}%
\definecolor{currentstroke}{rgb}{0.866667,0.517647,0.321569}%
\pgfsetstrokecolor{currentstroke}%
\pgfsetdash{}{0pt}%
\pgfsys@defobject{currentmarker}{\pgfqpoint{-0.046999in}{-0.046999in}}{\pgfqpoint{0.046999in}{0.046999in}}{%
\pgfpathmoveto{\pgfqpoint{0.000000in}{-0.046999in}}%
\pgfpathcurveto{\pgfqpoint{0.012464in}{-0.046999in}}{\pgfqpoint{0.024420in}{-0.042047in}}{\pgfqpoint{0.033234in}{-0.033234in}}%
\pgfpathcurveto{\pgfqpoint{0.042047in}{-0.024420in}}{\pgfqpoint{0.046999in}{-0.012464in}}{\pgfqpoint{0.046999in}{0.000000in}}%
\pgfpathcurveto{\pgfqpoint{0.046999in}{0.012464in}}{\pgfqpoint{0.042047in}{0.024420in}}{\pgfqpoint{0.033234in}{0.033234in}}%
\pgfpathcurveto{\pgfqpoint{0.024420in}{0.042047in}}{\pgfqpoint{0.012464in}{0.046999in}}{\pgfqpoint{0.000000in}{0.046999in}}%
\pgfpathcurveto{\pgfqpoint{-0.012464in}{0.046999in}}{\pgfqpoint{-0.024420in}{0.042047in}}{\pgfqpoint{-0.033234in}{0.033234in}}%
\pgfpathcurveto{\pgfqpoint{-0.042047in}{0.024420in}}{\pgfqpoint{-0.046999in}{0.012464in}}{\pgfqpoint{-0.046999in}{0.000000in}}%
\pgfpathcurveto{\pgfqpoint{-0.046999in}{-0.012464in}}{\pgfqpoint{-0.042047in}{-0.024420in}}{\pgfqpoint{-0.033234in}{-0.033234in}}%
\pgfpathcurveto{\pgfqpoint{-0.024420in}{-0.042047in}}{\pgfqpoint{-0.012464in}{-0.046999in}}{\pgfqpoint{0.000000in}{-0.046999in}}%
\pgfpathlineto{\pgfqpoint{0.000000in}{-0.046999in}}%
\pgfpathclose%
\pgfusepath{stroke,fill}%
}%
\begin{pgfscope}%
\pgfsys@transformshift{4.394805in}{2.723120in}%
\pgfsys@useobject{currentmarker}{}%
\end{pgfscope}%
\end{pgfscope}%
\begin{pgfscope}%
\definecolor{textcolor}{rgb}{0.150000,0.150000,0.150000}%
\pgfsetstrokecolor{textcolor}%
\pgfsetfillcolor{textcolor}%
\pgftext[x=4.669805in,y=2.683016in,left,base]{\color{textcolor}\sffamily\fontsize{11.000000}{13.200000}\selectfont F1 macro}%
\end{pgfscope}%
\end{pgfpicture}%
\makeatother%
\endgroup%
}
        \label{fig:node2vec:representation and window size}
    \end{subfigure}
    \hfill
    \begin{subfigure}[b]{0.49\textwidth}
        \centering
        \scalebox{.5}{%% Creator: Matplotlib, PGF backend
%%
%% To include the figure in your LaTeX document, write
%%   \input{<filename>.pgf}
%%
%% Make sure the required packages are loaded in your preamble
%%   \usepackage{pgf}
%%
%% Also ensure that all the required font packages are loaded; for instance,
%% the lmodern package is sometimes necessary when using math font.
%%   \usepackage{lmodern}
%%
%% Figures using additional raster images can only be included by \input if
%% they are in the same directory as the main LaTeX file. For loading figures
%% from other directories you can use the `import` package
%%   \usepackage{import}
%%
%% and then include the figures with
%%   \import{<path to file>}{<filename>.pgf}
%%
%% Matplotlib used the following preamble
%%   
%%   \makeatletter\@ifpackageloaded{underscore}{}{\usepackage[strings]{underscore}}\makeatother
%%
\begingroup%
\makeatletter%
\begin{pgfpicture}%
\pgfpathrectangle{\pgfpointorigin}{\pgfqpoint{5.590148in}{6.909803in}}%
\pgfusepath{use as bounding box, clip}%
\begin{pgfscope}%
\pgfsetbuttcap%
\pgfsetmiterjoin%
\definecolor{currentfill}{rgb}{1.000000,1.000000,1.000000}%
\pgfsetfillcolor{currentfill}%
\pgfsetlinewidth{0.000000pt}%
\definecolor{currentstroke}{rgb}{1.000000,1.000000,1.000000}%
\pgfsetstrokecolor{currentstroke}%
\pgfsetdash{}{0pt}%
\pgfpathmoveto{\pgfqpoint{0.000000in}{0.000000in}}%
\pgfpathlineto{\pgfqpoint{5.590148in}{0.000000in}}%
\pgfpathlineto{\pgfqpoint{5.590148in}{6.909803in}}%
\pgfpathlineto{\pgfqpoint{0.000000in}{6.909803in}}%
\pgfpathlineto{\pgfqpoint{0.000000in}{0.000000in}}%
\pgfpathclose%
\pgfusepath{fill}%
\end{pgfscope}%
\begin{pgfscope}%
\pgfsetbuttcap%
\pgfsetmiterjoin%
\definecolor{currentfill}{rgb}{0.917647,0.917647,0.949020}%
\pgfsetfillcolor{currentfill}%
\pgfsetlinewidth{0.000000pt}%
\definecolor{currentstroke}{rgb}{0.000000,0.000000,0.000000}%
\pgfsetstrokecolor{currentstroke}%
\pgfsetstrokeopacity{0.000000}%
\pgfsetdash{}{0pt}%
\pgfpathmoveto{\pgfqpoint{0.667731in}{4.015734in}}%
\pgfpathlineto{\pgfqpoint{5.410148in}{4.015734in}}%
\pgfpathlineto{\pgfqpoint{5.410148in}{6.729803in}}%
\pgfpathlineto{\pgfqpoint{0.667731in}{6.729803in}}%
\pgfpathlineto{\pgfqpoint{0.667731in}{4.015734in}}%
\pgfpathclose%
\pgfusepath{fill}%
\end{pgfscope}%
\begin{pgfscope}%
\definecolor{textcolor}{rgb}{0.150000,0.150000,0.150000}%
\pgfsetstrokecolor{textcolor}%
\pgfsetfillcolor{textcolor}%
\pgftext[x=1.141973in,y=3.883790in,,top]{\color{textcolor}\sffamily\fontsize{11.000000}{13.200000}\selectfont 10}%
\end{pgfscope}%
\begin{pgfscope}%
\definecolor{textcolor}{rgb}{0.150000,0.150000,0.150000}%
\pgfsetstrokecolor{textcolor}%
\pgfsetfillcolor{textcolor}%
\pgftext[x=2.090456in,y=3.883790in,,top]{\color{textcolor}\sffamily\fontsize{11.000000}{13.200000}\selectfont 20}%
\end{pgfscope}%
\begin{pgfscope}%
\definecolor{textcolor}{rgb}{0.150000,0.150000,0.150000}%
\pgfsetstrokecolor{textcolor}%
\pgfsetfillcolor{textcolor}%
\pgftext[x=3.038940in,y=3.883790in,,top]{\color{textcolor}\sffamily\fontsize{11.000000}{13.200000}\selectfont 40}%
\end{pgfscope}%
\begin{pgfscope}%
\definecolor{textcolor}{rgb}{0.150000,0.150000,0.150000}%
\pgfsetstrokecolor{textcolor}%
\pgfsetfillcolor{textcolor}%
\pgftext[x=3.987423in,y=3.883790in,,top]{\color{textcolor}\sffamily\fontsize{11.000000}{13.200000}\selectfont 60}%
\end{pgfscope}%
\begin{pgfscope}%
\definecolor{textcolor}{rgb}{0.150000,0.150000,0.150000}%
\pgfsetstrokecolor{textcolor}%
\pgfsetfillcolor{textcolor}%
\pgftext[x=4.935906in,y=3.883790in,,top]{\color{textcolor}\sffamily\fontsize{11.000000}{13.200000}\selectfont 80}%
\end{pgfscope}%
\begin{pgfscope}%
\definecolor{textcolor}{rgb}{0.150000,0.150000,0.150000}%
\pgfsetstrokecolor{textcolor}%
\pgfsetfillcolor{textcolor}%
\pgftext[x=3.038940in,y=3.693049in,,top]{\color{textcolor}\sffamily\fontsize{12.000000}{14.400000}\selectfont random walk length}%
\end{pgfscope}%
\begin{pgfscope}%
\pgfpathrectangle{\pgfqpoint{0.667731in}{4.015734in}}{\pgfqpoint{4.742417in}{2.714069in}}%
\pgfusepath{clip}%
\pgfsetroundcap%
\pgfsetroundjoin%
\pgfsetlinewidth{1.003750pt}%
\definecolor{currentstroke}{rgb}{1.000000,1.000000,1.000000}%
\pgfsetstrokecolor{currentstroke}%
\pgfsetdash{}{0pt}%
\pgfpathmoveto{\pgfqpoint{0.667731in}{4.340179in}}%
\pgfpathlineto{\pgfqpoint{5.410148in}{4.340179in}}%
\pgfusepath{stroke}%
\end{pgfscope}%
\begin{pgfscope}%
\definecolor{textcolor}{rgb}{0.150000,0.150000,0.150000}%
\pgfsetstrokecolor{textcolor}%
\pgfsetfillcolor{textcolor}%
\pgftext[x=0.383703in, y=4.287373in, left, base]{\color{textcolor}\sffamily\fontsize{11.000000}{13.200000}\selectfont \(\displaystyle {40}\)}%
\end{pgfscope}%
\begin{pgfscope}%
\pgfpathrectangle{\pgfqpoint{0.667731in}{4.015734in}}{\pgfqpoint{4.742417in}{2.714069in}}%
\pgfusepath{clip}%
\pgfsetroundcap%
\pgfsetroundjoin%
\pgfsetlinewidth{1.003750pt}%
\definecolor{currentstroke}{rgb}{1.000000,1.000000,1.000000}%
\pgfsetstrokecolor{currentstroke}%
\pgfsetdash{}{0pt}%
\pgfpathmoveto{\pgfqpoint{0.667731in}{4.714929in}}%
\pgfpathlineto{\pgfqpoint{5.410148in}{4.714929in}}%
\pgfusepath{stroke}%
\end{pgfscope}%
\begin{pgfscope}%
\definecolor{textcolor}{rgb}{0.150000,0.150000,0.150000}%
\pgfsetstrokecolor{textcolor}%
\pgfsetfillcolor{textcolor}%
\pgftext[x=0.383703in, y=4.662122in, left, base]{\color{textcolor}\sffamily\fontsize{11.000000}{13.200000}\selectfont \(\displaystyle {45}\)}%
\end{pgfscope}%
\begin{pgfscope}%
\pgfpathrectangle{\pgfqpoint{0.667731in}{4.015734in}}{\pgfqpoint{4.742417in}{2.714069in}}%
\pgfusepath{clip}%
\pgfsetroundcap%
\pgfsetroundjoin%
\pgfsetlinewidth{1.003750pt}%
\definecolor{currentstroke}{rgb}{1.000000,1.000000,1.000000}%
\pgfsetstrokecolor{currentstroke}%
\pgfsetdash{}{0pt}%
\pgfpathmoveto{\pgfqpoint{0.667731in}{5.089678in}}%
\pgfpathlineto{\pgfqpoint{5.410148in}{5.089678in}}%
\pgfusepath{stroke}%
\end{pgfscope}%
\begin{pgfscope}%
\definecolor{textcolor}{rgb}{0.150000,0.150000,0.150000}%
\pgfsetstrokecolor{textcolor}%
\pgfsetfillcolor{textcolor}%
\pgftext[x=0.383703in, y=5.036871in, left, base]{\color{textcolor}\sffamily\fontsize{11.000000}{13.200000}\selectfont \(\displaystyle {50}\)}%
\end{pgfscope}%
\begin{pgfscope}%
\pgfpathrectangle{\pgfqpoint{0.667731in}{4.015734in}}{\pgfqpoint{4.742417in}{2.714069in}}%
\pgfusepath{clip}%
\pgfsetroundcap%
\pgfsetroundjoin%
\pgfsetlinewidth{1.003750pt}%
\definecolor{currentstroke}{rgb}{1.000000,1.000000,1.000000}%
\pgfsetstrokecolor{currentstroke}%
\pgfsetdash{}{0pt}%
\pgfpathmoveto{\pgfqpoint{0.667731in}{5.464428in}}%
\pgfpathlineto{\pgfqpoint{5.410148in}{5.464428in}}%
\pgfusepath{stroke}%
\end{pgfscope}%
\begin{pgfscope}%
\definecolor{textcolor}{rgb}{0.150000,0.150000,0.150000}%
\pgfsetstrokecolor{textcolor}%
\pgfsetfillcolor{textcolor}%
\pgftext[x=0.383703in, y=5.411621in, left, base]{\color{textcolor}\sffamily\fontsize{11.000000}{13.200000}\selectfont \(\displaystyle {55}\)}%
\end{pgfscope}%
\begin{pgfscope}%
\pgfpathrectangle{\pgfqpoint{0.667731in}{4.015734in}}{\pgfqpoint{4.742417in}{2.714069in}}%
\pgfusepath{clip}%
\pgfsetroundcap%
\pgfsetroundjoin%
\pgfsetlinewidth{1.003750pt}%
\definecolor{currentstroke}{rgb}{1.000000,1.000000,1.000000}%
\pgfsetstrokecolor{currentstroke}%
\pgfsetdash{}{0pt}%
\pgfpathmoveto{\pgfqpoint{0.667731in}{5.839177in}}%
\pgfpathlineto{\pgfqpoint{5.410148in}{5.839177in}}%
\pgfusepath{stroke}%
\end{pgfscope}%
\begin{pgfscope}%
\definecolor{textcolor}{rgb}{0.150000,0.150000,0.150000}%
\pgfsetstrokecolor{textcolor}%
\pgfsetfillcolor{textcolor}%
\pgftext[x=0.383703in, y=5.786370in, left, base]{\color{textcolor}\sffamily\fontsize{11.000000}{13.200000}\selectfont \(\displaystyle {60}\)}%
\end{pgfscope}%
\begin{pgfscope}%
\pgfpathrectangle{\pgfqpoint{0.667731in}{4.015734in}}{\pgfqpoint{4.742417in}{2.714069in}}%
\pgfusepath{clip}%
\pgfsetroundcap%
\pgfsetroundjoin%
\pgfsetlinewidth{1.003750pt}%
\definecolor{currentstroke}{rgb}{1.000000,1.000000,1.000000}%
\pgfsetstrokecolor{currentstroke}%
\pgfsetdash{}{0pt}%
\pgfpathmoveto{\pgfqpoint{0.667731in}{6.213926in}}%
\pgfpathlineto{\pgfqpoint{5.410148in}{6.213926in}}%
\pgfusepath{stroke}%
\end{pgfscope}%
\begin{pgfscope}%
\definecolor{textcolor}{rgb}{0.150000,0.150000,0.150000}%
\pgfsetstrokecolor{textcolor}%
\pgfsetfillcolor{textcolor}%
\pgftext[x=0.383703in, y=6.161120in, left, base]{\color{textcolor}\sffamily\fontsize{11.000000}{13.200000}\selectfont \(\displaystyle {65}\)}%
\end{pgfscope}%
\begin{pgfscope}%
\pgfpathrectangle{\pgfqpoint{0.667731in}{4.015734in}}{\pgfqpoint{4.742417in}{2.714069in}}%
\pgfusepath{clip}%
\pgfsetroundcap%
\pgfsetroundjoin%
\pgfsetlinewidth{1.003750pt}%
\definecolor{currentstroke}{rgb}{1.000000,1.000000,1.000000}%
\pgfsetstrokecolor{currentstroke}%
\pgfsetdash{}{0pt}%
\pgfpathmoveto{\pgfqpoint{0.667731in}{6.588676in}}%
\pgfpathlineto{\pgfqpoint{5.410148in}{6.588676in}}%
\pgfusepath{stroke}%
\end{pgfscope}%
\begin{pgfscope}%
\definecolor{textcolor}{rgb}{0.150000,0.150000,0.150000}%
\pgfsetstrokecolor{textcolor}%
\pgfsetfillcolor{textcolor}%
\pgftext[x=0.383703in, y=6.535869in, left, base]{\color{textcolor}\sffamily\fontsize{11.000000}{13.200000}\selectfont \(\displaystyle {70}\)}%
\end{pgfscope}%
\begin{pgfscope}%
\definecolor{textcolor}{rgb}{0.150000,0.150000,0.150000}%
\pgfsetstrokecolor{textcolor}%
\pgfsetfillcolor{textcolor}%
\pgftext[x=0.328148in,y=5.372769in,,bottom,rotate=90.000000]{\color{textcolor}\sffamily\fontsize{12.000000}{14.400000}\selectfont Average score}%
\end{pgfscope}%
\begin{pgfscope}%
\pgfpathrectangle{\pgfqpoint{0.667731in}{4.015734in}}{\pgfqpoint{4.742417in}{2.714069in}}%
\pgfusepath{clip}%
\pgfsetbuttcap%
\pgfsetroundjoin%
\pgfsetlinewidth{2.710125pt}%
\definecolor{currentstroke}{rgb}{0.298039,0.447059,0.690196}%
\pgfsetstrokecolor{currentstroke}%
\pgfsetdash{{9.990000pt}{4.320000pt}}{0.000000pt}%
\pgfpathmoveto{\pgfqpoint{1.141973in}{5.668675in}}%
\pgfpathlineto{\pgfqpoint{2.090456in}{6.002575in}}%
\pgfpathlineto{\pgfqpoint{3.038940in}{6.435934in}}%
\pgfpathlineto{\pgfqpoint{3.987423in}{6.606436in}}%
\pgfpathlineto{\pgfqpoint{4.935906in}{6.354235in}}%
\pgfusepath{stroke}%
\end{pgfscope}%
\begin{pgfscope}%
\pgfpathrectangle{\pgfqpoint{0.667731in}{4.015734in}}{\pgfqpoint{4.742417in}{2.714069in}}%
\pgfusepath{clip}%
\pgfsetroundcap%
\pgfsetroundjoin%
\pgfsetlinewidth{2.710125pt}%
\definecolor{currentstroke}{rgb}{0.298039,0.447059,0.690196}%
\pgfsetstrokecolor{currentstroke}%
\pgfsetdash{}{0pt}%
\pgfusepath{stroke}%
\end{pgfscope}%
\begin{pgfscope}%
\pgfpathrectangle{\pgfqpoint{0.667731in}{4.015734in}}{\pgfqpoint{4.742417in}{2.714069in}}%
\pgfusepath{clip}%
\pgfsetroundcap%
\pgfsetroundjoin%
\pgfsetlinewidth{2.710125pt}%
\definecolor{currentstroke}{rgb}{0.298039,0.447059,0.690196}%
\pgfsetstrokecolor{currentstroke}%
\pgfsetdash{}{0pt}%
\pgfusepath{stroke}%
\end{pgfscope}%
\begin{pgfscope}%
\pgfpathrectangle{\pgfqpoint{0.667731in}{4.015734in}}{\pgfqpoint{4.742417in}{2.714069in}}%
\pgfusepath{clip}%
\pgfsetroundcap%
\pgfsetroundjoin%
\pgfsetlinewidth{2.710125pt}%
\definecolor{currentstroke}{rgb}{0.298039,0.447059,0.690196}%
\pgfsetstrokecolor{currentstroke}%
\pgfsetdash{}{0pt}%
\pgfusepath{stroke}%
\end{pgfscope}%
\begin{pgfscope}%
\pgfpathrectangle{\pgfqpoint{0.667731in}{4.015734in}}{\pgfqpoint{4.742417in}{2.714069in}}%
\pgfusepath{clip}%
\pgfsetroundcap%
\pgfsetroundjoin%
\pgfsetlinewidth{2.710125pt}%
\definecolor{currentstroke}{rgb}{0.298039,0.447059,0.690196}%
\pgfsetstrokecolor{currentstroke}%
\pgfsetdash{}{0pt}%
\pgfusepath{stroke}%
\end{pgfscope}%
\begin{pgfscope}%
\pgfpathrectangle{\pgfqpoint{0.667731in}{4.015734in}}{\pgfqpoint{4.742417in}{2.714069in}}%
\pgfusepath{clip}%
\pgfsetroundcap%
\pgfsetroundjoin%
\pgfsetlinewidth{2.710125pt}%
\definecolor{currentstroke}{rgb}{0.298039,0.447059,0.690196}%
\pgfsetstrokecolor{currentstroke}%
\pgfsetdash{}{0pt}%
\pgfusepath{stroke}%
\end{pgfscope}%
\begin{pgfscope}%
\pgfpathrectangle{\pgfqpoint{0.667731in}{4.015734in}}{\pgfqpoint{4.742417in}{2.714069in}}%
\pgfusepath{clip}%
\pgfsetbuttcap%
\pgfsetroundjoin%
\definecolor{currentfill}{rgb}{0.298039,0.447059,0.690196}%
\pgfsetfillcolor{currentfill}%
\pgfsetlinewidth{2.032594pt}%
\definecolor{currentstroke}{rgb}{0.298039,0.447059,0.690196}%
\pgfsetstrokecolor{currentstroke}%
\pgfsetdash{}{0pt}%
\pgfsys@defobject{currentmarker}{\pgfqpoint{-0.046999in}{-0.046999in}}{\pgfqpoint{0.046999in}{0.046999in}}{%
\pgfpathmoveto{\pgfqpoint{0.000000in}{-0.046999in}}%
\pgfpathcurveto{\pgfqpoint{0.012464in}{-0.046999in}}{\pgfqpoint{0.024420in}{-0.042047in}}{\pgfqpoint{0.033234in}{-0.033234in}}%
\pgfpathcurveto{\pgfqpoint{0.042047in}{-0.024420in}}{\pgfqpoint{0.046999in}{-0.012464in}}{\pgfqpoint{0.046999in}{0.000000in}}%
\pgfpathcurveto{\pgfqpoint{0.046999in}{0.012464in}}{\pgfqpoint{0.042047in}{0.024420in}}{\pgfqpoint{0.033234in}{0.033234in}}%
\pgfpathcurveto{\pgfqpoint{0.024420in}{0.042047in}}{\pgfqpoint{0.012464in}{0.046999in}}{\pgfqpoint{0.000000in}{0.046999in}}%
\pgfpathcurveto{\pgfqpoint{-0.012464in}{0.046999in}}{\pgfqpoint{-0.024420in}{0.042047in}}{\pgfqpoint{-0.033234in}{0.033234in}}%
\pgfpathcurveto{\pgfqpoint{-0.042047in}{0.024420in}}{\pgfqpoint{-0.046999in}{0.012464in}}{\pgfqpoint{-0.046999in}{0.000000in}}%
\pgfpathcurveto{\pgfqpoint{-0.046999in}{-0.012464in}}{\pgfqpoint{-0.042047in}{-0.024420in}}{\pgfqpoint{-0.033234in}{-0.033234in}}%
\pgfpathcurveto{\pgfqpoint{-0.024420in}{-0.042047in}}{\pgfqpoint{-0.012464in}{-0.046999in}}{\pgfqpoint{0.000000in}{-0.046999in}}%
\pgfpathlineto{\pgfqpoint{0.000000in}{-0.046999in}}%
\pgfpathclose%
\pgfusepath{stroke,fill}%
}%
\begin{pgfscope}%
\pgfsys@transformshift{1.141973in}{5.668675in}%
\pgfsys@useobject{currentmarker}{}%
\end{pgfscope}%
\begin{pgfscope}%
\pgfsys@transformshift{2.090456in}{6.002575in}%
\pgfsys@useobject{currentmarker}{}%
\end{pgfscope}%
\begin{pgfscope}%
\pgfsys@transformshift{3.038940in}{6.435934in}%
\pgfsys@useobject{currentmarker}{}%
\end{pgfscope}%
\begin{pgfscope}%
\pgfsys@transformshift{3.987423in}{6.606436in}%
\pgfsys@useobject{currentmarker}{}%
\end{pgfscope}%
\begin{pgfscope}%
\pgfsys@transformshift{4.935906in}{6.354235in}%
\pgfsys@useobject{currentmarker}{}%
\end{pgfscope}%
\end{pgfscope}%
\begin{pgfscope}%
\pgfpathrectangle{\pgfqpoint{0.667731in}{4.015734in}}{\pgfqpoint{4.742417in}{2.714069in}}%
\pgfusepath{clip}%
\pgfsetbuttcap%
\pgfsetroundjoin%
\pgfsetlinewidth{2.710125pt}%
\definecolor{currentstroke}{rgb}{0.866667,0.517647,0.321569}%
\pgfsetstrokecolor{currentstroke}%
\pgfsetdash{{9.990000pt}{4.320000pt}}{0.000000pt}%
\pgfpathmoveto{\pgfqpoint{1.141973in}{4.139101in}}%
\pgfpathlineto{\pgfqpoint{2.090456in}{5.328391in}}%
\pgfpathlineto{\pgfqpoint{3.038940in}{6.106245in}}%
\pgfpathlineto{\pgfqpoint{3.987423in}{6.387736in}}%
\pgfpathlineto{\pgfqpoint{4.935906in}{6.123920in}}%
\pgfusepath{stroke}%
\end{pgfscope}%
\begin{pgfscope}%
\pgfpathrectangle{\pgfqpoint{0.667731in}{4.015734in}}{\pgfqpoint{4.742417in}{2.714069in}}%
\pgfusepath{clip}%
\pgfsetroundcap%
\pgfsetroundjoin%
\pgfsetlinewidth{2.710125pt}%
\definecolor{currentstroke}{rgb}{0.866667,0.517647,0.321569}%
\pgfsetstrokecolor{currentstroke}%
\pgfsetdash{}{0pt}%
\pgfusepath{stroke}%
\end{pgfscope}%
\begin{pgfscope}%
\pgfpathrectangle{\pgfqpoint{0.667731in}{4.015734in}}{\pgfqpoint{4.742417in}{2.714069in}}%
\pgfusepath{clip}%
\pgfsetroundcap%
\pgfsetroundjoin%
\pgfsetlinewidth{2.710125pt}%
\definecolor{currentstroke}{rgb}{0.866667,0.517647,0.321569}%
\pgfsetstrokecolor{currentstroke}%
\pgfsetdash{}{0pt}%
\pgfusepath{stroke}%
\end{pgfscope}%
\begin{pgfscope}%
\pgfpathrectangle{\pgfqpoint{0.667731in}{4.015734in}}{\pgfqpoint{4.742417in}{2.714069in}}%
\pgfusepath{clip}%
\pgfsetroundcap%
\pgfsetroundjoin%
\pgfsetlinewidth{2.710125pt}%
\definecolor{currentstroke}{rgb}{0.866667,0.517647,0.321569}%
\pgfsetstrokecolor{currentstroke}%
\pgfsetdash{}{0pt}%
\pgfusepath{stroke}%
\end{pgfscope}%
\begin{pgfscope}%
\pgfpathrectangle{\pgfqpoint{0.667731in}{4.015734in}}{\pgfqpoint{4.742417in}{2.714069in}}%
\pgfusepath{clip}%
\pgfsetroundcap%
\pgfsetroundjoin%
\pgfsetlinewidth{2.710125pt}%
\definecolor{currentstroke}{rgb}{0.866667,0.517647,0.321569}%
\pgfsetstrokecolor{currentstroke}%
\pgfsetdash{}{0pt}%
\pgfusepath{stroke}%
\end{pgfscope}%
\begin{pgfscope}%
\pgfpathrectangle{\pgfqpoint{0.667731in}{4.015734in}}{\pgfqpoint{4.742417in}{2.714069in}}%
\pgfusepath{clip}%
\pgfsetroundcap%
\pgfsetroundjoin%
\pgfsetlinewidth{2.710125pt}%
\definecolor{currentstroke}{rgb}{0.866667,0.517647,0.321569}%
\pgfsetstrokecolor{currentstroke}%
\pgfsetdash{}{0pt}%
\pgfusepath{stroke}%
\end{pgfscope}%
\begin{pgfscope}%
\pgfpathrectangle{\pgfqpoint{0.667731in}{4.015734in}}{\pgfqpoint{4.742417in}{2.714069in}}%
\pgfusepath{clip}%
\pgfsetbuttcap%
\pgfsetroundjoin%
\definecolor{currentfill}{rgb}{0.866667,0.517647,0.321569}%
\pgfsetfillcolor{currentfill}%
\pgfsetlinewidth{2.032594pt}%
\definecolor{currentstroke}{rgb}{0.866667,0.517647,0.321569}%
\pgfsetstrokecolor{currentstroke}%
\pgfsetdash{}{0pt}%
\pgfsys@defobject{currentmarker}{\pgfqpoint{-0.046999in}{-0.046999in}}{\pgfqpoint{0.046999in}{0.046999in}}{%
\pgfpathmoveto{\pgfqpoint{0.000000in}{-0.046999in}}%
\pgfpathcurveto{\pgfqpoint{0.012464in}{-0.046999in}}{\pgfqpoint{0.024420in}{-0.042047in}}{\pgfqpoint{0.033234in}{-0.033234in}}%
\pgfpathcurveto{\pgfqpoint{0.042047in}{-0.024420in}}{\pgfqpoint{0.046999in}{-0.012464in}}{\pgfqpoint{0.046999in}{0.000000in}}%
\pgfpathcurveto{\pgfqpoint{0.046999in}{0.012464in}}{\pgfqpoint{0.042047in}{0.024420in}}{\pgfqpoint{0.033234in}{0.033234in}}%
\pgfpathcurveto{\pgfqpoint{0.024420in}{0.042047in}}{\pgfqpoint{0.012464in}{0.046999in}}{\pgfqpoint{0.000000in}{0.046999in}}%
\pgfpathcurveto{\pgfqpoint{-0.012464in}{0.046999in}}{\pgfqpoint{-0.024420in}{0.042047in}}{\pgfqpoint{-0.033234in}{0.033234in}}%
\pgfpathcurveto{\pgfqpoint{-0.042047in}{0.024420in}}{\pgfqpoint{-0.046999in}{0.012464in}}{\pgfqpoint{-0.046999in}{0.000000in}}%
\pgfpathcurveto{\pgfqpoint{-0.046999in}{-0.012464in}}{\pgfqpoint{-0.042047in}{-0.024420in}}{\pgfqpoint{-0.033234in}{-0.033234in}}%
\pgfpathcurveto{\pgfqpoint{-0.024420in}{-0.042047in}}{\pgfqpoint{-0.012464in}{-0.046999in}}{\pgfqpoint{0.000000in}{-0.046999in}}%
\pgfpathlineto{\pgfqpoint{0.000000in}{-0.046999in}}%
\pgfpathclose%
\pgfusepath{stroke,fill}%
}%
\begin{pgfscope}%
\pgfsys@transformshift{1.141973in}{4.139101in}%
\pgfsys@useobject{currentmarker}{}%
\end{pgfscope}%
\begin{pgfscope}%
\pgfsys@transformshift{2.090456in}{5.328391in}%
\pgfsys@useobject{currentmarker}{}%
\end{pgfscope}%
\begin{pgfscope}%
\pgfsys@transformshift{3.038940in}{6.106245in}%
\pgfsys@useobject{currentmarker}{}%
\end{pgfscope}%
\begin{pgfscope}%
\pgfsys@transformshift{3.987423in}{6.387736in}%
\pgfsys@useobject{currentmarker}{}%
\end{pgfscope}%
\begin{pgfscope}%
\pgfsys@transformshift{4.935906in}{6.123920in}%
\pgfsys@useobject{currentmarker}{}%
\end{pgfscope}%
\end{pgfscope}%
\begin{pgfscope}%
\pgfsetrectcap%
\pgfsetmiterjoin%
\pgfsetlinewidth{1.254687pt}%
\definecolor{currentstroke}{rgb}{1.000000,1.000000,1.000000}%
\pgfsetstrokecolor{currentstroke}%
\pgfsetdash{}{0pt}%
\pgfpathmoveto{\pgfqpoint{0.667731in}{4.015734in}}%
\pgfpathlineto{\pgfqpoint{0.667731in}{6.729803in}}%
\pgfusepath{stroke}%
\end{pgfscope}%
\begin{pgfscope}%
\pgfsetrectcap%
\pgfsetmiterjoin%
\pgfsetlinewidth{1.254687pt}%
\definecolor{currentstroke}{rgb}{1.000000,1.000000,1.000000}%
\pgfsetstrokecolor{currentstroke}%
\pgfsetdash{}{0pt}%
\pgfpathmoveto{\pgfqpoint{5.410148in}{4.015734in}}%
\pgfpathlineto{\pgfqpoint{5.410148in}{6.729803in}}%
\pgfusepath{stroke}%
\end{pgfscope}%
\begin{pgfscope}%
\pgfsetrectcap%
\pgfsetmiterjoin%
\pgfsetlinewidth{1.254687pt}%
\definecolor{currentstroke}{rgb}{1.000000,1.000000,1.000000}%
\pgfsetstrokecolor{currentstroke}%
\pgfsetdash{}{0pt}%
\pgfpathmoveto{\pgfqpoint{0.667731in}{4.015734in}}%
\pgfpathlineto{\pgfqpoint{5.410148in}{4.015734in}}%
\pgfusepath{stroke}%
\end{pgfscope}%
\begin{pgfscope}%
\pgfsetrectcap%
\pgfsetmiterjoin%
\pgfsetlinewidth{1.254687pt}%
\definecolor{currentstroke}{rgb}{1.000000,1.000000,1.000000}%
\pgfsetstrokecolor{currentstroke}%
\pgfsetdash{}{0pt}%
\pgfpathmoveto{\pgfqpoint{0.667731in}{6.729803in}}%
\pgfpathlineto{\pgfqpoint{5.410148in}{6.729803in}}%
\pgfusepath{stroke}%
\end{pgfscope}%
\begin{pgfscope}%
\pgfsetbuttcap%
\pgfsetmiterjoin%
\definecolor{currentfill}{rgb}{0.917647,0.917647,0.949020}%
\pgfsetfillcolor{currentfill}%
\pgfsetfillopacity{0.800000}%
\pgfsetlinewidth{1.003750pt}%
\definecolor{currentstroke}{rgb}{0.800000,0.800000,0.800000}%
\pgfsetstrokecolor{currentstroke}%
\pgfsetstrokeopacity{0.800000}%
\pgfsetdash{}{0pt}%
\pgfpathmoveto{\pgfqpoint{0.774675in}{5.957234in}}%
\pgfpathlineto{\pgfqpoint{1.866407in}{5.957234in}}%
\pgfpathquadraticcurveto{\pgfqpoint{1.896963in}{5.957234in}}{\pgfqpoint{1.896963in}{5.987790in}}%
\pgfpathlineto{\pgfqpoint{1.896963in}{6.622859in}}%
\pgfpathquadraticcurveto{\pgfqpoint{1.896963in}{6.653414in}}{\pgfqpoint{1.866407in}{6.653414in}}%
\pgfpathlineto{\pgfqpoint{0.774675in}{6.653414in}}%
\pgfpathquadraticcurveto{\pgfqpoint{0.744120in}{6.653414in}}{\pgfqpoint{0.744120in}{6.622859in}}%
\pgfpathlineto{\pgfqpoint{0.744120in}{5.987790in}}%
\pgfpathquadraticcurveto{\pgfqpoint{0.744120in}{5.957234in}}{\pgfqpoint{0.774675in}{5.957234in}}%
\pgfpathlineto{\pgfqpoint{0.774675in}{5.957234in}}%
\pgfpathclose%
\pgfusepath{stroke,fill}%
\end{pgfscope}%
\begin{pgfscope}%
\definecolor{textcolor}{rgb}{0.150000,0.150000,0.150000}%
\pgfsetstrokecolor{textcolor}%
\pgfsetfillcolor{textcolor}%
\pgftext[x=1.076929in,y=6.476563in,left,base]{\color{textcolor}\sffamily\fontsize{12.000000}{14.400000}\selectfont F-score}%
\end{pgfscope}%
\begin{pgfscope}%
\pgfsetbuttcap%
\pgfsetroundjoin%
\definecolor{currentfill}{rgb}{0.298039,0.447059,0.690196}%
\pgfsetfillcolor{currentfill}%
\pgfsetlinewidth{2.032594pt}%
\definecolor{currentstroke}{rgb}{0.298039,0.447059,0.690196}%
\pgfsetstrokecolor{currentstroke}%
\pgfsetdash{}{0pt}%
\pgfsys@defobject{currentmarker}{\pgfqpoint{-0.046999in}{-0.046999in}}{\pgfqpoint{0.046999in}{0.046999in}}{%
\pgfpathmoveto{\pgfqpoint{0.000000in}{-0.046999in}}%
\pgfpathcurveto{\pgfqpoint{0.012464in}{-0.046999in}}{\pgfqpoint{0.024420in}{-0.042047in}}{\pgfqpoint{0.033234in}{-0.033234in}}%
\pgfpathcurveto{\pgfqpoint{0.042047in}{-0.024420in}}{\pgfqpoint{0.046999in}{-0.012464in}}{\pgfqpoint{0.046999in}{0.000000in}}%
\pgfpathcurveto{\pgfqpoint{0.046999in}{0.012464in}}{\pgfqpoint{0.042047in}{0.024420in}}{\pgfqpoint{0.033234in}{0.033234in}}%
\pgfpathcurveto{\pgfqpoint{0.024420in}{0.042047in}}{\pgfqpoint{0.012464in}{0.046999in}}{\pgfqpoint{0.000000in}{0.046999in}}%
\pgfpathcurveto{\pgfqpoint{-0.012464in}{0.046999in}}{\pgfqpoint{-0.024420in}{0.042047in}}{\pgfqpoint{-0.033234in}{0.033234in}}%
\pgfpathcurveto{\pgfqpoint{-0.042047in}{0.024420in}}{\pgfqpoint{-0.046999in}{0.012464in}}{\pgfqpoint{-0.046999in}{0.000000in}}%
\pgfpathcurveto{\pgfqpoint{-0.046999in}{-0.012464in}}{\pgfqpoint{-0.042047in}{-0.024420in}}{\pgfqpoint{-0.033234in}{-0.033234in}}%
\pgfpathcurveto{\pgfqpoint{-0.024420in}{-0.042047in}}{\pgfqpoint{-0.012464in}{-0.046999in}}{\pgfqpoint{0.000000in}{-0.046999in}}%
\pgfpathlineto{\pgfqpoint{0.000000in}{-0.046999in}}%
\pgfpathclose%
\pgfusepath{stroke,fill}%
}%
\begin{pgfscope}%
\pgfsys@transformshift{0.958009in}{6.300926in}%
\pgfsys@useobject{currentmarker}{}%
\end{pgfscope}%
\end{pgfscope}%
\begin{pgfscope}%
\definecolor{textcolor}{rgb}{0.150000,0.150000,0.150000}%
\pgfsetstrokecolor{textcolor}%
\pgfsetfillcolor{textcolor}%
\pgftext[x=1.233009in,y=6.260822in,left,base]{\color{textcolor}\sffamily\fontsize{11.000000}{13.200000}\selectfont F1 micro}%
\end{pgfscope}%
\begin{pgfscope}%
\pgfsetbuttcap%
\pgfsetroundjoin%
\definecolor{currentfill}{rgb}{0.866667,0.517647,0.321569}%
\pgfsetfillcolor{currentfill}%
\pgfsetlinewidth{2.032594pt}%
\definecolor{currentstroke}{rgb}{0.866667,0.517647,0.321569}%
\pgfsetstrokecolor{currentstroke}%
\pgfsetdash{}{0pt}%
\pgfsys@defobject{currentmarker}{\pgfqpoint{-0.046999in}{-0.046999in}}{\pgfqpoint{0.046999in}{0.046999in}}{%
\pgfpathmoveto{\pgfqpoint{0.000000in}{-0.046999in}}%
\pgfpathcurveto{\pgfqpoint{0.012464in}{-0.046999in}}{\pgfqpoint{0.024420in}{-0.042047in}}{\pgfqpoint{0.033234in}{-0.033234in}}%
\pgfpathcurveto{\pgfqpoint{0.042047in}{-0.024420in}}{\pgfqpoint{0.046999in}{-0.012464in}}{\pgfqpoint{0.046999in}{0.000000in}}%
\pgfpathcurveto{\pgfqpoint{0.046999in}{0.012464in}}{\pgfqpoint{0.042047in}{0.024420in}}{\pgfqpoint{0.033234in}{0.033234in}}%
\pgfpathcurveto{\pgfqpoint{0.024420in}{0.042047in}}{\pgfqpoint{0.012464in}{0.046999in}}{\pgfqpoint{0.000000in}{0.046999in}}%
\pgfpathcurveto{\pgfqpoint{-0.012464in}{0.046999in}}{\pgfqpoint{-0.024420in}{0.042047in}}{\pgfqpoint{-0.033234in}{0.033234in}}%
\pgfpathcurveto{\pgfqpoint{-0.042047in}{0.024420in}}{\pgfqpoint{-0.046999in}{0.012464in}}{\pgfqpoint{-0.046999in}{0.000000in}}%
\pgfpathcurveto{\pgfqpoint{-0.046999in}{-0.012464in}}{\pgfqpoint{-0.042047in}{-0.024420in}}{\pgfqpoint{-0.033234in}{-0.033234in}}%
\pgfpathcurveto{\pgfqpoint{-0.024420in}{-0.042047in}}{\pgfqpoint{-0.012464in}{-0.046999in}}{\pgfqpoint{0.000000in}{-0.046999in}}%
\pgfpathlineto{\pgfqpoint{0.000000in}{-0.046999in}}%
\pgfpathclose%
\pgfusepath{stroke,fill}%
}%
\begin{pgfscope}%
\pgfsys@transformshift{0.958009in}{6.088021in}%
\pgfsys@useobject{currentmarker}{}%
\end{pgfscope}%
\end{pgfscope}%
\begin{pgfscope}%
\definecolor{textcolor}{rgb}{0.150000,0.150000,0.150000}%
\pgfsetstrokecolor{textcolor}%
\pgfsetfillcolor{textcolor}%
\pgftext[x=1.233009in,y=6.047917in,left,base]{\color{textcolor}\sffamily\fontsize{11.000000}{13.200000}\selectfont F1 macro}%
\end{pgfscope}%
\begin{pgfscope}%
\pgfsetbuttcap%
\pgfsetmiterjoin%
\definecolor{currentfill}{rgb}{0.917647,0.917647,0.949020}%
\pgfsetfillcolor{currentfill}%
\pgfsetlinewidth{0.000000pt}%
\definecolor{currentstroke}{rgb}{0.000000,0.000000,0.000000}%
\pgfsetstrokecolor{currentstroke}%
\pgfsetstrokeopacity{0.000000}%
\pgfsetdash{}{0pt}%
\pgfpathmoveto{\pgfqpoint{0.667731in}{0.650833in}}%
\pgfpathlineto{\pgfqpoint{5.410148in}{0.650833in}}%
\pgfpathlineto{\pgfqpoint{5.410148in}{3.364902in}}%
\pgfpathlineto{\pgfqpoint{0.667731in}{3.364902in}}%
\pgfpathlineto{\pgfqpoint{0.667731in}{0.650833in}}%
\pgfpathclose%
\pgfusepath{fill}%
\end{pgfscope}%
\begin{pgfscope}%
\definecolor{textcolor}{rgb}{0.150000,0.150000,0.150000}%
\pgfsetstrokecolor{textcolor}%
\pgfsetfillcolor{textcolor}%
\pgftext[x=1.062932in,y=0.518888in,,top]{\color{textcolor}\sffamily\fontsize{11.000000}{13.200000}\selectfont 10}%
\end{pgfscope}%
\begin{pgfscope}%
\definecolor{textcolor}{rgb}{0.150000,0.150000,0.150000}%
\pgfsetstrokecolor{textcolor}%
\pgfsetfillcolor{textcolor}%
\pgftext[x=1.853335in,y=0.518888in,,top]{\color{textcolor}\sffamily\fontsize{11.000000}{13.200000}\selectfont 20}%
\end{pgfscope}%
\begin{pgfscope}%
\definecolor{textcolor}{rgb}{0.150000,0.150000,0.150000}%
\pgfsetstrokecolor{textcolor}%
\pgfsetfillcolor{textcolor}%
\pgftext[x=2.643738in,y=0.518888in,,top]{\color{textcolor}\sffamily\fontsize{11.000000}{13.200000}\selectfont 30}%
\end{pgfscope}%
\begin{pgfscope}%
\definecolor{textcolor}{rgb}{0.150000,0.150000,0.150000}%
\pgfsetstrokecolor{textcolor}%
\pgfsetfillcolor{textcolor}%
\pgftext[x=3.434141in,y=0.518888in,,top]{\color{textcolor}\sffamily\fontsize{11.000000}{13.200000}\selectfont 40}%
\end{pgfscope}%
\begin{pgfscope}%
\definecolor{textcolor}{rgb}{0.150000,0.150000,0.150000}%
\pgfsetstrokecolor{textcolor}%
\pgfsetfillcolor{textcolor}%
\pgftext[x=4.224544in,y=0.518888in,,top]{\color{textcolor}\sffamily\fontsize{11.000000}{13.200000}\selectfont 60}%
\end{pgfscope}%
\begin{pgfscope}%
\definecolor{textcolor}{rgb}{0.150000,0.150000,0.150000}%
\pgfsetstrokecolor{textcolor}%
\pgfsetfillcolor{textcolor}%
\pgftext[x=5.014947in,y=0.518888in,,top]{\color{textcolor}\sffamily\fontsize{11.000000}{13.200000}\selectfont 80}%
\end{pgfscope}%
\begin{pgfscope}%
\definecolor{textcolor}{rgb}{0.150000,0.150000,0.150000}%
\pgfsetstrokecolor{textcolor}%
\pgfsetfillcolor{textcolor}%
\pgftext[x=3.038940in,y=0.328148in,,top]{\color{textcolor}\sffamily\fontsize{12.000000}{14.400000}\selectfont number of random walks}%
\end{pgfscope}%
\begin{pgfscope}%
\pgfpathrectangle{\pgfqpoint{0.667731in}{0.650833in}}{\pgfqpoint{4.742417in}{2.714069in}}%
\pgfusepath{clip}%
\pgfsetroundcap%
\pgfsetroundjoin%
\pgfsetlinewidth{1.003750pt}%
\definecolor{currentstroke}{rgb}{1.000000,1.000000,1.000000}%
\pgfsetstrokecolor{currentstroke}%
\pgfsetdash{}{0pt}%
\pgfpathmoveto{\pgfqpoint{0.667731in}{0.734454in}}%
\pgfpathlineto{\pgfqpoint{5.410148in}{0.734454in}}%
\pgfusepath{stroke}%
\end{pgfscope}%
\begin{pgfscope}%
\definecolor{textcolor}{rgb}{0.150000,0.150000,0.150000}%
\pgfsetstrokecolor{textcolor}%
\pgfsetfillcolor{textcolor}%
\pgftext[x=0.383703in, y=0.681647in, left, base]{\color{textcolor}\sffamily\fontsize{11.000000}{13.200000}\selectfont \(\displaystyle {62}\)}%
\end{pgfscope}%
\begin{pgfscope}%
\pgfpathrectangle{\pgfqpoint{0.667731in}{0.650833in}}{\pgfqpoint{4.742417in}{2.714069in}}%
\pgfusepath{clip}%
\pgfsetroundcap%
\pgfsetroundjoin%
\pgfsetlinewidth{1.003750pt}%
\definecolor{currentstroke}{rgb}{1.000000,1.000000,1.000000}%
\pgfsetstrokecolor{currentstroke}%
\pgfsetdash{}{0pt}%
\pgfpathmoveto{\pgfqpoint{0.667731in}{1.123993in}}%
\pgfpathlineto{\pgfqpoint{5.410148in}{1.123993in}}%
\pgfusepath{stroke}%
\end{pgfscope}%
\begin{pgfscope}%
\definecolor{textcolor}{rgb}{0.150000,0.150000,0.150000}%
\pgfsetstrokecolor{textcolor}%
\pgfsetfillcolor{textcolor}%
\pgftext[x=0.383703in, y=1.071186in, left, base]{\color{textcolor}\sffamily\fontsize{11.000000}{13.200000}\selectfont \(\displaystyle {63}\)}%
\end{pgfscope}%
\begin{pgfscope}%
\pgfpathrectangle{\pgfqpoint{0.667731in}{0.650833in}}{\pgfqpoint{4.742417in}{2.714069in}}%
\pgfusepath{clip}%
\pgfsetroundcap%
\pgfsetroundjoin%
\pgfsetlinewidth{1.003750pt}%
\definecolor{currentstroke}{rgb}{1.000000,1.000000,1.000000}%
\pgfsetstrokecolor{currentstroke}%
\pgfsetdash{}{0pt}%
\pgfpathmoveto{\pgfqpoint{0.667731in}{1.513532in}}%
\pgfpathlineto{\pgfqpoint{5.410148in}{1.513532in}}%
\pgfusepath{stroke}%
\end{pgfscope}%
\begin{pgfscope}%
\definecolor{textcolor}{rgb}{0.150000,0.150000,0.150000}%
\pgfsetstrokecolor{textcolor}%
\pgfsetfillcolor{textcolor}%
\pgftext[x=0.383703in, y=1.460725in, left, base]{\color{textcolor}\sffamily\fontsize{11.000000}{13.200000}\selectfont \(\displaystyle {64}\)}%
\end{pgfscope}%
\begin{pgfscope}%
\pgfpathrectangle{\pgfqpoint{0.667731in}{0.650833in}}{\pgfqpoint{4.742417in}{2.714069in}}%
\pgfusepath{clip}%
\pgfsetroundcap%
\pgfsetroundjoin%
\pgfsetlinewidth{1.003750pt}%
\definecolor{currentstroke}{rgb}{1.000000,1.000000,1.000000}%
\pgfsetstrokecolor{currentstroke}%
\pgfsetdash{}{0pt}%
\pgfpathmoveto{\pgfqpoint{0.667731in}{1.903071in}}%
\pgfpathlineto{\pgfqpoint{5.410148in}{1.903071in}}%
\pgfusepath{stroke}%
\end{pgfscope}%
\begin{pgfscope}%
\definecolor{textcolor}{rgb}{0.150000,0.150000,0.150000}%
\pgfsetstrokecolor{textcolor}%
\pgfsetfillcolor{textcolor}%
\pgftext[x=0.383703in, y=1.850264in, left, base]{\color{textcolor}\sffamily\fontsize{11.000000}{13.200000}\selectfont \(\displaystyle {65}\)}%
\end{pgfscope}%
\begin{pgfscope}%
\pgfpathrectangle{\pgfqpoint{0.667731in}{0.650833in}}{\pgfqpoint{4.742417in}{2.714069in}}%
\pgfusepath{clip}%
\pgfsetroundcap%
\pgfsetroundjoin%
\pgfsetlinewidth{1.003750pt}%
\definecolor{currentstroke}{rgb}{1.000000,1.000000,1.000000}%
\pgfsetstrokecolor{currentstroke}%
\pgfsetdash{}{0pt}%
\pgfpathmoveto{\pgfqpoint{0.667731in}{2.292610in}}%
\pgfpathlineto{\pgfqpoint{5.410148in}{2.292610in}}%
\pgfusepath{stroke}%
\end{pgfscope}%
\begin{pgfscope}%
\definecolor{textcolor}{rgb}{0.150000,0.150000,0.150000}%
\pgfsetstrokecolor{textcolor}%
\pgfsetfillcolor{textcolor}%
\pgftext[x=0.383703in, y=2.239803in, left, base]{\color{textcolor}\sffamily\fontsize{11.000000}{13.200000}\selectfont \(\displaystyle {66}\)}%
\end{pgfscope}%
\begin{pgfscope}%
\pgfpathrectangle{\pgfqpoint{0.667731in}{0.650833in}}{\pgfqpoint{4.742417in}{2.714069in}}%
\pgfusepath{clip}%
\pgfsetroundcap%
\pgfsetroundjoin%
\pgfsetlinewidth{1.003750pt}%
\definecolor{currentstroke}{rgb}{1.000000,1.000000,1.000000}%
\pgfsetstrokecolor{currentstroke}%
\pgfsetdash{}{0pt}%
\pgfpathmoveto{\pgfqpoint{0.667731in}{2.682149in}}%
\pgfpathlineto{\pgfqpoint{5.410148in}{2.682149in}}%
\pgfusepath{stroke}%
\end{pgfscope}%
\begin{pgfscope}%
\definecolor{textcolor}{rgb}{0.150000,0.150000,0.150000}%
\pgfsetstrokecolor{textcolor}%
\pgfsetfillcolor{textcolor}%
\pgftext[x=0.383703in, y=2.629343in, left, base]{\color{textcolor}\sffamily\fontsize{11.000000}{13.200000}\selectfont \(\displaystyle {67}\)}%
\end{pgfscope}%
\begin{pgfscope}%
\pgfpathrectangle{\pgfqpoint{0.667731in}{0.650833in}}{\pgfqpoint{4.742417in}{2.714069in}}%
\pgfusepath{clip}%
\pgfsetroundcap%
\pgfsetroundjoin%
\pgfsetlinewidth{1.003750pt}%
\definecolor{currentstroke}{rgb}{1.000000,1.000000,1.000000}%
\pgfsetstrokecolor{currentstroke}%
\pgfsetdash{}{0pt}%
\pgfpathmoveto{\pgfqpoint{0.667731in}{3.071688in}}%
\pgfpathlineto{\pgfqpoint{5.410148in}{3.071688in}}%
\pgfusepath{stroke}%
\end{pgfscope}%
\begin{pgfscope}%
\definecolor{textcolor}{rgb}{0.150000,0.150000,0.150000}%
\pgfsetstrokecolor{textcolor}%
\pgfsetfillcolor{textcolor}%
\pgftext[x=0.383703in, y=3.018882in, left, base]{\color{textcolor}\sffamily\fontsize{11.000000}{13.200000}\selectfont \(\displaystyle {68}\)}%
\end{pgfscope}%
\begin{pgfscope}%
\definecolor{textcolor}{rgb}{0.150000,0.150000,0.150000}%
\pgfsetstrokecolor{textcolor}%
\pgfsetfillcolor{textcolor}%
\pgftext[x=0.328148in,y=2.007867in,,bottom,rotate=90.000000]{\color{textcolor}\sffamily\fontsize{12.000000}{14.400000}\selectfont Average score}%
\end{pgfscope}%
\begin{pgfscope}%
\pgfpathrectangle{\pgfqpoint{0.667731in}{0.650833in}}{\pgfqpoint{4.742417in}{2.714069in}}%
\pgfusepath{clip}%
\pgfsetbuttcap%
\pgfsetroundjoin%
\pgfsetlinewidth{2.710125pt}%
\definecolor{currentstroke}{rgb}{0.298039,0.447059,0.690196}%
\pgfsetstrokecolor{currentstroke}%
\pgfsetdash{{9.990000pt}{4.320000pt}}{0.000000pt}%
\pgfpathmoveto{\pgfqpoint{1.062932in}{3.001534in}}%
\pgfpathlineto{\pgfqpoint{1.853335in}{3.241535in}}%
\pgfpathlineto{\pgfqpoint{2.643738in}{3.001534in}}%
\pgfpathlineto{\pgfqpoint{3.434141in}{2.669226in}}%
\pgfpathlineto{\pgfqpoint{4.224544in}{2.946150in}}%
\pgfpathlineto{\pgfqpoint{5.014947in}{1.524609in}}%
\pgfusepath{stroke}%
\end{pgfscope}%
\begin{pgfscope}%
\pgfpathrectangle{\pgfqpoint{0.667731in}{0.650833in}}{\pgfqpoint{4.742417in}{2.714069in}}%
\pgfusepath{clip}%
\pgfsetroundcap%
\pgfsetroundjoin%
\pgfsetlinewidth{2.710125pt}%
\definecolor{currentstroke}{rgb}{0.298039,0.447059,0.690196}%
\pgfsetstrokecolor{currentstroke}%
\pgfsetdash{}{0pt}%
\pgfusepath{stroke}%
\end{pgfscope}%
\begin{pgfscope}%
\pgfpathrectangle{\pgfqpoint{0.667731in}{0.650833in}}{\pgfqpoint{4.742417in}{2.714069in}}%
\pgfusepath{clip}%
\pgfsetroundcap%
\pgfsetroundjoin%
\pgfsetlinewidth{2.710125pt}%
\definecolor{currentstroke}{rgb}{0.298039,0.447059,0.690196}%
\pgfsetstrokecolor{currentstroke}%
\pgfsetdash{}{0pt}%
\pgfusepath{stroke}%
\end{pgfscope}%
\begin{pgfscope}%
\pgfpathrectangle{\pgfqpoint{0.667731in}{0.650833in}}{\pgfqpoint{4.742417in}{2.714069in}}%
\pgfusepath{clip}%
\pgfsetroundcap%
\pgfsetroundjoin%
\pgfsetlinewidth{2.710125pt}%
\definecolor{currentstroke}{rgb}{0.298039,0.447059,0.690196}%
\pgfsetstrokecolor{currentstroke}%
\pgfsetdash{}{0pt}%
\pgfusepath{stroke}%
\end{pgfscope}%
\begin{pgfscope}%
\pgfpathrectangle{\pgfqpoint{0.667731in}{0.650833in}}{\pgfqpoint{4.742417in}{2.714069in}}%
\pgfusepath{clip}%
\pgfsetroundcap%
\pgfsetroundjoin%
\pgfsetlinewidth{2.710125pt}%
\definecolor{currentstroke}{rgb}{0.298039,0.447059,0.690196}%
\pgfsetstrokecolor{currentstroke}%
\pgfsetdash{}{0pt}%
\pgfusepath{stroke}%
\end{pgfscope}%
\begin{pgfscope}%
\pgfpathrectangle{\pgfqpoint{0.667731in}{0.650833in}}{\pgfqpoint{4.742417in}{2.714069in}}%
\pgfusepath{clip}%
\pgfsetroundcap%
\pgfsetroundjoin%
\pgfsetlinewidth{2.710125pt}%
\definecolor{currentstroke}{rgb}{0.298039,0.447059,0.690196}%
\pgfsetstrokecolor{currentstroke}%
\pgfsetdash{}{0pt}%
\pgfusepath{stroke}%
\end{pgfscope}%
\begin{pgfscope}%
\pgfpathrectangle{\pgfqpoint{0.667731in}{0.650833in}}{\pgfqpoint{4.742417in}{2.714069in}}%
\pgfusepath{clip}%
\pgfsetroundcap%
\pgfsetroundjoin%
\pgfsetlinewidth{2.710125pt}%
\definecolor{currentstroke}{rgb}{0.298039,0.447059,0.690196}%
\pgfsetstrokecolor{currentstroke}%
\pgfsetdash{}{0pt}%
\pgfusepath{stroke}%
\end{pgfscope}%
\begin{pgfscope}%
\pgfpathrectangle{\pgfqpoint{0.667731in}{0.650833in}}{\pgfqpoint{4.742417in}{2.714069in}}%
\pgfusepath{clip}%
\pgfsetbuttcap%
\pgfsetroundjoin%
\definecolor{currentfill}{rgb}{0.298039,0.447059,0.690196}%
\pgfsetfillcolor{currentfill}%
\pgfsetlinewidth{2.032594pt}%
\definecolor{currentstroke}{rgb}{0.298039,0.447059,0.690196}%
\pgfsetstrokecolor{currentstroke}%
\pgfsetdash{}{0pt}%
\pgfsys@defobject{currentmarker}{\pgfqpoint{-0.046999in}{-0.046999in}}{\pgfqpoint{0.046999in}{0.046999in}}{%
\pgfpathmoveto{\pgfqpoint{0.000000in}{-0.046999in}}%
\pgfpathcurveto{\pgfqpoint{0.012464in}{-0.046999in}}{\pgfqpoint{0.024420in}{-0.042047in}}{\pgfqpoint{0.033234in}{-0.033234in}}%
\pgfpathcurveto{\pgfqpoint{0.042047in}{-0.024420in}}{\pgfqpoint{0.046999in}{-0.012464in}}{\pgfqpoint{0.046999in}{0.000000in}}%
\pgfpathcurveto{\pgfqpoint{0.046999in}{0.012464in}}{\pgfqpoint{0.042047in}{0.024420in}}{\pgfqpoint{0.033234in}{0.033234in}}%
\pgfpathcurveto{\pgfqpoint{0.024420in}{0.042047in}}{\pgfqpoint{0.012464in}{0.046999in}}{\pgfqpoint{0.000000in}{0.046999in}}%
\pgfpathcurveto{\pgfqpoint{-0.012464in}{0.046999in}}{\pgfqpoint{-0.024420in}{0.042047in}}{\pgfqpoint{-0.033234in}{0.033234in}}%
\pgfpathcurveto{\pgfqpoint{-0.042047in}{0.024420in}}{\pgfqpoint{-0.046999in}{0.012464in}}{\pgfqpoint{-0.046999in}{0.000000in}}%
\pgfpathcurveto{\pgfqpoint{-0.046999in}{-0.012464in}}{\pgfqpoint{-0.042047in}{-0.024420in}}{\pgfqpoint{-0.033234in}{-0.033234in}}%
\pgfpathcurveto{\pgfqpoint{-0.024420in}{-0.042047in}}{\pgfqpoint{-0.012464in}{-0.046999in}}{\pgfqpoint{0.000000in}{-0.046999in}}%
\pgfpathlineto{\pgfqpoint{0.000000in}{-0.046999in}}%
\pgfpathclose%
\pgfusepath{stroke,fill}%
}%
\begin{pgfscope}%
\pgfsys@transformshift{1.062932in}{3.001534in}%
\pgfsys@useobject{currentmarker}{}%
\end{pgfscope}%
\begin{pgfscope}%
\pgfsys@transformshift{1.853335in}{3.241535in}%
\pgfsys@useobject{currentmarker}{}%
\end{pgfscope}%
\begin{pgfscope}%
\pgfsys@transformshift{2.643738in}{3.001534in}%
\pgfsys@useobject{currentmarker}{}%
\end{pgfscope}%
\begin{pgfscope}%
\pgfsys@transformshift{3.434141in}{2.669226in}%
\pgfsys@useobject{currentmarker}{}%
\end{pgfscope}%
\begin{pgfscope}%
\pgfsys@transformshift{4.224544in}{2.946150in}%
\pgfsys@useobject{currentmarker}{}%
\end{pgfscope}%
\begin{pgfscope}%
\pgfsys@transformshift{5.014947in}{1.524609in}%
\pgfsys@useobject{currentmarker}{}%
\end{pgfscope}%
\end{pgfscope}%
\begin{pgfscope}%
\pgfpathrectangle{\pgfqpoint{0.667731in}{0.650833in}}{\pgfqpoint{4.742417in}{2.714069in}}%
\pgfusepath{clip}%
\pgfsetbuttcap%
\pgfsetroundjoin%
\pgfsetlinewidth{2.710125pt}%
\definecolor{currentstroke}{rgb}{0.866667,0.517647,0.321569}%
\pgfsetstrokecolor{currentstroke}%
\pgfsetdash{{9.990000pt}{4.320000pt}}{0.000000pt}%
\pgfpathmoveto{\pgfqpoint{1.062932in}{1.932167in}}%
\pgfpathlineto{\pgfqpoint{1.853335in}{2.457824in}}%
\pgfpathlineto{\pgfqpoint{2.643738in}{2.258992in}}%
\pgfpathlineto{\pgfqpoint{3.434141in}{2.182700in}}%
\pgfpathlineto{\pgfqpoint{4.224544in}{2.467052in}}%
\pgfpathlineto{\pgfqpoint{5.014947in}{0.774199in}}%
\pgfusepath{stroke}%
\end{pgfscope}%
\begin{pgfscope}%
\pgfpathrectangle{\pgfqpoint{0.667731in}{0.650833in}}{\pgfqpoint{4.742417in}{2.714069in}}%
\pgfusepath{clip}%
\pgfsetroundcap%
\pgfsetroundjoin%
\pgfsetlinewidth{2.710125pt}%
\definecolor{currentstroke}{rgb}{0.866667,0.517647,0.321569}%
\pgfsetstrokecolor{currentstroke}%
\pgfsetdash{}{0pt}%
\pgfusepath{stroke}%
\end{pgfscope}%
\begin{pgfscope}%
\pgfpathrectangle{\pgfqpoint{0.667731in}{0.650833in}}{\pgfqpoint{4.742417in}{2.714069in}}%
\pgfusepath{clip}%
\pgfsetroundcap%
\pgfsetroundjoin%
\pgfsetlinewidth{2.710125pt}%
\definecolor{currentstroke}{rgb}{0.866667,0.517647,0.321569}%
\pgfsetstrokecolor{currentstroke}%
\pgfsetdash{}{0pt}%
\pgfusepath{stroke}%
\end{pgfscope}%
\begin{pgfscope}%
\pgfpathrectangle{\pgfqpoint{0.667731in}{0.650833in}}{\pgfqpoint{4.742417in}{2.714069in}}%
\pgfusepath{clip}%
\pgfsetroundcap%
\pgfsetroundjoin%
\pgfsetlinewidth{2.710125pt}%
\definecolor{currentstroke}{rgb}{0.866667,0.517647,0.321569}%
\pgfsetstrokecolor{currentstroke}%
\pgfsetdash{}{0pt}%
\pgfusepath{stroke}%
\end{pgfscope}%
\begin{pgfscope}%
\pgfpathrectangle{\pgfqpoint{0.667731in}{0.650833in}}{\pgfqpoint{4.742417in}{2.714069in}}%
\pgfusepath{clip}%
\pgfsetroundcap%
\pgfsetroundjoin%
\pgfsetlinewidth{2.710125pt}%
\definecolor{currentstroke}{rgb}{0.866667,0.517647,0.321569}%
\pgfsetstrokecolor{currentstroke}%
\pgfsetdash{}{0pt}%
\pgfusepath{stroke}%
\end{pgfscope}%
\begin{pgfscope}%
\pgfpathrectangle{\pgfqpoint{0.667731in}{0.650833in}}{\pgfqpoint{4.742417in}{2.714069in}}%
\pgfusepath{clip}%
\pgfsetroundcap%
\pgfsetroundjoin%
\pgfsetlinewidth{2.710125pt}%
\definecolor{currentstroke}{rgb}{0.866667,0.517647,0.321569}%
\pgfsetstrokecolor{currentstroke}%
\pgfsetdash{}{0pt}%
\pgfusepath{stroke}%
\end{pgfscope}%
\begin{pgfscope}%
\pgfpathrectangle{\pgfqpoint{0.667731in}{0.650833in}}{\pgfqpoint{4.742417in}{2.714069in}}%
\pgfusepath{clip}%
\pgfsetroundcap%
\pgfsetroundjoin%
\pgfsetlinewidth{2.710125pt}%
\definecolor{currentstroke}{rgb}{0.866667,0.517647,0.321569}%
\pgfsetstrokecolor{currentstroke}%
\pgfsetdash{}{0pt}%
\pgfusepath{stroke}%
\end{pgfscope}%
\begin{pgfscope}%
\pgfpathrectangle{\pgfqpoint{0.667731in}{0.650833in}}{\pgfqpoint{4.742417in}{2.714069in}}%
\pgfusepath{clip}%
\pgfsetbuttcap%
\pgfsetroundjoin%
\definecolor{currentfill}{rgb}{0.866667,0.517647,0.321569}%
\pgfsetfillcolor{currentfill}%
\pgfsetlinewidth{2.032594pt}%
\definecolor{currentstroke}{rgb}{0.866667,0.517647,0.321569}%
\pgfsetstrokecolor{currentstroke}%
\pgfsetdash{}{0pt}%
\pgfsys@defobject{currentmarker}{\pgfqpoint{-0.046999in}{-0.046999in}}{\pgfqpoint{0.046999in}{0.046999in}}{%
\pgfpathmoveto{\pgfqpoint{0.000000in}{-0.046999in}}%
\pgfpathcurveto{\pgfqpoint{0.012464in}{-0.046999in}}{\pgfqpoint{0.024420in}{-0.042047in}}{\pgfqpoint{0.033234in}{-0.033234in}}%
\pgfpathcurveto{\pgfqpoint{0.042047in}{-0.024420in}}{\pgfqpoint{0.046999in}{-0.012464in}}{\pgfqpoint{0.046999in}{0.000000in}}%
\pgfpathcurveto{\pgfqpoint{0.046999in}{0.012464in}}{\pgfqpoint{0.042047in}{0.024420in}}{\pgfqpoint{0.033234in}{0.033234in}}%
\pgfpathcurveto{\pgfqpoint{0.024420in}{0.042047in}}{\pgfqpoint{0.012464in}{0.046999in}}{\pgfqpoint{0.000000in}{0.046999in}}%
\pgfpathcurveto{\pgfqpoint{-0.012464in}{0.046999in}}{\pgfqpoint{-0.024420in}{0.042047in}}{\pgfqpoint{-0.033234in}{0.033234in}}%
\pgfpathcurveto{\pgfqpoint{-0.042047in}{0.024420in}}{\pgfqpoint{-0.046999in}{0.012464in}}{\pgfqpoint{-0.046999in}{0.000000in}}%
\pgfpathcurveto{\pgfqpoint{-0.046999in}{-0.012464in}}{\pgfqpoint{-0.042047in}{-0.024420in}}{\pgfqpoint{-0.033234in}{-0.033234in}}%
\pgfpathcurveto{\pgfqpoint{-0.024420in}{-0.042047in}}{\pgfqpoint{-0.012464in}{-0.046999in}}{\pgfqpoint{0.000000in}{-0.046999in}}%
\pgfpathlineto{\pgfqpoint{0.000000in}{-0.046999in}}%
\pgfpathclose%
\pgfusepath{stroke,fill}%
}%
\begin{pgfscope}%
\pgfsys@transformshift{1.062932in}{1.932167in}%
\pgfsys@useobject{currentmarker}{}%
\end{pgfscope}%
\begin{pgfscope}%
\pgfsys@transformshift{1.853335in}{2.457824in}%
\pgfsys@useobject{currentmarker}{}%
\end{pgfscope}%
\begin{pgfscope}%
\pgfsys@transformshift{2.643738in}{2.258992in}%
\pgfsys@useobject{currentmarker}{}%
\end{pgfscope}%
\begin{pgfscope}%
\pgfsys@transformshift{3.434141in}{2.182700in}%
\pgfsys@useobject{currentmarker}{}%
\end{pgfscope}%
\begin{pgfscope}%
\pgfsys@transformshift{4.224544in}{2.467052in}%
\pgfsys@useobject{currentmarker}{}%
\end{pgfscope}%
\begin{pgfscope}%
\pgfsys@transformshift{5.014947in}{0.774199in}%
\pgfsys@useobject{currentmarker}{}%
\end{pgfscope}%
\end{pgfscope}%
\begin{pgfscope}%
\pgfsetrectcap%
\pgfsetmiterjoin%
\pgfsetlinewidth{1.254687pt}%
\definecolor{currentstroke}{rgb}{1.000000,1.000000,1.000000}%
\pgfsetstrokecolor{currentstroke}%
\pgfsetdash{}{0pt}%
\pgfpathmoveto{\pgfqpoint{0.667731in}{0.650833in}}%
\pgfpathlineto{\pgfqpoint{0.667731in}{3.364902in}}%
\pgfusepath{stroke}%
\end{pgfscope}%
\begin{pgfscope}%
\pgfsetrectcap%
\pgfsetmiterjoin%
\pgfsetlinewidth{1.254687pt}%
\definecolor{currentstroke}{rgb}{1.000000,1.000000,1.000000}%
\pgfsetstrokecolor{currentstroke}%
\pgfsetdash{}{0pt}%
\pgfpathmoveto{\pgfqpoint{5.410148in}{0.650833in}}%
\pgfpathlineto{\pgfqpoint{5.410148in}{3.364902in}}%
\pgfusepath{stroke}%
\end{pgfscope}%
\begin{pgfscope}%
\pgfsetrectcap%
\pgfsetmiterjoin%
\pgfsetlinewidth{1.254687pt}%
\definecolor{currentstroke}{rgb}{1.000000,1.000000,1.000000}%
\pgfsetstrokecolor{currentstroke}%
\pgfsetdash{}{0pt}%
\pgfpathmoveto{\pgfqpoint{0.667731in}{0.650833in}}%
\pgfpathlineto{\pgfqpoint{5.410148in}{0.650833in}}%
\pgfusepath{stroke}%
\end{pgfscope}%
\begin{pgfscope}%
\pgfsetrectcap%
\pgfsetmiterjoin%
\pgfsetlinewidth{1.254687pt}%
\definecolor{currentstroke}{rgb}{1.000000,1.000000,1.000000}%
\pgfsetstrokecolor{currentstroke}%
\pgfsetdash{}{0pt}%
\pgfpathmoveto{\pgfqpoint{0.667731in}{3.364902in}}%
\pgfpathlineto{\pgfqpoint{5.410148in}{3.364902in}}%
\pgfusepath{stroke}%
\end{pgfscope}%
\begin{pgfscope}%
\pgfsetbuttcap%
\pgfsetmiterjoin%
\definecolor{currentfill}{rgb}{0.917647,0.917647,0.949020}%
\pgfsetfillcolor{currentfill}%
\pgfsetfillopacity{0.800000}%
\pgfsetlinewidth{1.003750pt}%
\definecolor{currentstroke}{rgb}{0.800000,0.800000,0.800000}%
\pgfsetstrokecolor{currentstroke}%
\pgfsetstrokeopacity{0.800000}%
\pgfsetdash{}{0pt}%
\pgfpathmoveto{\pgfqpoint{0.774675in}{0.727222in}}%
\pgfpathlineto{\pgfqpoint{1.866407in}{0.727222in}}%
\pgfpathquadraticcurveto{\pgfqpoint{1.896963in}{0.727222in}}{\pgfqpoint{1.896963in}{0.757777in}}%
\pgfpathlineto{\pgfqpoint{1.896963in}{1.392846in}}%
\pgfpathquadraticcurveto{\pgfqpoint{1.896963in}{1.423402in}}{\pgfqpoint{1.866407in}{1.423402in}}%
\pgfpathlineto{\pgfqpoint{0.774675in}{1.423402in}}%
\pgfpathquadraticcurveto{\pgfqpoint{0.744120in}{1.423402in}}{\pgfqpoint{0.744120in}{1.392846in}}%
\pgfpathlineto{\pgfqpoint{0.744120in}{0.757777in}}%
\pgfpathquadraticcurveto{\pgfqpoint{0.744120in}{0.727222in}}{\pgfqpoint{0.774675in}{0.727222in}}%
\pgfpathlineto{\pgfqpoint{0.774675in}{0.727222in}}%
\pgfpathclose%
\pgfusepath{stroke,fill}%
\end{pgfscope}%
\begin{pgfscope}%
\definecolor{textcolor}{rgb}{0.150000,0.150000,0.150000}%
\pgfsetstrokecolor{textcolor}%
\pgfsetfillcolor{textcolor}%
\pgftext[x=1.076929in,y=1.246550in,left,base]{\color{textcolor}\sffamily\fontsize{12.000000}{14.400000}\selectfont F-score}%
\end{pgfscope}%
\begin{pgfscope}%
\pgfsetbuttcap%
\pgfsetroundjoin%
\definecolor{currentfill}{rgb}{0.298039,0.447059,0.690196}%
\pgfsetfillcolor{currentfill}%
\pgfsetlinewidth{2.032594pt}%
\definecolor{currentstroke}{rgb}{0.298039,0.447059,0.690196}%
\pgfsetstrokecolor{currentstroke}%
\pgfsetdash{}{0pt}%
\pgfsys@defobject{currentmarker}{\pgfqpoint{-0.046999in}{-0.046999in}}{\pgfqpoint{0.046999in}{0.046999in}}{%
\pgfpathmoveto{\pgfqpoint{0.000000in}{-0.046999in}}%
\pgfpathcurveto{\pgfqpoint{0.012464in}{-0.046999in}}{\pgfqpoint{0.024420in}{-0.042047in}}{\pgfqpoint{0.033234in}{-0.033234in}}%
\pgfpathcurveto{\pgfqpoint{0.042047in}{-0.024420in}}{\pgfqpoint{0.046999in}{-0.012464in}}{\pgfqpoint{0.046999in}{0.000000in}}%
\pgfpathcurveto{\pgfqpoint{0.046999in}{0.012464in}}{\pgfqpoint{0.042047in}{0.024420in}}{\pgfqpoint{0.033234in}{0.033234in}}%
\pgfpathcurveto{\pgfqpoint{0.024420in}{0.042047in}}{\pgfqpoint{0.012464in}{0.046999in}}{\pgfqpoint{0.000000in}{0.046999in}}%
\pgfpathcurveto{\pgfqpoint{-0.012464in}{0.046999in}}{\pgfqpoint{-0.024420in}{0.042047in}}{\pgfqpoint{-0.033234in}{0.033234in}}%
\pgfpathcurveto{\pgfqpoint{-0.042047in}{0.024420in}}{\pgfqpoint{-0.046999in}{0.012464in}}{\pgfqpoint{-0.046999in}{0.000000in}}%
\pgfpathcurveto{\pgfqpoint{-0.046999in}{-0.012464in}}{\pgfqpoint{-0.042047in}{-0.024420in}}{\pgfqpoint{-0.033234in}{-0.033234in}}%
\pgfpathcurveto{\pgfqpoint{-0.024420in}{-0.042047in}}{\pgfqpoint{-0.012464in}{-0.046999in}}{\pgfqpoint{0.000000in}{-0.046999in}}%
\pgfpathlineto{\pgfqpoint{0.000000in}{-0.046999in}}%
\pgfpathclose%
\pgfusepath{stroke,fill}%
}%
\begin{pgfscope}%
\pgfsys@transformshift{0.958009in}{1.070914in}%
\pgfsys@useobject{currentmarker}{}%
\end{pgfscope}%
\end{pgfscope}%
\begin{pgfscope}%
\definecolor{textcolor}{rgb}{0.150000,0.150000,0.150000}%
\pgfsetstrokecolor{textcolor}%
\pgfsetfillcolor{textcolor}%
\pgftext[x=1.233009in,y=1.030810in,left,base]{\color{textcolor}\sffamily\fontsize{11.000000}{13.200000}\selectfont F1 micro}%
\end{pgfscope}%
\begin{pgfscope}%
\pgfsetbuttcap%
\pgfsetroundjoin%
\definecolor{currentfill}{rgb}{0.866667,0.517647,0.321569}%
\pgfsetfillcolor{currentfill}%
\pgfsetlinewidth{2.032594pt}%
\definecolor{currentstroke}{rgb}{0.866667,0.517647,0.321569}%
\pgfsetstrokecolor{currentstroke}%
\pgfsetdash{}{0pt}%
\pgfsys@defobject{currentmarker}{\pgfqpoint{-0.046999in}{-0.046999in}}{\pgfqpoint{0.046999in}{0.046999in}}{%
\pgfpathmoveto{\pgfqpoint{0.000000in}{-0.046999in}}%
\pgfpathcurveto{\pgfqpoint{0.012464in}{-0.046999in}}{\pgfqpoint{0.024420in}{-0.042047in}}{\pgfqpoint{0.033234in}{-0.033234in}}%
\pgfpathcurveto{\pgfqpoint{0.042047in}{-0.024420in}}{\pgfqpoint{0.046999in}{-0.012464in}}{\pgfqpoint{0.046999in}{0.000000in}}%
\pgfpathcurveto{\pgfqpoint{0.046999in}{0.012464in}}{\pgfqpoint{0.042047in}{0.024420in}}{\pgfqpoint{0.033234in}{0.033234in}}%
\pgfpathcurveto{\pgfqpoint{0.024420in}{0.042047in}}{\pgfqpoint{0.012464in}{0.046999in}}{\pgfqpoint{0.000000in}{0.046999in}}%
\pgfpathcurveto{\pgfqpoint{-0.012464in}{0.046999in}}{\pgfqpoint{-0.024420in}{0.042047in}}{\pgfqpoint{-0.033234in}{0.033234in}}%
\pgfpathcurveto{\pgfqpoint{-0.042047in}{0.024420in}}{\pgfqpoint{-0.046999in}{0.012464in}}{\pgfqpoint{-0.046999in}{0.000000in}}%
\pgfpathcurveto{\pgfqpoint{-0.046999in}{-0.012464in}}{\pgfqpoint{-0.042047in}{-0.024420in}}{\pgfqpoint{-0.033234in}{-0.033234in}}%
\pgfpathcurveto{\pgfqpoint{-0.024420in}{-0.042047in}}{\pgfqpoint{-0.012464in}{-0.046999in}}{\pgfqpoint{0.000000in}{-0.046999in}}%
\pgfpathlineto{\pgfqpoint{0.000000in}{-0.046999in}}%
\pgfpathclose%
\pgfusepath{stroke,fill}%
}%
\begin{pgfscope}%
\pgfsys@transformshift{0.958009in}{0.858009in}%
\pgfsys@useobject{currentmarker}{}%
\end{pgfscope}%
\end{pgfscope}%
\begin{pgfscope}%
\definecolor{textcolor}{rgb}{0.150000,0.150000,0.150000}%
\pgfsetstrokecolor{textcolor}%
\pgfsetfillcolor{textcolor}%
\pgftext[x=1.233009in,y=0.817904in,left,base]{\color{textcolor}\sffamily\fontsize{11.000000}{13.200000}\selectfont F1 macro}%
\end{pgfscope}%
\end{pgfpicture}%
\makeatother%
\endgroup%
}
        \label{fig:node2vec:number and walk length}
    \end{subfigure}
    \caption{Node2vec results.}
    \label{fig:deepwalk:plots}
\end{figure}

\section{DeepWalk and Node2vec Results Discussion}
The parameters which resulted in the best score for node2vec on our disinformation dataset are $p=2, q=1, d=128, w=1, \gamma=60,t=20 $.

for node2vec, the best window size parameter was 1, compared with 10 for DeepWalk.
This means that node2vec consider only one possible node for the next step of the random walk.
We believe this strong difference in window size between the two algorithm could be due to the additional p and q hyperparameters of node2vec.
With a value of $p=2, q=1$, node2vec prioritize exploration over revisiting known nodes.
Because our graph is highly connected, network exploration means fewer but longer random walk, as we can see from $\gamma=60$ and $t=20$ for node2vec.
The hyperparameters $\gamma=20$ and $t=40$ shows that DeepWalk prioritize network exploration through randomness, with shorter but more numerous walks.

Compared with DeepWalk, we can a see slightly better performance improved by node2vec.
DeepWalk strictly performs the network exploration randomly. 
Node2vec has a small edge here, because its walks can be influenced through $p$ and $q$ parameters.
In a large network such as the misinformation dataset we use here, this improves the graph exploration slightly.
In our best result, we prioritize exploration of unknown nodes on the network over already visited ones.
This leads to a slight but significant improvement in the algorithm's network knowledge and thus a better representation in the resulting embeddings. 

\newpage
\section{Score Comparsion}

With our two algorithms best performance, we can now compare their performance compared to FANG.
In FANG's original paper, its performance was measured with Area Under the ROC Curve, AUC for short.
Conveniently FANG also produces a F1 score when we perform it on our dataset.
We will therefore use both metrics to compare the performance of these algorithms.

We measured the performance of Deepwalk and node2vec on an embedding performed with the best parameters, discussed previsouly in chapter~\ref{cha:Experiments}.
For FANG, we enabled all hyperparameters such that it uses all available context, therefore obtain the best performance possible.

\begin{table}[h]
    \centering
    \caption{Comparison between models, evaluated with F1 score and AUC.}
    \label{tab:f1_auc}
    \begin{tabular}{lcc}
        \toprule
        Model & F1 score (micro) & AUC \\
        \midrule
        % TODO update these values with BINARY F1, not micro!
        DeepWalk & 0.6810 & 0.6493 \\
        node2vec & 0.6844 & 0.6654 \\
        % these are the final values at 80%
        FANG & 0.6268 & 0.6701 \\
        \bottomrule
    \end{tabular}
\end{table}
We can now compare these metrics of both algorithm with the one from Factual News Graph (FANG).

\iffalse
\subsubsection{Notes}

Just like Deepwalk and node2vec, FANG produces both F1 score and AUC using the sklearn library.
The AUC uses the default ($average=macro$).

However the F1 produced uses $average=binary$.
We have F1 micro and macro score for DeepWalk and node2vec.
Which one should we use, or do we need to run the program with binary F1?

%%%%% RAW SCORE %%%%%
% DEEPWALk
% F1 score Average score: {'micro': 0.6810426540284359, 'macro': 0.6675884947364201}
% AUC Average score: {'micro': 0.6668246445497629, 'macro': 0.6492556721962208}

% NODE2VEC
% F1 score Average score: {'micro': 0.6843601895734597, 'macro': 0.6642412717978502}
% AUC Average score: {'micro': 0.6909952606635071, 'macro': 0.6654164938979849}
%%%%% RAW SCORE %%%%%

% 80% of the data would mean 843 news article to train on
\fi
