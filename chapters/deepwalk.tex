% !TEX root = ../main.tex

\chapter{Experiments} % (fold)
\label{cha:Experiments}

\section{Misinformation Dataset}

FakeNewsNet is dataset made of political \textit{tweets} collected from the Twitter social media.
The entities present in this social network are heterogeneous. 
An entity can be a user, a news agency or a source.

The data contains interactions and reactions to news article posted by various news account.
An edge between two entity represents an interaction.
Interactions can be between entities of the same or different type: user to user, news to user, source to news, and so on.

Users reaction to the information presented can be positive, neutral or negative.
Each news \textit{tweet} can be either fake or real. 
The corresponding labels are available for each news.

\section{Introduction to DeepWalk}

DeepWalk allows to to learn latent representations of vertices in a social network.
It performs random walks in a graph to learn latent features.
The skipgram model is used to perform word embedding, in a similar fashion as word2vec.

\section{From Heterogeneous to Homogeneous}

DeepWalk only performs the embedding on homogeneous entities represented by integers.
Because our data is heterogeneous, we first need to assign a unique value to each entity.

Using the available entity list, each entity is given a unique identifier.
This is then mapped to the edgelists of all entities interactions.

We then build the graph using the homogeneous network. 
This simplify computing the adjacency list for each entity.

\section{DeepWalk With Disinformation Data and Parameters Sensitivity}

DeepWalk can now use the adjacency list to perform the graph embedding.
The embedding is obtained with the entire network; 
scoring is however only performed with the news sources.

Once the embedding is obtained, the users and news entities are then removed from the embedding.
The embedding is then analyzed and compared with each news label.

To obtain the best performance, the parameters of the program are tuned one after the other.
First tuning one of them with all the others set to default, and then another one using the best value obtained for the previous run.

\begin{figure}
    \centering
    \begin{subfigure}[b]{0.49\textwidth}
        \centering
        \scalebox{.5}{\input{figures/plots/deepwalk/deepwalk-plot-dimensions-window-size.pgf}}
        \label{fig:deepwalk:representation and window size}
    \end{subfigure}
    \hfill
    \begin{subfigure}[b]{0.49\textwidth}
        \centering
        \scalebox{.5}{%% Creator: Matplotlib, PGF backend
%%
%% To include the figure in your LaTeX document, write
%%   \input{<filename>.pgf}
%%
%% Make sure the required packages are loaded in your preamble
%%   \usepackage{pgf}
%%
%% Also ensure that all the required font packages are loaded; for instance,
%% the lmodern package is sometimes necessary when using math font.
%%   \usepackage{lmodern}
%%
%% Figures using additional raster images can only be included by \input if
%% they are in the same directory as the main LaTeX file. For loading figures
%% from other directories you can use the `import` package
%%   \usepackage{import}
%%
%% and then include the figures with
%%   \import{<path to file>}{<filename>.pgf}
%%
%% Matplotlib used the following preamble
%%   
%%   \makeatletter\@ifpackageloaded{underscore}{}{\usepackage[strings]{underscore}}\makeatother
%%
\begingroup%
\makeatletter%
\begin{pgfpicture}%
\pgfpathrectangle{\pgfpointorigin}{\pgfqpoint{5.590148in}{6.909803in}}%
\pgfusepath{use as bounding box, clip}%
\begin{pgfscope}%
\pgfsetbuttcap%
\pgfsetmiterjoin%
\definecolor{currentfill}{rgb}{1.000000,1.000000,1.000000}%
\pgfsetfillcolor{currentfill}%
\pgfsetlinewidth{0.000000pt}%
\definecolor{currentstroke}{rgb}{1.000000,1.000000,1.000000}%
\pgfsetstrokecolor{currentstroke}%
\pgfsetdash{}{0pt}%
\pgfpathmoveto{\pgfqpoint{0.000000in}{0.000000in}}%
\pgfpathlineto{\pgfqpoint{5.590148in}{0.000000in}}%
\pgfpathlineto{\pgfqpoint{5.590148in}{6.909803in}}%
\pgfpathlineto{\pgfqpoint{0.000000in}{6.909803in}}%
\pgfpathlineto{\pgfqpoint{0.000000in}{0.000000in}}%
\pgfpathclose%
\pgfusepath{fill}%
\end{pgfscope}%
\begin{pgfscope}%
\pgfsetbuttcap%
\pgfsetmiterjoin%
\definecolor{currentfill}{rgb}{0.917647,0.917647,0.949020}%
\pgfsetfillcolor{currentfill}%
\pgfsetlinewidth{0.000000pt}%
\definecolor{currentstroke}{rgb}{0.000000,0.000000,0.000000}%
\pgfsetstrokecolor{currentstroke}%
\pgfsetstrokeopacity{0.000000}%
\pgfsetdash{}{0pt}%
\pgfpathmoveto{\pgfqpoint{0.667731in}{4.015734in}}%
\pgfpathlineto{\pgfqpoint{5.410148in}{4.015734in}}%
\pgfpathlineto{\pgfqpoint{5.410148in}{6.729803in}}%
\pgfpathlineto{\pgfqpoint{0.667731in}{6.729803in}}%
\pgfpathlineto{\pgfqpoint{0.667731in}{4.015734in}}%
\pgfpathclose%
\pgfusepath{fill}%
\end{pgfscope}%
\begin{pgfscope}%
\definecolor{textcolor}{rgb}{0.150000,0.150000,0.150000}%
\pgfsetstrokecolor{textcolor}%
\pgfsetfillcolor{textcolor}%
\pgftext[x=1.141973in,y=3.883790in,,top]{\color{textcolor}\sffamily\fontsize{11.000000}{13.200000}\selectfont 10}%
\end{pgfscope}%
\begin{pgfscope}%
\definecolor{textcolor}{rgb}{0.150000,0.150000,0.150000}%
\pgfsetstrokecolor{textcolor}%
\pgfsetfillcolor{textcolor}%
\pgftext[x=2.090456in,y=3.883790in,,top]{\color{textcolor}\sffamily\fontsize{11.000000}{13.200000}\selectfont 20}%
\end{pgfscope}%
\begin{pgfscope}%
\definecolor{textcolor}{rgb}{0.150000,0.150000,0.150000}%
\pgfsetstrokecolor{textcolor}%
\pgfsetfillcolor{textcolor}%
\pgftext[x=3.038940in,y=3.883790in,,top]{\color{textcolor}\sffamily\fontsize{11.000000}{13.200000}\selectfont 40}%
\end{pgfscope}%
\begin{pgfscope}%
\definecolor{textcolor}{rgb}{0.150000,0.150000,0.150000}%
\pgfsetstrokecolor{textcolor}%
\pgfsetfillcolor{textcolor}%
\pgftext[x=3.987423in,y=3.883790in,,top]{\color{textcolor}\sffamily\fontsize{11.000000}{13.200000}\selectfont 60}%
\end{pgfscope}%
\begin{pgfscope}%
\definecolor{textcolor}{rgb}{0.150000,0.150000,0.150000}%
\pgfsetstrokecolor{textcolor}%
\pgfsetfillcolor{textcolor}%
\pgftext[x=4.935906in,y=3.883790in,,top]{\color{textcolor}\sffamily\fontsize{11.000000}{13.200000}\selectfont 80}%
\end{pgfscope}%
\begin{pgfscope}%
\definecolor{textcolor}{rgb}{0.150000,0.150000,0.150000}%
\pgfsetstrokecolor{textcolor}%
\pgfsetfillcolor{textcolor}%
\pgftext[x=3.038940in,y=3.693049in,,top]{\color{textcolor}\sffamily\fontsize{12.000000}{14.400000}\selectfont random walk length}%
\end{pgfscope}%
\begin{pgfscope}%
\pgfpathrectangle{\pgfqpoint{0.667731in}{4.015734in}}{\pgfqpoint{4.742417in}{2.714069in}}%
\pgfusepath{clip}%
\pgfsetroundcap%
\pgfsetroundjoin%
\pgfsetlinewidth{1.003750pt}%
\definecolor{currentstroke}{rgb}{1.000000,1.000000,1.000000}%
\pgfsetstrokecolor{currentstroke}%
\pgfsetdash{}{0pt}%
\pgfpathmoveto{\pgfqpoint{0.667731in}{4.080654in}}%
\pgfpathlineto{\pgfqpoint{5.410148in}{4.080654in}}%
\pgfusepath{stroke}%
\end{pgfscope}%
\begin{pgfscope}%
\definecolor{textcolor}{rgb}{0.150000,0.150000,0.150000}%
\pgfsetstrokecolor{textcolor}%
\pgfsetfillcolor{textcolor}%
\pgftext[x=0.383703in, y=4.027848in, left, base]{\color{textcolor}\sffamily\fontsize{11.000000}{13.200000}\selectfont \(\displaystyle {62}\)}%
\end{pgfscope}%
\begin{pgfscope}%
\pgfpathrectangle{\pgfqpoint{0.667731in}{4.015734in}}{\pgfqpoint{4.742417in}{2.714069in}}%
\pgfusepath{clip}%
\pgfsetroundcap%
\pgfsetroundjoin%
\pgfsetlinewidth{1.003750pt}%
\definecolor{currentstroke}{rgb}{1.000000,1.000000,1.000000}%
\pgfsetstrokecolor{currentstroke}%
\pgfsetdash{}{0pt}%
\pgfpathmoveto{\pgfqpoint{0.667731in}{4.429265in}}%
\pgfpathlineto{\pgfqpoint{5.410148in}{4.429265in}}%
\pgfusepath{stroke}%
\end{pgfscope}%
\begin{pgfscope}%
\definecolor{textcolor}{rgb}{0.150000,0.150000,0.150000}%
\pgfsetstrokecolor{textcolor}%
\pgfsetfillcolor{textcolor}%
\pgftext[x=0.383703in, y=4.376458in, left, base]{\color{textcolor}\sffamily\fontsize{11.000000}{13.200000}\selectfont \(\displaystyle {63}\)}%
\end{pgfscope}%
\begin{pgfscope}%
\pgfpathrectangle{\pgfqpoint{0.667731in}{4.015734in}}{\pgfqpoint{4.742417in}{2.714069in}}%
\pgfusepath{clip}%
\pgfsetroundcap%
\pgfsetroundjoin%
\pgfsetlinewidth{1.003750pt}%
\definecolor{currentstroke}{rgb}{1.000000,1.000000,1.000000}%
\pgfsetstrokecolor{currentstroke}%
\pgfsetdash{}{0pt}%
\pgfpathmoveto{\pgfqpoint{0.667731in}{4.777875in}}%
\pgfpathlineto{\pgfqpoint{5.410148in}{4.777875in}}%
\pgfusepath{stroke}%
\end{pgfscope}%
\begin{pgfscope}%
\definecolor{textcolor}{rgb}{0.150000,0.150000,0.150000}%
\pgfsetstrokecolor{textcolor}%
\pgfsetfillcolor{textcolor}%
\pgftext[x=0.383703in, y=4.725069in, left, base]{\color{textcolor}\sffamily\fontsize{11.000000}{13.200000}\selectfont \(\displaystyle {64}\)}%
\end{pgfscope}%
\begin{pgfscope}%
\pgfpathrectangle{\pgfqpoint{0.667731in}{4.015734in}}{\pgfqpoint{4.742417in}{2.714069in}}%
\pgfusepath{clip}%
\pgfsetroundcap%
\pgfsetroundjoin%
\pgfsetlinewidth{1.003750pt}%
\definecolor{currentstroke}{rgb}{1.000000,1.000000,1.000000}%
\pgfsetstrokecolor{currentstroke}%
\pgfsetdash{}{0pt}%
\pgfpathmoveto{\pgfqpoint{0.667731in}{5.126486in}}%
\pgfpathlineto{\pgfqpoint{5.410148in}{5.126486in}}%
\pgfusepath{stroke}%
\end{pgfscope}%
\begin{pgfscope}%
\definecolor{textcolor}{rgb}{0.150000,0.150000,0.150000}%
\pgfsetstrokecolor{textcolor}%
\pgfsetfillcolor{textcolor}%
\pgftext[x=0.383703in, y=5.073679in, left, base]{\color{textcolor}\sffamily\fontsize{11.000000}{13.200000}\selectfont \(\displaystyle {65}\)}%
\end{pgfscope}%
\begin{pgfscope}%
\pgfpathrectangle{\pgfqpoint{0.667731in}{4.015734in}}{\pgfqpoint{4.742417in}{2.714069in}}%
\pgfusepath{clip}%
\pgfsetroundcap%
\pgfsetroundjoin%
\pgfsetlinewidth{1.003750pt}%
\definecolor{currentstroke}{rgb}{1.000000,1.000000,1.000000}%
\pgfsetstrokecolor{currentstroke}%
\pgfsetdash{}{0pt}%
\pgfpathmoveto{\pgfqpoint{0.667731in}{5.475096in}}%
\pgfpathlineto{\pgfqpoint{5.410148in}{5.475096in}}%
\pgfusepath{stroke}%
\end{pgfscope}%
\begin{pgfscope}%
\definecolor{textcolor}{rgb}{0.150000,0.150000,0.150000}%
\pgfsetstrokecolor{textcolor}%
\pgfsetfillcolor{textcolor}%
\pgftext[x=0.383703in, y=5.422290in, left, base]{\color{textcolor}\sffamily\fontsize{11.000000}{13.200000}\selectfont \(\displaystyle {66}\)}%
\end{pgfscope}%
\begin{pgfscope}%
\pgfpathrectangle{\pgfqpoint{0.667731in}{4.015734in}}{\pgfqpoint{4.742417in}{2.714069in}}%
\pgfusepath{clip}%
\pgfsetroundcap%
\pgfsetroundjoin%
\pgfsetlinewidth{1.003750pt}%
\definecolor{currentstroke}{rgb}{1.000000,1.000000,1.000000}%
\pgfsetstrokecolor{currentstroke}%
\pgfsetdash{}{0pt}%
\pgfpathmoveto{\pgfqpoint{0.667731in}{5.823707in}}%
\pgfpathlineto{\pgfqpoint{5.410148in}{5.823707in}}%
\pgfusepath{stroke}%
\end{pgfscope}%
\begin{pgfscope}%
\definecolor{textcolor}{rgb}{0.150000,0.150000,0.150000}%
\pgfsetstrokecolor{textcolor}%
\pgfsetfillcolor{textcolor}%
\pgftext[x=0.383703in, y=5.770900in, left, base]{\color{textcolor}\sffamily\fontsize{11.000000}{13.200000}\selectfont \(\displaystyle {67}\)}%
\end{pgfscope}%
\begin{pgfscope}%
\pgfpathrectangle{\pgfqpoint{0.667731in}{4.015734in}}{\pgfqpoint{4.742417in}{2.714069in}}%
\pgfusepath{clip}%
\pgfsetroundcap%
\pgfsetroundjoin%
\pgfsetlinewidth{1.003750pt}%
\definecolor{currentstroke}{rgb}{1.000000,1.000000,1.000000}%
\pgfsetstrokecolor{currentstroke}%
\pgfsetdash{}{0pt}%
\pgfpathmoveto{\pgfqpoint{0.667731in}{6.172317in}}%
\pgfpathlineto{\pgfqpoint{5.410148in}{6.172317in}}%
\pgfusepath{stroke}%
\end{pgfscope}%
\begin{pgfscope}%
\definecolor{textcolor}{rgb}{0.150000,0.150000,0.150000}%
\pgfsetstrokecolor{textcolor}%
\pgfsetfillcolor{textcolor}%
\pgftext[x=0.383703in, y=6.119511in, left, base]{\color{textcolor}\sffamily\fontsize{11.000000}{13.200000}\selectfont \(\displaystyle {68}\)}%
\end{pgfscope}%
\begin{pgfscope}%
\pgfpathrectangle{\pgfqpoint{0.667731in}{4.015734in}}{\pgfqpoint{4.742417in}{2.714069in}}%
\pgfusepath{clip}%
\pgfsetroundcap%
\pgfsetroundjoin%
\pgfsetlinewidth{1.003750pt}%
\definecolor{currentstroke}{rgb}{1.000000,1.000000,1.000000}%
\pgfsetstrokecolor{currentstroke}%
\pgfsetdash{}{0pt}%
\pgfpathmoveto{\pgfqpoint{0.667731in}{6.520928in}}%
\pgfpathlineto{\pgfqpoint{5.410148in}{6.520928in}}%
\pgfusepath{stroke}%
\end{pgfscope}%
\begin{pgfscope}%
\definecolor{textcolor}{rgb}{0.150000,0.150000,0.150000}%
\pgfsetstrokecolor{textcolor}%
\pgfsetfillcolor{textcolor}%
\pgftext[x=0.383703in, y=6.468121in, left, base]{\color{textcolor}\sffamily\fontsize{11.000000}{13.200000}\selectfont \(\displaystyle {69}\)}%
\end{pgfscope}%
\begin{pgfscope}%
\definecolor{textcolor}{rgb}{0.150000,0.150000,0.150000}%
\pgfsetstrokecolor{textcolor}%
\pgfsetfillcolor{textcolor}%
\pgftext[x=0.328148in,y=5.372769in,,bottom,rotate=90.000000]{\color{textcolor}\sffamily\fontsize{12.000000}{14.400000}\selectfont Average score}%
\end{pgfscope}%
\begin{pgfscope}%
\pgfpathrectangle{\pgfqpoint{0.667731in}{4.015734in}}{\pgfqpoint{4.742417in}{2.714069in}}%
\pgfusepath{clip}%
\pgfsetbuttcap%
\pgfsetroundjoin%
\pgfsetlinewidth{2.710125pt}%
\definecolor{currentstroke}{rgb}{0.298039,0.447059,0.690196}%
\pgfsetstrokecolor{currentstroke}%
\pgfsetdash{{9.990000pt}{4.320000pt}}{0.000000pt}%
\pgfpathmoveto{\pgfqpoint{1.141973in}{5.981568in}}%
\pgfpathlineto{\pgfqpoint{2.090456in}{6.606436in}}%
\pgfpathlineto{\pgfqpoint{3.038940in}{5.455364in}}%
\pgfpathlineto{\pgfqpoint{3.987423in}{5.027823in}}%
\pgfpathlineto{\pgfqpoint{4.935906in}{4.962047in}}%
\pgfusepath{stroke}%
\end{pgfscope}%
\begin{pgfscope}%
\pgfpathrectangle{\pgfqpoint{0.667731in}{4.015734in}}{\pgfqpoint{4.742417in}{2.714069in}}%
\pgfusepath{clip}%
\pgfsetroundcap%
\pgfsetroundjoin%
\pgfsetlinewidth{2.710125pt}%
\definecolor{currentstroke}{rgb}{0.298039,0.447059,0.690196}%
\pgfsetstrokecolor{currentstroke}%
\pgfsetdash{}{0pt}%
\pgfusepath{stroke}%
\end{pgfscope}%
\begin{pgfscope}%
\pgfpathrectangle{\pgfqpoint{0.667731in}{4.015734in}}{\pgfqpoint{4.742417in}{2.714069in}}%
\pgfusepath{clip}%
\pgfsetroundcap%
\pgfsetroundjoin%
\pgfsetlinewidth{2.710125pt}%
\definecolor{currentstroke}{rgb}{0.298039,0.447059,0.690196}%
\pgfsetstrokecolor{currentstroke}%
\pgfsetdash{}{0pt}%
\pgfusepath{stroke}%
\end{pgfscope}%
\begin{pgfscope}%
\pgfpathrectangle{\pgfqpoint{0.667731in}{4.015734in}}{\pgfqpoint{4.742417in}{2.714069in}}%
\pgfusepath{clip}%
\pgfsetroundcap%
\pgfsetroundjoin%
\pgfsetlinewidth{2.710125pt}%
\definecolor{currentstroke}{rgb}{0.298039,0.447059,0.690196}%
\pgfsetstrokecolor{currentstroke}%
\pgfsetdash{}{0pt}%
\pgfusepath{stroke}%
\end{pgfscope}%
\begin{pgfscope}%
\pgfpathrectangle{\pgfqpoint{0.667731in}{4.015734in}}{\pgfqpoint{4.742417in}{2.714069in}}%
\pgfusepath{clip}%
\pgfsetroundcap%
\pgfsetroundjoin%
\pgfsetlinewidth{2.710125pt}%
\definecolor{currentstroke}{rgb}{0.298039,0.447059,0.690196}%
\pgfsetstrokecolor{currentstroke}%
\pgfsetdash{}{0pt}%
\pgfusepath{stroke}%
\end{pgfscope}%
\begin{pgfscope}%
\pgfpathrectangle{\pgfqpoint{0.667731in}{4.015734in}}{\pgfqpoint{4.742417in}{2.714069in}}%
\pgfusepath{clip}%
\pgfsetroundcap%
\pgfsetroundjoin%
\pgfsetlinewidth{2.710125pt}%
\definecolor{currentstroke}{rgb}{0.298039,0.447059,0.690196}%
\pgfsetstrokecolor{currentstroke}%
\pgfsetdash{}{0pt}%
\pgfusepath{stroke}%
\end{pgfscope}%
\begin{pgfscope}%
\pgfpathrectangle{\pgfqpoint{0.667731in}{4.015734in}}{\pgfqpoint{4.742417in}{2.714069in}}%
\pgfusepath{clip}%
\pgfsetbuttcap%
\pgfsetroundjoin%
\definecolor{currentfill}{rgb}{0.298039,0.447059,0.690196}%
\pgfsetfillcolor{currentfill}%
\pgfsetlinewidth{2.032594pt}%
\definecolor{currentstroke}{rgb}{0.298039,0.447059,0.690196}%
\pgfsetstrokecolor{currentstroke}%
\pgfsetdash{}{0pt}%
\pgfsys@defobject{currentmarker}{\pgfqpoint{-0.046999in}{-0.046999in}}{\pgfqpoint{0.046999in}{0.046999in}}{%
\pgfpathmoveto{\pgfqpoint{0.000000in}{-0.046999in}}%
\pgfpathcurveto{\pgfqpoint{0.012464in}{-0.046999in}}{\pgfqpoint{0.024420in}{-0.042047in}}{\pgfqpoint{0.033234in}{-0.033234in}}%
\pgfpathcurveto{\pgfqpoint{0.042047in}{-0.024420in}}{\pgfqpoint{0.046999in}{-0.012464in}}{\pgfqpoint{0.046999in}{0.000000in}}%
\pgfpathcurveto{\pgfqpoint{0.046999in}{0.012464in}}{\pgfqpoint{0.042047in}{0.024420in}}{\pgfqpoint{0.033234in}{0.033234in}}%
\pgfpathcurveto{\pgfqpoint{0.024420in}{0.042047in}}{\pgfqpoint{0.012464in}{0.046999in}}{\pgfqpoint{0.000000in}{0.046999in}}%
\pgfpathcurveto{\pgfqpoint{-0.012464in}{0.046999in}}{\pgfqpoint{-0.024420in}{0.042047in}}{\pgfqpoint{-0.033234in}{0.033234in}}%
\pgfpathcurveto{\pgfqpoint{-0.042047in}{0.024420in}}{\pgfqpoint{-0.046999in}{0.012464in}}{\pgfqpoint{-0.046999in}{0.000000in}}%
\pgfpathcurveto{\pgfqpoint{-0.046999in}{-0.012464in}}{\pgfqpoint{-0.042047in}{-0.024420in}}{\pgfqpoint{-0.033234in}{-0.033234in}}%
\pgfpathcurveto{\pgfqpoint{-0.024420in}{-0.042047in}}{\pgfqpoint{-0.012464in}{-0.046999in}}{\pgfqpoint{0.000000in}{-0.046999in}}%
\pgfpathlineto{\pgfqpoint{0.000000in}{-0.046999in}}%
\pgfpathclose%
\pgfusepath{stroke,fill}%
}%
\begin{pgfscope}%
\pgfsys@transformshift{1.141973in}{5.981568in}%
\pgfsys@useobject{currentmarker}{}%
\end{pgfscope}%
\begin{pgfscope}%
\pgfsys@transformshift{2.090456in}{6.606436in}%
\pgfsys@useobject{currentmarker}{}%
\end{pgfscope}%
\begin{pgfscope}%
\pgfsys@transformshift{3.038940in}{5.455364in}%
\pgfsys@useobject{currentmarker}{}%
\end{pgfscope}%
\begin{pgfscope}%
\pgfsys@transformshift{3.987423in}{5.027823in}%
\pgfsys@useobject{currentmarker}{}%
\end{pgfscope}%
\begin{pgfscope}%
\pgfsys@transformshift{4.935906in}{4.962047in}%
\pgfsys@useobject{currentmarker}{}%
\end{pgfscope}%
\end{pgfscope}%
\begin{pgfscope}%
\pgfpathrectangle{\pgfqpoint{0.667731in}{4.015734in}}{\pgfqpoint{4.742417in}{2.714069in}}%
\pgfusepath{clip}%
\pgfsetbuttcap%
\pgfsetroundjoin%
\pgfsetlinewidth{2.710125pt}%
\definecolor{currentstroke}{rgb}{0.866667,0.517647,0.321569}%
\pgfsetstrokecolor{currentstroke}%
\pgfsetdash{{9.990000pt}{4.320000pt}}{0.000000pt}%
\pgfpathmoveto{\pgfqpoint{1.141973in}{5.134986in}}%
\pgfpathlineto{\pgfqpoint{2.090456in}{5.929632in}}%
\pgfpathlineto{\pgfqpoint{3.038940in}{4.981911in}}%
\pgfpathlineto{\pgfqpoint{3.987423in}{4.592100in}}%
\pgfpathlineto{\pgfqpoint{4.935906in}{4.139101in}}%
\pgfusepath{stroke}%
\end{pgfscope}%
\begin{pgfscope}%
\pgfpathrectangle{\pgfqpoint{0.667731in}{4.015734in}}{\pgfqpoint{4.742417in}{2.714069in}}%
\pgfusepath{clip}%
\pgfsetroundcap%
\pgfsetroundjoin%
\pgfsetlinewidth{2.710125pt}%
\definecolor{currentstroke}{rgb}{0.866667,0.517647,0.321569}%
\pgfsetstrokecolor{currentstroke}%
\pgfsetdash{}{0pt}%
\pgfusepath{stroke}%
\end{pgfscope}%
\begin{pgfscope}%
\pgfpathrectangle{\pgfqpoint{0.667731in}{4.015734in}}{\pgfqpoint{4.742417in}{2.714069in}}%
\pgfusepath{clip}%
\pgfsetroundcap%
\pgfsetroundjoin%
\pgfsetlinewidth{2.710125pt}%
\definecolor{currentstroke}{rgb}{0.866667,0.517647,0.321569}%
\pgfsetstrokecolor{currentstroke}%
\pgfsetdash{}{0pt}%
\pgfusepath{stroke}%
\end{pgfscope}%
\begin{pgfscope}%
\pgfpathrectangle{\pgfqpoint{0.667731in}{4.015734in}}{\pgfqpoint{4.742417in}{2.714069in}}%
\pgfusepath{clip}%
\pgfsetroundcap%
\pgfsetroundjoin%
\pgfsetlinewidth{2.710125pt}%
\definecolor{currentstroke}{rgb}{0.866667,0.517647,0.321569}%
\pgfsetstrokecolor{currentstroke}%
\pgfsetdash{}{0pt}%
\pgfusepath{stroke}%
\end{pgfscope}%
\begin{pgfscope}%
\pgfpathrectangle{\pgfqpoint{0.667731in}{4.015734in}}{\pgfqpoint{4.742417in}{2.714069in}}%
\pgfusepath{clip}%
\pgfsetroundcap%
\pgfsetroundjoin%
\pgfsetlinewidth{2.710125pt}%
\definecolor{currentstroke}{rgb}{0.866667,0.517647,0.321569}%
\pgfsetstrokecolor{currentstroke}%
\pgfsetdash{}{0pt}%
\pgfusepath{stroke}%
\end{pgfscope}%
\begin{pgfscope}%
\pgfpathrectangle{\pgfqpoint{0.667731in}{4.015734in}}{\pgfqpoint{4.742417in}{2.714069in}}%
\pgfusepath{clip}%
\pgfsetroundcap%
\pgfsetroundjoin%
\pgfsetlinewidth{2.710125pt}%
\definecolor{currentstroke}{rgb}{0.866667,0.517647,0.321569}%
\pgfsetstrokecolor{currentstroke}%
\pgfsetdash{}{0pt}%
\pgfusepath{stroke}%
\end{pgfscope}%
\begin{pgfscope}%
\pgfpathrectangle{\pgfqpoint{0.667731in}{4.015734in}}{\pgfqpoint{4.742417in}{2.714069in}}%
\pgfusepath{clip}%
\pgfsetbuttcap%
\pgfsetroundjoin%
\definecolor{currentfill}{rgb}{0.866667,0.517647,0.321569}%
\pgfsetfillcolor{currentfill}%
\pgfsetlinewidth{2.032594pt}%
\definecolor{currentstroke}{rgb}{0.866667,0.517647,0.321569}%
\pgfsetstrokecolor{currentstroke}%
\pgfsetdash{}{0pt}%
\pgfsys@defobject{currentmarker}{\pgfqpoint{-0.046999in}{-0.046999in}}{\pgfqpoint{0.046999in}{0.046999in}}{%
\pgfpathmoveto{\pgfqpoint{0.000000in}{-0.046999in}}%
\pgfpathcurveto{\pgfqpoint{0.012464in}{-0.046999in}}{\pgfqpoint{0.024420in}{-0.042047in}}{\pgfqpoint{0.033234in}{-0.033234in}}%
\pgfpathcurveto{\pgfqpoint{0.042047in}{-0.024420in}}{\pgfqpoint{0.046999in}{-0.012464in}}{\pgfqpoint{0.046999in}{0.000000in}}%
\pgfpathcurveto{\pgfqpoint{0.046999in}{0.012464in}}{\pgfqpoint{0.042047in}{0.024420in}}{\pgfqpoint{0.033234in}{0.033234in}}%
\pgfpathcurveto{\pgfqpoint{0.024420in}{0.042047in}}{\pgfqpoint{0.012464in}{0.046999in}}{\pgfqpoint{0.000000in}{0.046999in}}%
\pgfpathcurveto{\pgfqpoint{-0.012464in}{0.046999in}}{\pgfqpoint{-0.024420in}{0.042047in}}{\pgfqpoint{-0.033234in}{0.033234in}}%
\pgfpathcurveto{\pgfqpoint{-0.042047in}{0.024420in}}{\pgfqpoint{-0.046999in}{0.012464in}}{\pgfqpoint{-0.046999in}{0.000000in}}%
\pgfpathcurveto{\pgfqpoint{-0.046999in}{-0.012464in}}{\pgfqpoint{-0.042047in}{-0.024420in}}{\pgfqpoint{-0.033234in}{-0.033234in}}%
\pgfpathcurveto{\pgfqpoint{-0.024420in}{-0.042047in}}{\pgfqpoint{-0.012464in}{-0.046999in}}{\pgfqpoint{0.000000in}{-0.046999in}}%
\pgfpathlineto{\pgfqpoint{0.000000in}{-0.046999in}}%
\pgfpathclose%
\pgfusepath{stroke,fill}%
}%
\begin{pgfscope}%
\pgfsys@transformshift{1.141973in}{5.134986in}%
\pgfsys@useobject{currentmarker}{}%
\end{pgfscope}%
\begin{pgfscope}%
\pgfsys@transformshift{2.090456in}{5.929632in}%
\pgfsys@useobject{currentmarker}{}%
\end{pgfscope}%
\begin{pgfscope}%
\pgfsys@transformshift{3.038940in}{4.981911in}%
\pgfsys@useobject{currentmarker}{}%
\end{pgfscope}%
\begin{pgfscope}%
\pgfsys@transformshift{3.987423in}{4.592100in}%
\pgfsys@useobject{currentmarker}{}%
\end{pgfscope}%
\begin{pgfscope}%
\pgfsys@transformshift{4.935906in}{4.139101in}%
\pgfsys@useobject{currentmarker}{}%
\end{pgfscope}%
\end{pgfscope}%
\begin{pgfscope}%
\pgfsetrectcap%
\pgfsetmiterjoin%
\pgfsetlinewidth{1.254687pt}%
\definecolor{currentstroke}{rgb}{1.000000,1.000000,1.000000}%
\pgfsetstrokecolor{currentstroke}%
\pgfsetdash{}{0pt}%
\pgfpathmoveto{\pgfqpoint{0.667731in}{4.015734in}}%
\pgfpathlineto{\pgfqpoint{0.667731in}{6.729803in}}%
\pgfusepath{stroke}%
\end{pgfscope}%
\begin{pgfscope}%
\pgfsetrectcap%
\pgfsetmiterjoin%
\pgfsetlinewidth{1.254687pt}%
\definecolor{currentstroke}{rgb}{1.000000,1.000000,1.000000}%
\pgfsetstrokecolor{currentstroke}%
\pgfsetdash{}{0pt}%
\pgfpathmoveto{\pgfqpoint{5.410148in}{4.015734in}}%
\pgfpathlineto{\pgfqpoint{5.410148in}{6.729803in}}%
\pgfusepath{stroke}%
\end{pgfscope}%
\begin{pgfscope}%
\pgfsetrectcap%
\pgfsetmiterjoin%
\pgfsetlinewidth{1.254687pt}%
\definecolor{currentstroke}{rgb}{1.000000,1.000000,1.000000}%
\pgfsetstrokecolor{currentstroke}%
\pgfsetdash{}{0pt}%
\pgfpathmoveto{\pgfqpoint{0.667731in}{4.015734in}}%
\pgfpathlineto{\pgfqpoint{5.410148in}{4.015734in}}%
\pgfusepath{stroke}%
\end{pgfscope}%
\begin{pgfscope}%
\pgfsetrectcap%
\pgfsetmiterjoin%
\pgfsetlinewidth{1.254687pt}%
\definecolor{currentstroke}{rgb}{1.000000,1.000000,1.000000}%
\pgfsetstrokecolor{currentstroke}%
\pgfsetdash{}{0pt}%
\pgfpathmoveto{\pgfqpoint{0.667731in}{6.729803in}}%
\pgfpathlineto{\pgfqpoint{5.410148in}{6.729803in}}%
\pgfusepath{stroke}%
\end{pgfscope}%
\begin{pgfscope}%
\pgfsetbuttcap%
\pgfsetmiterjoin%
\definecolor{currentfill}{rgb}{0.917647,0.917647,0.949020}%
\pgfsetfillcolor{currentfill}%
\pgfsetfillopacity{0.800000}%
\pgfsetlinewidth{1.003750pt}%
\definecolor{currentstroke}{rgb}{0.800000,0.800000,0.800000}%
\pgfsetstrokecolor{currentstroke}%
\pgfsetstrokeopacity{0.800000}%
\pgfsetdash{}{0pt}%
\pgfpathmoveto{\pgfqpoint{4.211472in}{5.957234in}}%
\pgfpathlineto{\pgfqpoint{5.303204in}{5.957234in}}%
\pgfpathquadraticcurveto{\pgfqpoint{5.333759in}{5.957234in}}{\pgfqpoint{5.333759in}{5.987790in}}%
\pgfpathlineto{\pgfqpoint{5.333759in}{6.622859in}}%
\pgfpathquadraticcurveto{\pgfqpoint{5.333759in}{6.653414in}}{\pgfqpoint{5.303204in}{6.653414in}}%
\pgfpathlineto{\pgfqpoint{4.211472in}{6.653414in}}%
\pgfpathquadraticcurveto{\pgfqpoint{4.180916in}{6.653414in}}{\pgfqpoint{4.180916in}{6.622859in}}%
\pgfpathlineto{\pgfqpoint{4.180916in}{5.987790in}}%
\pgfpathquadraticcurveto{\pgfqpoint{4.180916in}{5.957234in}}{\pgfqpoint{4.211472in}{5.957234in}}%
\pgfpathlineto{\pgfqpoint{4.211472in}{5.957234in}}%
\pgfpathclose%
\pgfusepath{stroke,fill}%
\end{pgfscope}%
\begin{pgfscope}%
\definecolor{textcolor}{rgb}{0.150000,0.150000,0.150000}%
\pgfsetstrokecolor{textcolor}%
\pgfsetfillcolor{textcolor}%
\pgftext[x=4.513726in,y=6.476563in,left,base]{\color{textcolor}\sffamily\fontsize{12.000000}{14.400000}\selectfont F-score}%
\end{pgfscope}%
\begin{pgfscope}%
\pgfsetbuttcap%
\pgfsetroundjoin%
\definecolor{currentfill}{rgb}{0.298039,0.447059,0.690196}%
\pgfsetfillcolor{currentfill}%
\pgfsetlinewidth{2.032594pt}%
\definecolor{currentstroke}{rgb}{0.298039,0.447059,0.690196}%
\pgfsetstrokecolor{currentstroke}%
\pgfsetdash{}{0pt}%
\pgfsys@defobject{currentmarker}{\pgfqpoint{-0.046999in}{-0.046999in}}{\pgfqpoint{0.046999in}{0.046999in}}{%
\pgfpathmoveto{\pgfqpoint{0.000000in}{-0.046999in}}%
\pgfpathcurveto{\pgfqpoint{0.012464in}{-0.046999in}}{\pgfqpoint{0.024420in}{-0.042047in}}{\pgfqpoint{0.033234in}{-0.033234in}}%
\pgfpathcurveto{\pgfqpoint{0.042047in}{-0.024420in}}{\pgfqpoint{0.046999in}{-0.012464in}}{\pgfqpoint{0.046999in}{0.000000in}}%
\pgfpathcurveto{\pgfqpoint{0.046999in}{0.012464in}}{\pgfqpoint{0.042047in}{0.024420in}}{\pgfqpoint{0.033234in}{0.033234in}}%
\pgfpathcurveto{\pgfqpoint{0.024420in}{0.042047in}}{\pgfqpoint{0.012464in}{0.046999in}}{\pgfqpoint{0.000000in}{0.046999in}}%
\pgfpathcurveto{\pgfqpoint{-0.012464in}{0.046999in}}{\pgfqpoint{-0.024420in}{0.042047in}}{\pgfqpoint{-0.033234in}{0.033234in}}%
\pgfpathcurveto{\pgfqpoint{-0.042047in}{0.024420in}}{\pgfqpoint{-0.046999in}{0.012464in}}{\pgfqpoint{-0.046999in}{0.000000in}}%
\pgfpathcurveto{\pgfqpoint{-0.046999in}{-0.012464in}}{\pgfqpoint{-0.042047in}{-0.024420in}}{\pgfqpoint{-0.033234in}{-0.033234in}}%
\pgfpathcurveto{\pgfqpoint{-0.024420in}{-0.042047in}}{\pgfqpoint{-0.012464in}{-0.046999in}}{\pgfqpoint{0.000000in}{-0.046999in}}%
\pgfpathlineto{\pgfqpoint{0.000000in}{-0.046999in}}%
\pgfpathclose%
\pgfusepath{stroke,fill}%
}%
\begin{pgfscope}%
\pgfsys@transformshift{4.394805in}{6.300926in}%
\pgfsys@useobject{currentmarker}{}%
\end{pgfscope}%
\end{pgfscope}%
\begin{pgfscope}%
\definecolor{textcolor}{rgb}{0.150000,0.150000,0.150000}%
\pgfsetstrokecolor{textcolor}%
\pgfsetfillcolor{textcolor}%
\pgftext[x=4.669805in,y=6.260822in,left,base]{\color{textcolor}\sffamily\fontsize{11.000000}{13.200000}\selectfont F1 micro}%
\end{pgfscope}%
\begin{pgfscope}%
\pgfsetbuttcap%
\pgfsetroundjoin%
\definecolor{currentfill}{rgb}{0.866667,0.517647,0.321569}%
\pgfsetfillcolor{currentfill}%
\pgfsetlinewidth{2.032594pt}%
\definecolor{currentstroke}{rgb}{0.866667,0.517647,0.321569}%
\pgfsetstrokecolor{currentstroke}%
\pgfsetdash{}{0pt}%
\pgfsys@defobject{currentmarker}{\pgfqpoint{-0.046999in}{-0.046999in}}{\pgfqpoint{0.046999in}{0.046999in}}{%
\pgfpathmoveto{\pgfqpoint{0.000000in}{-0.046999in}}%
\pgfpathcurveto{\pgfqpoint{0.012464in}{-0.046999in}}{\pgfqpoint{0.024420in}{-0.042047in}}{\pgfqpoint{0.033234in}{-0.033234in}}%
\pgfpathcurveto{\pgfqpoint{0.042047in}{-0.024420in}}{\pgfqpoint{0.046999in}{-0.012464in}}{\pgfqpoint{0.046999in}{0.000000in}}%
\pgfpathcurveto{\pgfqpoint{0.046999in}{0.012464in}}{\pgfqpoint{0.042047in}{0.024420in}}{\pgfqpoint{0.033234in}{0.033234in}}%
\pgfpathcurveto{\pgfqpoint{0.024420in}{0.042047in}}{\pgfqpoint{0.012464in}{0.046999in}}{\pgfqpoint{0.000000in}{0.046999in}}%
\pgfpathcurveto{\pgfqpoint{-0.012464in}{0.046999in}}{\pgfqpoint{-0.024420in}{0.042047in}}{\pgfqpoint{-0.033234in}{0.033234in}}%
\pgfpathcurveto{\pgfqpoint{-0.042047in}{0.024420in}}{\pgfqpoint{-0.046999in}{0.012464in}}{\pgfqpoint{-0.046999in}{0.000000in}}%
\pgfpathcurveto{\pgfqpoint{-0.046999in}{-0.012464in}}{\pgfqpoint{-0.042047in}{-0.024420in}}{\pgfqpoint{-0.033234in}{-0.033234in}}%
\pgfpathcurveto{\pgfqpoint{-0.024420in}{-0.042047in}}{\pgfqpoint{-0.012464in}{-0.046999in}}{\pgfqpoint{0.000000in}{-0.046999in}}%
\pgfpathlineto{\pgfqpoint{0.000000in}{-0.046999in}}%
\pgfpathclose%
\pgfusepath{stroke,fill}%
}%
\begin{pgfscope}%
\pgfsys@transformshift{4.394805in}{6.088021in}%
\pgfsys@useobject{currentmarker}{}%
\end{pgfscope}%
\end{pgfscope}%
\begin{pgfscope}%
\definecolor{textcolor}{rgb}{0.150000,0.150000,0.150000}%
\pgfsetstrokecolor{textcolor}%
\pgfsetfillcolor{textcolor}%
\pgftext[x=4.669805in,y=6.047917in,left,base]{\color{textcolor}\sffamily\fontsize{11.000000}{13.200000}\selectfont F1 macro}%
\end{pgfscope}%
\begin{pgfscope}%
\pgfsetbuttcap%
\pgfsetmiterjoin%
\definecolor{currentfill}{rgb}{0.917647,0.917647,0.949020}%
\pgfsetfillcolor{currentfill}%
\pgfsetlinewidth{0.000000pt}%
\definecolor{currentstroke}{rgb}{0.000000,0.000000,0.000000}%
\pgfsetstrokecolor{currentstroke}%
\pgfsetstrokeopacity{0.000000}%
\pgfsetdash{}{0pt}%
\pgfpathmoveto{\pgfqpoint{0.667731in}{0.650833in}}%
\pgfpathlineto{\pgfqpoint{5.410148in}{0.650833in}}%
\pgfpathlineto{\pgfqpoint{5.410148in}{3.364902in}}%
\pgfpathlineto{\pgfqpoint{0.667731in}{3.364902in}}%
\pgfpathlineto{\pgfqpoint{0.667731in}{0.650833in}}%
\pgfpathclose%
\pgfusepath{fill}%
\end{pgfscope}%
\begin{pgfscope}%
\definecolor{textcolor}{rgb}{0.150000,0.150000,0.150000}%
\pgfsetstrokecolor{textcolor}%
\pgfsetfillcolor{textcolor}%
\pgftext[x=0.964132in,y=0.518888in,,top]{\color{textcolor}\sffamily\fontsize{11.000000}{13.200000}\selectfont 5}%
\end{pgfscope}%
\begin{pgfscope}%
\definecolor{textcolor}{rgb}{0.150000,0.150000,0.150000}%
\pgfsetstrokecolor{textcolor}%
\pgfsetfillcolor{textcolor}%
\pgftext[x=1.556934in,y=0.518888in,,top]{\color{textcolor}\sffamily\fontsize{11.000000}{13.200000}\selectfont 10}%
\end{pgfscope}%
\begin{pgfscope}%
\definecolor{textcolor}{rgb}{0.150000,0.150000,0.150000}%
\pgfsetstrokecolor{textcolor}%
\pgfsetfillcolor{textcolor}%
\pgftext[x=2.149736in,y=0.518888in,,top]{\color{textcolor}\sffamily\fontsize{11.000000}{13.200000}\selectfont 15}%
\end{pgfscope}%
\begin{pgfscope}%
\definecolor{textcolor}{rgb}{0.150000,0.150000,0.150000}%
\pgfsetstrokecolor{textcolor}%
\pgfsetfillcolor{textcolor}%
\pgftext[x=2.742538in,y=0.518888in,,top]{\color{textcolor}\sffamily\fontsize{11.000000}{13.200000}\selectfont 20}%
\end{pgfscope}%
\begin{pgfscope}%
\definecolor{textcolor}{rgb}{0.150000,0.150000,0.150000}%
\pgfsetstrokecolor{textcolor}%
\pgfsetfillcolor{textcolor}%
\pgftext[x=3.335341in,y=0.518888in,,top]{\color{textcolor}\sffamily\fontsize{11.000000}{13.200000}\selectfont 30}%
\end{pgfscope}%
\begin{pgfscope}%
\definecolor{textcolor}{rgb}{0.150000,0.150000,0.150000}%
\pgfsetstrokecolor{textcolor}%
\pgfsetfillcolor{textcolor}%
\pgftext[x=3.928143in,y=0.518888in,,top]{\color{textcolor}\sffamily\fontsize{11.000000}{13.200000}\selectfont 40}%
\end{pgfscope}%
\begin{pgfscope}%
\definecolor{textcolor}{rgb}{0.150000,0.150000,0.150000}%
\pgfsetstrokecolor{textcolor}%
\pgfsetfillcolor{textcolor}%
\pgftext[x=4.520945in,y=0.518888in,,top]{\color{textcolor}\sffamily\fontsize{11.000000}{13.200000}\selectfont 60}%
\end{pgfscope}%
\begin{pgfscope}%
\definecolor{textcolor}{rgb}{0.150000,0.150000,0.150000}%
\pgfsetstrokecolor{textcolor}%
\pgfsetfillcolor{textcolor}%
\pgftext[x=5.113747in,y=0.518888in,,top]{\color{textcolor}\sffamily\fontsize{11.000000}{13.200000}\selectfont 80}%
\end{pgfscope}%
\begin{pgfscope}%
\definecolor{textcolor}{rgb}{0.150000,0.150000,0.150000}%
\pgfsetstrokecolor{textcolor}%
\pgfsetfillcolor{textcolor}%
\pgftext[x=3.038940in,y=0.328148in,,top]{\color{textcolor}\sffamily\fontsize{12.000000}{14.400000}\selectfont number of random walks}%
\end{pgfscope}%
\begin{pgfscope}%
\pgfpathrectangle{\pgfqpoint{0.667731in}{0.650833in}}{\pgfqpoint{4.742417in}{2.714069in}}%
\pgfusepath{clip}%
\pgfsetroundcap%
\pgfsetroundjoin%
\pgfsetlinewidth{1.003750pt}%
\definecolor{currentstroke}{rgb}{1.000000,1.000000,1.000000}%
\pgfsetstrokecolor{currentstroke}%
\pgfsetdash{}{0pt}%
\pgfpathmoveto{\pgfqpoint{0.667731in}{0.691163in}}%
\pgfpathlineto{\pgfqpoint{5.410148in}{0.691163in}}%
\pgfusepath{stroke}%
\end{pgfscope}%
\begin{pgfscope}%
\definecolor{textcolor}{rgb}{0.150000,0.150000,0.150000}%
\pgfsetstrokecolor{textcolor}%
\pgfsetfillcolor{textcolor}%
\pgftext[x=0.383703in, y=0.638357in, left, base]{\color{textcolor}\sffamily\fontsize{11.000000}{13.200000}\selectfont \(\displaystyle {63}\)}%
\end{pgfscope}%
\begin{pgfscope}%
\pgfpathrectangle{\pgfqpoint{0.667731in}{0.650833in}}{\pgfqpoint{4.742417in}{2.714069in}}%
\pgfusepath{clip}%
\pgfsetroundcap%
\pgfsetroundjoin%
\pgfsetlinewidth{1.003750pt}%
\definecolor{currentstroke}{rgb}{1.000000,1.000000,1.000000}%
\pgfsetstrokecolor{currentstroke}%
\pgfsetdash{}{0pt}%
\pgfpathmoveto{\pgfqpoint{0.667731in}{1.093454in}}%
\pgfpathlineto{\pgfqpoint{5.410148in}{1.093454in}}%
\pgfusepath{stroke}%
\end{pgfscope}%
\begin{pgfscope}%
\definecolor{textcolor}{rgb}{0.150000,0.150000,0.150000}%
\pgfsetstrokecolor{textcolor}%
\pgfsetfillcolor{textcolor}%
\pgftext[x=0.383703in, y=1.040648in, left, base]{\color{textcolor}\sffamily\fontsize{11.000000}{13.200000}\selectfont \(\displaystyle {64}\)}%
\end{pgfscope}%
\begin{pgfscope}%
\pgfpathrectangle{\pgfqpoint{0.667731in}{0.650833in}}{\pgfqpoint{4.742417in}{2.714069in}}%
\pgfusepath{clip}%
\pgfsetroundcap%
\pgfsetroundjoin%
\pgfsetlinewidth{1.003750pt}%
\definecolor{currentstroke}{rgb}{1.000000,1.000000,1.000000}%
\pgfsetstrokecolor{currentstroke}%
\pgfsetdash{}{0pt}%
\pgfpathmoveto{\pgfqpoint{0.667731in}{1.495745in}}%
\pgfpathlineto{\pgfqpoint{5.410148in}{1.495745in}}%
\pgfusepath{stroke}%
\end{pgfscope}%
\begin{pgfscope}%
\definecolor{textcolor}{rgb}{0.150000,0.150000,0.150000}%
\pgfsetstrokecolor{textcolor}%
\pgfsetfillcolor{textcolor}%
\pgftext[x=0.383703in, y=1.442938in, left, base]{\color{textcolor}\sffamily\fontsize{11.000000}{13.200000}\selectfont \(\displaystyle {65}\)}%
\end{pgfscope}%
\begin{pgfscope}%
\pgfpathrectangle{\pgfqpoint{0.667731in}{0.650833in}}{\pgfqpoint{4.742417in}{2.714069in}}%
\pgfusepath{clip}%
\pgfsetroundcap%
\pgfsetroundjoin%
\pgfsetlinewidth{1.003750pt}%
\definecolor{currentstroke}{rgb}{1.000000,1.000000,1.000000}%
\pgfsetstrokecolor{currentstroke}%
\pgfsetdash{}{0pt}%
\pgfpathmoveto{\pgfqpoint{0.667731in}{1.898036in}}%
\pgfpathlineto{\pgfqpoint{5.410148in}{1.898036in}}%
\pgfusepath{stroke}%
\end{pgfscope}%
\begin{pgfscope}%
\definecolor{textcolor}{rgb}{0.150000,0.150000,0.150000}%
\pgfsetstrokecolor{textcolor}%
\pgfsetfillcolor{textcolor}%
\pgftext[x=0.383703in, y=1.845229in, left, base]{\color{textcolor}\sffamily\fontsize{11.000000}{13.200000}\selectfont \(\displaystyle {66}\)}%
\end{pgfscope}%
\begin{pgfscope}%
\pgfpathrectangle{\pgfqpoint{0.667731in}{0.650833in}}{\pgfqpoint{4.742417in}{2.714069in}}%
\pgfusepath{clip}%
\pgfsetroundcap%
\pgfsetroundjoin%
\pgfsetlinewidth{1.003750pt}%
\definecolor{currentstroke}{rgb}{1.000000,1.000000,1.000000}%
\pgfsetstrokecolor{currentstroke}%
\pgfsetdash{}{0pt}%
\pgfpathmoveto{\pgfqpoint{0.667731in}{2.300326in}}%
\pgfpathlineto{\pgfqpoint{5.410148in}{2.300326in}}%
\pgfusepath{stroke}%
\end{pgfscope}%
\begin{pgfscope}%
\definecolor{textcolor}{rgb}{0.150000,0.150000,0.150000}%
\pgfsetstrokecolor{textcolor}%
\pgfsetfillcolor{textcolor}%
\pgftext[x=0.383703in, y=2.247520in, left, base]{\color{textcolor}\sffamily\fontsize{11.000000}{13.200000}\selectfont \(\displaystyle {67}\)}%
\end{pgfscope}%
\begin{pgfscope}%
\pgfpathrectangle{\pgfqpoint{0.667731in}{0.650833in}}{\pgfqpoint{4.742417in}{2.714069in}}%
\pgfusepath{clip}%
\pgfsetroundcap%
\pgfsetroundjoin%
\pgfsetlinewidth{1.003750pt}%
\definecolor{currentstroke}{rgb}{1.000000,1.000000,1.000000}%
\pgfsetstrokecolor{currentstroke}%
\pgfsetdash{}{0pt}%
\pgfpathmoveto{\pgfqpoint{0.667731in}{2.702617in}}%
\pgfpathlineto{\pgfqpoint{5.410148in}{2.702617in}}%
\pgfusepath{stroke}%
\end{pgfscope}%
\begin{pgfscope}%
\definecolor{textcolor}{rgb}{0.150000,0.150000,0.150000}%
\pgfsetstrokecolor{textcolor}%
\pgfsetfillcolor{textcolor}%
\pgftext[x=0.383703in, y=2.649810in, left, base]{\color{textcolor}\sffamily\fontsize{11.000000}{13.200000}\selectfont \(\displaystyle {68}\)}%
\end{pgfscope}%
\begin{pgfscope}%
\pgfpathrectangle{\pgfqpoint{0.667731in}{0.650833in}}{\pgfqpoint{4.742417in}{2.714069in}}%
\pgfusepath{clip}%
\pgfsetroundcap%
\pgfsetroundjoin%
\pgfsetlinewidth{1.003750pt}%
\definecolor{currentstroke}{rgb}{1.000000,1.000000,1.000000}%
\pgfsetstrokecolor{currentstroke}%
\pgfsetdash{}{0pt}%
\pgfpathmoveto{\pgfqpoint{0.667731in}{3.104908in}}%
\pgfpathlineto{\pgfqpoint{5.410148in}{3.104908in}}%
\pgfusepath{stroke}%
\end{pgfscope}%
\begin{pgfscope}%
\definecolor{textcolor}{rgb}{0.150000,0.150000,0.150000}%
\pgfsetstrokecolor{textcolor}%
\pgfsetfillcolor{textcolor}%
\pgftext[x=0.383703in, y=3.052101in, left, base]{\color{textcolor}\sffamily\fontsize{11.000000}{13.200000}\selectfont \(\displaystyle {69}\)}%
\end{pgfscope}%
\begin{pgfscope}%
\definecolor{textcolor}{rgb}{0.150000,0.150000,0.150000}%
\pgfsetstrokecolor{textcolor}%
\pgfsetfillcolor{textcolor}%
\pgftext[x=0.328148in,y=2.007867in,,bottom,rotate=90.000000]{\color{textcolor}\sffamily\fontsize{12.000000}{14.400000}\selectfont Average score}%
\end{pgfscope}%
\begin{pgfscope}%
\pgfpathrectangle{\pgfqpoint{0.667731in}{0.650833in}}{\pgfqpoint{4.742417in}{2.714069in}}%
\pgfusepath{clip}%
\pgfsetbuttcap%
\pgfsetroundjoin%
\pgfsetlinewidth{2.710125pt}%
\definecolor{currentstroke}{rgb}{0.298039,0.447059,0.690196}%
\pgfsetstrokecolor{currentstroke}%
\pgfsetdash{{9.990000pt}{4.320000pt}}{0.000000pt}%
\pgfpathmoveto{\pgfqpoint{0.964132in}{1.419841in}}%
\pgfpathlineto{\pgfqpoint{1.556934in}{2.406592in}}%
\pgfpathlineto{\pgfqpoint{2.149736in}{1.647553in}}%
\pgfpathlineto{\pgfqpoint{2.742538in}{1.799361in}}%
\pgfpathlineto{\pgfqpoint{3.335341in}{3.013823in}}%
\pgfpathlineto{\pgfqpoint{3.928143in}{3.241535in}}%
\pgfpathlineto{\pgfqpoint{4.520945in}{2.216832in}}%
\pgfpathlineto{\pgfqpoint{5.113747in}{2.824063in}}%
\pgfusepath{stroke}%
\end{pgfscope}%
\begin{pgfscope}%
\pgfpathrectangle{\pgfqpoint{0.667731in}{0.650833in}}{\pgfqpoint{4.742417in}{2.714069in}}%
\pgfusepath{clip}%
\pgfsetroundcap%
\pgfsetroundjoin%
\pgfsetlinewidth{2.710125pt}%
\definecolor{currentstroke}{rgb}{0.298039,0.447059,0.690196}%
\pgfsetstrokecolor{currentstroke}%
\pgfsetdash{}{0pt}%
\pgfusepath{stroke}%
\end{pgfscope}%
\begin{pgfscope}%
\pgfpathrectangle{\pgfqpoint{0.667731in}{0.650833in}}{\pgfqpoint{4.742417in}{2.714069in}}%
\pgfusepath{clip}%
\pgfsetroundcap%
\pgfsetroundjoin%
\pgfsetlinewidth{2.710125pt}%
\definecolor{currentstroke}{rgb}{0.298039,0.447059,0.690196}%
\pgfsetstrokecolor{currentstroke}%
\pgfsetdash{}{0pt}%
\pgfusepath{stroke}%
\end{pgfscope}%
\begin{pgfscope}%
\pgfpathrectangle{\pgfqpoint{0.667731in}{0.650833in}}{\pgfqpoint{4.742417in}{2.714069in}}%
\pgfusepath{clip}%
\pgfsetroundcap%
\pgfsetroundjoin%
\pgfsetlinewidth{2.710125pt}%
\definecolor{currentstroke}{rgb}{0.298039,0.447059,0.690196}%
\pgfsetstrokecolor{currentstroke}%
\pgfsetdash{}{0pt}%
\pgfusepath{stroke}%
\end{pgfscope}%
\begin{pgfscope}%
\pgfpathrectangle{\pgfqpoint{0.667731in}{0.650833in}}{\pgfqpoint{4.742417in}{2.714069in}}%
\pgfusepath{clip}%
\pgfsetroundcap%
\pgfsetroundjoin%
\pgfsetlinewidth{2.710125pt}%
\definecolor{currentstroke}{rgb}{0.298039,0.447059,0.690196}%
\pgfsetstrokecolor{currentstroke}%
\pgfsetdash{}{0pt}%
\pgfusepath{stroke}%
\end{pgfscope}%
\begin{pgfscope}%
\pgfpathrectangle{\pgfqpoint{0.667731in}{0.650833in}}{\pgfqpoint{4.742417in}{2.714069in}}%
\pgfusepath{clip}%
\pgfsetroundcap%
\pgfsetroundjoin%
\pgfsetlinewidth{2.710125pt}%
\definecolor{currentstroke}{rgb}{0.298039,0.447059,0.690196}%
\pgfsetstrokecolor{currentstroke}%
\pgfsetdash{}{0pt}%
\pgfusepath{stroke}%
\end{pgfscope}%
\begin{pgfscope}%
\pgfpathrectangle{\pgfqpoint{0.667731in}{0.650833in}}{\pgfqpoint{4.742417in}{2.714069in}}%
\pgfusepath{clip}%
\pgfsetroundcap%
\pgfsetroundjoin%
\pgfsetlinewidth{2.710125pt}%
\definecolor{currentstroke}{rgb}{0.298039,0.447059,0.690196}%
\pgfsetstrokecolor{currentstroke}%
\pgfsetdash{}{0pt}%
\pgfusepath{stroke}%
\end{pgfscope}%
\begin{pgfscope}%
\pgfpathrectangle{\pgfqpoint{0.667731in}{0.650833in}}{\pgfqpoint{4.742417in}{2.714069in}}%
\pgfusepath{clip}%
\pgfsetroundcap%
\pgfsetroundjoin%
\pgfsetlinewidth{2.710125pt}%
\definecolor{currentstroke}{rgb}{0.298039,0.447059,0.690196}%
\pgfsetstrokecolor{currentstroke}%
\pgfsetdash{}{0pt}%
\pgfusepath{stroke}%
\end{pgfscope}%
\begin{pgfscope}%
\pgfpathrectangle{\pgfqpoint{0.667731in}{0.650833in}}{\pgfqpoint{4.742417in}{2.714069in}}%
\pgfusepath{clip}%
\pgfsetroundcap%
\pgfsetroundjoin%
\pgfsetlinewidth{2.710125pt}%
\definecolor{currentstroke}{rgb}{0.298039,0.447059,0.690196}%
\pgfsetstrokecolor{currentstroke}%
\pgfsetdash{}{0pt}%
\pgfusepath{stroke}%
\end{pgfscope}%
\begin{pgfscope}%
\pgfpathrectangle{\pgfqpoint{0.667731in}{0.650833in}}{\pgfqpoint{4.742417in}{2.714069in}}%
\pgfusepath{clip}%
\pgfsetbuttcap%
\pgfsetroundjoin%
\definecolor{currentfill}{rgb}{0.298039,0.447059,0.690196}%
\pgfsetfillcolor{currentfill}%
\pgfsetlinewidth{2.032594pt}%
\definecolor{currentstroke}{rgb}{0.298039,0.447059,0.690196}%
\pgfsetstrokecolor{currentstroke}%
\pgfsetdash{}{0pt}%
\pgfsys@defobject{currentmarker}{\pgfqpoint{-0.046999in}{-0.046999in}}{\pgfqpoint{0.046999in}{0.046999in}}{%
\pgfpathmoveto{\pgfqpoint{0.000000in}{-0.046999in}}%
\pgfpathcurveto{\pgfqpoint{0.012464in}{-0.046999in}}{\pgfqpoint{0.024420in}{-0.042047in}}{\pgfqpoint{0.033234in}{-0.033234in}}%
\pgfpathcurveto{\pgfqpoint{0.042047in}{-0.024420in}}{\pgfqpoint{0.046999in}{-0.012464in}}{\pgfqpoint{0.046999in}{0.000000in}}%
\pgfpathcurveto{\pgfqpoint{0.046999in}{0.012464in}}{\pgfqpoint{0.042047in}{0.024420in}}{\pgfqpoint{0.033234in}{0.033234in}}%
\pgfpathcurveto{\pgfqpoint{0.024420in}{0.042047in}}{\pgfqpoint{0.012464in}{0.046999in}}{\pgfqpoint{0.000000in}{0.046999in}}%
\pgfpathcurveto{\pgfqpoint{-0.012464in}{0.046999in}}{\pgfqpoint{-0.024420in}{0.042047in}}{\pgfqpoint{-0.033234in}{0.033234in}}%
\pgfpathcurveto{\pgfqpoint{-0.042047in}{0.024420in}}{\pgfqpoint{-0.046999in}{0.012464in}}{\pgfqpoint{-0.046999in}{0.000000in}}%
\pgfpathcurveto{\pgfqpoint{-0.046999in}{-0.012464in}}{\pgfqpoint{-0.042047in}{-0.024420in}}{\pgfqpoint{-0.033234in}{-0.033234in}}%
\pgfpathcurveto{\pgfqpoint{-0.024420in}{-0.042047in}}{\pgfqpoint{-0.012464in}{-0.046999in}}{\pgfqpoint{0.000000in}{-0.046999in}}%
\pgfpathlineto{\pgfqpoint{0.000000in}{-0.046999in}}%
\pgfpathclose%
\pgfusepath{stroke,fill}%
}%
\begin{pgfscope}%
\pgfsys@transformshift{0.964132in}{1.419841in}%
\pgfsys@useobject{currentmarker}{}%
\end{pgfscope}%
\begin{pgfscope}%
\pgfsys@transformshift{1.556934in}{2.406592in}%
\pgfsys@useobject{currentmarker}{}%
\end{pgfscope}%
\begin{pgfscope}%
\pgfsys@transformshift{2.149736in}{1.647553in}%
\pgfsys@useobject{currentmarker}{}%
\end{pgfscope}%
\begin{pgfscope}%
\pgfsys@transformshift{2.742538in}{1.799361in}%
\pgfsys@useobject{currentmarker}{}%
\end{pgfscope}%
\begin{pgfscope}%
\pgfsys@transformshift{3.335341in}{3.013823in}%
\pgfsys@useobject{currentmarker}{}%
\end{pgfscope}%
\begin{pgfscope}%
\pgfsys@transformshift{3.928143in}{3.241535in}%
\pgfsys@useobject{currentmarker}{}%
\end{pgfscope}%
\begin{pgfscope}%
\pgfsys@transformshift{4.520945in}{2.216832in}%
\pgfsys@useobject{currentmarker}{}%
\end{pgfscope}%
\begin{pgfscope}%
\pgfsys@transformshift{5.113747in}{2.824063in}%
\pgfsys@useobject{currentmarker}{}%
\end{pgfscope}%
\end{pgfscope}%
\begin{pgfscope}%
\pgfpathrectangle{\pgfqpoint{0.667731in}{0.650833in}}{\pgfqpoint{4.742417in}{2.714069in}}%
\pgfusepath{clip}%
\pgfsetbuttcap%
\pgfsetroundjoin%
\pgfsetlinewidth{2.710125pt}%
\definecolor{currentstroke}{rgb}{0.866667,0.517647,0.321569}%
\pgfsetstrokecolor{currentstroke}%
\pgfsetdash{{9.990000pt}{4.320000pt}}{0.000000pt}%
\pgfpathmoveto{\pgfqpoint{0.964132in}{0.774199in}}%
\pgfpathlineto{\pgfqpoint{1.556934in}{1.375127in}}%
\pgfpathlineto{\pgfqpoint{2.149736in}{0.807932in}}%
\pgfpathlineto{\pgfqpoint{2.742538in}{1.052932in}}%
\pgfpathlineto{\pgfqpoint{3.335341in}{2.339383in}}%
\pgfpathlineto{\pgfqpoint{3.928143in}{2.641892in}}%
\pgfpathlineto{\pgfqpoint{4.520945in}{1.468794in}}%
\pgfpathlineto{\pgfqpoint{5.113747in}{2.204825in}}%
\pgfusepath{stroke}%
\end{pgfscope}%
\begin{pgfscope}%
\pgfpathrectangle{\pgfqpoint{0.667731in}{0.650833in}}{\pgfqpoint{4.742417in}{2.714069in}}%
\pgfusepath{clip}%
\pgfsetroundcap%
\pgfsetroundjoin%
\pgfsetlinewidth{2.710125pt}%
\definecolor{currentstroke}{rgb}{0.866667,0.517647,0.321569}%
\pgfsetstrokecolor{currentstroke}%
\pgfsetdash{}{0pt}%
\pgfusepath{stroke}%
\end{pgfscope}%
\begin{pgfscope}%
\pgfpathrectangle{\pgfqpoint{0.667731in}{0.650833in}}{\pgfqpoint{4.742417in}{2.714069in}}%
\pgfusepath{clip}%
\pgfsetroundcap%
\pgfsetroundjoin%
\pgfsetlinewidth{2.710125pt}%
\definecolor{currentstroke}{rgb}{0.866667,0.517647,0.321569}%
\pgfsetstrokecolor{currentstroke}%
\pgfsetdash{}{0pt}%
\pgfusepath{stroke}%
\end{pgfscope}%
\begin{pgfscope}%
\pgfpathrectangle{\pgfqpoint{0.667731in}{0.650833in}}{\pgfqpoint{4.742417in}{2.714069in}}%
\pgfusepath{clip}%
\pgfsetroundcap%
\pgfsetroundjoin%
\pgfsetlinewidth{2.710125pt}%
\definecolor{currentstroke}{rgb}{0.866667,0.517647,0.321569}%
\pgfsetstrokecolor{currentstroke}%
\pgfsetdash{}{0pt}%
\pgfusepath{stroke}%
\end{pgfscope}%
\begin{pgfscope}%
\pgfpathrectangle{\pgfqpoint{0.667731in}{0.650833in}}{\pgfqpoint{4.742417in}{2.714069in}}%
\pgfusepath{clip}%
\pgfsetroundcap%
\pgfsetroundjoin%
\pgfsetlinewidth{2.710125pt}%
\definecolor{currentstroke}{rgb}{0.866667,0.517647,0.321569}%
\pgfsetstrokecolor{currentstroke}%
\pgfsetdash{}{0pt}%
\pgfusepath{stroke}%
\end{pgfscope}%
\begin{pgfscope}%
\pgfpathrectangle{\pgfqpoint{0.667731in}{0.650833in}}{\pgfqpoint{4.742417in}{2.714069in}}%
\pgfusepath{clip}%
\pgfsetroundcap%
\pgfsetroundjoin%
\pgfsetlinewidth{2.710125pt}%
\definecolor{currentstroke}{rgb}{0.866667,0.517647,0.321569}%
\pgfsetstrokecolor{currentstroke}%
\pgfsetdash{}{0pt}%
\pgfusepath{stroke}%
\end{pgfscope}%
\begin{pgfscope}%
\pgfpathrectangle{\pgfqpoint{0.667731in}{0.650833in}}{\pgfqpoint{4.742417in}{2.714069in}}%
\pgfusepath{clip}%
\pgfsetroundcap%
\pgfsetroundjoin%
\pgfsetlinewidth{2.710125pt}%
\definecolor{currentstroke}{rgb}{0.866667,0.517647,0.321569}%
\pgfsetstrokecolor{currentstroke}%
\pgfsetdash{}{0pt}%
\pgfusepath{stroke}%
\end{pgfscope}%
\begin{pgfscope}%
\pgfpathrectangle{\pgfqpoint{0.667731in}{0.650833in}}{\pgfqpoint{4.742417in}{2.714069in}}%
\pgfusepath{clip}%
\pgfsetroundcap%
\pgfsetroundjoin%
\pgfsetlinewidth{2.710125pt}%
\definecolor{currentstroke}{rgb}{0.866667,0.517647,0.321569}%
\pgfsetstrokecolor{currentstroke}%
\pgfsetdash{}{0pt}%
\pgfusepath{stroke}%
\end{pgfscope}%
\begin{pgfscope}%
\pgfpathrectangle{\pgfqpoint{0.667731in}{0.650833in}}{\pgfqpoint{4.742417in}{2.714069in}}%
\pgfusepath{clip}%
\pgfsetroundcap%
\pgfsetroundjoin%
\pgfsetlinewidth{2.710125pt}%
\definecolor{currentstroke}{rgb}{0.866667,0.517647,0.321569}%
\pgfsetstrokecolor{currentstroke}%
\pgfsetdash{}{0pt}%
\pgfusepath{stroke}%
\end{pgfscope}%
\begin{pgfscope}%
\pgfpathrectangle{\pgfqpoint{0.667731in}{0.650833in}}{\pgfqpoint{4.742417in}{2.714069in}}%
\pgfusepath{clip}%
\pgfsetbuttcap%
\pgfsetroundjoin%
\definecolor{currentfill}{rgb}{0.866667,0.517647,0.321569}%
\pgfsetfillcolor{currentfill}%
\pgfsetlinewidth{2.032594pt}%
\definecolor{currentstroke}{rgb}{0.866667,0.517647,0.321569}%
\pgfsetstrokecolor{currentstroke}%
\pgfsetdash{}{0pt}%
\pgfsys@defobject{currentmarker}{\pgfqpoint{-0.046999in}{-0.046999in}}{\pgfqpoint{0.046999in}{0.046999in}}{%
\pgfpathmoveto{\pgfqpoint{0.000000in}{-0.046999in}}%
\pgfpathcurveto{\pgfqpoint{0.012464in}{-0.046999in}}{\pgfqpoint{0.024420in}{-0.042047in}}{\pgfqpoint{0.033234in}{-0.033234in}}%
\pgfpathcurveto{\pgfqpoint{0.042047in}{-0.024420in}}{\pgfqpoint{0.046999in}{-0.012464in}}{\pgfqpoint{0.046999in}{0.000000in}}%
\pgfpathcurveto{\pgfqpoint{0.046999in}{0.012464in}}{\pgfqpoint{0.042047in}{0.024420in}}{\pgfqpoint{0.033234in}{0.033234in}}%
\pgfpathcurveto{\pgfqpoint{0.024420in}{0.042047in}}{\pgfqpoint{0.012464in}{0.046999in}}{\pgfqpoint{0.000000in}{0.046999in}}%
\pgfpathcurveto{\pgfqpoint{-0.012464in}{0.046999in}}{\pgfqpoint{-0.024420in}{0.042047in}}{\pgfqpoint{-0.033234in}{0.033234in}}%
\pgfpathcurveto{\pgfqpoint{-0.042047in}{0.024420in}}{\pgfqpoint{-0.046999in}{0.012464in}}{\pgfqpoint{-0.046999in}{0.000000in}}%
\pgfpathcurveto{\pgfqpoint{-0.046999in}{-0.012464in}}{\pgfqpoint{-0.042047in}{-0.024420in}}{\pgfqpoint{-0.033234in}{-0.033234in}}%
\pgfpathcurveto{\pgfqpoint{-0.024420in}{-0.042047in}}{\pgfqpoint{-0.012464in}{-0.046999in}}{\pgfqpoint{0.000000in}{-0.046999in}}%
\pgfpathlineto{\pgfqpoint{0.000000in}{-0.046999in}}%
\pgfpathclose%
\pgfusepath{stroke,fill}%
}%
\begin{pgfscope}%
\pgfsys@transformshift{0.964132in}{0.774199in}%
\pgfsys@useobject{currentmarker}{}%
\end{pgfscope}%
\begin{pgfscope}%
\pgfsys@transformshift{1.556934in}{1.375127in}%
\pgfsys@useobject{currentmarker}{}%
\end{pgfscope}%
\begin{pgfscope}%
\pgfsys@transformshift{2.149736in}{0.807932in}%
\pgfsys@useobject{currentmarker}{}%
\end{pgfscope}%
\begin{pgfscope}%
\pgfsys@transformshift{2.742538in}{1.052932in}%
\pgfsys@useobject{currentmarker}{}%
\end{pgfscope}%
\begin{pgfscope}%
\pgfsys@transformshift{3.335341in}{2.339383in}%
\pgfsys@useobject{currentmarker}{}%
\end{pgfscope}%
\begin{pgfscope}%
\pgfsys@transformshift{3.928143in}{2.641892in}%
\pgfsys@useobject{currentmarker}{}%
\end{pgfscope}%
\begin{pgfscope}%
\pgfsys@transformshift{4.520945in}{1.468794in}%
\pgfsys@useobject{currentmarker}{}%
\end{pgfscope}%
\begin{pgfscope}%
\pgfsys@transformshift{5.113747in}{2.204825in}%
\pgfsys@useobject{currentmarker}{}%
\end{pgfscope}%
\end{pgfscope}%
\begin{pgfscope}%
\pgfsetrectcap%
\pgfsetmiterjoin%
\pgfsetlinewidth{1.254687pt}%
\definecolor{currentstroke}{rgb}{1.000000,1.000000,1.000000}%
\pgfsetstrokecolor{currentstroke}%
\pgfsetdash{}{0pt}%
\pgfpathmoveto{\pgfqpoint{0.667731in}{0.650833in}}%
\pgfpathlineto{\pgfqpoint{0.667731in}{3.364902in}}%
\pgfusepath{stroke}%
\end{pgfscope}%
\begin{pgfscope}%
\pgfsetrectcap%
\pgfsetmiterjoin%
\pgfsetlinewidth{1.254687pt}%
\definecolor{currentstroke}{rgb}{1.000000,1.000000,1.000000}%
\pgfsetstrokecolor{currentstroke}%
\pgfsetdash{}{0pt}%
\pgfpathmoveto{\pgfqpoint{5.410148in}{0.650833in}}%
\pgfpathlineto{\pgfqpoint{5.410148in}{3.364902in}}%
\pgfusepath{stroke}%
\end{pgfscope}%
\begin{pgfscope}%
\pgfsetrectcap%
\pgfsetmiterjoin%
\pgfsetlinewidth{1.254687pt}%
\definecolor{currentstroke}{rgb}{1.000000,1.000000,1.000000}%
\pgfsetstrokecolor{currentstroke}%
\pgfsetdash{}{0pt}%
\pgfpathmoveto{\pgfqpoint{0.667731in}{0.650833in}}%
\pgfpathlineto{\pgfqpoint{5.410148in}{0.650833in}}%
\pgfusepath{stroke}%
\end{pgfscope}%
\begin{pgfscope}%
\pgfsetrectcap%
\pgfsetmiterjoin%
\pgfsetlinewidth{1.254687pt}%
\definecolor{currentstroke}{rgb}{1.000000,1.000000,1.000000}%
\pgfsetstrokecolor{currentstroke}%
\pgfsetdash{}{0pt}%
\pgfpathmoveto{\pgfqpoint{0.667731in}{3.364902in}}%
\pgfpathlineto{\pgfqpoint{5.410148in}{3.364902in}}%
\pgfusepath{stroke}%
\end{pgfscope}%
\begin{pgfscope}%
\pgfsetbuttcap%
\pgfsetmiterjoin%
\definecolor{currentfill}{rgb}{0.917647,0.917647,0.949020}%
\pgfsetfillcolor{currentfill}%
\pgfsetfillopacity{0.800000}%
\pgfsetlinewidth{1.003750pt}%
\definecolor{currentstroke}{rgb}{0.800000,0.800000,0.800000}%
\pgfsetstrokecolor{currentstroke}%
\pgfsetstrokeopacity{0.800000}%
\pgfsetdash{}{0pt}%
\pgfpathmoveto{\pgfqpoint{0.774675in}{2.592333in}}%
\pgfpathlineto{\pgfqpoint{1.866407in}{2.592333in}}%
\pgfpathquadraticcurveto{\pgfqpoint{1.896963in}{2.592333in}}{\pgfqpoint{1.896963in}{2.622888in}}%
\pgfpathlineto{\pgfqpoint{1.896963in}{3.257957in}}%
\pgfpathquadraticcurveto{\pgfqpoint{1.896963in}{3.288513in}}{\pgfqpoint{1.866407in}{3.288513in}}%
\pgfpathlineto{\pgfqpoint{0.774675in}{3.288513in}}%
\pgfpathquadraticcurveto{\pgfqpoint{0.744120in}{3.288513in}}{\pgfqpoint{0.744120in}{3.257957in}}%
\pgfpathlineto{\pgfqpoint{0.744120in}{2.622888in}}%
\pgfpathquadraticcurveto{\pgfqpoint{0.744120in}{2.592333in}}{\pgfqpoint{0.774675in}{2.592333in}}%
\pgfpathlineto{\pgfqpoint{0.774675in}{2.592333in}}%
\pgfpathclose%
\pgfusepath{stroke,fill}%
\end{pgfscope}%
\begin{pgfscope}%
\definecolor{textcolor}{rgb}{0.150000,0.150000,0.150000}%
\pgfsetstrokecolor{textcolor}%
\pgfsetfillcolor{textcolor}%
\pgftext[x=1.076929in,y=3.111661in,left,base]{\color{textcolor}\sffamily\fontsize{12.000000}{14.400000}\selectfont F-score}%
\end{pgfscope}%
\begin{pgfscope}%
\pgfsetbuttcap%
\pgfsetroundjoin%
\definecolor{currentfill}{rgb}{0.298039,0.447059,0.690196}%
\pgfsetfillcolor{currentfill}%
\pgfsetlinewidth{2.032594pt}%
\definecolor{currentstroke}{rgb}{0.298039,0.447059,0.690196}%
\pgfsetstrokecolor{currentstroke}%
\pgfsetdash{}{0pt}%
\pgfsys@defobject{currentmarker}{\pgfqpoint{-0.046999in}{-0.046999in}}{\pgfqpoint{0.046999in}{0.046999in}}{%
\pgfpathmoveto{\pgfqpoint{0.000000in}{-0.046999in}}%
\pgfpathcurveto{\pgfqpoint{0.012464in}{-0.046999in}}{\pgfqpoint{0.024420in}{-0.042047in}}{\pgfqpoint{0.033234in}{-0.033234in}}%
\pgfpathcurveto{\pgfqpoint{0.042047in}{-0.024420in}}{\pgfqpoint{0.046999in}{-0.012464in}}{\pgfqpoint{0.046999in}{0.000000in}}%
\pgfpathcurveto{\pgfqpoint{0.046999in}{0.012464in}}{\pgfqpoint{0.042047in}{0.024420in}}{\pgfqpoint{0.033234in}{0.033234in}}%
\pgfpathcurveto{\pgfqpoint{0.024420in}{0.042047in}}{\pgfqpoint{0.012464in}{0.046999in}}{\pgfqpoint{0.000000in}{0.046999in}}%
\pgfpathcurveto{\pgfqpoint{-0.012464in}{0.046999in}}{\pgfqpoint{-0.024420in}{0.042047in}}{\pgfqpoint{-0.033234in}{0.033234in}}%
\pgfpathcurveto{\pgfqpoint{-0.042047in}{0.024420in}}{\pgfqpoint{-0.046999in}{0.012464in}}{\pgfqpoint{-0.046999in}{0.000000in}}%
\pgfpathcurveto{\pgfqpoint{-0.046999in}{-0.012464in}}{\pgfqpoint{-0.042047in}{-0.024420in}}{\pgfqpoint{-0.033234in}{-0.033234in}}%
\pgfpathcurveto{\pgfqpoint{-0.024420in}{-0.042047in}}{\pgfqpoint{-0.012464in}{-0.046999in}}{\pgfqpoint{0.000000in}{-0.046999in}}%
\pgfpathlineto{\pgfqpoint{0.000000in}{-0.046999in}}%
\pgfpathclose%
\pgfusepath{stroke,fill}%
}%
\begin{pgfscope}%
\pgfsys@transformshift{0.958009in}{2.936025in}%
\pgfsys@useobject{currentmarker}{}%
\end{pgfscope}%
\end{pgfscope}%
\begin{pgfscope}%
\definecolor{textcolor}{rgb}{0.150000,0.150000,0.150000}%
\pgfsetstrokecolor{textcolor}%
\pgfsetfillcolor{textcolor}%
\pgftext[x=1.233009in,y=2.895921in,left,base]{\color{textcolor}\sffamily\fontsize{11.000000}{13.200000}\selectfont F1 micro}%
\end{pgfscope}%
\begin{pgfscope}%
\pgfsetbuttcap%
\pgfsetroundjoin%
\definecolor{currentfill}{rgb}{0.866667,0.517647,0.321569}%
\pgfsetfillcolor{currentfill}%
\pgfsetlinewidth{2.032594pt}%
\definecolor{currentstroke}{rgb}{0.866667,0.517647,0.321569}%
\pgfsetstrokecolor{currentstroke}%
\pgfsetdash{}{0pt}%
\pgfsys@defobject{currentmarker}{\pgfqpoint{-0.046999in}{-0.046999in}}{\pgfqpoint{0.046999in}{0.046999in}}{%
\pgfpathmoveto{\pgfqpoint{0.000000in}{-0.046999in}}%
\pgfpathcurveto{\pgfqpoint{0.012464in}{-0.046999in}}{\pgfqpoint{0.024420in}{-0.042047in}}{\pgfqpoint{0.033234in}{-0.033234in}}%
\pgfpathcurveto{\pgfqpoint{0.042047in}{-0.024420in}}{\pgfqpoint{0.046999in}{-0.012464in}}{\pgfqpoint{0.046999in}{0.000000in}}%
\pgfpathcurveto{\pgfqpoint{0.046999in}{0.012464in}}{\pgfqpoint{0.042047in}{0.024420in}}{\pgfqpoint{0.033234in}{0.033234in}}%
\pgfpathcurveto{\pgfqpoint{0.024420in}{0.042047in}}{\pgfqpoint{0.012464in}{0.046999in}}{\pgfqpoint{0.000000in}{0.046999in}}%
\pgfpathcurveto{\pgfqpoint{-0.012464in}{0.046999in}}{\pgfqpoint{-0.024420in}{0.042047in}}{\pgfqpoint{-0.033234in}{0.033234in}}%
\pgfpathcurveto{\pgfqpoint{-0.042047in}{0.024420in}}{\pgfqpoint{-0.046999in}{0.012464in}}{\pgfqpoint{-0.046999in}{0.000000in}}%
\pgfpathcurveto{\pgfqpoint{-0.046999in}{-0.012464in}}{\pgfqpoint{-0.042047in}{-0.024420in}}{\pgfqpoint{-0.033234in}{-0.033234in}}%
\pgfpathcurveto{\pgfqpoint{-0.024420in}{-0.042047in}}{\pgfqpoint{-0.012464in}{-0.046999in}}{\pgfqpoint{0.000000in}{-0.046999in}}%
\pgfpathlineto{\pgfqpoint{0.000000in}{-0.046999in}}%
\pgfpathclose%
\pgfusepath{stroke,fill}%
}%
\begin{pgfscope}%
\pgfsys@transformshift{0.958009in}{2.723120in}%
\pgfsys@useobject{currentmarker}{}%
\end{pgfscope}%
\end{pgfscope}%
\begin{pgfscope}%
\definecolor{textcolor}{rgb}{0.150000,0.150000,0.150000}%
\pgfsetstrokecolor{textcolor}%
\pgfsetfillcolor{textcolor}%
\pgftext[x=1.233009in,y=2.683016in,left,base]{\color{textcolor}\sffamily\fontsize{11.000000}{13.200000}\selectfont F1 macro}%
\end{pgfscope}%
\end{pgfpicture}%
\makeatother%
\endgroup%
}
        \label{fig:deepwalk:number and walk length}
    \end{subfigure}
    \caption{DeepWalk and FakeNewsNet results.}
    \label{fig:deepwalk:plots}
\end{figure}

\section{Node2vec}
Similarly to DeepWalk, node2vec also allows to represent a network as a spatial representation.

\begin{figure}
    \centering
    \begin{subfigure}[b]{0.49\textwidth}
        \centering
        \scalebox{.5}{%% Creator: Matplotlib, PGF backend
%%
%% To include the figure in your LaTeX document, write
%%   \input{<filename>.pgf}
%%
%% Make sure the required packages are loaded in your preamble
%%   \usepackage{pgf}
%%
%% Also ensure that all the required font packages are loaded; for instance,
%% the lmodern package is sometimes necessary when using math font.
%%   \usepackage{lmodern}
%%
%% Figures using additional raster images can only be included by \input if
%% they are in the same directory as the main LaTeX file. For loading figures
%% from other directories you can use the `import` package
%%   \usepackage{import}
%%
%% and then include the figures with
%%   \import{<path to file>}{<filename>.pgf}
%%
%% Matplotlib used the following preamble
%%   
%%   \makeatletter\@ifpackageloaded{underscore}{}{\usepackage[strings]{underscore}}\makeatother
%%
\begingroup%
\makeatletter%
\begin{pgfpicture}%
\pgfpathrectangle{\pgfpointorigin}{\pgfqpoint{5.590148in}{6.909803in}}%
\pgfusepath{use as bounding box, clip}%
\begin{pgfscope}%
\pgfsetbuttcap%
\pgfsetmiterjoin%
\definecolor{currentfill}{rgb}{1.000000,1.000000,1.000000}%
\pgfsetfillcolor{currentfill}%
\pgfsetlinewidth{0.000000pt}%
\definecolor{currentstroke}{rgb}{1.000000,1.000000,1.000000}%
\pgfsetstrokecolor{currentstroke}%
\pgfsetdash{}{0pt}%
\pgfpathmoveto{\pgfqpoint{0.000000in}{0.000000in}}%
\pgfpathlineto{\pgfqpoint{5.590148in}{0.000000in}}%
\pgfpathlineto{\pgfqpoint{5.590148in}{6.909803in}}%
\pgfpathlineto{\pgfqpoint{0.000000in}{6.909803in}}%
\pgfpathlineto{\pgfqpoint{0.000000in}{0.000000in}}%
\pgfpathclose%
\pgfusepath{fill}%
\end{pgfscope}%
\begin{pgfscope}%
\pgfsetbuttcap%
\pgfsetmiterjoin%
\definecolor{currentfill}{rgb}{0.917647,0.917647,0.949020}%
\pgfsetfillcolor{currentfill}%
\pgfsetlinewidth{0.000000pt}%
\definecolor{currentstroke}{rgb}{0.000000,0.000000,0.000000}%
\pgfsetstrokecolor{currentstroke}%
\pgfsetstrokeopacity{0.000000}%
\pgfsetdash{}{0pt}%
\pgfpathmoveto{\pgfqpoint{0.667731in}{4.015734in}}%
\pgfpathlineto{\pgfqpoint{5.410148in}{4.015734in}}%
\pgfpathlineto{\pgfqpoint{5.410148in}{6.729803in}}%
\pgfpathlineto{\pgfqpoint{0.667731in}{6.729803in}}%
\pgfpathlineto{\pgfqpoint{0.667731in}{4.015734in}}%
\pgfpathclose%
\pgfusepath{fill}%
\end{pgfscope}%
\begin{pgfscope}%
\definecolor{textcolor}{rgb}{0.150000,0.150000,0.150000}%
\pgfsetstrokecolor{textcolor}%
\pgfsetfillcolor{textcolor}%
\pgftext[x=1.260533in,y=3.883790in,,top]{\color{textcolor}\sffamily\fontsize{11.000000}{13.200000}\selectfont 32}%
\end{pgfscope}%
\begin{pgfscope}%
\definecolor{textcolor}{rgb}{0.150000,0.150000,0.150000}%
\pgfsetstrokecolor{textcolor}%
\pgfsetfillcolor{textcolor}%
\pgftext[x=2.446137in,y=3.883790in,,top]{\color{textcolor}\sffamily\fontsize{11.000000}{13.200000}\selectfont 64}%
\end{pgfscope}%
\begin{pgfscope}%
\definecolor{textcolor}{rgb}{0.150000,0.150000,0.150000}%
\pgfsetstrokecolor{textcolor}%
\pgfsetfillcolor{textcolor}%
\pgftext[x=3.631742in,y=3.883790in,,top]{\color{textcolor}\sffamily\fontsize{11.000000}{13.200000}\selectfont 128}%
\end{pgfscope}%
\begin{pgfscope}%
\definecolor{textcolor}{rgb}{0.150000,0.150000,0.150000}%
\pgfsetstrokecolor{textcolor}%
\pgfsetfillcolor{textcolor}%
\pgftext[x=4.817346in,y=3.883790in,,top]{\color{textcolor}\sffamily\fontsize{11.000000}{13.200000}\selectfont 256}%
\end{pgfscope}%
\begin{pgfscope}%
\definecolor{textcolor}{rgb}{0.150000,0.150000,0.150000}%
\pgfsetstrokecolor{textcolor}%
\pgfsetfillcolor{textcolor}%
\pgftext[x=3.038940in,y=3.693049in,,top]{\color{textcolor}\sffamily\fontsize{12.000000}{14.400000}\selectfont number of dimensions}%
\end{pgfscope}%
\begin{pgfscope}%
\pgfpathrectangle{\pgfqpoint{0.667731in}{4.015734in}}{\pgfqpoint{4.742417in}{2.714069in}}%
\pgfusepath{clip}%
\pgfsetroundcap%
\pgfsetroundjoin%
\pgfsetlinewidth{1.003750pt}%
\definecolor{currentstroke}{rgb}{1.000000,1.000000,1.000000}%
\pgfsetstrokecolor{currentstroke}%
\pgfsetdash{}{0pt}%
\pgfpathmoveto{\pgfqpoint{0.667731in}{4.039097in}}%
\pgfpathlineto{\pgfqpoint{5.410148in}{4.039097in}}%
\pgfusepath{stroke}%
\end{pgfscope}%
\begin{pgfscope}%
\definecolor{textcolor}{rgb}{0.150000,0.150000,0.150000}%
\pgfsetstrokecolor{textcolor}%
\pgfsetfillcolor{textcolor}%
\pgftext[x=0.383703in, y=3.986290in, left, base]{\color{textcolor}\sffamily\fontsize{11.000000}{13.200000}\selectfont \(\displaystyle {63}\)}%
\end{pgfscope}%
\begin{pgfscope}%
\pgfpathrectangle{\pgfqpoint{0.667731in}{4.015734in}}{\pgfqpoint{4.742417in}{2.714069in}}%
\pgfusepath{clip}%
\pgfsetroundcap%
\pgfsetroundjoin%
\pgfsetlinewidth{1.003750pt}%
\definecolor{currentstroke}{rgb}{1.000000,1.000000,1.000000}%
\pgfsetstrokecolor{currentstroke}%
\pgfsetdash{}{0pt}%
\pgfpathmoveto{\pgfqpoint{0.667731in}{4.532101in}}%
\pgfpathlineto{\pgfqpoint{5.410148in}{4.532101in}}%
\pgfusepath{stroke}%
\end{pgfscope}%
\begin{pgfscope}%
\definecolor{textcolor}{rgb}{0.150000,0.150000,0.150000}%
\pgfsetstrokecolor{textcolor}%
\pgfsetfillcolor{textcolor}%
\pgftext[x=0.383703in, y=4.479294in, left, base]{\color{textcolor}\sffamily\fontsize{11.000000}{13.200000}\selectfont \(\displaystyle {64}\)}%
\end{pgfscope}%
\begin{pgfscope}%
\pgfpathrectangle{\pgfqpoint{0.667731in}{4.015734in}}{\pgfqpoint{4.742417in}{2.714069in}}%
\pgfusepath{clip}%
\pgfsetroundcap%
\pgfsetroundjoin%
\pgfsetlinewidth{1.003750pt}%
\definecolor{currentstroke}{rgb}{1.000000,1.000000,1.000000}%
\pgfsetstrokecolor{currentstroke}%
\pgfsetdash{}{0pt}%
\pgfpathmoveto{\pgfqpoint{0.667731in}{5.025104in}}%
\pgfpathlineto{\pgfqpoint{5.410148in}{5.025104in}}%
\pgfusepath{stroke}%
\end{pgfscope}%
\begin{pgfscope}%
\definecolor{textcolor}{rgb}{0.150000,0.150000,0.150000}%
\pgfsetstrokecolor{textcolor}%
\pgfsetfillcolor{textcolor}%
\pgftext[x=0.383703in, y=4.972297in, left, base]{\color{textcolor}\sffamily\fontsize{11.000000}{13.200000}\selectfont \(\displaystyle {65}\)}%
\end{pgfscope}%
\begin{pgfscope}%
\pgfpathrectangle{\pgfqpoint{0.667731in}{4.015734in}}{\pgfqpoint{4.742417in}{2.714069in}}%
\pgfusepath{clip}%
\pgfsetroundcap%
\pgfsetroundjoin%
\pgfsetlinewidth{1.003750pt}%
\definecolor{currentstroke}{rgb}{1.000000,1.000000,1.000000}%
\pgfsetstrokecolor{currentstroke}%
\pgfsetdash{}{0pt}%
\pgfpathmoveto{\pgfqpoint{0.667731in}{5.518108in}}%
\pgfpathlineto{\pgfqpoint{5.410148in}{5.518108in}}%
\pgfusepath{stroke}%
\end{pgfscope}%
\begin{pgfscope}%
\definecolor{textcolor}{rgb}{0.150000,0.150000,0.150000}%
\pgfsetstrokecolor{textcolor}%
\pgfsetfillcolor{textcolor}%
\pgftext[x=0.383703in, y=5.465301in, left, base]{\color{textcolor}\sffamily\fontsize{11.000000}{13.200000}\selectfont \(\displaystyle {66}\)}%
\end{pgfscope}%
\begin{pgfscope}%
\pgfpathrectangle{\pgfqpoint{0.667731in}{4.015734in}}{\pgfqpoint{4.742417in}{2.714069in}}%
\pgfusepath{clip}%
\pgfsetroundcap%
\pgfsetroundjoin%
\pgfsetlinewidth{1.003750pt}%
\definecolor{currentstroke}{rgb}{1.000000,1.000000,1.000000}%
\pgfsetstrokecolor{currentstroke}%
\pgfsetdash{}{0pt}%
\pgfpathmoveto{\pgfqpoint{0.667731in}{6.011111in}}%
\pgfpathlineto{\pgfqpoint{5.410148in}{6.011111in}}%
\pgfusepath{stroke}%
\end{pgfscope}%
\begin{pgfscope}%
\definecolor{textcolor}{rgb}{0.150000,0.150000,0.150000}%
\pgfsetstrokecolor{textcolor}%
\pgfsetfillcolor{textcolor}%
\pgftext[x=0.383703in, y=5.958305in, left, base]{\color{textcolor}\sffamily\fontsize{11.000000}{13.200000}\selectfont \(\displaystyle {67}\)}%
\end{pgfscope}%
\begin{pgfscope}%
\pgfpathrectangle{\pgfqpoint{0.667731in}{4.015734in}}{\pgfqpoint{4.742417in}{2.714069in}}%
\pgfusepath{clip}%
\pgfsetroundcap%
\pgfsetroundjoin%
\pgfsetlinewidth{1.003750pt}%
\definecolor{currentstroke}{rgb}{1.000000,1.000000,1.000000}%
\pgfsetstrokecolor{currentstroke}%
\pgfsetdash{}{0pt}%
\pgfpathmoveto{\pgfqpoint{0.667731in}{6.504115in}}%
\pgfpathlineto{\pgfqpoint{5.410148in}{6.504115in}}%
\pgfusepath{stroke}%
\end{pgfscope}%
\begin{pgfscope}%
\definecolor{textcolor}{rgb}{0.150000,0.150000,0.150000}%
\pgfsetstrokecolor{textcolor}%
\pgfsetfillcolor{textcolor}%
\pgftext[x=0.383703in, y=6.451308in, left, base]{\color{textcolor}\sffamily\fontsize{11.000000}{13.200000}\selectfont \(\displaystyle {68}\)}%
\end{pgfscope}%
\begin{pgfscope}%
\definecolor{textcolor}{rgb}{0.150000,0.150000,0.150000}%
\pgfsetstrokecolor{textcolor}%
\pgfsetfillcolor{textcolor}%
\pgftext[x=0.328148in,y=5.372769in,,bottom,rotate=90.000000]{\color{textcolor}\sffamily\fontsize{12.000000}{14.400000}\selectfont Average score}%
\end{pgfscope}%
\begin{pgfscope}%
\pgfpathrectangle{\pgfqpoint{0.667731in}{4.015734in}}{\pgfqpoint{4.742417in}{2.714069in}}%
\pgfusepath{clip}%
\pgfsetbuttcap%
\pgfsetroundjoin%
\pgfsetlinewidth{2.710125pt}%
\definecolor{currentstroke}{rgb}{0.298039,0.447059,0.690196}%
\pgfsetstrokecolor{currentstroke}%
\pgfsetdash{{9.990000pt}{4.320000pt}}{0.000000pt}%
\pgfpathmoveto{\pgfqpoint{1.260533in}{6.606436in}}%
\pgfpathlineto{\pgfqpoint{2.446137in}{6.141339in}}%
\pgfpathlineto{\pgfqpoint{3.631742in}{5.211143in}}%
\pgfpathlineto{\pgfqpoint{4.817346in}{6.187848in}}%
\pgfusepath{stroke}%
\end{pgfscope}%
\begin{pgfscope}%
\pgfpathrectangle{\pgfqpoint{0.667731in}{4.015734in}}{\pgfqpoint{4.742417in}{2.714069in}}%
\pgfusepath{clip}%
\pgfsetroundcap%
\pgfsetroundjoin%
\pgfsetlinewidth{2.710125pt}%
\definecolor{currentstroke}{rgb}{0.298039,0.447059,0.690196}%
\pgfsetstrokecolor{currentstroke}%
\pgfsetdash{}{0pt}%
\pgfusepath{stroke}%
\end{pgfscope}%
\begin{pgfscope}%
\pgfpathrectangle{\pgfqpoint{0.667731in}{4.015734in}}{\pgfqpoint{4.742417in}{2.714069in}}%
\pgfusepath{clip}%
\pgfsetroundcap%
\pgfsetroundjoin%
\pgfsetlinewidth{2.710125pt}%
\definecolor{currentstroke}{rgb}{0.298039,0.447059,0.690196}%
\pgfsetstrokecolor{currentstroke}%
\pgfsetdash{}{0pt}%
\pgfusepath{stroke}%
\end{pgfscope}%
\begin{pgfscope}%
\pgfpathrectangle{\pgfqpoint{0.667731in}{4.015734in}}{\pgfqpoint{4.742417in}{2.714069in}}%
\pgfusepath{clip}%
\pgfsetroundcap%
\pgfsetroundjoin%
\pgfsetlinewidth{2.710125pt}%
\definecolor{currentstroke}{rgb}{0.298039,0.447059,0.690196}%
\pgfsetstrokecolor{currentstroke}%
\pgfsetdash{}{0pt}%
\pgfusepath{stroke}%
\end{pgfscope}%
\begin{pgfscope}%
\pgfpathrectangle{\pgfqpoint{0.667731in}{4.015734in}}{\pgfqpoint{4.742417in}{2.714069in}}%
\pgfusepath{clip}%
\pgfsetroundcap%
\pgfsetroundjoin%
\pgfsetlinewidth{2.710125pt}%
\definecolor{currentstroke}{rgb}{0.298039,0.447059,0.690196}%
\pgfsetstrokecolor{currentstroke}%
\pgfsetdash{}{0pt}%
\pgfusepath{stroke}%
\end{pgfscope}%
\begin{pgfscope}%
\pgfpathrectangle{\pgfqpoint{0.667731in}{4.015734in}}{\pgfqpoint{4.742417in}{2.714069in}}%
\pgfusepath{clip}%
\pgfsetbuttcap%
\pgfsetroundjoin%
\definecolor{currentfill}{rgb}{0.298039,0.447059,0.690196}%
\pgfsetfillcolor{currentfill}%
\pgfsetlinewidth{2.032594pt}%
\definecolor{currentstroke}{rgb}{0.298039,0.447059,0.690196}%
\pgfsetstrokecolor{currentstroke}%
\pgfsetdash{}{0pt}%
\pgfsys@defobject{currentmarker}{\pgfqpoint{-0.046999in}{-0.046999in}}{\pgfqpoint{0.046999in}{0.046999in}}{%
\pgfpathmoveto{\pgfqpoint{0.000000in}{-0.046999in}}%
\pgfpathcurveto{\pgfqpoint{0.012464in}{-0.046999in}}{\pgfqpoint{0.024420in}{-0.042047in}}{\pgfqpoint{0.033234in}{-0.033234in}}%
\pgfpathcurveto{\pgfqpoint{0.042047in}{-0.024420in}}{\pgfqpoint{0.046999in}{-0.012464in}}{\pgfqpoint{0.046999in}{0.000000in}}%
\pgfpathcurveto{\pgfqpoint{0.046999in}{0.012464in}}{\pgfqpoint{0.042047in}{0.024420in}}{\pgfqpoint{0.033234in}{0.033234in}}%
\pgfpathcurveto{\pgfqpoint{0.024420in}{0.042047in}}{\pgfqpoint{0.012464in}{0.046999in}}{\pgfqpoint{0.000000in}{0.046999in}}%
\pgfpathcurveto{\pgfqpoint{-0.012464in}{0.046999in}}{\pgfqpoint{-0.024420in}{0.042047in}}{\pgfqpoint{-0.033234in}{0.033234in}}%
\pgfpathcurveto{\pgfqpoint{-0.042047in}{0.024420in}}{\pgfqpoint{-0.046999in}{0.012464in}}{\pgfqpoint{-0.046999in}{0.000000in}}%
\pgfpathcurveto{\pgfqpoint{-0.046999in}{-0.012464in}}{\pgfqpoint{-0.042047in}{-0.024420in}}{\pgfqpoint{-0.033234in}{-0.033234in}}%
\pgfpathcurveto{\pgfqpoint{-0.024420in}{-0.042047in}}{\pgfqpoint{-0.012464in}{-0.046999in}}{\pgfqpoint{0.000000in}{-0.046999in}}%
\pgfpathlineto{\pgfqpoint{0.000000in}{-0.046999in}}%
\pgfpathclose%
\pgfusepath{stroke,fill}%
}%
\begin{pgfscope}%
\pgfsys@transformshift{1.260533in}{6.606436in}%
\pgfsys@useobject{currentmarker}{}%
\end{pgfscope}%
\begin{pgfscope}%
\pgfsys@transformshift{2.446137in}{6.141339in}%
\pgfsys@useobject{currentmarker}{}%
\end{pgfscope}%
\begin{pgfscope}%
\pgfsys@transformshift{3.631742in}{5.211143in}%
\pgfsys@useobject{currentmarker}{}%
\end{pgfscope}%
\begin{pgfscope}%
\pgfsys@transformshift{4.817346in}{6.187848in}%
\pgfsys@useobject{currentmarker}{}%
\end{pgfscope}%
\end{pgfscope}%
\begin{pgfscope}%
\pgfpathrectangle{\pgfqpoint{0.667731in}{4.015734in}}{\pgfqpoint{4.742417in}{2.714069in}}%
\pgfusepath{clip}%
\pgfsetbuttcap%
\pgfsetroundjoin%
\pgfsetlinewidth{2.710125pt}%
\definecolor{currentstroke}{rgb}{0.866667,0.517647,0.321569}%
\pgfsetstrokecolor{currentstroke}%
\pgfsetdash{{9.990000pt}{4.320000pt}}{0.000000pt}%
\pgfpathmoveto{\pgfqpoint{1.260533in}{5.748086in}}%
\pgfpathlineto{\pgfqpoint{2.446137in}{4.807854in}}%
\pgfpathlineto{\pgfqpoint{3.631742in}{4.139101in}}%
\pgfpathlineto{\pgfqpoint{4.817346in}{4.980048in}}%
\pgfusepath{stroke}%
\end{pgfscope}%
\begin{pgfscope}%
\pgfpathrectangle{\pgfqpoint{0.667731in}{4.015734in}}{\pgfqpoint{4.742417in}{2.714069in}}%
\pgfusepath{clip}%
\pgfsetroundcap%
\pgfsetroundjoin%
\pgfsetlinewidth{2.710125pt}%
\definecolor{currentstroke}{rgb}{0.866667,0.517647,0.321569}%
\pgfsetstrokecolor{currentstroke}%
\pgfsetdash{}{0pt}%
\pgfusepath{stroke}%
\end{pgfscope}%
\begin{pgfscope}%
\pgfpathrectangle{\pgfqpoint{0.667731in}{4.015734in}}{\pgfqpoint{4.742417in}{2.714069in}}%
\pgfusepath{clip}%
\pgfsetroundcap%
\pgfsetroundjoin%
\pgfsetlinewidth{2.710125pt}%
\definecolor{currentstroke}{rgb}{0.866667,0.517647,0.321569}%
\pgfsetstrokecolor{currentstroke}%
\pgfsetdash{}{0pt}%
\pgfusepath{stroke}%
\end{pgfscope}%
\begin{pgfscope}%
\pgfpathrectangle{\pgfqpoint{0.667731in}{4.015734in}}{\pgfqpoint{4.742417in}{2.714069in}}%
\pgfusepath{clip}%
\pgfsetroundcap%
\pgfsetroundjoin%
\pgfsetlinewidth{2.710125pt}%
\definecolor{currentstroke}{rgb}{0.866667,0.517647,0.321569}%
\pgfsetstrokecolor{currentstroke}%
\pgfsetdash{}{0pt}%
\pgfusepath{stroke}%
\end{pgfscope}%
\begin{pgfscope}%
\pgfpathrectangle{\pgfqpoint{0.667731in}{4.015734in}}{\pgfqpoint{4.742417in}{2.714069in}}%
\pgfusepath{clip}%
\pgfsetroundcap%
\pgfsetroundjoin%
\pgfsetlinewidth{2.710125pt}%
\definecolor{currentstroke}{rgb}{0.866667,0.517647,0.321569}%
\pgfsetstrokecolor{currentstroke}%
\pgfsetdash{}{0pt}%
\pgfusepath{stroke}%
\end{pgfscope}%
\begin{pgfscope}%
\pgfpathrectangle{\pgfqpoint{0.667731in}{4.015734in}}{\pgfqpoint{4.742417in}{2.714069in}}%
\pgfusepath{clip}%
\pgfsetbuttcap%
\pgfsetroundjoin%
\definecolor{currentfill}{rgb}{0.866667,0.517647,0.321569}%
\pgfsetfillcolor{currentfill}%
\pgfsetlinewidth{2.032594pt}%
\definecolor{currentstroke}{rgb}{0.866667,0.517647,0.321569}%
\pgfsetstrokecolor{currentstroke}%
\pgfsetdash{}{0pt}%
\pgfsys@defobject{currentmarker}{\pgfqpoint{-0.046999in}{-0.046999in}}{\pgfqpoint{0.046999in}{0.046999in}}{%
\pgfpathmoveto{\pgfqpoint{0.000000in}{-0.046999in}}%
\pgfpathcurveto{\pgfqpoint{0.012464in}{-0.046999in}}{\pgfqpoint{0.024420in}{-0.042047in}}{\pgfqpoint{0.033234in}{-0.033234in}}%
\pgfpathcurveto{\pgfqpoint{0.042047in}{-0.024420in}}{\pgfqpoint{0.046999in}{-0.012464in}}{\pgfqpoint{0.046999in}{0.000000in}}%
\pgfpathcurveto{\pgfqpoint{0.046999in}{0.012464in}}{\pgfqpoint{0.042047in}{0.024420in}}{\pgfqpoint{0.033234in}{0.033234in}}%
\pgfpathcurveto{\pgfqpoint{0.024420in}{0.042047in}}{\pgfqpoint{0.012464in}{0.046999in}}{\pgfqpoint{0.000000in}{0.046999in}}%
\pgfpathcurveto{\pgfqpoint{-0.012464in}{0.046999in}}{\pgfqpoint{-0.024420in}{0.042047in}}{\pgfqpoint{-0.033234in}{0.033234in}}%
\pgfpathcurveto{\pgfqpoint{-0.042047in}{0.024420in}}{\pgfqpoint{-0.046999in}{0.012464in}}{\pgfqpoint{-0.046999in}{0.000000in}}%
\pgfpathcurveto{\pgfqpoint{-0.046999in}{-0.012464in}}{\pgfqpoint{-0.042047in}{-0.024420in}}{\pgfqpoint{-0.033234in}{-0.033234in}}%
\pgfpathcurveto{\pgfqpoint{-0.024420in}{-0.042047in}}{\pgfqpoint{-0.012464in}{-0.046999in}}{\pgfqpoint{0.000000in}{-0.046999in}}%
\pgfpathlineto{\pgfqpoint{0.000000in}{-0.046999in}}%
\pgfpathclose%
\pgfusepath{stroke,fill}%
}%
\begin{pgfscope}%
\pgfsys@transformshift{1.260533in}{5.748086in}%
\pgfsys@useobject{currentmarker}{}%
\end{pgfscope}%
\begin{pgfscope}%
\pgfsys@transformshift{2.446137in}{4.807854in}%
\pgfsys@useobject{currentmarker}{}%
\end{pgfscope}%
\begin{pgfscope}%
\pgfsys@transformshift{3.631742in}{4.139101in}%
\pgfsys@useobject{currentmarker}{}%
\end{pgfscope}%
\begin{pgfscope}%
\pgfsys@transformshift{4.817346in}{4.980048in}%
\pgfsys@useobject{currentmarker}{}%
\end{pgfscope}%
\end{pgfscope}%
\begin{pgfscope}%
\pgfsetrectcap%
\pgfsetmiterjoin%
\pgfsetlinewidth{1.254687pt}%
\definecolor{currentstroke}{rgb}{1.000000,1.000000,1.000000}%
\pgfsetstrokecolor{currentstroke}%
\pgfsetdash{}{0pt}%
\pgfpathmoveto{\pgfqpoint{0.667731in}{4.015734in}}%
\pgfpathlineto{\pgfqpoint{0.667731in}{6.729803in}}%
\pgfusepath{stroke}%
\end{pgfscope}%
\begin{pgfscope}%
\pgfsetrectcap%
\pgfsetmiterjoin%
\pgfsetlinewidth{1.254687pt}%
\definecolor{currentstroke}{rgb}{1.000000,1.000000,1.000000}%
\pgfsetstrokecolor{currentstroke}%
\pgfsetdash{}{0pt}%
\pgfpathmoveto{\pgfqpoint{5.410148in}{4.015734in}}%
\pgfpathlineto{\pgfqpoint{5.410148in}{6.729803in}}%
\pgfusepath{stroke}%
\end{pgfscope}%
\begin{pgfscope}%
\pgfsetrectcap%
\pgfsetmiterjoin%
\pgfsetlinewidth{1.254687pt}%
\definecolor{currentstroke}{rgb}{1.000000,1.000000,1.000000}%
\pgfsetstrokecolor{currentstroke}%
\pgfsetdash{}{0pt}%
\pgfpathmoveto{\pgfqpoint{0.667731in}{4.015734in}}%
\pgfpathlineto{\pgfqpoint{5.410148in}{4.015734in}}%
\pgfusepath{stroke}%
\end{pgfscope}%
\begin{pgfscope}%
\pgfsetrectcap%
\pgfsetmiterjoin%
\pgfsetlinewidth{1.254687pt}%
\definecolor{currentstroke}{rgb}{1.000000,1.000000,1.000000}%
\pgfsetstrokecolor{currentstroke}%
\pgfsetdash{}{0pt}%
\pgfpathmoveto{\pgfqpoint{0.667731in}{6.729803in}}%
\pgfpathlineto{\pgfqpoint{5.410148in}{6.729803in}}%
\pgfusepath{stroke}%
\end{pgfscope}%
\begin{pgfscope}%
\pgfsetbuttcap%
\pgfsetmiterjoin%
\definecolor{currentfill}{rgb}{0.917647,0.917647,0.949020}%
\pgfsetfillcolor{currentfill}%
\pgfsetfillopacity{0.800000}%
\pgfsetlinewidth{1.003750pt}%
\definecolor{currentstroke}{rgb}{0.800000,0.800000,0.800000}%
\pgfsetstrokecolor{currentstroke}%
\pgfsetstrokeopacity{0.800000}%
\pgfsetdash{}{0pt}%
\pgfpathmoveto{\pgfqpoint{0.774675in}{4.092123in}}%
\pgfpathlineto{\pgfqpoint{1.866407in}{4.092123in}}%
\pgfpathquadraticcurveto{\pgfqpoint{1.896963in}{4.092123in}}{\pgfqpoint{1.896963in}{4.122679in}}%
\pgfpathlineto{\pgfqpoint{1.896963in}{4.757747in}}%
\pgfpathquadraticcurveto{\pgfqpoint{1.896963in}{4.788303in}}{\pgfqpoint{1.866407in}{4.788303in}}%
\pgfpathlineto{\pgfqpoint{0.774675in}{4.788303in}}%
\pgfpathquadraticcurveto{\pgfqpoint{0.744120in}{4.788303in}}{\pgfqpoint{0.744120in}{4.757747in}}%
\pgfpathlineto{\pgfqpoint{0.744120in}{4.122679in}}%
\pgfpathquadraticcurveto{\pgfqpoint{0.744120in}{4.092123in}}{\pgfqpoint{0.774675in}{4.092123in}}%
\pgfpathlineto{\pgfqpoint{0.774675in}{4.092123in}}%
\pgfpathclose%
\pgfusepath{stroke,fill}%
\end{pgfscope}%
\begin{pgfscope}%
\definecolor{textcolor}{rgb}{0.150000,0.150000,0.150000}%
\pgfsetstrokecolor{textcolor}%
\pgfsetfillcolor{textcolor}%
\pgftext[x=1.076929in,y=4.611452in,left,base]{\color{textcolor}\sffamily\fontsize{12.000000}{14.400000}\selectfont F-score}%
\end{pgfscope}%
\begin{pgfscope}%
\pgfsetbuttcap%
\pgfsetroundjoin%
\definecolor{currentfill}{rgb}{0.298039,0.447059,0.690196}%
\pgfsetfillcolor{currentfill}%
\pgfsetlinewidth{2.032594pt}%
\definecolor{currentstroke}{rgb}{0.298039,0.447059,0.690196}%
\pgfsetstrokecolor{currentstroke}%
\pgfsetdash{}{0pt}%
\pgfsys@defobject{currentmarker}{\pgfqpoint{-0.046999in}{-0.046999in}}{\pgfqpoint{0.046999in}{0.046999in}}{%
\pgfpathmoveto{\pgfqpoint{0.000000in}{-0.046999in}}%
\pgfpathcurveto{\pgfqpoint{0.012464in}{-0.046999in}}{\pgfqpoint{0.024420in}{-0.042047in}}{\pgfqpoint{0.033234in}{-0.033234in}}%
\pgfpathcurveto{\pgfqpoint{0.042047in}{-0.024420in}}{\pgfqpoint{0.046999in}{-0.012464in}}{\pgfqpoint{0.046999in}{0.000000in}}%
\pgfpathcurveto{\pgfqpoint{0.046999in}{0.012464in}}{\pgfqpoint{0.042047in}{0.024420in}}{\pgfqpoint{0.033234in}{0.033234in}}%
\pgfpathcurveto{\pgfqpoint{0.024420in}{0.042047in}}{\pgfqpoint{0.012464in}{0.046999in}}{\pgfqpoint{0.000000in}{0.046999in}}%
\pgfpathcurveto{\pgfqpoint{-0.012464in}{0.046999in}}{\pgfqpoint{-0.024420in}{0.042047in}}{\pgfqpoint{-0.033234in}{0.033234in}}%
\pgfpathcurveto{\pgfqpoint{-0.042047in}{0.024420in}}{\pgfqpoint{-0.046999in}{0.012464in}}{\pgfqpoint{-0.046999in}{0.000000in}}%
\pgfpathcurveto{\pgfqpoint{-0.046999in}{-0.012464in}}{\pgfqpoint{-0.042047in}{-0.024420in}}{\pgfqpoint{-0.033234in}{-0.033234in}}%
\pgfpathcurveto{\pgfqpoint{-0.024420in}{-0.042047in}}{\pgfqpoint{-0.012464in}{-0.046999in}}{\pgfqpoint{0.000000in}{-0.046999in}}%
\pgfpathlineto{\pgfqpoint{0.000000in}{-0.046999in}}%
\pgfpathclose%
\pgfusepath{stroke,fill}%
}%
\begin{pgfscope}%
\pgfsys@transformshift{0.958009in}{4.435815in}%
\pgfsys@useobject{currentmarker}{}%
\end{pgfscope}%
\end{pgfscope}%
\begin{pgfscope}%
\definecolor{textcolor}{rgb}{0.150000,0.150000,0.150000}%
\pgfsetstrokecolor{textcolor}%
\pgfsetfillcolor{textcolor}%
\pgftext[x=1.233009in,y=4.395711in,left,base]{\color{textcolor}\sffamily\fontsize{11.000000}{13.200000}\selectfont F1 micro}%
\end{pgfscope}%
\begin{pgfscope}%
\pgfsetbuttcap%
\pgfsetroundjoin%
\definecolor{currentfill}{rgb}{0.866667,0.517647,0.321569}%
\pgfsetfillcolor{currentfill}%
\pgfsetlinewidth{2.032594pt}%
\definecolor{currentstroke}{rgb}{0.866667,0.517647,0.321569}%
\pgfsetstrokecolor{currentstroke}%
\pgfsetdash{}{0pt}%
\pgfsys@defobject{currentmarker}{\pgfqpoint{-0.046999in}{-0.046999in}}{\pgfqpoint{0.046999in}{0.046999in}}{%
\pgfpathmoveto{\pgfqpoint{0.000000in}{-0.046999in}}%
\pgfpathcurveto{\pgfqpoint{0.012464in}{-0.046999in}}{\pgfqpoint{0.024420in}{-0.042047in}}{\pgfqpoint{0.033234in}{-0.033234in}}%
\pgfpathcurveto{\pgfqpoint{0.042047in}{-0.024420in}}{\pgfqpoint{0.046999in}{-0.012464in}}{\pgfqpoint{0.046999in}{0.000000in}}%
\pgfpathcurveto{\pgfqpoint{0.046999in}{0.012464in}}{\pgfqpoint{0.042047in}{0.024420in}}{\pgfqpoint{0.033234in}{0.033234in}}%
\pgfpathcurveto{\pgfqpoint{0.024420in}{0.042047in}}{\pgfqpoint{0.012464in}{0.046999in}}{\pgfqpoint{0.000000in}{0.046999in}}%
\pgfpathcurveto{\pgfqpoint{-0.012464in}{0.046999in}}{\pgfqpoint{-0.024420in}{0.042047in}}{\pgfqpoint{-0.033234in}{0.033234in}}%
\pgfpathcurveto{\pgfqpoint{-0.042047in}{0.024420in}}{\pgfqpoint{-0.046999in}{0.012464in}}{\pgfqpoint{-0.046999in}{0.000000in}}%
\pgfpathcurveto{\pgfqpoint{-0.046999in}{-0.012464in}}{\pgfqpoint{-0.042047in}{-0.024420in}}{\pgfqpoint{-0.033234in}{-0.033234in}}%
\pgfpathcurveto{\pgfqpoint{-0.024420in}{-0.042047in}}{\pgfqpoint{-0.012464in}{-0.046999in}}{\pgfqpoint{0.000000in}{-0.046999in}}%
\pgfpathlineto{\pgfqpoint{0.000000in}{-0.046999in}}%
\pgfpathclose%
\pgfusepath{stroke,fill}%
}%
\begin{pgfscope}%
\pgfsys@transformshift{0.958009in}{4.222910in}%
\pgfsys@useobject{currentmarker}{}%
\end{pgfscope}%
\end{pgfscope}%
\begin{pgfscope}%
\definecolor{textcolor}{rgb}{0.150000,0.150000,0.150000}%
\pgfsetstrokecolor{textcolor}%
\pgfsetfillcolor{textcolor}%
\pgftext[x=1.233009in,y=4.182806in,left,base]{\color{textcolor}\sffamily\fontsize{11.000000}{13.200000}\selectfont F1 macro}%
\end{pgfscope}%
\begin{pgfscope}%
\pgfsetbuttcap%
\pgfsetmiterjoin%
\definecolor{currentfill}{rgb}{0.917647,0.917647,0.949020}%
\pgfsetfillcolor{currentfill}%
\pgfsetlinewidth{0.000000pt}%
\definecolor{currentstroke}{rgb}{0.000000,0.000000,0.000000}%
\pgfsetstrokecolor{currentstroke}%
\pgfsetstrokeopacity{0.000000}%
\pgfsetdash{}{0pt}%
\pgfpathmoveto{\pgfqpoint{0.667731in}{0.650833in}}%
\pgfpathlineto{\pgfqpoint{5.410148in}{0.650833in}}%
\pgfpathlineto{\pgfqpoint{5.410148in}{3.364902in}}%
\pgfpathlineto{\pgfqpoint{0.667731in}{3.364902in}}%
\pgfpathlineto{\pgfqpoint{0.667731in}{0.650833in}}%
\pgfpathclose%
\pgfusepath{fill}%
\end{pgfscope}%
\begin{pgfscope}%
\definecolor{textcolor}{rgb}{0.150000,0.150000,0.150000}%
\pgfsetstrokecolor{textcolor}%
\pgfsetfillcolor{textcolor}%
\pgftext[x=1.141973in,y=0.518888in,,top]{\color{textcolor}\sffamily\fontsize{11.000000}{13.200000}\selectfont 1}%
\end{pgfscope}%
\begin{pgfscope}%
\definecolor{textcolor}{rgb}{0.150000,0.150000,0.150000}%
\pgfsetstrokecolor{textcolor}%
\pgfsetfillcolor{textcolor}%
\pgftext[x=2.090456in,y=0.518888in,,top]{\color{textcolor}\sffamily\fontsize{11.000000}{13.200000}\selectfont 5}%
\end{pgfscope}%
\begin{pgfscope}%
\definecolor{textcolor}{rgb}{0.150000,0.150000,0.150000}%
\pgfsetstrokecolor{textcolor}%
\pgfsetfillcolor{textcolor}%
\pgftext[x=3.038940in,y=0.518888in,,top]{\color{textcolor}\sffamily\fontsize{11.000000}{13.200000}\selectfont 10}%
\end{pgfscope}%
\begin{pgfscope}%
\definecolor{textcolor}{rgb}{0.150000,0.150000,0.150000}%
\pgfsetstrokecolor{textcolor}%
\pgfsetfillcolor{textcolor}%
\pgftext[x=3.987423in,y=0.518888in,,top]{\color{textcolor}\sffamily\fontsize{11.000000}{13.200000}\selectfont 15}%
\end{pgfscope}%
\begin{pgfscope}%
\definecolor{textcolor}{rgb}{0.150000,0.150000,0.150000}%
\pgfsetstrokecolor{textcolor}%
\pgfsetfillcolor{textcolor}%
\pgftext[x=4.935906in,y=0.518888in,,top]{\color{textcolor}\sffamily\fontsize{11.000000}{13.200000}\selectfont 20}%
\end{pgfscope}%
\begin{pgfscope}%
\definecolor{textcolor}{rgb}{0.150000,0.150000,0.150000}%
\pgfsetstrokecolor{textcolor}%
\pgfsetfillcolor{textcolor}%
\pgftext[x=3.038940in,y=0.328148in,,top]{\color{textcolor}\sffamily\fontsize{12.000000}{14.400000}\selectfont window size}%
\end{pgfscope}%
\begin{pgfscope}%
\pgfpathrectangle{\pgfqpoint{0.667731in}{0.650833in}}{\pgfqpoint{4.742417in}{2.714069in}}%
\pgfusepath{clip}%
\pgfsetroundcap%
\pgfsetroundjoin%
\pgfsetlinewidth{1.003750pt}%
\definecolor{currentstroke}{rgb}{1.000000,1.000000,1.000000}%
\pgfsetstrokecolor{currentstroke}%
\pgfsetdash{}{0pt}%
\pgfpathmoveto{\pgfqpoint{0.667731in}{0.776442in}}%
\pgfpathlineto{\pgfqpoint{5.410148in}{0.776442in}}%
\pgfusepath{stroke}%
\end{pgfscope}%
\begin{pgfscope}%
\definecolor{textcolor}{rgb}{0.150000,0.150000,0.150000}%
\pgfsetstrokecolor{textcolor}%
\pgfsetfillcolor{textcolor}%
\pgftext[x=0.383703in, y=0.723635in, left, base]{\color{textcolor}\sffamily\fontsize{11.000000}{13.200000}\selectfont \(\displaystyle {63}\)}%
\end{pgfscope}%
\begin{pgfscope}%
\pgfpathrectangle{\pgfqpoint{0.667731in}{0.650833in}}{\pgfqpoint{4.742417in}{2.714069in}}%
\pgfusepath{clip}%
\pgfsetroundcap%
\pgfsetroundjoin%
\pgfsetlinewidth{1.003750pt}%
\definecolor{currentstroke}{rgb}{1.000000,1.000000,1.000000}%
\pgfsetstrokecolor{currentstroke}%
\pgfsetdash{}{0pt}%
\pgfpathmoveto{\pgfqpoint{0.667731in}{1.177209in}}%
\pgfpathlineto{\pgfqpoint{5.410148in}{1.177209in}}%
\pgfusepath{stroke}%
\end{pgfscope}%
\begin{pgfscope}%
\definecolor{textcolor}{rgb}{0.150000,0.150000,0.150000}%
\pgfsetstrokecolor{textcolor}%
\pgfsetfillcolor{textcolor}%
\pgftext[x=0.383703in, y=1.124402in, left, base]{\color{textcolor}\sffamily\fontsize{11.000000}{13.200000}\selectfont \(\displaystyle {64}\)}%
\end{pgfscope}%
\begin{pgfscope}%
\pgfpathrectangle{\pgfqpoint{0.667731in}{0.650833in}}{\pgfqpoint{4.742417in}{2.714069in}}%
\pgfusepath{clip}%
\pgfsetroundcap%
\pgfsetroundjoin%
\pgfsetlinewidth{1.003750pt}%
\definecolor{currentstroke}{rgb}{1.000000,1.000000,1.000000}%
\pgfsetstrokecolor{currentstroke}%
\pgfsetdash{}{0pt}%
\pgfpathmoveto{\pgfqpoint{0.667731in}{1.577975in}}%
\pgfpathlineto{\pgfqpoint{5.410148in}{1.577975in}}%
\pgfusepath{stroke}%
\end{pgfscope}%
\begin{pgfscope}%
\definecolor{textcolor}{rgb}{0.150000,0.150000,0.150000}%
\pgfsetstrokecolor{textcolor}%
\pgfsetfillcolor{textcolor}%
\pgftext[x=0.383703in, y=1.525169in, left, base]{\color{textcolor}\sffamily\fontsize{11.000000}{13.200000}\selectfont \(\displaystyle {65}\)}%
\end{pgfscope}%
\begin{pgfscope}%
\pgfpathrectangle{\pgfqpoint{0.667731in}{0.650833in}}{\pgfqpoint{4.742417in}{2.714069in}}%
\pgfusepath{clip}%
\pgfsetroundcap%
\pgfsetroundjoin%
\pgfsetlinewidth{1.003750pt}%
\definecolor{currentstroke}{rgb}{1.000000,1.000000,1.000000}%
\pgfsetstrokecolor{currentstroke}%
\pgfsetdash{}{0pt}%
\pgfpathmoveto{\pgfqpoint{0.667731in}{1.978742in}}%
\pgfpathlineto{\pgfqpoint{5.410148in}{1.978742in}}%
\pgfusepath{stroke}%
\end{pgfscope}%
\begin{pgfscope}%
\definecolor{textcolor}{rgb}{0.150000,0.150000,0.150000}%
\pgfsetstrokecolor{textcolor}%
\pgfsetfillcolor{textcolor}%
\pgftext[x=0.383703in, y=1.925935in, left, base]{\color{textcolor}\sffamily\fontsize{11.000000}{13.200000}\selectfont \(\displaystyle {66}\)}%
\end{pgfscope}%
\begin{pgfscope}%
\pgfpathrectangle{\pgfqpoint{0.667731in}{0.650833in}}{\pgfqpoint{4.742417in}{2.714069in}}%
\pgfusepath{clip}%
\pgfsetroundcap%
\pgfsetroundjoin%
\pgfsetlinewidth{1.003750pt}%
\definecolor{currentstroke}{rgb}{1.000000,1.000000,1.000000}%
\pgfsetstrokecolor{currentstroke}%
\pgfsetdash{}{0pt}%
\pgfpathmoveto{\pgfqpoint{0.667731in}{2.379508in}}%
\pgfpathlineto{\pgfqpoint{5.410148in}{2.379508in}}%
\pgfusepath{stroke}%
\end{pgfscope}%
\begin{pgfscope}%
\definecolor{textcolor}{rgb}{0.150000,0.150000,0.150000}%
\pgfsetstrokecolor{textcolor}%
\pgfsetfillcolor{textcolor}%
\pgftext[x=0.383703in, y=2.326702in, left, base]{\color{textcolor}\sffamily\fontsize{11.000000}{13.200000}\selectfont \(\displaystyle {67}\)}%
\end{pgfscope}%
\begin{pgfscope}%
\pgfpathrectangle{\pgfqpoint{0.667731in}{0.650833in}}{\pgfqpoint{4.742417in}{2.714069in}}%
\pgfusepath{clip}%
\pgfsetroundcap%
\pgfsetroundjoin%
\pgfsetlinewidth{1.003750pt}%
\definecolor{currentstroke}{rgb}{1.000000,1.000000,1.000000}%
\pgfsetstrokecolor{currentstroke}%
\pgfsetdash{}{0pt}%
\pgfpathmoveto{\pgfqpoint{0.667731in}{2.780275in}}%
\pgfpathlineto{\pgfqpoint{5.410148in}{2.780275in}}%
\pgfusepath{stroke}%
\end{pgfscope}%
\begin{pgfscope}%
\definecolor{textcolor}{rgb}{0.150000,0.150000,0.150000}%
\pgfsetstrokecolor{textcolor}%
\pgfsetfillcolor{textcolor}%
\pgftext[x=0.383703in, y=2.727468in, left, base]{\color{textcolor}\sffamily\fontsize{11.000000}{13.200000}\selectfont \(\displaystyle {68}\)}%
\end{pgfscope}%
\begin{pgfscope}%
\pgfpathrectangle{\pgfqpoint{0.667731in}{0.650833in}}{\pgfqpoint{4.742417in}{2.714069in}}%
\pgfusepath{clip}%
\pgfsetroundcap%
\pgfsetroundjoin%
\pgfsetlinewidth{1.003750pt}%
\definecolor{currentstroke}{rgb}{1.000000,1.000000,1.000000}%
\pgfsetstrokecolor{currentstroke}%
\pgfsetdash{}{0pt}%
\pgfpathmoveto{\pgfqpoint{0.667731in}{3.181042in}}%
\pgfpathlineto{\pgfqpoint{5.410148in}{3.181042in}}%
\pgfusepath{stroke}%
\end{pgfscope}%
\begin{pgfscope}%
\definecolor{textcolor}{rgb}{0.150000,0.150000,0.150000}%
\pgfsetstrokecolor{textcolor}%
\pgfsetfillcolor{textcolor}%
\pgftext[x=0.383703in, y=3.128235in, left, base]{\color{textcolor}\sffamily\fontsize{11.000000}{13.200000}\selectfont \(\displaystyle {69}\)}%
\end{pgfscope}%
\begin{pgfscope}%
\definecolor{textcolor}{rgb}{0.150000,0.150000,0.150000}%
\pgfsetstrokecolor{textcolor}%
\pgfsetfillcolor{textcolor}%
\pgftext[x=0.328148in,y=2.007867in,,bottom,rotate=90.000000]{\color{textcolor}\sffamily\fontsize{12.000000}{14.400000}\selectfont Average score}%
\end{pgfscope}%
\begin{pgfscope}%
\pgfpathrectangle{\pgfqpoint{0.667731in}{0.650833in}}{\pgfqpoint{4.742417in}{2.714069in}}%
\pgfusepath{clip}%
\pgfsetbuttcap%
\pgfsetroundjoin%
\pgfsetlinewidth{2.710125pt}%
\definecolor{currentstroke}{rgb}{0.298039,0.447059,0.690196}%
\pgfsetstrokecolor{currentstroke}%
\pgfsetdash{{9.990000pt}{4.320000pt}}{0.000000pt}%
\pgfpathmoveto{\pgfqpoint{1.141973in}{2.939069in}}%
\pgfpathlineto{\pgfqpoint{2.090456in}{3.241535in}}%
\pgfpathlineto{\pgfqpoint{3.038940in}{2.031673in}}%
\pgfpathlineto{\pgfqpoint{3.987423in}{2.107290in}}%
\pgfpathlineto{\pgfqpoint{4.935906in}{2.334139in}}%
\pgfusepath{stroke}%
\end{pgfscope}%
\begin{pgfscope}%
\pgfpathrectangle{\pgfqpoint{0.667731in}{0.650833in}}{\pgfqpoint{4.742417in}{2.714069in}}%
\pgfusepath{clip}%
\pgfsetroundcap%
\pgfsetroundjoin%
\pgfsetlinewidth{2.710125pt}%
\definecolor{currentstroke}{rgb}{0.298039,0.447059,0.690196}%
\pgfsetstrokecolor{currentstroke}%
\pgfsetdash{}{0pt}%
\pgfusepath{stroke}%
\end{pgfscope}%
\begin{pgfscope}%
\pgfpathrectangle{\pgfqpoint{0.667731in}{0.650833in}}{\pgfqpoint{4.742417in}{2.714069in}}%
\pgfusepath{clip}%
\pgfsetroundcap%
\pgfsetroundjoin%
\pgfsetlinewidth{2.710125pt}%
\definecolor{currentstroke}{rgb}{0.298039,0.447059,0.690196}%
\pgfsetstrokecolor{currentstroke}%
\pgfsetdash{}{0pt}%
\pgfusepath{stroke}%
\end{pgfscope}%
\begin{pgfscope}%
\pgfpathrectangle{\pgfqpoint{0.667731in}{0.650833in}}{\pgfqpoint{4.742417in}{2.714069in}}%
\pgfusepath{clip}%
\pgfsetroundcap%
\pgfsetroundjoin%
\pgfsetlinewidth{2.710125pt}%
\definecolor{currentstroke}{rgb}{0.298039,0.447059,0.690196}%
\pgfsetstrokecolor{currentstroke}%
\pgfsetdash{}{0pt}%
\pgfusepath{stroke}%
\end{pgfscope}%
\begin{pgfscope}%
\pgfpathrectangle{\pgfqpoint{0.667731in}{0.650833in}}{\pgfqpoint{4.742417in}{2.714069in}}%
\pgfusepath{clip}%
\pgfsetroundcap%
\pgfsetroundjoin%
\pgfsetlinewidth{2.710125pt}%
\definecolor{currentstroke}{rgb}{0.298039,0.447059,0.690196}%
\pgfsetstrokecolor{currentstroke}%
\pgfsetdash{}{0pt}%
\pgfusepath{stroke}%
\end{pgfscope}%
\begin{pgfscope}%
\pgfpathrectangle{\pgfqpoint{0.667731in}{0.650833in}}{\pgfqpoint{4.742417in}{2.714069in}}%
\pgfusepath{clip}%
\pgfsetroundcap%
\pgfsetroundjoin%
\pgfsetlinewidth{2.710125pt}%
\definecolor{currentstroke}{rgb}{0.298039,0.447059,0.690196}%
\pgfsetstrokecolor{currentstroke}%
\pgfsetdash{}{0pt}%
\pgfusepath{stroke}%
\end{pgfscope}%
\begin{pgfscope}%
\pgfpathrectangle{\pgfqpoint{0.667731in}{0.650833in}}{\pgfqpoint{4.742417in}{2.714069in}}%
\pgfusepath{clip}%
\pgfsetbuttcap%
\pgfsetroundjoin%
\definecolor{currentfill}{rgb}{0.298039,0.447059,0.690196}%
\pgfsetfillcolor{currentfill}%
\pgfsetlinewidth{2.032594pt}%
\definecolor{currentstroke}{rgb}{0.298039,0.447059,0.690196}%
\pgfsetstrokecolor{currentstroke}%
\pgfsetdash{}{0pt}%
\pgfsys@defobject{currentmarker}{\pgfqpoint{-0.046999in}{-0.046999in}}{\pgfqpoint{0.046999in}{0.046999in}}{%
\pgfpathmoveto{\pgfqpoint{0.000000in}{-0.046999in}}%
\pgfpathcurveto{\pgfqpoint{0.012464in}{-0.046999in}}{\pgfqpoint{0.024420in}{-0.042047in}}{\pgfqpoint{0.033234in}{-0.033234in}}%
\pgfpathcurveto{\pgfqpoint{0.042047in}{-0.024420in}}{\pgfqpoint{0.046999in}{-0.012464in}}{\pgfqpoint{0.046999in}{0.000000in}}%
\pgfpathcurveto{\pgfqpoint{0.046999in}{0.012464in}}{\pgfqpoint{0.042047in}{0.024420in}}{\pgfqpoint{0.033234in}{0.033234in}}%
\pgfpathcurveto{\pgfqpoint{0.024420in}{0.042047in}}{\pgfqpoint{0.012464in}{0.046999in}}{\pgfqpoint{0.000000in}{0.046999in}}%
\pgfpathcurveto{\pgfqpoint{-0.012464in}{0.046999in}}{\pgfqpoint{-0.024420in}{0.042047in}}{\pgfqpoint{-0.033234in}{0.033234in}}%
\pgfpathcurveto{\pgfqpoint{-0.042047in}{0.024420in}}{\pgfqpoint{-0.046999in}{0.012464in}}{\pgfqpoint{-0.046999in}{0.000000in}}%
\pgfpathcurveto{\pgfqpoint{-0.046999in}{-0.012464in}}{\pgfqpoint{-0.042047in}{-0.024420in}}{\pgfqpoint{-0.033234in}{-0.033234in}}%
\pgfpathcurveto{\pgfqpoint{-0.024420in}{-0.042047in}}{\pgfqpoint{-0.012464in}{-0.046999in}}{\pgfqpoint{0.000000in}{-0.046999in}}%
\pgfpathlineto{\pgfqpoint{0.000000in}{-0.046999in}}%
\pgfpathclose%
\pgfusepath{stroke,fill}%
}%
\begin{pgfscope}%
\pgfsys@transformshift{1.141973in}{2.939069in}%
\pgfsys@useobject{currentmarker}{}%
\end{pgfscope}%
\begin{pgfscope}%
\pgfsys@transformshift{2.090456in}{3.241535in}%
\pgfsys@useobject{currentmarker}{}%
\end{pgfscope}%
\begin{pgfscope}%
\pgfsys@transformshift{3.038940in}{2.031673in}%
\pgfsys@useobject{currentmarker}{}%
\end{pgfscope}%
\begin{pgfscope}%
\pgfsys@transformshift{3.987423in}{2.107290in}%
\pgfsys@useobject{currentmarker}{}%
\end{pgfscope}%
\begin{pgfscope}%
\pgfsys@transformshift{4.935906in}{2.334139in}%
\pgfsys@useobject{currentmarker}{}%
\end{pgfscope}%
\end{pgfscope}%
\begin{pgfscope}%
\pgfpathrectangle{\pgfqpoint{0.667731in}{0.650833in}}{\pgfqpoint{4.742417in}{2.714069in}}%
\pgfusepath{clip}%
\pgfsetbuttcap%
\pgfsetroundjoin%
\pgfsetlinewidth{2.710125pt}%
\definecolor{currentstroke}{rgb}{0.866667,0.517647,0.321569}%
\pgfsetstrokecolor{currentstroke}%
\pgfsetdash{{9.990000pt}{4.320000pt}}{0.000000pt}%
\pgfpathmoveto{\pgfqpoint{1.141973in}{1.797423in}}%
\pgfpathlineto{\pgfqpoint{2.090456in}{2.087194in}}%
\pgfpathlineto{\pgfqpoint{3.038940in}{0.863657in}}%
\pgfpathlineto{\pgfqpoint{3.987423in}{0.774199in}}%
\pgfpathlineto{\pgfqpoint{4.935906in}{1.093456in}}%
\pgfusepath{stroke}%
\end{pgfscope}%
\begin{pgfscope}%
\pgfpathrectangle{\pgfqpoint{0.667731in}{0.650833in}}{\pgfqpoint{4.742417in}{2.714069in}}%
\pgfusepath{clip}%
\pgfsetroundcap%
\pgfsetroundjoin%
\pgfsetlinewidth{2.710125pt}%
\definecolor{currentstroke}{rgb}{0.866667,0.517647,0.321569}%
\pgfsetstrokecolor{currentstroke}%
\pgfsetdash{}{0pt}%
\pgfusepath{stroke}%
\end{pgfscope}%
\begin{pgfscope}%
\pgfpathrectangle{\pgfqpoint{0.667731in}{0.650833in}}{\pgfqpoint{4.742417in}{2.714069in}}%
\pgfusepath{clip}%
\pgfsetroundcap%
\pgfsetroundjoin%
\pgfsetlinewidth{2.710125pt}%
\definecolor{currentstroke}{rgb}{0.866667,0.517647,0.321569}%
\pgfsetstrokecolor{currentstroke}%
\pgfsetdash{}{0pt}%
\pgfusepath{stroke}%
\end{pgfscope}%
\begin{pgfscope}%
\pgfpathrectangle{\pgfqpoint{0.667731in}{0.650833in}}{\pgfqpoint{4.742417in}{2.714069in}}%
\pgfusepath{clip}%
\pgfsetroundcap%
\pgfsetroundjoin%
\pgfsetlinewidth{2.710125pt}%
\definecolor{currentstroke}{rgb}{0.866667,0.517647,0.321569}%
\pgfsetstrokecolor{currentstroke}%
\pgfsetdash{}{0pt}%
\pgfusepath{stroke}%
\end{pgfscope}%
\begin{pgfscope}%
\pgfpathrectangle{\pgfqpoint{0.667731in}{0.650833in}}{\pgfqpoint{4.742417in}{2.714069in}}%
\pgfusepath{clip}%
\pgfsetroundcap%
\pgfsetroundjoin%
\pgfsetlinewidth{2.710125pt}%
\definecolor{currentstroke}{rgb}{0.866667,0.517647,0.321569}%
\pgfsetstrokecolor{currentstroke}%
\pgfsetdash{}{0pt}%
\pgfusepath{stroke}%
\end{pgfscope}%
\begin{pgfscope}%
\pgfpathrectangle{\pgfqpoint{0.667731in}{0.650833in}}{\pgfqpoint{4.742417in}{2.714069in}}%
\pgfusepath{clip}%
\pgfsetroundcap%
\pgfsetroundjoin%
\pgfsetlinewidth{2.710125pt}%
\definecolor{currentstroke}{rgb}{0.866667,0.517647,0.321569}%
\pgfsetstrokecolor{currentstroke}%
\pgfsetdash{}{0pt}%
\pgfusepath{stroke}%
\end{pgfscope}%
\begin{pgfscope}%
\pgfpathrectangle{\pgfqpoint{0.667731in}{0.650833in}}{\pgfqpoint{4.742417in}{2.714069in}}%
\pgfusepath{clip}%
\pgfsetbuttcap%
\pgfsetroundjoin%
\definecolor{currentfill}{rgb}{0.866667,0.517647,0.321569}%
\pgfsetfillcolor{currentfill}%
\pgfsetlinewidth{2.032594pt}%
\definecolor{currentstroke}{rgb}{0.866667,0.517647,0.321569}%
\pgfsetstrokecolor{currentstroke}%
\pgfsetdash{}{0pt}%
\pgfsys@defobject{currentmarker}{\pgfqpoint{-0.046999in}{-0.046999in}}{\pgfqpoint{0.046999in}{0.046999in}}{%
\pgfpathmoveto{\pgfqpoint{0.000000in}{-0.046999in}}%
\pgfpathcurveto{\pgfqpoint{0.012464in}{-0.046999in}}{\pgfqpoint{0.024420in}{-0.042047in}}{\pgfqpoint{0.033234in}{-0.033234in}}%
\pgfpathcurveto{\pgfqpoint{0.042047in}{-0.024420in}}{\pgfqpoint{0.046999in}{-0.012464in}}{\pgfqpoint{0.046999in}{0.000000in}}%
\pgfpathcurveto{\pgfqpoint{0.046999in}{0.012464in}}{\pgfqpoint{0.042047in}{0.024420in}}{\pgfqpoint{0.033234in}{0.033234in}}%
\pgfpathcurveto{\pgfqpoint{0.024420in}{0.042047in}}{\pgfqpoint{0.012464in}{0.046999in}}{\pgfqpoint{0.000000in}{0.046999in}}%
\pgfpathcurveto{\pgfqpoint{-0.012464in}{0.046999in}}{\pgfqpoint{-0.024420in}{0.042047in}}{\pgfqpoint{-0.033234in}{0.033234in}}%
\pgfpathcurveto{\pgfqpoint{-0.042047in}{0.024420in}}{\pgfqpoint{-0.046999in}{0.012464in}}{\pgfqpoint{-0.046999in}{0.000000in}}%
\pgfpathcurveto{\pgfqpoint{-0.046999in}{-0.012464in}}{\pgfqpoint{-0.042047in}{-0.024420in}}{\pgfqpoint{-0.033234in}{-0.033234in}}%
\pgfpathcurveto{\pgfqpoint{-0.024420in}{-0.042047in}}{\pgfqpoint{-0.012464in}{-0.046999in}}{\pgfqpoint{0.000000in}{-0.046999in}}%
\pgfpathlineto{\pgfqpoint{0.000000in}{-0.046999in}}%
\pgfpathclose%
\pgfusepath{stroke,fill}%
}%
\begin{pgfscope}%
\pgfsys@transformshift{1.141973in}{1.797423in}%
\pgfsys@useobject{currentmarker}{}%
\end{pgfscope}%
\begin{pgfscope}%
\pgfsys@transformshift{2.090456in}{2.087194in}%
\pgfsys@useobject{currentmarker}{}%
\end{pgfscope}%
\begin{pgfscope}%
\pgfsys@transformshift{3.038940in}{0.863657in}%
\pgfsys@useobject{currentmarker}{}%
\end{pgfscope}%
\begin{pgfscope}%
\pgfsys@transformshift{3.987423in}{0.774199in}%
\pgfsys@useobject{currentmarker}{}%
\end{pgfscope}%
\begin{pgfscope}%
\pgfsys@transformshift{4.935906in}{1.093456in}%
\pgfsys@useobject{currentmarker}{}%
\end{pgfscope}%
\end{pgfscope}%
\begin{pgfscope}%
\pgfsetrectcap%
\pgfsetmiterjoin%
\pgfsetlinewidth{1.254687pt}%
\definecolor{currentstroke}{rgb}{1.000000,1.000000,1.000000}%
\pgfsetstrokecolor{currentstroke}%
\pgfsetdash{}{0pt}%
\pgfpathmoveto{\pgfqpoint{0.667731in}{0.650833in}}%
\pgfpathlineto{\pgfqpoint{0.667731in}{3.364902in}}%
\pgfusepath{stroke}%
\end{pgfscope}%
\begin{pgfscope}%
\pgfsetrectcap%
\pgfsetmiterjoin%
\pgfsetlinewidth{1.254687pt}%
\definecolor{currentstroke}{rgb}{1.000000,1.000000,1.000000}%
\pgfsetstrokecolor{currentstroke}%
\pgfsetdash{}{0pt}%
\pgfpathmoveto{\pgfqpoint{5.410148in}{0.650833in}}%
\pgfpathlineto{\pgfqpoint{5.410148in}{3.364902in}}%
\pgfusepath{stroke}%
\end{pgfscope}%
\begin{pgfscope}%
\pgfsetrectcap%
\pgfsetmiterjoin%
\pgfsetlinewidth{1.254687pt}%
\definecolor{currentstroke}{rgb}{1.000000,1.000000,1.000000}%
\pgfsetstrokecolor{currentstroke}%
\pgfsetdash{}{0pt}%
\pgfpathmoveto{\pgfqpoint{0.667731in}{0.650833in}}%
\pgfpathlineto{\pgfqpoint{5.410148in}{0.650833in}}%
\pgfusepath{stroke}%
\end{pgfscope}%
\begin{pgfscope}%
\pgfsetrectcap%
\pgfsetmiterjoin%
\pgfsetlinewidth{1.254687pt}%
\definecolor{currentstroke}{rgb}{1.000000,1.000000,1.000000}%
\pgfsetstrokecolor{currentstroke}%
\pgfsetdash{}{0pt}%
\pgfpathmoveto{\pgfqpoint{0.667731in}{3.364902in}}%
\pgfpathlineto{\pgfqpoint{5.410148in}{3.364902in}}%
\pgfusepath{stroke}%
\end{pgfscope}%
\begin{pgfscope}%
\pgfsetbuttcap%
\pgfsetmiterjoin%
\definecolor{currentfill}{rgb}{0.917647,0.917647,0.949020}%
\pgfsetfillcolor{currentfill}%
\pgfsetfillopacity{0.800000}%
\pgfsetlinewidth{1.003750pt}%
\definecolor{currentstroke}{rgb}{0.800000,0.800000,0.800000}%
\pgfsetstrokecolor{currentstroke}%
\pgfsetstrokeopacity{0.800000}%
\pgfsetdash{}{0pt}%
\pgfpathmoveto{\pgfqpoint{4.211472in}{2.592333in}}%
\pgfpathlineto{\pgfqpoint{5.303204in}{2.592333in}}%
\pgfpathquadraticcurveto{\pgfqpoint{5.333759in}{2.592333in}}{\pgfqpoint{5.333759in}{2.622888in}}%
\pgfpathlineto{\pgfqpoint{5.333759in}{3.257957in}}%
\pgfpathquadraticcurveto{\pgfqpoint{5.333759in}{3.288513in}}{\pgfqpoint{5.303204in}{3.288513in}}%
\pgfpathlineto{\pgfqpoint{4.211472in}{3.288513in}}%
\pgfpathquadraticcurveto{\pgfqpoint{4.180916in}{3.288513in}}{\pgfqpoint{4.180916in}{3.257957in}}%
\pgfpathlineto{\pgfqpoint{4.180916in}{2.622888in}}%
\pgfpathquadraticcurveto{\pgfqpoint{4.180916in}{2.592333in}}{\pgfqpoint{4.211472in}{2.592333in}}%
\pgfpathlineto{\pgfqpoint{4.211472in}{2.592333in}}%
\pgfpathclose%
\pgfusepath{stroke,fill}%
\end{pgfscope}%
\begin{pgfscope}%
\definecolor{textcolor}{rgb}{0.150000,0.150000,0.150000}%
\pgfsetstrokecolor{textcolor}%
\pgfsetfillcolor{textcolor}%
\pgftext[x=4.513726in,y=3.111661in,left,base]{\color{textcolor}\sffamily\fontsize{12.000000}{14.400000}\selectfont F-score}%
\end{pgfscope}%
\begin{pgfscope}%
\pgfsetbuttcap%
\pgfsetroundjoin%
\definecolor{currentfill}{rgb}{0.298039,0.447059,0.690196}%
\pgfsetfillcolor{currentfill}%
\pgfsetlinewidth{2.032594pt}%
\definecolor{currentstroke}{rgb}{0.298039,0.447059,0.690196}%
\pgfsetstrokecolor{currentstroke}%
\pgfsetdash{}{0pt}%
\pgfsys@defobject{currentmarker}{\pgfqpoint{-0.046999in}{-0.046999in}}{\pgfqpoint{0.046999in}{0.046999in}}{%
\pgfpathmoveto{\pgfqpoint{0.000000in}{-0.046999in}}%
\pgfpathcurveto{\pgfqpoint{0.012464in}{-0.046999in}}{\pgfqpoint{0.024420in}{-0.042047in}}{\pgfqpoint{0.033234in}{-0.033234in}}%
\pgfpathcurveto{\pgfqpoint{0.042047in}{-0.024420in}}{\pgfqpoint{0.046999in}{-0.012464in}}{\pgfqpoint{0.046999in}{0.000000in}}%
\pgfpathcurveto{\pgfqpoint{0.046999in}{0.012464in}}{\pgfqpoint{0.042047in}{0.024420in}}{\pgfqpoint{0.033234in}{0.033234in}}%
\pgfpathcurveto{\pgfqpoint{0.024420in}{0.042047in}}{\pgfqpoint{0.012464in}{0.046999in}}{\pgfqpoint{0.000000in}{0.046999in}}%
\pgfpathcurveto{\pgfqpoint{-0.012464in}{0.046999in}}{\pgfqpoint{-0.024420in}{0.042047in}}{\pgfqpoint{-0.033234in}{0.033234in}}%
\pgfpathcurveto{\pgfqpoint{-0.042047in}{0.024420in}}{\pgfqpoint{-0.046999in}{0.012464in}}{\pgfqpoint{-0.046999in}{0.000000in}}%
\pgfpathcurveto{\pgfqpoint{-0.046999in}{-0.012464in}}{\pgfqpoint{-0.042047in}{-0.024420in}}{\pgfqpoint{-0.033234in}{-0.033234in}}%
\pgfpathcurveto{\pgfqpoint{-0.024420in}{-0.042047in}}{\pgfqpoint{-0.012464in}{-0.046999in}}{\pgfqpoint{0.000000in}{-0.046999in}}%
\pgfpathlineto{\pgfqpoint{0.000000in}{-0.046999in}}%
\pgfpathclose%
\pgfusepath{stroke,fill}%
}%
\begin{pgfscope}%
\pgfsys@transformshift{4.394805in}{2.936025in}%
\pgfsys@useobject{currentmarker}{}%
\end{pgfscope}%
\end{pgfscope}%
\begin{pgfscope}%
\definecolor{textcolor}{rgb}{0.150000,0.150000,0.150000}%
\pgfsetstrokecolor{textcolor}%
\pgfsetfillcolor{textcolor}%
\pgftext[x=4.669805in,y=2.895921in,left,base]{\color{textcolor}\sffamily\fontsize{11.000000}{13.200000}\selectfont F1 micro}%
\end{pgfscope}%
\begin{pgfscope}%
\pgfsetbuttcap%
\pgfsetroundjoin%
\definecolor{currentfill}{rgb}{0.866667,0.517647,0.321569}%
\pgfsetfillcolor{currentfill}%
\pgfsetlinewidth{2.032594pt}%
\definecolor{currentstroke}{rgb}{0.866667,0.517647,0.321569}%
\pgfsetstrokecolor{currentstroke}%
\pgfsetdash{}{0pt}%
\pgfsys@defobject{currentmarker}{\pgfqpoint{-0.046999in}{-0.046999in}}{\pgfqpoint{0.046999in}{0.046999in}}{%
\pgfpathmoveto{\pgfqpoint{0.000000in}{-0.046999in}}%
\pgfpathcurveto{\pgfqpoint{0.012464in}{-0.046999in}}{\pgfqpoint{0.024420in}{-0.042047in}}{\pgfqpoint{0.033234in}{-0.033234in}}%
\pgfpathcurveto{\pgfqpoint{0.042047in}{-0.024420in}}{\pgfqpoint{0.046999in}{-0.012464in}}{\pgfqpoint{0.046999in}{0.000000in}}%
\pgfpathcurveto{\pgfqpoint{0.046999in}{0.012464in}}{\pgfqpoint{0.042047in}{0.024420in}}{\pgfqpoint{0.033234in}{0.033234in}}%
\pgfpathcurveto{\pgfqpoint{0.024420in}{0.042047in}}{\pgfqpoint{0.012464in}{0.046999in}}{\pgfqpoint{0.000000in}{0.046999in}}%
\pgfpathcurveto{\pgfqpoint{-0.012464in}{0.046999in}}{\pgfqpoint{-0.024420in}{0.042047in}}{\pgfqpoint{-0.033234in}{0.033234in}}%
\pgfpathcurveto{\pgfqpoint{-0.042047in}{0.024420in}}{\pgfqpoint{-0.046999in}{0.012464in}}{\pgfqpoint{-0.046999in}{0.000000in}}%
\pgfpathcurveto{\pgfqpoint{-0.046999in}{-0.012464in}}{\pgfqpoint{-0.042047in}{-0.024420in}}{\pgfqpoint{-0.033234in}{-0.033234in}}%
\pgfpathcurveto{\pgfqpoint{-0.024420in}{-0.042047in}}{\pgfqpoint{-0.012464in}{-0.046999in}}{\pgfqpoint{0.000000in}{-0.046999in}}%
\pgfpathlineto{\pgfqpoint{0.000000in}{-0.046999in}}%
\pgfpathclose%
\pgfusepath{stroke,fill}%
}%
\begin{pgfscope}%
\pgfsys@transformshift{4.394805in}{2.723120in}%
\pgfsys@useobject{currentmarker}{}%
\end{pgfscope}%
\end{pgfscope}%
\begin{pgfscope}%
\definecolor{textcolor}{rgb}{0.150000,0.150000,0.150000}%
\pgfsetstrokecolor{textcolor}%
\pgfsetfillcolor{textcolor}%
\pgftext[x=4.669805in,y=2.683016in,left,base]{\color{textcolor}\sffamily\fontsize{11.000000}{13.200000}\selectfont F1 macro}%
\end{pgfscope}%
\end{pgfpicture}%
\makeatother%
\endgroup%
}
        \label{fig:node2vec:representation and window size}
    \end{subfigure}
    \hfill
    \begin{subfigure}[b]{0.49\textwidth}
        \centering
        \scalebox{.5}{%% Creator: Matplotlib, PGF backend
%%
%% To include the figure in your LaTeX document, write
%%   \input{<filename>.pgf}
%%
%% Make sure the required packages are loaded in your preamble
%%   \usepackage{pgf}
%%
%% Also ensure that all the required font packages are loaded; for instance,
%% the lmodern package is sometimes necessary when using math font.
%%   \usepackage{lmodern}
%%
%% Figures using additional raster images can only be included by \input if
%% they are in the same directory as the main LaTeX file. For loading figures
%% from other directories you can use the `import` package
%%   \usepackage{import}
%%
%% and then include the figures with
%%   \import{<path to file>}{<filename>.pgf}
%%
%% Matplotlib used the following preamble
%%   
%%   \makeatletter\@ifpackageloaded{underscore}{}{\usepackage[strings]{underscore}}\makeatother
%%
\begingroup%
\makeatletter%
\begin{pgfpicture}%
\pgfpathrectangle{\pgfpointorigin}{\pgfqpoint{5.590148in}{6.909803in}}%
\pgfusepath{use as bounding box, clip}%
\begin{pgfscope}%
\pgfsetbuttcap%
\pgfsetmiterjoin%
\definecolor{currentfill}{rgb}{1.000000,1.000000,1.000000}%
\pgfsetfillcolor{currentfill}%
\pgfsetlinewidth{0.000000pt}%
\definecolor{currentstroke}{rgb}{1.000000,1.000000,1.000000}%
\pgfsetstrokecolor{currentstroke}%
\pgfsetdash{}{0pt}%
\pgfpathmoveto{\pgfqpoint{0.000000in}{0.000000in}}%
\pgfpathlineto{\pgfqpoint{5.590148in}{0.000000in}}%
\pgfpathlineto{\pgfqpoint{5.590148in}{6.909803in}}%
\pgfpathlineto{\pgfqpoint{0.000000in}{6.909803in}}%
\pgfpathlineto{\pgfqpoint{0.000000in}{0.000000in}}%
\pgfpathclose%
\pgfusepath{fill}%
\end{pgfscope}%
\begin{pgfscope}%
\pgfsetbuttcap%
\pgfsetmiterjoin%
\definecolor{currentfill}{rgb}{0.917647,0.917647,0.949020}%
\pgfsetfillcolor{currentfill}%
\pgfsetlinewidth{0.000000pt}%
\definecolor{currentstroke}{rgb}{0.000000,0.000000,0.000000}%
\pgfsetstrokecolor{currentstroke}%
\pgfsetstrokeopacity{0.000000}%
\pgfsetdash{}{0pt}%
\pgfpathmoveto{\pgfqpoint{0.667731in}{4.015734in}}%
\pgfpathlineto{\pgfqpoint{5.410148in}{4.015734in}}%
\pgfpathlineto{\pgfqpoint{5.410148in}{6.729803in}}%
\pgfpathlineto{\pgfqpoint{0.667731in}{6.729803in}}%
\pgfpathlineto{\pgfqpoint{0.667731in}{4.015734in}}%
\pgfpathclose%
\pgfusepath{fill}%
\end{pgfscope}%
\begin{pgfscope}%
\definecolor{textcolor}{rgb}{0.150000,0.150000,0.150000}%
\pgfsetstrokecolor{textcolor}%
\pgfsetfillcolor{textcolor}%
\pgftext[x=1.260533in,y=3.883790in,,top]{\color{textcolor}\sffamily\fontsize{11.000000}{13.200000}\selectfont 32}%
\end{pgfscope}%
\begin{pgfscope}%
\definecolor{textcolor}{rgb}{0.150000,0.150000,0.150000}%
\pgfsetstrokecolor{textcolor}%
\pgfsetfillcolor{textcolor}%
\pgftext[x=2.446137in,y=3.883790in,,top]{\color{textcolor}\sffamily\fontsize{11.000000}{13.200000}\selectfont 64}%
\end{pgfscope}%
\begin{pgfscope}%
\definecolor{textcolor}{rgb}{0.150000,0.150000,0.150000}%
\pgfsetstrokecolor{textcolor}%
\pgfsetfillcolor{textcolor}%
\pgftext[x=3.631742in,y=3.883790in,,top]{\color{textcolor}\sffamily\fontsize{11.000000}{13.200000}\selectfont 128}%
\end{pgfscope}%
\begin{pgfscope}%
\definecolor{textcolor}{rgb}{0.150000,0.150000,0.150000}%
\pgfsetstrokecolor{textcolor}%
\pgfsetfillcolor{textcolor}%
\pgftext[x=4.817346in,y=3.883790in,,top]{\color{textcolor}\sffamily\fontsize{11.000000}{13.200000}\selectfont 256}%
\end{pgfscope}%
\begin{pgfscope}%
\definecolor{textcolor}{rgb}{0.150000,0.150000,0.150000}%
\pgfsetstrokecolor{textcolor}%
\pgfsetfillcolor{textcolor}%
\pgftext[x=3.038940in,y=3.693049in,,top]{\color{textcolor}\sffamily\fontsize{12.000000}{14.400000}\selectfont number of dimensions}%
\end{pgfscope}%
\begin{pgfscope}%
\pgfpathrectangle{\pgfqpoint{0.667731in}{4.015734in}}{\pgfqpoint{4.742417in}{2.714069in}}%
\pgfusepath{clip}%
\pgfsetroundcap%
\pgfsetroundjoin%
\pgfsetlinewidth{1.003750pt}%
\definecolor{currentstroke}{rgb}{1.000000,1.000000,1.000000}%
\pgfsetstrokecolor{currentstroke}%
\pgfsetdash{}{0pt}%
\pgfpathmoveto{\pgfqpoint{0.667731in}{4.039097in}}%
\pgfpathlineto{\pgfqpoint{5.410148in}{4.039097in}}%
\pgfusepath{stroke}%
\end{pgfscope}%
\begin{pgfscope}%
\definecolor{textcolor}{rgb}{0.150000,0.150000,0.150000}%
\pgfsetstrokecolor{textcolor}%
\pgfsetfillcolor{textcolor}%
\pgftext[x=0.383703in, y=3.986290in, left, base]{\color{textcolor}\sffamily\fontsize{11.000000}{13.200000}\selectfont \(\displaystyle {63}\)}%
\end{pgfscope}%
\begin{pgfscope}%
\pgfpathrectangle{\pgfqpoint{0.667731in}{4.015734in}}{\pgfqpoint{4.742417in}{2.714069in}}%
\pgfusepath{clip}%
\pgfsetroundcap%
\pgfsetroundjoin%
\pgfsetlinewidth{1.003750pt}%
\definecolor{currentstroke}{rgb}{1.000000,1.000000,1.000000}%
\pgfsetstrokecolor{currentstroke}%
\pgfsetdash{}{0pt}%
\pgfpathmoveto{\pgfqpoint{0.667731in}{4.532101in}}%
\pgfpathlineto{\pgfqpoint{5.410148in}{4.532101in}}%
\pgfusepath{stroke}%
\end{pgfscope}%
\begin{pgfscope}%
\definecolor{textcolor}{rgb}{0.150000,0.150000,0.150000}%
\pgfsetstrokecolor{textcolor}%
\pgfsetfillcolor{textcolor}%
\pgftext[x=0.383703in, y=4.479294in, left, base]{\color{textcolor}\sffamily\fontsize{11.000000}{13.200000}\selectfont \(\displaystyle {64}\)}%
\end{pgfscope}%
\begin{pgfscope}%
\pgfpathrectangle{\pgfqpoint{0.667731in}{4.015734in}}{\pgfqpoint{4.742417in}{2.714069in}}%
\pgfusepath{clip}%
\pgfsetroundcap%
\pgfsetroundjoin%
\pgfsetlinewidth{1.003750pt}%
\definecolor{currentstroke}{rgb}{1.000000,1.000000,1.000000}%
\pgfsetstrokecolor{currentstroke}%
\pgfsetdash{}{0pt}%
\pgfpathmoveto{\pgfqpoint{0.667731in}{5.025104in}}%
\pgfpathlineto{\pgfqpoint{5.410148in}{5.025104in}}%
\pgfusepath{stroke}%
\end{pgfscope}%
\begin{pgfscope}%
\definecolor{textcolor}{rgb}{0.150000,0.150000,0.150000}%
\pgfsetstrokecolor{textcolor}%
\pgfsetfillcolor{textcolor}%
\pgftext[x=0.383703in, y=4.972297in, left, base]{\color{textcolor}\sffamily\fontsize{11.000000}{13.200000}\selectfont \(\displaystyle {65}\)}%
\end{pgfscope}%
\begin{pgfscope}%
\pgfpathrectangle{\pgfqpoint{0.667731in}{4.015734in}}{\pgfqpoint{4.742417in}{2.714069in}}%
\pgfusepath{clip}%
\pgfsetroundcap%
\pgfsetroundjoin%
\pgfsetlinewidth{1.003750pt}%
\definecolor{currentstroke}{rgb}{1.000000,1.000000,1.000000}%
\pgfsetstrokecolor{currentstroke}%
\pgfsetdash{}{0pt}%
\pgfpathmoveto{\pgfqpoint{0.667731in}{5.518108in}}%
\pgfpathlineto{\pgfqpoint{5.410148in}{5.518108in}}%
\pgfusepath{stroke}%
\end{pgfscope}%
\begin{pgfscope}%
\definecolor{textcolor}{rgb}{0.150000,0.150000,0.150000}%
\pgfsetstrokecolor{textcolor}%
\pgfsetfillcolor{textcolor}%
\pgftext[x=0.383703in, y=5.465301in, left, base]{\color{textcolor}\sffamily\fontsize{11.000000}{13.200000}\selectfont \(\displaystyle {66}\)}%
\end{pgfscope}%
\begin{pgfscope}%
\pgfpathrectangle{\pgfqpoint{0.667731in}{4.015734in}}{\pgfqpoint{4.742417in}{2.714069in}}%
\pgfusepath{clip}%
\pgfsetroundcap%
\pgfsetroundjoin%
\pgfsetlinewidth{1.003750pt}%
\definecolor{currentstroke}{rgb}{1.000000,1.000000,1.000000}%
\pgfsetstrokecolor{currentstroke}%
\pgfsetdash{}{0pt}%
\pgfpathmoveto{\pgfqpoint{0.667731in}{6.011111in}}%
\pgfpathlineto{\pgfqpoint{5.410148in}{6.011111in}}%
\pgfusepath{stroke}%
\end{pgfscope}%
\begin{pgfscope}%
\definecolor{textcolor}{rgb}{0.150000,0.150000,0.150000}%
\pgfsetstrokecolor{textcolor}%
\pgfsetfillcolor{textcolor}%
\pgftext[x=0.383703in, y=5.958305in, left, base]{\color{textcolor}\sffamily\fontsize{11.000000}{13.200000}\selectfont \(\displaystyle {67}\)}%
\end{pgfscope}%
\begin{pgfscope}%
\pgfpathrectangle{\pgfqpoint{0.667731in}{4.015734in}}{\pgfqpoint{4.742417in}{2.714069in}}%
\pgfusepath{clip}%
\pgfsetroundcap%
\pgfsetroundjoin%
\pgfsetlinewidth{1.003750pt}%
\definecolor{currentstroke}{rgb}{1.000000,1.000000,1.000000}%
\pgfsetstrokecolor{currentstroke}%
\pgfsetdash{}{0pt}%
\pgfpathmoveto{\pgfqpoint{0.667731in}{6.504115in}}%
\pgfpathlineto{\pgfqpoint{5.410148in}{6.504115in}}%
\pgfusepath{stroke}%
\end{pgfscope}%
\begin{pgfscope}%
\definecolor{textcolor}{rgb}{0.150000,0.150000,0.150000}%
\pgfsetstrokecolor{textcolor}%
\pgfsetfillcolor{textcolor}%
\pgftext[x=0.383703in, y=6.451308in, left, base]{\color{textcolor}\sffamily\fontsize{11.000000}{13.200000}\selectfont \(\displaystyle {68}\)}%
\end{pgfscope}%
\begin{pgfscope}%
\definecolor{textcolor}{rgb}{0.150000,0.150000,0.150000}%
\pgfsetstrokecolor{textcolor}%
\pgfsetfillcolor{textcolor}%
\pgftext[x=0.328148in,y=5.372769in,,bottom,rotate=90.000000]{\color{textcolor}\sffamily\fontsize{12.000000}{14.400000}\selectfont Average score}%
\end{pgfscope}%
\begin{pgfscope}%
\pgfpathrectangle{\pgfqpoint{0.667731in}{4.015734in}}{\pgfqpoint{4.742417in}{2.714069in}}%
\pgfusepath{clip}%
\pgfsetbuttcap%
\pgfsetroundjoin%
\pgfsetlinewidth{2.710125pt}%
\definecolor{currentstroke}{rgb}{0.298039,0.447059,0.690196}%
\pgfsetstrokecolor{currentstroke}%
\pgfsetdash{{9.990000pt}{4.320000pt}}{0.000000pt}%
\pgfpathmoveto{\pgfqpoint{1.260533in}{6.606436in}}%
\pgfpathlineto{\pgfqpoint{2.446137in}{6.141339in}}%
\pgfpathlineto{\pgfqpoint{3.631742in}{5.211143in}}%
\pgfpathlineto{\pgfqpoint{4.817346in}{6.187848in}}%
\pgfusepath{stroke}%
\end{pgfscope}%
\begin{pgfscope}%
\pgfpathrectangle{\pgfqpoint{0.667731in}{4.015734in}}{\pgfqpoint{4.742417in}{2.714069in}}%
\pgfusepath{clip}%
\pgfsetroundcap%
\pgfsetroundjoin%
\pgfsetlinewidth{2.710125pt}%
\definecolor{currentstroke}{rgb}{0.298039,0.447059,0.690196}%
\pgfsetstrokecolor{currentstroke}%
\pgfsetdash{}{0pt}%
\pgfusepath{stroke}%
\end{pgfscope}%
\begin{pgfscope}%
\pgfpathrectangle{\pgfqpoint{0.667731in}{4.015734in}}{\pgfqpoint{4.742417in}{2.714069in}}%
\pgfusepath{clip}%
\pgfsetroundcap%
\pgfsetroundjoin%
\pgfsetlinewidth{2.710125pt}%
\definecolor{currentstroke}{rgb}{0.298039,0.447059,0.690196}%
\pgfsetstrokecolor{currentstroke}%
\pgfsetdash{}{0pt}%
\pgfusepath{stroke}%
\end{pgfscope}%
\begin{pgfscope}%
\pgfpathrectangle{\pgfqpoint{0.667731in}{4.015734in}}{\pgfqpoint{4.742417in}{2.714069in}}%
\pgfusepath{clip}%
\pgfsetroundcap%
\pgfsetroundjoin%
\pgfsetlinewidth{2.710125pt}%
\definecolor{currentstroke}{rgb}{0.298039,0.447059,0.690196}%
\pgfsetstrokecolor{currentstroke}%
\pgfsetdash{}{0pt}%
\pgfusepath{stroke}%
\end{pgfscope}%
\begin{pgfscope}%
\pgfpathrectangle{\pgfqpoint{0.667731in}{4.015734in}}{\pgfqpoint{4.742417in}{2.714069in}}%
\pgfusepath{clip}%
\pgfsetroundcap%
\pgfsetroundjoin%
\pgfsetlinewidth{2.710125pt}%
\definecolor{currentstroke}{rgb}{0.298039,0.447059,0.690196}%
\pgfsetstrokecolor{currentstroke}%
\pgfsetdash{}{0pt}%
\pgfusepath{stroke}%
\end{pgfscope}%
\begin{pgfscope}%
\pgfpathrectangle{\pgfqpoint{0.667731in}{4.015734in}}{\pgfqpoint{4.742417in}{2.714069in}}%
\pgfusepath{clip}%
\pgfsetbuttcap%
\pgfsetroundjoin%
\definecolor{currentfill}{rgb}{0.298039,0.447059,0.690196}%
\pgfsetfillcolor{currentfill}%
\pgfsetlinewidth{2.032594pt}%
\definecolor{currentstroke}{rgb}{0.298039,0.447059,0.690196}%
\pgfsetstrokecolor{currentstroke}%
\pgfsetdash{}{0pt}%
\pgfsys@defobject{currentmarker}{\pgfqpoint{-0.046999in}{-0.046999in}}{\pgfqpoint{0.046999in}{0.046999in}}{%
\pgfpathmoveto{\pgfqpoint{0.000000in}{-0.046999in}}%
\pgfpathcurveto{\pgfqpoint{0.012464in}{-0.046999in}}{\pgfqpoint{0.024420in}{-0.042047in}}{\pgfqpoint{0.033234in}{-0.033234in}}%
\pgfpathcurveto{\pgfqpoint{0.042047in}{-0.024420in}}{\pgfqpoint{0.046999in}{-0.012464in}}{\pgfqpoint{0.046999in}{0.000000in}}%
\pgfpathcurveto{\pgfqpoint{0.046999in}{0.012464in}}{\pgfqpoint{0.042047in}{0.024420in}}{\pgfqpoint{0.033234in}{0.033234in}}%
\pgfpathcurveto{\pgfqpoint{0.024420in}{0.042047in}}{\pgfqpoint{0.012464in}{0.046999in}}{\pgfqpoint{0.000000in}{0.046999in}}%
\pgfpathcurveto{\pgfqpoint{-0.012464in}{0.046999in}}{\pgfqpoint{-0.024420in}{0.042047in}}{\pgfqpoint{-0.033234in}{0.033234in}}%
\pgfpathcurveto{\pgfqpoint{-0.042047in}{0.024420in}}{\pgfqpoint{-0.046999in}{0.012464in}}{\pgfqpoint{-0.046999in}{0.000000in}}%
\pgfpathcurveto{\pgfqpoint{-0.046999in}{-0.012464in}}{\pgfqpoint{-0.042047in}{-0.024420in}}{\pgfqpoint{-0.033234in}{-0.033234in}}%
\pgfpathcurveto{\pgfqpoint{-0.024420in}{-0.042047in}}{\pgfqpoint{-0.012464in}{-0.046999in}}{\pgfqpoint{0.000000in}{-0.046999in}}%
\pgfpathlineto{\pgfqpoint{0.000000in}{-0.046999in}}%
\pgfpathclose%
\pgfusepath{stroke,fill}%
}%
\begin{pgfscope}%
\pgfsys@transformshift{1.260533in}{6.606436in}%
\pgfsys@useobject{currentmarker}{}%
\end{pgfscope}%
\begin{pgfscope}%
\pgfsys@transformshift{2.446137in}{6.141339in}%
\pgfsys@useobject{currentmarker}{}%
\end{pgfscope}%
\begin{pgfscope}%
\pgfsys@transformshift{3.631742in}{5.211143in}%
\pgfsys@useobject{currentmarker}{}%
\end{pgfscope}%
\begin{pgfscope}%
\pgfsys@transformshift{4.817346in}{6.187848in}%
\pgfsys@useobject{currentmarker}{}%
\end{pgfscope}%
\end{pgfscope}%
\begin{pgfscope}%
\pgfpathrectangle{\pgfqpoint{0.667731in}{4.015734in}}{\pgfqpoint{4.742417in}{2.714069in}}%
\pgfusepath{clip}%
\pgfsetbuttcap%
\pgfsetroundjoin%
\pgfsetlinewidth{2.710125pt}%
\definecolor{currentstroke}{rgb}{0.866667,0.517647,0.321569}%
\pgfsetstrokecolor{currentstroke}%
\pgfsetdash{{9.990000pt}{4.320000pt}}{0.000000pt}%
\pgfpathmoveto{\pgfqpoint{1.260533in}{5.748086in}}%
\pgfpathlineto{\pgfqpoint{2.446137in}{4.807854in}}%
\pgfpathlineto{\pgfqpoint{3.631742in}{4.139101in}}%
\pgfpathlineto{\pgfqpoint{4.817346in}{4.980048in}}%
\pgfusepath{stroke}%
\end{pgfscope}%
\begin{pgfscope}%
\pgfpathrectangle{\pgfqpoint{0.667731in}{4.015734in}}{\pgfqpoint{4.742417in}{2.714069in}}%
\pgfusepath{clip}%
\pgfsetroundcap%
\pgfsetroundjoin%
\pgfsetlinewidth{2.710125pt}%
\definecolor{currentstroke}{rgb}{0.866667,0.517647,0.321569}%
\pgfsetstrokecolor{currentstroke}%
\pgfsetdash{}{0pt}%
\pgfusepath{stroke}%
\end{pgfscope}%
\begin{pgfscope}%
\pgfpathrectangle{\pgfqpoint{0.667731in}{4.015734in}}{\pgfqpoint{4.742417in}{2.714069in}}%
\pgfusepath{clip}%
\pgfsetroundcap%
\pgfsetroundjoin%
\pgfsetlinewidth{2.710125pt}%
\definecolor{currentstroke}{rgb}{0.866667,0.517647,0.321569}%
\pgfsetstrokecolor{currentstroke}%
\pgfsetdash{}{0pt}%
\pgfusepath{stroke}%
\end{pgfscope}%
\begin{pgfscope}%
\pgfpathrectangle{\pgfqpoint{0.667731in}{4.015734in}}{\pgfqpoint{4.742417in}{2.714069in}}%
\pgfusepath{clip}%
\pgfsetroundcap%
\pgfsetroundjoin%
\pgfsetlinewidth{2.710125pt}%
\definecolor{currentstroke}{rgb}{0.866667,0.517647,0.321569}%
\pgfsetstrokecolor{currentstroke}%
\pgfsetdash{}{0pt}%
\pgfusepath{stroke}%
\end{pgfscope}%
\begin{pgfscope}%
\pgfpathrectangle{\pgfqpoint{0.667731in}{4.015734in}}{\pgfqpoint{4.742417in}{2.714069in}}%
\pgfusepath{clip}%
\pgfsetroundcap%
\pgfsetroundjoin%
\pgfsetlinewidth{2.710125pt}%
\definecolor{currentstroke}{rgb}{0.866667,0.517647,0.321569}%
\pgfsetstrokecolor{currentstroke}%
\pgfsetdash{}{0pt}%
\pgfusepath{stroke}%
\end{pgfscope}%
\begin{pgfscope}%
\pgfpathrectangle{\pgfqpoint{0.667731in}{4.015734in}}{\pgfqpoint{4.742417in}{2.714069in}}%
\pgfusepath{clip}%
\pgfsetbuttcap%
\pgfsetroundjoin%
\definecolor{currentfill}{rgb}{0.866667,0.517647,0.321569}%
\pgfsetfillcolor{currentfill}%
\pgfsetlinewidth{2.032594pt}%
\definecolor{currentstroke}{rgb}{0.866667,0.517647,0.321569}%
\pgfsetstrokecolor{currentstroke}%
\pgfsetdash{}{0pt}%
\pgfsys@defobject{currentmarker}{\pgfqpoint{-0.046999in}{-0.046999in}}{\pgfqpoint{0.046999in}{0.046999in}}{%
\pgfpathmoveto{\pgfqpoint{0.000000in}{-0.046999in}}%
\pgfpathcurveto{\pgfqpoint{0.012464in}{-0.046999in}}{\pgfqpoint{0.024420in}{-0.042047in}}{\pgfqpoint{0.033234in}{-0.033234in}}%
\pgfpathcurveto{\pgfqpoint{0.042047in}{-0.024420in}}{\pgfqpoint{0.046999in}{-0.012464in}}{\pgfqpoint{0.046999in}{0.000000in}}%
\pgfpathcurveto{\pgfqpoint{0.046999in}{0.012464in}}{\pgfqpoint{0.042047in}{0.024420in}}{\pgfqpoint{0.033234in}{0.033234in}}%
\pgfpathcurveto{\pgfqpoint{0.024420in}{0.042047in}}{\pgfqpoint{0.012464in}{0.046999in}}{\pgfqpoint{0.000000in}{0.046999in}}%
\pgfpathcurveto{\pgfqpoint{-0.012464in}{0.046999in}}{\pgfqpoint{-0.024420in}{0.042047in}}{\pgfqpoint{-0.033234in}{0.033234in}}%
\pgfpathcurveto{\pgfqpoint{-0.042047in}{0.024420in}}{\pgfqpoint{-0.046999in}{0.012464in}}{\pgfqpoint{-0.046999in}{0.000000in}}%
\pgfpathcurveto{\pgfqpoint{-0.046999in}{-0.012464in}}{\pgfqpoint{-0.042047in}{-0.024420in}}{\pgfqpoint{-0.033234in}{-0.033234in}}%
\pgfpathcurveto{\pgfqpoint{-0.024420in}{-0.042047in}}{\pgfqpoint{-0.012464in}{-0.046999in}}{\pgfqpoint{0.000000in}{-0.046999in}}%
\pgfpathlineto{\pgfqpoint{0.000000in}{-0.046999in}}%
\pgfpathclose%
\pgfusepath{stroke,fill}%
}%
\begin{pgfscope}%
\pgfsys@transformshift{1.260533in}{5.748086in}%
\pgfsys@useobject{currentmarker}{}%
\end{pgfscope}%
\begin{pgfscope}%
\pgfsys@transformshift{2.446137in}{4.807854in}%
\pgfsys@useobject{currentmarker}{}%
\end{pgfscope}%
\begin{pgfscope}%
\pgfsys@transformshift{3.631742in}{4.139101in}%
\pgfsys@useobject{currentmarker}{}%
\end{pgfscope}%
\begin{pgfscope}%
\pgfsys@transformshift{4.817346in}{4.980048in}%
\pgfsys@useobject{currentmarker}{}%
\end{pgfscope}%
\end{pgfscope}%
\begin{pgfscope}%
\pgfsetrectcap%
\pgfsetmiterjoin%
\pgfsetlinewidth{1.254687pt}%
\definecolor{currentstroke}{rgb}{1.000000,1.000000,1.000000}%
\pgfsetstrokecolor{currentstroke}%
\pgfsetdash{}{0pt}%
\pgfpathmoveto{\pgfqpoint{0.667731in}{4.015734in}}%
\pgfpathlineto{\pgfqpoint{0.667731in}{6.729803in}}%
\pgfusepath{stroke}%
\end{pgfscope}%
\begin{pgfscope}%
\pgfsetrectcap%
\pgfsetmiterjoin%
\pgfsetlinewidth{1.254687pt}%
\definecolor{currentstroke}{rgb}{1.000000,1.000000,1.000000}%
\pgfsetstrokecolor{currentstroke}%
\pgfsetdash{}{0pt}%
\pgfpathmoveto{\pgfqpoint{5.410148in}{4.015734in}}%
\pgfpathlineto{\pgfqpoint{5.410148in}{6.729803in}}%
\pgfusepath{stroke}%
\end{pgfscope}%
\begin{pgfscope}%
\pgfsetrectcap%
\pgfsetmiterjoin%
\pgfsetlinewidth{1.254687pt}%
\definecolor{currentstroke}{rgb}{1.000000,1.000000,1.000000}%
\pgfsetstrokecolor{currentstroke}%
\pgfsetdash{}{0pt}%
\pgfpathmoveto{\pgfqpoint{0.667731in}{4.015734in}}%
\pgfpathlineto{\pgfqpoint{5.410148in}{4.015734in}}%
\pgfusepath{stroke}%
\end{pgfscope}%
\begin{pgfscope}%
\pgfsetrectcap%
\pgfsetmiterjoin%
\pgfsetlinewidth{1.254687pt}%
\definecolor{currentstroke}{rgb}{1.000000,1.000000,1.000000}%
\pgfsetstrokecolor{currentstroke}%
\pgfsetdash{}{0pt}%
\pgfpathmoveto{\pgfqpoint{0.667731in}{6.729803in}}%
\pgfpathlineto{\pgfqpoint{5.410148in}{6.729803in}}%
\pgfusepath{stroke}%
\end{pgfscope}%
\begin{pgfscope}%
\pgfsetbuttcap%
\pgfsetmiterjoin%
\definecolor{currentfill}{rgb}{0.917647,0.917647,0.949020}%
\pgfsetfillcolor{currentfill}%
\pgfsetfillopacity{0.800000}%
\pgfsetlinewidth{1.003750pt}%
\definecolor{currentstroke}{rgb}{0.800000,0.800000,0.800000}%
\pgfsetstrokecolor{currentstroke}%
\pgfsetstrokeopacity{0.800000}%
\pgfsetdash{}{0pt}%
\pgfpathmoveto{\pgfqpoint{0.774675in}{4.092123in}}%
\pgfpathlineto{\pgfqpoint{1.866407in}{4.092123in}}%
\pgfpathquadraticcurveto{\pgfqpoint{1.896963in}{4.092123in}}{\pgfqpoint{1.896963in}{4.122679in}}%
\pgfpathlineto{\pgfqpoint{1.896963in}{4.757747in}}%
\pgfpathquadraticcurveto{\pgfqpoint{1.896963in}{4.788303in}}{\pgfqpoint{1.866407in}{4.788303in}}%
\pgfpathlineto{\pgfqpoint{0.774675in}{4.788303in}}%
\pgfpathquadraticcurveto{\pgfqpoint{0.744120in}{4.788303in}}{\pgfqpoint{0.744120in}{4.757747in}}%
\pgfpathlineto{\pgfqpoint{0.744120in}{4.122679in}}%
\pgfpathquadraticcurveto{\pgfqpoint{0.744120in}{4.092123in}}{\pgfqpoint{0.774675in}{4.092123in}}%
\pgfpathlineto{\pgfqpoint{0.774675in}{4.092123in}}%
\pgfpathclose%
\pgfusepath{stroke,fill}%
\end{pgfscope}%
\begin{pgfscope}%
\definecolor{textcolor}{rgb}{0.150000,0.150000,0.150000}%
\pgfsetstrokecolor{textcolor}%
\pgfsetfillcolor{textcolor}%
\pgftext[x=1.076929in,y=4.611452in,left,base]{\color{textcolor}\sffamily\fontsize{12.000000}{14.400000}\selectfont F-score}%
\end{pgfscope}%
\begin{pgfscope}%
\pgfsetbuttcap%
\pgfsetroundjoin%
\definecolor{currentfill}{rgb}{0.298039,0.447059,0.690196}%
\pgfsetfillcolor{currentfill}%
\pgfsetlinewidth{2.032594pt}%
\definecolor{currentstroke}{rgb}{0.298039,0.447059,0.690196}%
\pgfsetstrokecolor{currentstroke}%
\pgfsetdash{}{0pt}%
\pgfsys@defobject{currentmarker}{\pgfqpoint{-0.046999in}{-0.046999in}}{\pgfqpoint{0.046999in}{0.046999in}}{%
\pgfpathmoveto{\pgfqpoint{0.000000in}{-0.046999in}}%
\pgfpathcurveto{\pgfqpoint{0.012464in}{-0.046999in}}{\pgfqpoint{0.024420in}{-0.042047in}}{\pgfqpoint{0.033234in}{-0.033234in}}%
\pgfpathcurveto{\pgfqpoint{0.042047in}{-0.024420in}}{\pgfqpoint{0.046999in}{-0.012464in}}{\pgfqpoint{0.046999in}{0.000000in}}%
\pgfpathcurveto{\pgfqpoint{0.046999in}{0.012464in}}{\pgfqpoint{0.042047in}{0.024420in}}{\pgfqpoint{0.033234in}{0.033234in}}%
\pgfpathcurveto{\pgfqpoint{0.024420in}{0.042047in}}{\pgfqpoint{0.012464in}{0.046999in}}{\pgfqpoint{0.000000in}{0.046999in}}%
\pgfpathcurveto{\pgfqpoint{-0.012464in}{0.046999in}}{\pgfqpoint{-0.024420in}{0.042047in}}{\pgfqpoint{-0.033234in}{0.033234in}}%
\pgfpathcurveto{\pgfqpoint{-0.042047in}{0.024420in}}{\pgfqpoint{-0.046999in}{0.012464in}}{\pgfqpoint{-0.046999in}{0.000000in}}%
\pgfpathcurveto{\pgfqpoint{-0.046999in}{-0.012464in}}{\pgfqpoint{-0.042047in}{-0.024420in}}{\pgfqpoint{-0.033234in}{-0.033234in}}%
\pgfpathcurveto{\pgfqpoint{-0.024420in}{-0.042047in}}{\pgfqpoint{-0.012464in}{-0.046999in}}{\pgfqpoint{0.000000in}{-0.046999in}}%
\pgfpathlineto{\pgfqpoint{0.000000in}{-0.046999in}}%
\pgfpathclose%
\pgfusepath{stroke,fill}%
}%
\begin{pgfscope}%
\pgfsys@transformshift{0.958009in}{4.435815in}%
\pgfsys@useobject{currentmarker}{}%
\end{pgfscope}%
\end{pgfscope}%
\begin{pgfscope}%
\definecolor{textcolor}{rgb}{0.150000,0.150000,0.150000}%
\pgfsetstrokecolor{textcolor}%
\pgfsetfillcolor{textcolor}%
\pgftext[x=1.233009in,y=4.395711in,left,base]{\color{textcolor}\sffamily\fontsize{11.000000}{13.200000}\selectfont F1 micro}%
\end{pgfscope}%
\begin{pgfscope}%
\pgfsetbuttcap%
\pgfsetroundjoin%
\definecolor{currentfill}{rgb}{0.866667,0.517647,0.321569}%
\pgfsetfillcolor{currentfill}%
\pgfsetlinewidth{2.032594pt}%
\definecolor{currentstroke}{rgb}{0.866667,0.517647,0.321569}%
\pgfsetstrokecolor{currentstroke}%
\pgfsetdash{}{0pt}%
\pgfsys@defobject{currentmarker}{\pgfqpoint{-0.046999in}{-0.046999in}}{\pgfqpoint{0.046999in}{0.046999in}}{%
\pgfpathmoveto{\pgfqpoint{0.000000in}{-0.046999in}}%
\pgfpathcurveto{\pgfqpoint{0.012464in}{-0.046999in}}{\pgfqpoint{0.024420in}{-0.042047in}}{\pgfqpoint{0.033234in}{-0.033234in}}%
\pgfpathcurveto{\pgfqpoint{0.042047in}{-0.024420in}}{\pgfqpoint{0.046999in}{-0.012464in}}{\pgfqpoint{0.046999in}{0.000000in}}%
\pgfpathcurveto{\pgfqpoint{0.046999in}{0.012464in}}{\pgfqpoint{0.042047in}{0.024420in}}{\pgfqpoint{0.033234in}{0.033234in}}%
\pgfpathcurveto{\pgfqpoint{0.024420in}{0.042047in}}{\pgfqpoint{0.012464in}{0.046999in}}{\pgfqpoint{0.000000in}{0.046999in}}%
\pgfpathcurveto{\pgfqpoint{-0.012464in}{0.046999in}}{\pgfqpoint{-0.024420in}{0.042047in}}{\pgfqpoint{-0.033234in}{0.033234in}}%
\pgfpathcurveto{\pgfqpoint{-0.042047in}{0.024420in}}{\pgfqpoint{-0.046999in}{0.012464in}}{\pgfqpoint{-0.046999in}{0.000000in}}%
\pgfpathcurveto{\pgfqpoint{-0.046999in}{-0.012464in}}{\pgfqpoint{-0.042047in}{-0.024420in}}{\pgfqpoint{-0.033234in}{-0.033234in}}%
\pgfpathcurveto{\pgfqpoint{-0.024420in}{-0.042047in}}{\pgfqpoint{-0.012464in}{-0.046999in}}{\pgfqpoint{0.000000in}{-0.046999in}}%
\pgfpathlineto{\pgfqpoint{0.000000in}{-0.046999in}}%
\pgfpathclose%
\pgfusepath{stroke,fill}%
}%
\begin{pgfscope}%
\pgfsys@transformshift{0.958009in}{4.222910in}%
\pgfsys@useobject{currentmarker}{}%
\end{pgfscope}%
\end{pgfscope}%
\begin{pgfscope}%
\definecolor{textcolor}{rgb}{0.150000,0.150000,0.150000}%
\pgfsetstrokecolor{textcolor}%
\pgfsetfillcolor{textcolor}%
\pgftext[x=1.233009in,y=4.182806in,left,base]{\color{textcolor}\sffamily\fontsize{11.000000}{13.200000}\selectfont F1 macro}%
\end{pgfscope}%
\begin{pgfscope}%
\pgfsetbuttcap%
\pgfsetmiterjoin%
\definecolor{currentfill}{rgb}{0.917647,0.917647,0.949020}%
\pgfsetfillcolor{currentfill}%
\pgfsetlinewidth{0.000000pt}%
\definecolor{currentstroke}{rgb}{0.000000,0.000000,0.000000}%
\pgfsetstrokecolor{currentstroke}%
\pgfsetstrokeopacity{0.000000}%
\pgfsetdash{}{0pt}%
\pgfpathmoveto{\pgfqpoint{0.667731in}{0.650833in}}%
\pgfpathlineto{\pgfqpoint{5.410148in}{0.650833in}}%
\pgfpathlineto{\pgfqpoint{5.410148in}{3.364902in}}%
\pgfpathlineto{\pgfqpoint{0.667731in}{3.364902in}}%
\pgfpathlineto{\pgfqpoint{0.667731in}{0.650833in}}%
\pgfpathclose%
\pgfusepath{fill}%
\end{pgfscope}%
\begin{pgfscope}%
\definecolor{textcolor}{rgb}{0.150000,0.150000,0.150000}%
\pgfsetstrokecolor{textcolor}%
\pgfsetfillcolor{textcolor}%
\pgftext[x=1.141973in,y=0.518888in,,top]{\color{textcolor}\sffamily\fontsize{11.000000}{13.200000}\selectfont 1}%
\end{pgfscope}%
\begin{pgfscope}%
\definecolor{textcolor}{rgb}{0.150000,0.150000,0.150000}%
\pgfsetstrokecolor{textcolor}%
\pgfsetfillcolor{textcolor}%
\pgftext[x=2.090456in,y=0.518888in,,top]{\color{textcolor}\sffamily\fontsize{11.000000}{13.200000}\selectfont 5}%
\end{pgfscope}%
\begin{pgfscope}%
\definecolor{textcolor}{rgb}{0.150000,0.150000,0.150000}%
\pgfsetstrokecolor{textcolor}%
\pgfsetfillcolor{textcolor}%
\pgftext[x=3.038940in,y=0.518888in,,top]{\color{textcolor}\sffamily\fontsize{11.000000}{13.200000}\selectfont 10}%
\end{pgfscope}%
\begin{pgfscope}%
\definecolor{textcolor}{rgb}{0.150000,0.150000,0.150000}%
\pgfsetstrokecolor{textcolor}%
\pgfsetfillcolor{textcolor}%
\pgftext[x=3.987423in,y=0.518888in,,top]{\color{textcolor}\sffamily\fontsize{11.000000}{13.200000}\selectfont 15}%
\end{pgfscope}%
\begin{pgfscope}%
\definecolor{textcolor}{rgb}{0.150000,0.150000,0.150000}%
\pgfsetstrokecolor{textcolor}%
\pgfsetfillcolor{textcolor}%
\pgftext[x=4.935906in,y=0.518888in,,top]{\color{textcolor}\sffamily\fontsize{11.000000}{13.200000}\selectfont 20}%
\end{pgfscope}%
\begin{pgfscope}%
\definecolor{textcolor}{rgb}{0.150000,0.150000,0.150000}%
\pgfsetstrokecolor{textcolor}%
\pgfsetfillcolor{textcolor}%
\pgftext[x=3.038940in,y=0.328148in,,top]{\color{textcolor}\sffamily\fontsize{12.000000}{14.400000}\selectfont window size}%
\end{pgfscope}%
\begin{pgfscope}%
\pgfpathrectangle{\pgfqpoint{0.667731in}{0.650833in}}{\pgfqpoint{4.742417in}{2.714069in}}%
\pgfusepath{clip}%
\pgfsetroundcap%
\pgfsetroundjoin%
\pgfsetlinewidth{1.003750pt}%
\definecolor{currentstroke}{rgb}{1.000000,1.000000,1.000000}%
\pgfsetstrokecolor{currentstroke}%
\pgfsetdash{}{0pt}%
\pgfpathmoveto{\pgfqpoint{0.667731in}{0.776442in}}%
\pgfpathlineto{\pgfqpoint{5.410148in}{0.776442in}}%
\pgfusepath{stroke}%
\end{pgfscope}%
\begin{pgfscope}%
\definecolor{textcolor}{rgb}{0.150000,0.150000,0.150000}%
\pgfsetstrokecolor{textcolor}%
\pgfsetfillcolor{textcolor}%
\pgftext[x=0.383703in, y=0.723635in, left, base]{\color{textcolor}\sffamily\fontsize{11.000000}{13.200000}\selectfont \(\displaystyle {63}\)}%
\end{pgfscope}%
\begin{pgfscope}%
\pgfpathrectangle{\pgfqpoint{0.667731in}{0.650833in}}{\pgfqpoint{4.742417in}{2.714069in}}%
\pgfusepath{clip}%
\pgfsetroundcap%
\pgfsetroundjoin%
\pgfsetlinewidth{1.003750pt}%
\definecolor{currentstroke}{rgb}{1.000000,1.000000,1.000000}%
\pgfsetstrokecolor{currentstroke}%
\pgfsetdash{}{0pt}%
\pgfpathmoveto{\pgfqpoint{0.667731in}{1.177209in}}%
\pgfpathlineto{\pgfqpoint{5.410148in}{1.177209in}}%
\pgfusepath{stroke}%
\end{pgfscope}%
\begin{pgfscope}%
\definecolor{textcolor}{rgb}{0.150000,0.150000,0.150000}%
\pgfsetstrokecolor{textcolor}%
\pgfsetfillcolor{textcolor}%
\pgftext[x=0.383703in, y=1.124402in, left, base]{\color{textcolor}\sffamily\fontsize{11.000000}{13.200000}\selectfont \(\displaystyle {64}\)}%
\end{pgfscope}%
\begin{pgfscope}%
\pgfpathrectangle{\pgfqpoint{0.667731in}{0.650833in}}{\pgfqpoint{4.742417in}{2.714069in}}%
\pgfusepath{clip}%
\pgfsetroundcap%
\pgfsetroundjoin%
\pgfsetlinewidth{1.003750pt}%
\definecolor{currentstroke}{rgb}{1.000000,1.000000,1.000000}%
\pgfsetstrokecolor{currentstroke}%
\pgfsetdash{}{0pt}%
\pgfpathmoveto{\pgfqpoint{0.667731in}{1.577975in}}%
\pgfpathlineto{\pgfqpoint{5.410148in}{1.577975in}}%
\pgfusepath{stroke}%
\end{pgfscope}%
\begin{pgfscope}%
\definecolor{textcolor}{rgb}{0.150000,0.150000,0.150000}%
\pgfsetstrokecolor{textcolor}%
\pgfsetfillcolor{textcolor}%
\pgftext[x=0.383703in, y=1.525169in, left, base]{\color{textcolor}\sffamily\fontsize{11.000000}{13.200000}\selectfont \(\displaystyle {65}\)}%
\end{pgfscope}%
\begin{pgfscope}%
\pgfpathrectangle{\pgfqpoint{0.667731in}{0.650833in}}{\pgfqpoint{4.742417in}{2.714069in}}%
\pgfusepath{clip}%
\pgfsetroundcap%
\pgfsetroundjoin%
\pgfsetlinewidth{1.003750pt}%
\definecolor{currentstroke}{rgb}{1.000000,1.000000,1.000000}%
\pgfsetstrokecolor{currentstroke}%
\pgfsetdash{}{0pt}%
\pgfpathmoveto{\pgfqpoint{0.667731in}{1.978742in}}%
\pgfpathlineto{\pgfqpoint{5.410148in}{1.978742in}}%
\pgfusepath{stroke}%
\end{pgfscope}%
\begin{pgfscope}%
\definecolor{textcolor}{rgb}{0.150000,0.150000,0.150000}%
\pgfsetstrokecolor{textcolor}%
\pgfsetfillcolor{textcolor}%
\pgftext[x=0.383703in, y=1.925935in, left, base]{\color{textcolor}\sffamily\fontsize{11.000000}{13.200000}\selectfont \(\displaystyle {66}\)}%
\end{pgfscope}%
\begin{pgfscope}%
\pgfpathrectangle{\pgfqpoint{0.667731in}{0.650833in}}{\pgfqpoint{4.742417in}{2.714069in}}%
\pgfusepath{clip}%
\pgfsetroundcap%
\pgfsetroundjoin%
\pgfsetlinewidth{1.003750pt}%
\definecolor{currentstroke}{rgb}{1.000000,1.000000,1.000000}%
\pgfsetstrokecolor{currentstroke}%
\pgfsetdash{}{0pt}%
\pgfpathmoveto{\pgfqpoint{0.667731in}{2.379508in}}%
\pgfpathlineto{\pgfqpoint{5.410148in}{2.379508in}}%
\pgfusepath{stroke}%
\end{pgfscope}%
\begin{pgfscope}%
\definecolor{textcolor}{rgb}{0.150000,0.150000,0.150000}%
\pgfsetstrokecolor{textcolor}%
\pgfsetfillcolor{textcolor}%
\pgftext[x=0.383703in, y=2.326702in, left, base]{\color{textcolor}\sffamily\fontsize{11.000000}{13.200000}\selectfont \(\displaystyle {67}\)}%
\end{pgfscope}%
\begin{pgfscope}%
\pgfpathrectangle{\pgfqpoint{0.667731in}{0.650833in}}{\pgfqpoint{4.742417in}{2.714069in}}%
\pgfusepath{clip}%
\pgfsetroundcap%
\pgfsetroundjoin%
\pgfsetlinewidth{1.003750pt}%
\definecolor{currentstroke}{rgb}{1.000000,1.000000,1.000000}%
\pgfsetstrokecolor{currentstroke}%
\pgfsetdash{}{0pt}%
\pgfpathmoveto{\pgfqpoint{0.667731in}{2.780275in}}%
\pgfpathlineto{\pgfqpoint{5.410148in}{2.780275in}}%
\pgfusepath{stroke}%
\end{pgfscope}%
\begin{pgfscope}%
\definecolor{textcolor}{rgb}{0.150000,0.150000,0.150000}%
\pgfsetstrokecolor{textcolor}%
\pgfsetfillcolor{textcolor}%
\pgftext[x=0.383703in, y=2.727468in, left, base]{\color{textcolor}\sffamily\fontsize{11.000000}{13.200000}\selectfont \(\displaystyle {68}\)}%
\end{pgfscope}%
\begin{pgfscope}%
\pgfpathrectangle{\pgfqpoint{0.667731in}{0.650833in}}{\pgfqpoint{4.742417in}{2.714069in}}%
\pgfusepath{clip}%
\pgfsetroundcap%
\pgfsetroundjoin%
\pgfsetlinewidth{1.003750pt}%
\definecolor{currentstroke}{rgb}{1.000000,1.000000,1.000000}%
\pgfsetstrokecolor{currentstroke}%
\pgfsetdash{}{0pt}%
\pgfpathmoveto{\pgfqpoint{0.667731in}{3.181042in}}%
\pgfpathlineto{\pgfqpoint{5.410148in}{3.181042in}}%
\pgfusepath{stroke}%
\end{pgfscope}%
\begin{pgfscope}%
\definecolor{textcolor}{rgb}{0.150000,0.150000,0.150000}%
\pgfsetstrokecolor{textcolor}%
\pgfsetfillcolor{textcolor}%
\pgftext[x=0.383703in, y=3.128235in, left, base]{\color{textcolor}\sffamily\fontsize{11.000000}{13.200000}\selectfont \(\displaystyle {69}\)}%
\end{pgfscope}%
\begin{pgfscope}%
\definecolor{textcolor}{rgb}{0.150000,0.150000,0.150000}%
\pgfsetstrokecolor{textcolor}%
\pgfsetfillcolor{textcolor}%
\pgftext[x=0.328148in,y=2.007867in,,bottom,rotate=90.000000]{\color{textcolor}\sffamily\fontsize{12.000000}{14.400000}\selectfont Average score}%
\end{pgfscope}%
\begin{pgfscope}%
\pgfpathrectangle{\pgfqpoint{0.667731in}{0.650833in}}{\pgfqpoint{4.742417in}{2.714069in}}%
\pgfusepath{clip}%
\pgfsetbuttcap%
\pgfsetroundjoin%
\pgfsetlinewidth{2.710125pt}%
\definecolor{currentstroke}{rgb}{0.298039,0.447059,0.690196}%
\pgfsetstrokecolor{currentstroke}%
\pgfsetdash{{9.990000pt}{4.320000pt}}{0.000000pt}%
\pgfpathmoveto{\pgfqpoint{1.141973in}{2.939069in}}%
\pgfpathlineto{\pgfqpoint{2.090456in}{3.241535in}}%
\pgfpathlineto{\pgfqpoint{3.038940in}{2.031673in}}%
\pgfpathlineto{\pgfqpoint{3.987423in}{2.107290in}}%
\pgfpathlineto{\pgfqpoint{4.935906in}{2.334139in}}%
\pgfusepath{stroke}%
\end{pgfscope}%
\begin{pgfscope}%
\pgfpathrectangle{\pgfqpoint{0.667731in}{0.650833in}}{\pgfqpoint{4.742417in}{2.714069in}}%
\pgfusepath{clip}%
\pgfsetroundcap%
\pgfsetroundjoin%
\pgfsetlinewidth{2.710125pt}%
\definecolor{currentstroke}{rgb}{0.298039,0.447059,0.690196}%
\pgfsetstrokecolor{currentstroke}%
\pgfsetdash{}{0pt}%
\pgfusepath{stroke}%
\end{pgfscope}%
\begin{pgfscope}%
\pgfpathrectangle{\pgfqpoint{0.667731in}{0.650833in}}{\pgfqpoint{4.742417in}{2.714069in}}%
\pgfusepath{clip}%
\pgfsetroundcap%
\pgfsetroundjoin%
\pgfsetlinewidth{2.710125pt}%
\definecolor{currentstroke}{rgb}{0.298039,0.447059,0.690196}%
\pgfsetstrokecolor{currentstroke}%
\pgfsetdash{}{0pt}%
\pgfusepath{stroke}%
\end{pgfscope}%
\begin{pgfscope}%
\pgfpathrectangle{\pgfqpoint{0.667731in}{0.650833in}}{\pgfqpoint{4.742417in}{2.714069in}}%
\pgfusepath{clip}%
\pgfsetroundcap%
\pgfsetroundjoin%
\pgfsetlinewidth{2.710125pt}%
\definecolor{currentstroke}{rgb}{0.298039,0.447059,0.690196}%
\pgfsetstrokecolor{currentstroke}%
\pgfsetdash{}{0pt}%
\pgfusepath{stroke}%
\end{pgfscope}%
\begin{pgfscope}%
\pgfpathrectangle{\pgfqpoint{0.667731in}{0.650833in}}{\pgfqpoint{4.742417in}{2.714069in}}%
\pgfusepath{clip}%
\pgfsetroundcap%
\pgfsetroundjoin%
\pgfsetlinewidth{2.710125pt}%
\definecolor{currentstroke}{rgb}{0.298039,0.447059,0.690196}%
\pgfsetstrokecolor{currentstroke}%
\pgfsetdash{}{0pt}%
\pgfusepath{stroke}%
\end{pgfscope}%
\begin{pgfscope}%
\pgfpathrectangle{\pgfqpoint{0.667731in}{0.650833in}}{\pgfqpoint{4.742417in}{2.714069in}}%
\pgfusepath{clip}%
\pgfsetroundcap%
\pgfsetroundjoin%
\pgfsetlinewidth{2.710125pt}%
\definecolor{currentstroke}{rgb}{0.298039,0.447059,0.690196}%
\pgfsetstrokecolor{currentstroke}%
\pgfsetdash{}{0pt}%
\pgfusepath{stroke}%
\end{pgfscope}%
\begin{pgfscope}%
\pgfpathrectangle{\pgfqpoint{0.667731in}{0.650833in}}{\pgfqpoint{4.742417in}{2.714069in}}%
\pgfusepath{clip}%
\pgfsetbuttcap%
\pgfsetroundjoin%
\definecolor{currentfill}{rgb}{0.298039,0.447059,0.690196}%
\pgfsetfillcolor{currentfill}%
\pgfsetlinewidth{2.032594pt}%
\definecolor{currentstroke}{rgb}{0.298039,0.447059,0.690196}%
\pgfsetstrokecolor{currentstroke}%
\pgfsetdash{}{0pt}%
\pgfsys@defobject{currentmarker}{\pgfqpoint{-0.046999in}{-0.046999in}}{\pgfqpoint{0.046999in}{0.046999in}}{%
\pgfpathmoveto{\pgfqpoint{0.000000in}{-0.046999in}}%
\pgfpathcurveto{\pgfqpoint{0.012464in}{-0.046999in}}{\pgfqpoint{0.024420in}{-0.042047in}}{\pgfqpoint{0.033234in}{-0.033234in}}%
\pgfpathcurveto{\pgfqpoint{0.042047in}{-0.024420in}}{\pgfqpoint{0.046999in}{-0.012464in}}{\pgfqpoint{0.046999in}{0.000000in}}%
\pgfpathcurveto{\pgfqpoint{0.046999in}{0.012464in}}{\pgfqpoint{0.042047in}{0.024420in}}{\pgfqpoint{0.033234in}{0.033234in}}%
\pgfpathcurveto{\pgfqpoint{0.024420in}{0.042047in}}{\pgfqpoint{0.012464in}{0.046999in}}{\pgfqpoint{0.000000in}{0.046999in}}%
\pgfpathcurveto{\pgfqpoint{-0.012464in}{0.046999in}}{\pgfqpoint{-0.024420in}{0.042047in}}{\pgfqpoint{-0.033234in}{0.033234in}}%
\pgfpathcurveto{\pgfqpoint{-0.042047in}{0.024420in}}{\pgfqpoint{-0.046999in}{0.012464in}}{\pgfqpoint{-0.046999in}{0.000000in}}%
\pgfpathcurveto{\pgfqpoint{-0.046999in}{-0.012464in}}{\pgfqpoint{-0.042047in}{-0.024420in}}{\pgfqpoint{-0.033234in}{-0.033234in}}%
\pgfpathcurveto{\pgfqpoint{-0.024420in}{-0.042047in}}{\pgfqpoint{-0.012464in}{-0.046999in}}{\pgfqpoint{0.000000in}{-0.046999in}}%
\pgfpathlineto{\pgfqpoint{0.000000in}{-0.046999in}}%
\pgfpathclose%
\pgfusepath{stroke,fill}%
}%
\begin{pgfscope}%
\pgfsys@transformshift{1.141973in}{2.939069in}%
\pgfsys@useobject{currentmarker}{}%
\end{pgfscope}%
\begin{pgfscope}%
\pgfsys@transformshift{2.090456in}{3.241535in}%
\pgfsys@useobject{currentmarker}{}%
\end{pgfscope}%
\begin{pgfscope}%
\pgfsys@transformshift{3.038940in}{2.031673in}%
\pgfsys@useobject{currentmarker}{}%
\end{pgfscope}%
\begin{pgfscope}%
\pgfsys@transformshift{3.987423in}{2.107290in}%
\pgfsys@useobject{currentmarker}{}%
\end{pgfscope}%
\begin{pgfscope}%
\pgfsys@transformshift{4.935906in}{2.334139in}%
\pgfsys@useobject{currentmarker}{}%
\end{pgfscope}%
\end{pgfscope}%
\begin{pgfscope}%
\pgfpathrectangle{\pgfqpoint{0.667731in}{0.650833in}}{\pgfqpoint{4.742417in}{2.714069in}}%
\pgfusepath{clip}%
\pgfsetbuttcap%
\pgfsetroundjoin%
\pgfsetlinewidth{2.710125pt}%
\definecolor{currentstroke}{rgb}{0.866667,0.517647,0.321569}%
\pgfsetstrokecolor{currentstroke}%
\pgfsetdash{{9.990000pt}{4.320000pt}}{0.000000pt}%
\pgfpathmoveto{\pgfqpoint{1.141973in}{1.797423in}}%
\pgfpathlineto{\pgfqpoint{2.090456in}{2.087194in}}%
\pgfpathlineto{\pgfqpoint{3.038940in}{0.863657in}}%
\pgfpathlineto{\pgfqpoint{3.987423in}{0.774199in}}%
\pgfpathlineto{\pgfqpoint{4.935906in}{1.093456in}}%
\pgfusepath{stroke}%
\end{pgfscope}%
\begin{pgfscope}%
\pgfpathrectangle{\pgfqpoint{0.667731in}{0.650833in}}{\pgfqpoint{4.742417in}{2.714069in}}%
\pgfusepath{clip}%
\pgfsetroundcap%
\pgfsetroundjoin%
\pgfsetlinewidth{2.710125pt}%
\definecolor{currentstroke}{rgb}{0.866667,0.517647,0.321569}%
\pgfsetstrokecolor{currentstroke}%
\pgfsetdash{}{0pt}%
\pgfusepath{stroke}%
\end{pgfscope}%
\begin{pgfscope}%
\pgfpathrectangle{\pgfqpoint{0.667731in}{0.650833in}}{\pgfqpoint{4.742417in}{2.714069in}}%
\pgfusepath{clip}%
\pgfsetroundcap%
\pgfsetroundjoin%
\pgfsetlinewidth{2.710125pt}%
\definecolor{currentstroke}{rgb}{0.866667,0.517647,0.321569}%
\pgfsetstrokecolor{currentstroke}%
\pgfsetdash{}{0pt}%
\pgfusepath{stroke}%
\end{pgfscope}%
\begin{pgfscope}%
\pgfpathrectangle{\pgfqpoint{0.667731in}{0.650833in}}{\pgfqpoint{4.742417in}{2.714069in}}%
\pgfusepath{clip}%
\pgfsetroundcap%
\pgfsetroundjoin%
\pgfsetlinewidth{2.710125pt}%
\definecolor{currentstroke}{rgb}{0.866667,0.517647,0.321569}%
\pgfsetstrokecolor{currentstroke}%
\pgfsetdash{}{0pt}%
\pgfusepath{stroke}%
\end{pgfscope}%
\begin{pgfscope}%
\pgfpathrectangle{\pgfqpoint{0.667731in}{0.650833in}}{\pgfqpoint{4.742417in}{2.714069in}}%
\pgfusepath{clip}%
\pgfsetroundcap%
\pgfsetroundjoin%
\pgfsetlinewidth{2.710125pt}%
\definecolor{currentstroke}{rgb}{0.866667,0.517647,0.321569}%
\pgfsetstrokecolor{currentstroke}%
\pgfsetdash{}{0pt}%
\pgfusepath{stroke}%
\end{pgfscope}%
\begin{pgfscope}%
\pgfpathrectangle{\pgfqpoint{0.667731in}{0.650833in}}{\pgfqpoint{4.742417in}{2.714069in}}%
\pgfusepath{clip}%
\pgfsetroundcap%
\pgfsetroundjoin%
\pgfsetlinewidth{2.710125pt}%
\definecolor{currentstroke}{rgb}{0.866667,0.517647,0.321569}%
\pgfsetstrokecolor{currentstroke}%
\pgfsetdash{}{0pt}%
\pgfusepath{stroke}%
\end{pgfscope}%
\begin{pgfscope}%
\pgfpathrectangle{\pgfqpoint{0.667731in}{0.650833in}}{\pgfqpoint{4.742417in}{2.714069in}}%
\pgfusepath{clip}%
\pgfsetbuttcap%
\pgfsetroundjoin%
\definecolor{currentfill}{rgb}{0.866667,0.517647,0.321569}%
\pgfsetfillcolor{currentfill}%
\pgfsetlinewidth{2.032594pt}%
\definecolor{currentstroke}{rgb}{0.866667,0.517647,0.321569}%
\pgfsetstrokecolor{currentstroke}%
\pgfsetdash{}{0pt}%
\pgfsys@defobject{currentmarker}{\pgfqpoint{-0.046999in}{-0.046999in}}{\pgfqpoint{0.046999in}{0.046999in}}{%
\pgfpathmoveto{\pgfqpoint{0.000000in}{-0.046999in}}%
\pgfpathcurveto{\pgfqpoint{0.012464in}{-0.046999in}}{\pgfqpoint{0.024420in}{-0.042047in}}{\pgfqpoint{0.033234in}{-0.033234in}}%
\pgfpathcurveto{\pgfqpoint{0.042047in}{-0.024420in}}{\pgfqpoint{0.046999in}{-0.012464in}}{\pgfqpoint{0.046999in}{0.000000in}}%
\pgfpathcurveto{\pgfqpoint{0.046999in}{0.012464in}}{\pgfqpoint{0.042047in}{0.024420in}}{\pgfqpoint{0.033234in}{0.033234in}}%
\pgfpathcurveto{\pgfqpoint{0.024420in}{0.042047in}}{\pgfqpoint{0.012464in}{0.046999in}}{\pgfqpoint{0.000000in}{0.046999in}}%
\pgfpathcurveto{\pgfqpoint{-0.012464in}{0.046999in}}{\pgfqpoint{-0.024420in}{0.042047in}}{\pgfqpoint{-0.033234in}{0.033234in}}%
\pgfpathcurveto{\pgfqpoint{-0.042047in}{0.024420in}}{\pgfqpoint{-0.046999in}{0.012464in}}{\pgfqpoint{-0.046999in}{0.000000in}}%
\pgfpathcurveto{\pgfqpoint{-0.046999in}{-0.012464in}}{\pgfqpoint{-0.042047in}{-0.024420in}}{\pgfqpoint{-0.033234in}{-0.033234in}}%
\pgfpathcurveto{\pgfqpoint{-0.024420in}{-0.042047in}}{\pgfqpoint{-0.012464in}{-0.046999in}}{\pgfqpoint{0.000000in}{-0.046999in}}%
\pgfpathlineto{\pgfqpoint{0.000000in}{-0.046999in}}%
\pgfpathclose%
\pgfusepath{stroke,fill}%
}%
\begin{pgfscope}%
\pgfsys@transformshift{1.141973in}{1.797423in}%
\pgfsys@useobject{currentmarker}{}%
\end{pgfscope}%
\begin{pgfscope}%
\pgfsys@transformshift{2.090456in}{2.087194in}%
\pgfsys@useobject{currentmarker}{}%
\end{pgfscope}%
\begin{pgfscope}%
\pgfsys@transformshift{3.038940in}{0.863657in}%
\pgfsys@useobject{currentmarker}{}%
\end{pgfscope}%
\begin{pgfscope}%
\pgfsys@transformshift{3.987423in}{0.774199in}%
\pgfsys@useobject{currentmarker}{}%
\end{pgfscope}%
\begin{pgfscope}%
\pgfsys@transformshift{4.935906in}{1.093456in}%
\pgfsys@useobject{currentmarker}{}%
\end{pgfscope}%
\end{pgfscope}%
\begin{pgfscope}%
\pgfsetrectcap%
\pgfsetmiterjoin%
\pgfsetlinewidth{1.254687pt}%
\definecolor{currentstroke}{rgb}{1.000000,1.000000,1.000000}%
\pgfsetstrokecolor{currentstroke}%
\pgfsetdash{}{0pt}%
\pgfpathmoveto{\pgfqpoint{0.667731in}{0.650833in}}%
\pgfpathlineto{\pgfqpoint{0.667731in}{3.364902in}}%
\pgfusepath{stroke}%
\end{pgfscope}%
\begin{pgfscope}%
\pgfsetrectcap%
\pgfsetmiterjoin%
\pgfsetlinewidth{1.254687pt}%
\definecolor{currentstroke}{rgb}{1.000000,1.000000,1.000000}%
\pgfsetstrokecolor{currentstroke}%
\pgfsetdash{}{0pt}%
\pgfpathmoveto{\pgfqpoint{5.410148in}{0.650833in}}%
\pgfpathlineto{\pgfqpoint{5.410148in}{3.364902in}}%
\pgfusepath{stroke}%
\end{pgfscope}%
\begin{pgfscope}%
\pgfsetrectcap%
\pgfsetmiterjoin%
\pgfsetlinewidth{1.254687pt}%
\definecolor{currentstroke}{rgb}{1.000000,1.000000,1.000000}%
\pgfsetstrokecolor{currentstroke}%
\pgfsetdash{}{0pt}%
\pgfpathmoveto{\pgfqpoint{0.667731in}{0.650833in}}%
\pgfpathlineto{\pgfqpoint{5.410148in}{0.650833in}}%
\pgfusepath{stroke}%
\end{pgfscope}%
\begin{pgfscope}%
\pgfsetrectcap%
\pgfsetmiterjoin%
\pgfsetlinewidth{1.254687pt}%
\definecolor{currentstroke}{rgb}{1.000000,1.000000,1.000000}%
\pgfsetstrokecolor{currentstroke}%
\pgfsetdash{}{0pt}%
\pgfpathmoveto{\pgfqpoint{0.667731in}{3.364902in}}%
\pgfpathlineto{\pgfqpoint{5.410148in}{3.364902in}}%
\pgfusepath{stroke}%
\end{pgfscope}%
\begin{pgfscope}%
\pgfsetbuttcap%
\pgfsetmiterjoin%
\definecolor{currentfill}{rgb}{0.917647,0.917647,0.949020}%
\pgfsetfillcolor{currentfill}%
\pgfsetfillopacity{0.800000}%
\pgfsetlinewidth{1.003750pt}%
\definecolor{currentstroke}{rgb}{0.800000,0.800000,0.800000}%
\pgfsetstrokecolor{currentstroke}%
\pgfsetstrokeopacity{0.800000}%
\pgfsetdash{}{0pt}%
\pgfpathmoveto{\pgfqpoint{4.211472in}{2.592333in}}%
\pgfpathlineto{\pgfqpoint{5.303204in}{2.592333in}}%
\pgfpathquadraticcurveto{\pgfqpoint{5.333759in}{2.592333in}}{\pgfqpoint{5.333759in}{2.622888in}}%
\pgfpathlineto{\pgfqpoint{5.333759in}{3.257957in}}%
\pgfpathquadraticcurveto{\pgfqpoint{5.333759in}{3.288513in}}{\pgfqpoint{5.303204in}{3.288513in}}%
\pgfpathlineto{\pgfqpoint{4.211472in}{3.288513in}}%
\pgfpathquadraticcurveto{\pgfqpoint{4.180916in}{3.288513in}}{\pgfqpoint{4.180916in}{3.257957in}}%
\pgfpathlineto{\pgfqpoint{4.180916in}{2.622888in}}%
\pgfpathquadraticcurveto{\pgfqpoint{4.180916in}{2.592333in}}{\pgfqpoint{4.211472in}{2.592333in}}%
\pgfpathlineto{\pgfqpoint{4.211472in}{2.592333in}}%
\pgfpathclose%
\pgfusepath{stroke,fill}%
\end{pgfscope}%
\begin{pgfscope}%
\definecolor{textcolor}{rgb}{0.150000,0.150000,0.150000}%
\pgfsetstrokecolor{textcolor}%
\pgfsetfillcolor{textcolor}%
\pgftext[x=4.513726in,y=3.111661in,left,base]{\color{textcolor}\sffamily\fontsize{12.000000}{14.400000}\selectfont F-score}%
\end{pgfscope}%
\begin{pgfscope}%
\pgfsetbuttcap%
\pgfsetroundjoin%
\definecolor{currentfill}{rgb}{0.298039,0.447059,0.690196}%
\pgfsetfillcolor{currentfill}%
\pgfsetlinewidth{2.032594pt}%
\definecolor{currentstroke}{rgb}{0.298039,0.447059,0.690196}%
\pgfsetstrokecolor{currentstroke}%
\pgfsetdash{}{0pt}%
\pgfsys@defobject{currentmarker}{\pgfqpoint{-0.046999in}{-0.046999in}}{\pgfqpoint{0.046999in}{0.046999in}}{%
\pgfpathmoveto{\pgfqpoint{0.000000in}{-0.046999in}}%
\pgfpathcurveto{\pgfqpoint{0.012464in}{-0.046999in}}{\pgfqpoint{0.024420in}{-0.042047in}}{\pgfqpoint{0.033234in}{-0.033234in}}%
\pgfpathcurveto{\pgfqpoint{0.042047in}{-0.024420in}}{\pgfqpoint{0.046999in}{-0.012464in}}{\pgfqpoint{0.046999in}{0.000000in}}%
\pgfpathcurveto{\pgfqpoint{0.046999in}{0.012464in}}{\pgfqpoint{0.042047in}{0.024420in}}{\pgfqpoint{0.033234in}{0.033234in}}%
\pgfpathcurveto{\pgfqpoint{0.024420in}{0.042047in}}{\pgfqpoint{0.012464in}{0.046999in}}{\pgfqpoint{0.000000in}{0.046999in}}%
\pgfpathcurveto{\pgfqpoint{-0.012464in}{0.046999in}}{\pgfqpoint{-0.024420in}{0.042047in}}{\pgfqpoint{-0.033234in}{0.033234in}}%
\pgfpathcurveto{\pgfqpoint{-0.042047in}{0.024420in}}{\pgfqpoint{-0.046999in}{0.012464in}}{\pgfqpoint{-0.046999in}{0.000000in}}%
\pgfpathcurveto{\pgfqpoint{-0.046999in}{-0.012464in}}{\pgfqpoint{-0.042047in}{-0.024420in}}{\pgfqpoint{-0.033234in}{-0.033234in}}%
\pgfpathcurveto{\pgfqpoint{-0.024420in}{-0.042047in}}{\pgfqpoint{-0.012464in}{-0.046999in}}{\pgfqpoint{0.000000in}{-0.046999in}}%
\pgfpathlineto{\pgfqpoint{0.000000in}{-0.046999in}}%
\pgfpathclose%
\pgfusepath{stroke,fill}%
}%
\begin{pgfscope}%
\pgfsys@transformshift{4.394805in}{2.936025in}%
\pgfsys@useobject{currentmarker}{}%
\end{pgfscope}%
\end{pgfscope}%
\begin{pgfscope}%
\definecolor{textcolor}{rgb}{0.150000,0.150000,0.150000}%
\pgfsetstrokecolor{textcolor}%
\pgfsetfillcolor{textcolor}%
\pgftext[x=4.669805in,y=2.895921in,left,base]{\color{textcolor}\sffamily\fontsize{11.000000}{13.200000}\selectfont F1 micro}%
\end{pgfscope}%
\begin{pgfscope}%
\pgfsetbuttcap%
\pgfsetroundjoin%
\definecolor{currentfill}{rgb}{0.866667,0.517647,0.321569}%
\pgfsetfillcolor{currentfill}%
\pgfsetlinewidth{2.032594pt}%
\definecolor{currentstroke}{rgb}{0.866667,0.517647,0.321569}%
\pgfsetstrokecolor{currentstroke}%
\pgfsetdash{}{0pt}%
\pgfsys@defobject{currentmarker}{\pgfqpoint{-0.046999in}{-0.046999in}}{\pgfqpoint{0.046999in}{0.046999in}}{%
\pgfpathmoveto{\pgfqpoint{0.000000in}{-0.046999in}}%
\pgfpathcurveto{\pgfqpoint{0.012464in}{-0.046999in}}{\pgfqpoint{0.024420in}{-0.042047in}}{\pgfqpoint{0.033234in}{-0.033234in}}%
\pgfpathcurveto{\pgfqpoint{0.042047in}{-0.024420in}}{\pgfqpoint{0.046999in}{-0.012464in}}{\pgfqpoint{0.046999in}{0.000000in}}%
\pgfpathcurveto{\pgfqpoint{0.046999in}{0.012464in}}{\pgfqpoint{0.042047in}{0.024420in}}{\pgfqpoint{0.033234in}{0.033234in}}%
\pgfpathcurveto{\pgfqpoint{0.024420in}{0.042047in}}{\pgfqpoint{0.012464in}{0.046999in}}{\pgfqpoint{0.000000in}{0.046999in}}%
\pgfpathcurveto{\pgfqpoint{-0.012464in}{0.046999in}}{\pgfqpoint{-0.024420in}{0.042047in}}{\pgfqpoint{-0.033234in}{0.033234in}}%
\pgfpathcurveto{\pgfqpoint{-0.042047in}{0.024420in}}{\pgfqpoint{-0.046999in}{0.012464in}}{\pgfqpoint{-0.046999in}{0.000000in}}%
\pgfpathcurveto{\pgfqpoint{-0.046999in}{-0.012464in}}{\pgfqpoint{-0.042047in}{-0.024420in}}{\pgfqpoint{-0.033234in}{-0.033234in}}%
\pgfpathcurveto{\pgfqpoint{-0.024420in}{-0.042047in}}{\pgfqpoint{-0.012464in}{-0.046999in}}{\pgfqpoint{0.000000in}{-0.046999in}}%
\pgfpathlineto{\pgfqpoint{0.000000in}{-0.046999in}}%
\pgfpathclose%
\pgfusepath{stroke,fill}%
}%
\begin{pgfscope}%
\pgfsys@transformshift{4.394805in}{2.723120in}%
\pgfsys@useobject{currentmarker}{}%
\end{pgfscope}%
\end{pgfscope}%
\begin{pgfscope}%
\definecolor{textcolor}{rgb}{0.150000,0.150000,0.150000}%
\pgfsetstrokecolor{textcolor}%
\pgfsetfillcolor{textcolor}%
\pgftext[x=4.669805in,y=2.683016in,left,base]{\color{textcolor}\sffamily\fontsize{11.000000}{13.200000}\selectfont F1 macro}%
\end{pgfscope}%
\end{pgfpicture}%
\makeatother%
\endgroup%
}
        \label{fig:node2vec:number and walk length}
        \caption{Placeholder since plot is missing.}
    \end{subfigure}
    \caption{Node2vec and FakeNewsNet results.}
    \label{fig:deepwalk:plots}
\end{figure}

